

%------------------------------------------------------------------------
%------------------------------------------------------------------------
%------------------------------------------------------------------------

\chapter{Introduction}
\shwlabel{ChapterIntroduction}

 The calibration of the SNO detector can be divided into several
aspects:
\begin{enumerate}
\item Calibration of the low level electronics channels (ECA).
\item Calibration of phototube timing (PCA).
\item Calibration of the detector's physics response.
\end{enumerate}
The low level electronics calibrations (ECA) are done via calibration
circuitry built into the electronics hardware and are not considered
further here.  The scope of this document is the equipment and procedures
used to do the phototube calibrations (PCA) and the multitude of 
physics calibrations.
  
  The phototube and physics calibration of the SNO detector consists
of placing standard sources of light or radioactivity (gamma rays and
electrons) at known locations within the detector.  These sources
are used to extract the calibration parameters for both the individual
phototubes (timing resolution, charge vs time, charge resolution,
efficiency) and bulk properties of the detector (wavelength
dependent light attenuation of materials, global detection efficiency).

  To achieve the calibrations there are special calibration {\em sources}
prepared for SNO.  These sources are light sources such as the 
{\bf laserball} or radioactive sources such as the 
{\bf $^{16}$N Gamma Ray Source},
the {\bf $^{8}$Li Electron Source}, 
or the {\bf $^{252}$Cf Neutron Source}.
Each source has unique physical properties but they all share the
requirements of being constructed of materials with little or no
radioactivity (accept for the source material in the case of active
sources).  These sources are positioned within the SNO detector
using the {\bf Calibration Manipulator System}.  The manipulator 
is a computer controlled system that is designed to place a source
within either the inner detector volume (the AV) or in the light
water between the AV and PSUP.  The manipulator's nominal positioning
accuracy is 5cm. 



%=====================================================================

\section{Overview of the Calibration Apparatus}
\shwlabel{SecOverview}

  Calibration sources are deployed into the SNO detector using
the {\em Calibration Source Manipulator} system shown schematically
in figure \ref{figmansystem}. 
\begin{figure}[htbp]
\begin{center}
\leavevmode
%\epsfysize=0.85\textheight
\epsfxsize=5.0in
\epsfbox{figures/mansystem.ps}
\caption[Schmatic layout of the SNO manipulator]{
  SNO manipulator System (Not to Scale).  The calibration
  source manipulator is a system of ropes the allow the positioning
  of calibration sources inside SNO detector.  Either within the AV
  or between the AV and the PSUP.
  \shwlabel{figmansystem}}
\end{center}
\end{figure}
  Sources
are mounted on {\em Umbilical Retreival Mechanisms} ({\em URMs})
and then mounted on either the {\em Glovebox} or {\em Calibration
Guide Tubes} and then lowered into the detector.  The apparatus is
designed to maintain the {\em Cover Gas} seal on the detector.  Prior
to opening valves to the detector volume, air is flushed out of the
URMs with pure N$_2$ gas derived from liquid nitrogen dewars located
in the Junction.  The apparatus also maintains the light seal on the
detector allowing the deployment of sources (with the acception of
the calibration camera) with the detector turned on.


  The manipulator and calibration sources are located in the 
{\em Deck Clean Room} which is located at the centre of the deck
over the detector.  It is shown in figure \ref{FigDcr}. 
\begin{figure}
\begin{center}
\leavevmode
\epsfxsize=7in
\epsfbox{figures/sno_dcr.ps}
~\\
\caption[Deck Clean Room]
        {The Deck Clean Room
         \shwlabel{FigureDcr}
        }
\end{center}
\end{figure} 
Inside the DCR are the URMs, the manipulator control computer,
the calibration laser, storage cabinets for the sources, lifting
mechanisms to move the URMs on and off the glovebox and carts for
moving the URMs about in the DCR. 

  There are presently three URMs in use for the calibration sources.
URM1 is kept for ``guest sources'' that are not used on a regular basis.
It is usually mounted on gatevalve 1 on the glove box (10'' gatevalve)
or on the calibration guide tubes.  URM2 is used for the Laserball
optical calibration source and is usually mounted on gatevalve 1 on
the glovebox.  The design of the Laserball hardware is such that
it can be quickly removed and acrylic encapsulated radiactive sources
mounted instead.  This is occasionally done on URM2 but it is usually
restored to the Laserball configuration.  URM3 is the permanent home
of the  $^{16}$N gamma-ray source and is usually mounted on gatevalve
3 (4'' valve) on the glovebox.  Because disassembly of the $^{16}$N
source is and involved process, this URM is rarely used for other
purposes.



%=====================================================================
\newpage
\section{Contacts}

Below are names and phone numbers for people familiar with different
parts of the calibration system.  However, if there is any question
about the state and safety of the system, contact the {\bf On Call Expert}
(OCE)
who is the person carrying the calibration pager.  Prior to any running
of the manipulator the head of the calibration group or his designate will
indicate who the On Call Expert is.

\begin{center}
\begin{tabular}{|l|l|rl|}
\hline
         &           &             &                             \\
Calibration Pagers   &             &  (705)669-8813 &  Sudbury  \\
         &                         &  (613)548-2739 &  Kingston   \\
         &             &               & \\
\hline
Peter Skensved & Manipulator & (613)533-2676 &  Queen's Office \\
               & Laser       & (613)376-3491 &  Home  \\
               & N16         & (705)692-5892 & House Lively \\
               &             & (613)888-9573 & Cell Phone \\
               &             &          x221 & Office on Site \\
\hline
Fraser Duncan & Manipulator &          x205 & Office on Site \\
              & Laser       & (613)549-3360 & Home Kingston\\
              & N16         & (705)522-4893 & Home Sudbury \\
              &             & (705)561-0453 & Cell phone \\
\hline
Phil Harvey    & Manipulator  & (613)533-6000 \  x77788     & Office Queen's   \\
               &                      & (613)389-9154 &  Home  \\
\hline
Aksel Hallin  & Manipulator & (613)533-6766 & Office Queen's \\
              & Laser       & (613)634-1571 & Home \\
              & N16         &               &  \\
\hline
\end{tabular}
\end{center}   


\newpage
\section{Rules of (Dis)Engagement}

  Chapter \ref{ChapterTroubleShooting} of these procedures describe 
possible problems
with the manipulator system and how to correct them.  Some problems are
straight forward and are of little consequence.  These problems can be
corrected by the  operator.  Some problems are significant and the
On Call Expert (OCE) should be contacted before addressing them.  
The OCE is in turn required to inform the Head of the Calibration Group
or dhis designate 
of any significant problem (here significant is defined as a problem
that could potentially put at risk the manipulator system, the detector
or the heavy water). 
  
{\bf
\begin{enumerate}
\item Contact the OCE for any problems that the trouble shooting 
  section indicate the OCE should be contacted.
\item Contact the OCE if you have doubts about a procedure.
\item Contact the OCE if you have had to execute any of the 
  emergency shutdown procedures.
\item Contact the OCE if any abnormal situation occurs.
\end{enumerate}
}
