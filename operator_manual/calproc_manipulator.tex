  
%------------------------------------------------------------------------
%------------------------------------------------------------------------
%------------------------------------------------------------------------
\section{Manipulator Operation Procedures}
\shwlabel{secprocman}
 
  This section contains procedures related to the normal operation
of the manipulator system.  Procedures for specific sources or
tasks may supercede some of the procedures here.



%------------------------------------------------------------
\newpage
\subsection{Manipulator System Shutdown}


\newprocedure{CalProcManipulatorShutdown}
             {Manipulator System Shutdown}
             {Fraser Duncan/ Peter S.}{Sept. 2004}{1}


  
  The purpose of this procedure is to shutdown the calibration manipulator
electronics in an orderly fashion.  The circumstances when this should be
done are when there is a scheduled power outage to the underground lab.
Except in an obvious emergency, the manipulator computer should only be
shutdown with the permission of the Calibration Group {\bf and} the AV group.
  
\noindent
{\bf Outline of Procedure:}
\begin{itemize}
  
\item Stop the manipulator control program.
  
\item Turn off the manipulator computer.
  
\item Turn off the manipulator computer monitor.
  
\item Turn off the data concentrator.
  
\item Turn off the watchdog timer box.

\end{itemize}  

\noindent  
{\bf Prior to Starting this procedure:}
\begin{itemize}
   
\item Obtain permission to shutdown the manipulator computer from 
    the Calibration Group and the AV Group.
  
\item Verify that access to the DCR can be obtained.
\end{itemize}
  
\noindent
{\bf Procedure:}
  
\begin{enumerate}
\checkitem Enter the DCR.  The Manipulator electronics are in "Aksel's Garage",
     the alcove immediately to the right of the entry way to the DCR.  
  
\checkitem If the monitor is off, turn it on.  
     The manip program should be running.  This can be seen by the
     {\small\begin{verbatim} 
         manip>
     \end{verbatim}}
     prompt at the bottom of the screen.
    
\checkitem At the prompt, type the command
     {\small\begin{verbatim}
         quit
     \end{verbatim}}
     The manip program should shutdown either returning to a DOS prompt,
     {\small\begin{verbatim}
         C:\MOTORS.3_8\MANIP>
     \end{verbatim}}
     or it will display a message stating that the program will restart
     in 5 seconds.  You wish to prevent the program from restarting so
     type the letter "n" at the prompt. 

\small
{\em
Note that the version number may not
be exactly what is indicated above - this is ok. Also note
that sometimes the program stalls while exiting ( it has hung network connections ).
  Regardless of this procede to step 4 after waiting some 10 seconds. }
  \normalsize
  
\checkitem Press RESET and then turn off the manip computer as soon as the BIOS screen appear.  The power button is located on 
    the front.\\
     {\small\em Hitting RESET interrupts the hardware ( disks especially ) cleanly thus minimizing any
potential damage.
         The computer lights may be covered with aluminum tape, you may have to lift
         the tape to see the lights. Alternatively, check the monitor screen.
     }
  
\checkitem Turn off the monitor.
  
\checkitem Turn off the data concentrator box.  This is the box on the top
     shelf of the equipment rack.  When standing in front of it (the
     side with the label and cables) the power switch is on the back at
     the top left.  Off is in the down position.
  
\checkitem Turn off the watchdog timer box.  This is the box on the 
  second shelf of the equipment rack beside the computer.  Looking at the front
  of the box (the side with the label and the cables) the power switch
  is located on the back left side at the bottom.   Off is in the
  down position.
 
\end{enumerate}


{\small
~\\
~\\
\noindent
{\bf Revision History:}\\
\begin{tabular}{llll}
Rev. & Date & Author & Comments\\

0             & 
?    & 
Fraser Duncan &
\parbox[t]{3.0in}{
  First Draft
}
\end{tabular}
}





%===========================================================================
%===========================================================================
%===========================================================================

\newpage
\subsection{Manipulator System Startup}

\newprocedure{CalProcManipulatorStartup}
             {Manipulator System Startup}
             {Fraser Duncan/ Peter S}{?}{1}

  
  The purpose of this procedure is to start the manipulator control computer
after it has been turned off.  This procedure should only be done with
the permission of the Calibration Group.
    
\noindent
{\bf Outline of Procedure:}
\begin{itemize}  
\item Turn on the manipulator computer monitor.
  
\item Turn on the data concentrator.
  
\item Turn on the watchdog timer.
  
\item Turn on the computer.
  
\item Verify that the manip program has started correctly.
  
\item Verify that the CMA system has connected to the manipulator
    computer. 
\end{itemize}  
  
\noindent
{\bf Prior to Starting this procedure:}
\begin{itemize}   
\item Obtain permission to start the manipulator computer from 
    the Calibration Group.
  
\item Verify that access to the DCR can be obtained.
\end{itemize}
  

\noindent
{\bf Procedure:}
\begin{enumerate}  
\checkitem Enter the DCR.  The Manipulator electronics are in "Aksel's Garage",
     the alcove immediately to the right of the entry way to the DCR.  
  
\checkitem If the monitor is off, turn it on.
  
\checkitem Turn on the Data Concentrator Box.  This is the box on the top
     shelf of the equipment rack.  When standing in front of it (the
     side with the label and cables) the power switch is on the back at
     the top left.  On is in the up position.
  
\checkitem Turn on the Watchdog Timer Box.   This is the box on the second shelf
     of the equipment rack beside the computer.  Looking at the front
     of the box (the side with the label and the cables) the power switch
     is located on the back left side at the bottom.   On is in the
     up position.
  
\checkitem Get ready to turn on the Manipulator Computer.  This is the computer on the
     second shelf of the equipment rack.  The power switch is on the
     front.  When turned on the computer will go through the normal BIOS and DOS startup procedure
     and then ask you to verify that the hardware ( Data Concentrator and Watchdog Timer )
    has been turned on. Answer appropriately. Note that you have to respond within about
10 seconds. It is {\bf always} safe to answer N ( for no ) to this question.

\small
{\em Turning on manip without turning on the hardware will usually lead to a loss of 
important and vital calibration  information ( like rpoe positions ). 
 }
\normalsize

\checkitem Turn on the Manipulator Computer and answer the question about the hardware state.
Once running the screen should
     display the manipulator status and have a 
     \begin{verbatim}
         MANIP>
\end{verbatim}
     prompt at the bottom.


\small

{\em If you for any reason answered ``N'' the computer will end up with a normal DOS prompt. 
Simply type {\tt MANRUN} to continue. }

\normalsize



\checkitem Watch for the CMA computer to connect to the manipulator 
     computer.  This will be indicated by a message at the bottom of
     the screen indicating that a connection has been made from the
     IP address,
                      142.51.70.153
     The connection should be made within 20 seconds of the start of
     the program.  If the connection is not made within a minute or
     so, inform the AV group.
   
\end{enumerate}




{\small
~\\
~\\
\noindent
{\bf Revision History:}\\
\begin{tabular}{llll}
Rev. & Date & Author & Comments\\

0             & 
?    & 
Fraser Duncan &
\parbox[t]{3.0in}{
  First Draft
}
\end{tabular}
}



%====================================================================
\newpage
\subsection{Remote Reboot of the Calibration Manipulator Computer}

\newprocedure{ProcManipReboot}
             {Remote Reboot of the Calibration Manipulator Computer}
             {P. Skensved}{2002/10/26}{0}


  This procedure is used to reboot the calibration manipulator computer,
{\em manip}.  It is used if the computer freezes up or most commonly
if it refuses a TCP connection.  Rebooting the {\em manip} computer
should only be done with the approval of the SIC, SAC or the Calibration
Lead Hand.

\begin{enumerate}
\checkitem Log on to either {\em alcor}, {\em crag1}, {\em crag2},
  {\em crug1} or  {\em crug2}.

\checkitem run the {\bf wakelan} program by typing
  \begin{verbatim}
          wakelan 0004e21cc5eb
  \end{verbatim}

\checkitem Wait for {\em manip} to reboot (takes approximately 30 seconds).
  
\end{enumerate}


{\small
~\\
~\\
\noindent
{\bf Revision History:}\\
\begin{tabular}{llll}
Rev. & Date & Author & Comments\\
0           & 
2002/10/26  & 
Peter Skensved &
\parbox[t]{3.0in}{
  First Draft
}\\
\end{tabular}
}






%=================================================================
\newpage
\subsection{URM Light Leak Check}
\newprocedure{CalProcLightLeakCheck}
             {URM Light Leak Check}
             {Fraser Duncan}{2002/11/10}{1}


  After a URM has been opened up (cover plate taken off or removed
from the  glovebox), it is necessary to do a check for light  leaks.
This is done using the 24 OWL tube light monitor on the detector
itself.  This monitor consists of doing singles rate reads of the
top 24 OWL tubes that look up towards the deck and the DCR and 
glovebox.  


\begin{enumerate}
\checkitem Contact detector operator, verify that either the detector
  is in a maintenance run or has the UC bit set.
\checkitem Turn on Owl Tube Light Monitor
\checkitem Turn off DCR lights.
\checkitem While Watching the Owl Tube Light Monitor:
  \begin{enumerate}
  \item Open the gate valve for the URM being lightleak checked.
  \item Shine flashlight around the gate valve, 
         and then around all seals on URM
  \end{enumerate}
  %----------------------
  \small
  {\em
    Note that the light monitor only updates once a second.  The
    flashlight must be moved at an appropriate speed.
  }
  \normalsize
  %----------------------
\end{enumerate}
 
  



%=================================================
{\small
~\\
~\\
\noindent
{\bf Revision History:}\\
\begin{tabular}{llll}
Rev. & Date & Author & Comments\\
0           & 
?  & 
Fraser Duncan &
\parbox[t]{3.0in}{
  First draft
}\\

1             & 
2002/11/10    & 
Fraser Duncan &
\parbox[t]{3.0in}{
  Made procedure more general.
}
\end{tabular}
}



%=================================================================
\newpage
\subsection{URM Central Rope Length Calibration}
\newprocedure{CalProcCentralRope}
             {URM Central Rope Length Calibration}
             {Fraser Duncan}{2002/11/10}{1}

  The length calibration of the central rope for each URM is determined
by sighting the pivot of the manipulator carriage 
(see figure \ref{figmancarriage}) against a fiducial line scribed on the
window of the source tube viewing port. 


\begin{center}
\begin{tabular}{|l|c|}
\hline
   & \\
 URM2  on GV 1 with wide 4'' Source Tube & 1559.9 \\
  & \\
\hline
  & \\
 URM3  on GV 3 with normal 4'' Source Tube & 1558.5 \\
  & \\
\hline
\end{tabular}
\end{center}    


 The height of the fiducial
mark is also indicated on the side of each source tube.  If the number on the source tube
differs from the one listed above use the number written on the tube.





\noindent
{\bf Prior to Procedure:}
\begin{enumerate}
\item Source is above gate valve.
\item Gate valve is {\bf closed}.
\item Gate valve is locked or handle is removed.
\end{enumerate}

\noindent
{\bf Procedure:}
\begin{enumerate}
\checkitem Verify gatevalve  on glovebox below source tube is
  locked in the {\bf CLOSED} position.

\checkitem Open view port on source tube.  Requires 7/16" wrench.

\checkitem Operate manipulator until the centre of the manipulator
      carriage is at the horizontal line marked on window.\\
      {\small\em Note: The example below 
       assumes the n16 source.  for the laserball or
        a different source replace the object name {\tt n16} below as
        appropriate.}\\
      From the manip console:
  \begin{center}
  \begin{tabular}{|l|l|}
  \hline
  console & {\tt manip$>$ n16 by $<$dx$>$ $<$dy$>$ $<$dz$>$ } \\
  \hline
  \end{tabular}
  \end{center}  
  For example,
  \begin{verbatim}
                    n16 by  0 0 2
  \end{verbatim}
        moves the n16 2 cm up, and
   \begin{verbatim}
                    n16 by 0 0 -0.5
   \end{verbatim}
   moves the n16 0.5 cm down.

\checkitem Set the calibration in the manip program:
  \begin{center}
  \begin{tabular}{|l|l|}
  \hline
  console & {\tt manip$>$ n16 locate 0 0 1558.5 } \\
  \hline
  \end{tabular}
  \end{center}  
  %--------------------------
  \small
  {\em
    The position 1558.5 is the location of the calibration mark on
    the view port window.  It was determined by measuring the height
    of the source tube and the location of the AV below deck.  
  }
  \normalsize
  %--------------------------

\checkitem Reseal view port window.

\checkitem When approriate (after the URM has been radon flushed) perform
  a light leak check (see procedure \ref{CalProcLightLeakCheck}.

\end{enumerate}


%=================================================
{\small
~\\
~\\
\noindent
{\bf Revision History:}\\
\begin{tabular}{llll}
Rev. & Date & Author & Comments\\
0           & 
?  & 
Fraser Duncan &
\parbox[t]{3.0in}{
  First draft
}\\

1             & 
2002/11/10    & 
Fraser Duncan &
\parbox[t]{3.0in}{
  Made procedure more general.  Corrected fiducial mark for
  URM2.
}
\end{tabular}
}






%==========================================================================
%==========================================================================
%==========================================================================
%==========================================================================



\newpage
\subsection{Calibrating East/West Side Ropes}
\shwlabel{secsideropes}
\vspace*{0.25in}
\noindent
\newprocedure{CalProcEastWestSideRopes}
             {Calibrating East/West Side Ropes}
             {Fraser Duncan/Peter S.}{Sept 2004}{2}

Before each use the side ropes require both a tension calibration
and a length calibration.

\subsubsection{Calibrating Side Rope Tension Offsets}
\shwlabel{secsidtension}
The load cells that measure rope tension are prone to have their
offsets drift.  I.e. although the slope of the loadcell calibration
does not change, the apparent zero tension point drifts.  This
is potentially very bad since when operating the manipulator with
side ropes on, it is necessary to go down to low tension (low is
on the order of 5N or less).\\
{\bf
  If the ropes are operated at zero tension they will unspool from
  the takeup reels in the motor units resulting in tangling and
  requiring major intervention.\\
}
Therefore this procedure to reset the zeros on the loadcells is
important.  Unfortunately it involves taking all the tension off
the rope units and thus risks the same problem it is trying to 
prevent.\\
{\bf
  Extreme care must be taken when performing this procedure. 
}

\vspace*{0.25in}
\noindent
{\bf Calibrating the ( East ) Side Rope Tension Offset}

 In the following we outline  the steps needed to calibrate the East Rope. The 
calibration of the other ropes is identical except for the obvious change of object name.
Note that the offset on the southrope is weird and that one may have to ``lie'' to
it to get the proper offset. Contact the OCE if you're trying to calibrate the South Rope
and do not understand what to do.


\begin{enumerate}


\checkitem Obtain permission from OCE to (re- ) calibrate the sideropes.

\checkitem Go to expert mode at the manipulator console

  \begin{center}
  \begin{tabular}{|l|l|}
  \hline
  console & {\tt manip$>$ expert room601} \\
  \hline
  \end{tabular}
  \end{center}
  {\small\em Note: Expert mode has a 30 minute time out.  If 
   you take longer than this it will be necessary to reenter expert mode.}

\checkitem  Drive out 30 to 40 cm of rope under constant tension.
\begin{itemize}
\item
     Have one person apply tension to the rope in question while another sets the
tope in constant tension mode. For eaxmple `:
  \begin{center}
  \begin{tabular}{|l|l|}
  \hline
  console & {\tt manip$>$ eastrope tension 15} \\
  \hline
  \end{tabular}
  \end{center}
\small
{\em  In tension mode the motor will attempt to keep the rope under constant tension. However,
motors have a maximum speed of 3 cm per second so whatever you do {\bf do it slowly  !}
 Note that a STOP command ( or really low tension ) causes MANIP to exit tension mode. }
\normalsize
\item The person at the glovebox can now pull out the desired amount of rope. {\bf Do it
slowly !}
\item Wait for the motor to stop. ( Listen )
\item The person at the console hits the ESC key or types STOP.
\item The person holding the rope may now slack it off. {\bf Do it gently !}
\end{itemize}
\small
{\em If the rope is not completely slack repeat the above steps }
\normalsize


%While one person maintains tension on the rope by hand 
%  an operator at the {\bf manip} console plays out
%  approximately 30-40cm of rope.
%  \begin{center}
%  \begin{tabular}{|l|l|}
%  \hline
%  console & {\tt manip$>$ eastrope down 30} \\
%%  \hline
%  \end{tabular}
%  \end{center}

%\checkitem {\bf Gently} release the tension from the rope. \\
%   {\em Do not push the rope back up into the motor mount.\\
%   If the the rope does not go slack, re-apply tension
%   by hand and  play out more rope.\\

\checkitem Check what the tension is by doing a   
  \begin{center}
  \begin{tabular}{|l|l|}
  \hline
  console & {\tt manip$>$ eastrope monitor} \\
  \hline
  \end{tabular}
  \end{center}
 If the tension is within 0.2 N of zero there is no need to do the next two steps


\checkitem Calibrate the loadcell offset. {\bf Make sure the rope really has zero tension !} \\
\small
{\em The {\tt calibrate} command is a ``toggle'' command}
\normalsize

  \begin{center}
  \begin{tabular}{|l|l|}
  \hline
  console & {\tt manip$>$ eastrope calibrate} \\
          & {\tt manip$>$ eastrope point 0 N} \\
          & {\tt manip$>$ eastrope calibrate} \\
  \hline
  \end{tabular}
  \end{center}
\small
  {\em It is important that only {\bf one} calibration point is used while the rope is in
{\tt calibrate} mode. Two or more points will change the slope of the calibration as well.
Thus, if you happen to mistype the {\tt point 0N} do {\bf not } under any circumstances
just retype the {\tt point}  command !  Instead, complete the calibration ( ie. exit {\tt calibration}  mode and re-do all three steps.}
\normalsize 

\checkitem Check that the rope tension now reads 0 by using the

  monitor command,
  \begin{center}
  \begin{tabular}{|l|l|}
  \hline
  console & {\tt manip$>$ eastrope monitor} \\
  \hline
  \end{tabular}
  \end{center}
  and reading the rope tension.

\checkitem Wind the rope back in under constant tension.
\begin{itemize}
\item The person at the glovebox applies tension to the rope.
\item Set the rope in constant tension mode
  \begin{center}
  \begin{tabular}{|l|l|}
  \hline
  console & {\tt manip$>$ eastrope tension 15} \\
  \hline
  \end{tabular}
  \end{center}
\item The person holding the rope may now {\bf gently} let the motor take in the excess
rope. {\bf Remember : slow movements only !} 
\item Once the excess rope has been taken up the person at the console types STOP ( or hits the ESC key )
\small
{\em Listen to the motor while you do this. Wait until it settles down before stopping}
\normalsize

\end{itemize}
%  \begin{center}
%  \begin{tabular}{|l|l|}
%  \hline
%  console & {\tt manip$>$ eastrope up 30} \\
%  \hline
%  \end{tabular}
%  \end{center}
%  The self tension of the rope should be about 10N.

  
\end{enumerate}


\vspace*{0.25in}
\noindent
{\bf Calibrating the  Side Rope Tension Offset}

The procedure for the west rope is identical to that for
the east rope.

  
%%\end{enumerate}




\subsubsection{Calibrating the Side Rope Lengths}
\shwlabel{secsidelength}
  Now that the side rope tension offsets have been
calibrated, it is necessary to calibrate the rope
lengths.  This is done by calculating the rope length
based on the positions of the rope attachment points.
Before the actual calibration of the length is done,
the side ropes are pulled tight to high tension and
then relaxed.  This is to prestretch the ropes.
 
\begin{enumerate}

\checkitem Go to expert mode at the manipulator console

  \begin{center}
  \begin{tabular}{|l|l|}
  \hline
  console & {\tt manip$>$ expert room601} \\
  \hline
  \end{tabular}
  \end{center}
  {\small\em Note: Expert mode has a 30 minute time out.  If 
   you take longer than this it will be necessary to reenter expert mode.}

\checkitem Run the siderope calibration command file from the 
  {\bf manip} console  
  \begin{center}
  \begin{tabular}{|l|l|}
  \hline
  console & {\tt manip$>$ calew} \\
  \hline
  \end{tabular}
  \end{center}
  %--------------------------
  \small
  {\em
    The command calew is actually a command file that executes
    a series of commands.  First the ropes are wound to 90N tension
    and held there for 30 seconds.  Then the ropes are relaxed to
    10N and then the rope lengths are set.
  }
  \normalsize
  %--------------------------

\checkitem At the end of the calibration process, the change in rope lengths
  are reported.  If either of the changes in rope lengths are greater
  than 0.5 cm, repeat the calibration process. Record the change in rope length
in the calibration logbook.


\end{enumerate}

%=================================================
{\small
~\\
~\\
\noindent
{\bf Revision History:}\\
\begin{tabular}{llll}
Rev. & Date & Author & Comments\\
0           &  ?           &  Fraser Duncan & \parbox[t]{3.0in}{   First draft }  \\

1             & 2002/11/10    & Fraser Duncan &  \parbox[t]{3.0in}
                                           {   Added steps to go to expert mode. }\\
2          &   2004/08/10 & Peter Skensved & \parbox[t]{3.0in}
     {  Use constant tension mode }\\

\end{tabular}
}






%=========================================================================
%=========================================================================
%=========================================================================

\newpage
\subsection{Attaching the East/West Side Ropes}
\shwlabel{secattachropes}
\newprocedure{CalProcAttachingEastWestRopes}
             {Attaching the East/West Side Ropes}
             {Fraser Duncan}{2002/11/10}{1}
 
  Connecting or disconnecting the side ropes to the source is the most
delicate part of the calibration procedure.  A mistake
in the procedure could easily destroy the laserball and
drop fragments of it into the detector.  \\
{\bf
This procedure should only be done under the supervision of
an experienced operator. Before embarking ensure that everybody involved
is aware of what is about to happen. 
}

\noindent
{\em Note:  This procedure assumes you are connecting the sideropes
  to the laserball.  If a different source such as the {\tt N16} source
  is being used replace {\tt laserball} with the appropriate source
  name in  the  following procedure.
}
  
\vspace*{0.25in}
\noindent
{\bf Prior to this Procedure}\\
The Source has been deployed into the glovebox from the source
tube with the pivot located at approximately $z_{pivot} = 1380$.

  The N16 source is further away from the center of the glovebox
than the laserball. Some people find it easier to attach the side ropes
to this source if is is at a slightly lower pivot position like
1370. Also, for the N16 source the primary operator sits at the west
gloveports.

  If at some point in the procedure you find that any of
the operators cannot reach the side ropes or are unable to safely
pass the ropes undo all the steps completed in reverse order and contact
the OCE before regrouping and trying again.

\begin{enumerate}


\checkitem Go to expert mode at the manipulator console

  \begin{center}
  \begin{tabular}{|l|l|}
  \hline
  console & {\tt manip$>$ expert room601} \\
  \hline
  \end{tabular}
  \end{center}
  {\small\em Note: Expert mode has a 30 minute time out.  If 
   you take longer than this it will be necessary to reenter expert mode.}

\checkitem Open the glove ports on the glovebox.


\checkitem Operator at {\bf manip} console puts the
  side ropes in constant tension mode,
  \begin{center}
  \begin{tabular}{|l|l|}
  \hline
  console & {\tt manip$>$ moveew} \\
  \hline
  \end{tabular}
  \end{center}
  %-----------------------
  \small
  {\em
    The command {\tt moveew} is a command file that puts the east and
    west side ropes into a constant tension mode.  This mode
    causes the manipulator to try and maintain a constant ( 12 N )
    on each rope.  If an operator pulls on the rope and increases the
    tension, the manipulator plays out more rope to decrease the tension
    back to 12 N.  This allows the operator to pull the ropes about
    and have the manipulator ``follow''. Note that the ropes cannot move
    faster than 3 cm per second. So - wheneever you move the ropes { \bf do
    it slowly ! } 
  }
  \normalsize
  %-----------------------
\checkitem Primary operator at south  glove ports reaches in and grasps source
  at the lower part of the carriage.  The rope slot   on the source carriage should face south unless
  the OCE has given different instructions. Make sure your hand is low enough that the pulleys will pivot.

\small
{\em 
    During this procedure the source will be pulled away from its normal
    vertical position under the gatevalve. This means that the source will swing
if the operator lets go of it causing damage to both the detector and the source.
 {\bf Be extremely careful !}
}
\normalsize

\checkitem Operator at the west glove port reaches in and checks that
  the side rope is in constant tension mode by pulling on it.  The
  west rope motor unit on the rope should activate to play out more 
  rope.


\checkitem  The south port operator holds the source with his or her right hand while 
the  west port operator moves the west side rope  to a position within
 easy reach of the south port operator.

  \small
  {\em
    A good way to move the side ropes is to think of them as
    bow strings as in a bow and arrow.  The way to move the rope
    is to hook it with a finger and slowly pull it sideways.  What
    the operator should try avoiding is pulling down on the rope
    such that it goes slack down in the vessel.
  }
  \normalsize
  %---------------------
  
\checkitem The west port operator hands the west rope to the south port operator
{\bf but does not let go of the rope until the south port operator confirms
that he or she has hold of it.}

\small
{\em The handover must be done in a controlled manner with tension on the siderope at
all times. Make sure the other person
is aware of what is about to happen. Ask and recieve confirmation before proceeding
with each step of the handover. }
\normalsize

\checkitem The south port operator attaches the west rope to the source.

\small
{\em The easiest way to do this is to hold the rope above the carriage
with a tiny amount of slack in the rope below. Gently work the slack line into
the slot and slowly let the motor unit take up the slack before
letting go of the rope. Make sure the rope runs correctly over the pulley.
}
\normalsize


\checkitem The south port operator switches the source to his or her left
hand. Make sure the source doen't swing and make sure the west rope
stays put.

\checkitem Either a third operator goes to the east ports or the west
  port operator moves over to the east ports.


\checkitem The east port operator checks that the east rope is in
  constant tension mode by gently pulling on the rope and checking
  that the rope motor unit plays out more rope.

\checkitem The east port operator
  moves the east rope to a position within easy reach of the south port
 operator.

\checkitem The east port operator hands the east rope to the south port
operator {\bf but does not let go of the rope until the south port operator confirms
that he or she has hold of it.}

\small
{\em The handover must be done in a controlled manner with tension on the siderope at
all times. Make sure the other person
is aware of what is about to happen. Ask and recieve confirmation before proceeding
with each step of the handover. }
\normalsize


\checkitem The south port operator attaches the east rope to the source.

\small
{\em Hold the source with your left hand on  the lower part of the
carriage. Make sure there is room for the pulleys to pivot. Hold the tensioned east
rope with your right hand above the pulley with a tiny amount of slack below and work
the slack part of the rope into the slot. Then let the east motor unit take up
the slack ( gently ! ). 
}
\normalsize

  
\checkitem The south port operator checks that the side ropes are sitting
  securely on their pullies.
  
\checkitem The south port operator finally moves the source back towards
the center of the glovebox. {\bf Do it slowly and don't let the source swing !}

\small
{\em Hold the source with the palm of your hand {\bf behind} the source as you
move it towards the center of the glovebox. This way you will not pull the source
too far to the other side.
}
\normalsize

\checkitem Close all glove ports on glovebox.
  
\checkitem Console operator logically connects the side ropes to the laserball object
  in the manipulator code.
  \begin{center}
  \begin{tabular}{|l|l|}
  \hline
  console & {\tt manip$>$ laserball connect eastrope westrope} \\
  \hline
  \end{tabular}
  \end{center}
\small
{\em Once the ( logical ) connection is made the side rope in question will
show up on the display.
  Note that the ropes can be connected one at a time if so desired.

}
\normalsize
\checkitem Set the source orientation ( laserball only ). The orientaion is
a number between 0 and 4 . To list the possible orientation codes do a :

\begin{center}
\begin{tabular}{|l|l|}
\hline
console & {  \tt manip$>$ laserball orientation }\\

\hline
\end{tabular}
\end{center}

 To set the orientation to EAST do a 

\begin{center}
\begin{tabular}{|l|l|}
\hline
console & {  \tt manip$>$ laserball orientation 2 }\\

\hline
\end{tabular}

\end{center}


\end{enumerate}




%=================================================
{\small
~\\
~\\
\noindent
{\bf Revision History:}\\
\begin{tabular}{llll}
Rev. & Date & Author & Comments\\
0           &  ?  & Fraser Duncan &
\parbox[t]{3.0in}{
  First draft
}\\

1             & 2002/11/10    & Fraser Duncan &
\parbox[t]{3.0in}{
  Added steps to go to expert mode.
}\\

2                & Oct. 2004 & P. Skensved &
\parbox[t]{3.0in}{
  Add more detailed instructions. 
} \\

\end{tabular}
}

 



%========================================================================== 
%========================================================================== 
%========================================================================== 
%========================================================================== 

\newpage
\subsection{Detaching Side Ropes}
\shwlabel{secdetachside}
\newprocedure{CalProcDettachingEastWestRopes}
             {Dettaching the East/West Side Ropes}
             {Fraser Duncan}{2002/11/10}{1}

 
  Connecting or disconnecting the side ropes to the source is the most
delicate part of the calibration procedure.  A mistake
in the procedure could easily destroy the laserball and
drop fragments of it into the detector.  
{\bf
This procedure should only be done under the supervision of
an experienced operator. Before embarking ensure that everybody involved
is aware of what is about to happen. 
}
 
\noindent
{\em Note:  This procedure assumes that  the sideropes are connected 
  to the laserball.  If a different source such as the {\tt N16} source
  is being used replace {\tt laserball} with the appropriate source
  name in  the  following procedure.
}
  
\vspace*{0.25in}
\noindent
{\bf Prior to this Procedure}\\
The Source has been deployed into the glovebox from the source
tube with the pivot located at approximately $z_{pivot} = 1380$.

  The N16 source is further away from the center of the glovebox
than the laserball. Some people find it easier to detach the side ropes
from this source if is is at a slightly lower pivot position like
1370. Also, for the N16 source the primary operator sits at the west
gloveports.

  If at some point in the procedure you find that any of
the operators cannot reach the side ropes or are unable to safely
pass the ropes undo all the steps completed in reverse order and contact
the OCE before regrouping and trying again.


  
\begin{enumerate}

\checkitem Go to expert mode at the manipulator console

  \begin{center}
  \begin{tabular}{|l|l|}
  \hline
  console & {\tt manip$>$ expert room601} \\
  \hline
  \end{tabular}
  \end{center}
  {\small\em Note: Expert mode has a 30 minute time out.  If 
   you take longer than this it will be necessary to reenter expert mode.}

\checkitem Open glove ports on the glovebox.

\checkitem Verify source pivot is located at approximately $z_{pivot}=1380$.

\checkitem Locate primary operator at south ports, hands in gloves.

\checkitem Operator at {\bf manip} console logically detaches the side ropes
  from the laserball object.
  \begin{center}
  \begin{tabular}{|l|l|}
  \hline
  console & {\tt manip$>$ laserball disconnects eastrope westrope} \\
  \hline
  \end{tabular}
  \end{center}
  %-------------------
  \small
  {\em 
     The east and west ropes should disappear from the display.
    The laserball at this point becomes a single axis source
    (but the side ropes are still physically attached).
  }
  \normalsize
  %--------------------


  
\checkitem Operator at the console puts the east and west side ropes
  in constant tension mode to allow the physical detachment.
  \begin{center}
  \begin{tabular}{|l|l|}
  \hline
  console & {\tt manip$>$ moveew} \\
  \hline
  \end{tabular}
  \end{center}




\checkitem The primary operator at the south gloveports reaches in and
grasps the source at the lower part of the  carriage. Make sure your
hand is low enough that the pulleys will pivot.

\small
{em 
    During this procedure the source will be pulled away from its normal
    vertical position under the gatevalve. This means that the source will swing
if the operator lets go of it causing damage to both the detector and the source.
 {\bf Be extremely careful !}
}
\normalsize

\checkitem Another operator at the east gloveports hold his or her hand 
near the source  ready to receive  the east rope .



\checkitem The southrope operator holds the source with his or her left hand and
detaches the east rope with his or her right hand.

\small
{\em The easiest way to do this is to hold the rope above the carriage
with a tiny amount of slack in the rope below. Gently work the slack line out of
the slot.
}
\normalsize



\checkitem The south prot operator now passes the east rope  to the east port operator
{\bf but does not let go of the rope until the east
 port operator confirms that he or she has hold of it.}

\small
{\em The handover must be done in a controlled manner with tension on the siderope at
all times. Make sure the other person
is aware of what is about to happen. Ask and recieve confirmation before proceeding
with each step of the handover. }
\normalsize


\checkitem The east port operator gently moves the east rope back to its vertical
resting position while keeping tension on the rope at all times.

  \small
  {\em
    A good way to move the side ropes is to think of them as
    bow strings as in a bow and arrow.  The way to move the rope
    is to hook it with a finger and slowly pull it sideways.  What
    the operator should try avoiding is pulling down on the rope
    such that it goes slack down in the vessel.
  }
  \normalsize



\checkitem The south port operator switches the source to his or her right hand.

\checkitem The east port operator or a third operator gets ready to receive
the west rope from the west side ports.



\checkitem South port operator detaches west rope 

\small
{\em Hold the source with your right hand on  the lower part of the
carriage. Make sure there is room for the pulleys to pivot. Hold the tensioned west
rope with your left hand above the pulley with a tiny amount of slack below and work
the slack part of the rope out of the slot.
}
\normalsize

\checkitem Hand the west rope  to the
  west port operator who allows the rope to slowly relax to
  its resting position.

\small
{\em The handover must be done in a controlled manner with tension on the siderope at
all times. Make sure the other person
is aware of what is about to happen. Ask and recieve confirmation before proceeding
with each step of the handover. }
\normalsize


  
\checkitem South port operator moves the source back to its resting position
under the gatevalve.
{\bf Do it slowly and don't let the source swing !}

\small
{\em Hold the source with the palm of your hand {\bf behind} the source as you
move it towards the gatevalvex. This way you will not pull the source
too far to the other side.
}
\normalsize


\checkitem  Console operator takes the side ropes out of constant tension
  mode by pressing the STOP button (the ESC key) or by typing the
  command:
  \begin{center}
  \begin{tabular}{|l|l|}
  \hline
  console & {\tt manip$>$ stop} \\
  \hline
  \end{tabular}
  \end{center}

\checkitem Close all glove ports on glovebox.


\end{enumerate}  


%=================================================
{\small
~\\
~\\
\noindent
{\bf Revision History:}\\
\begin{tabular}{llll}
Rev. & Date & Author & Comments\\
0           &  ?  &Fraser Duncan &
\parbox[t]{3.0in}{
  First draft
}\\

1             & 2002/11/10    & Fraser Duncan &
\parbox[t]{3.0in}{
  Added steps to go to expert mode.
}\\

2   & Oct. 2004 & P. Skensved &
\parbox[t]{3.0in}{
 Added more detail to the procedure
}
\end{tabular}
}

 
  



