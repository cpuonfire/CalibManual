  
\chapter{Misc}
  

   
\begin{figure}
\begin{center}
\leavevmode
%\epsfysize=0.85\textheight
\epsfxsize=5in
\epsfbox{figures/catenaries.ps}
~\\\
\caption[Shape of Umbilical]
        {Shape of Umbilical
         \shwlabel{figumbilical}
        }
\end{center}
\end{figure} 
  
\section{Control Hardware}
  
  
  Each of the ropes is referred to as an ``axis'' and each axis has a
\begin{itemize}
\item stepping motor to wind rope in or out.
\item shaft encoder to measure length of rope.
\item load cell to measure tension of rope.
\end{itemize}
The stepping motors are controlled from a National Instruments TIO 10 card
in the control PC that has many clock signals.  These signals are fanned
out through the motor fanout box to the individual motor controllers.
The readback from the axis is the load cell measuring the tension of the
rope and the shaft encoder measuring the length of the rope.  The input
signals go into the counter boards which are gray boxes 
which are daisy chained together and are indvidually 
addressable.    The address for the counters are set with jumpers on 
the boards as is the address for the analog circuit.  Note that the
analog circuit has a different address from the counter circuit.
These boxes are read out by the data concentrator which is
read by the computer.

\section{Using the {\tt see} editor on the PC}
start it by typing
\begin{verbatim}
  see wiring.dat
\end{verbatim}
Now in command mode, go to insert mode by typing 
\begin{verbatim}
    i
\end{verbatim}
To leave insert mode type 
\begin{verbatim}
    <esc>
\end{verbatim}
When in command mode, the space bar scrolls through the commands.  The
page up and and page down buttons allow paging through text file.
\section{Data files}
\begin{verbatim}
On PC
-----
c:\motors\manip\
  
   wiring.dat  -- TIO 10 wiring map
                   counter board wiring map
                   motor fanout wiring map
  
       -- both the manipulator and the AV position sensors
  
   motor.dat    -- physical parameters for motors
  
   encoder.dat  -- physical parameters for encoders
  
   loadcell.dat -- physical parameters for load cells
  
   axis.dat     -- combines motor loadcell and encoder infor plus other
                   stuff (i.e. wire tension etc) to form info on axis
  
   polyaxis.dat -- combines 3 axes into the manipulator
  
   av.dat       -- information on acrylic vesel geometry

\end{verbatim}
  


  
%-------------------------------------------------------------------------
%-------------------------------------------------------------------------
%-------------------------------------------------------------------------
%\appendix
  
\chapter{Water Level Measurement With Manipulator}
  
  On 5 Oct 1998 at approximately 14:00, Aksel Hallin, Peter Skensved
and Fraser Duncan deployed the manipulator into the D2O in the Acrylic
vessel.  It could easily be seen to the order of 2mm when the top of
the manipulator weight cylinder contacted the D2O surface.  In the
manipulator coordinate system, the manipulator carriage pivot was at
 -285 cm when this happened.  The weight top plate was located 2.5"
(6.35cm) below the pivot.  When the manipulator was returned to the glovebox
it's calibration was checked and was found to be 1mm off.    When the
measurement was taken, the centre of the AV as reported by the neck monitors
was located at 3.04cm in the global coordinate system.  The manipulator
was calibrated such that the bottom of the AV is located at -600.53 cm
(at nominal lab temperature --- the AV is now getting colder).
Thus the depth of water in the AV (distance from water surface to bottom
of AV) as measured by the manipulator is:
\[  
    -285 - 6.35 - ( -600.53 + 3.04) = 306.14 cm
\] 
The measurements used to calibrate the manipulator position had an
uncertainty of 0.16 cm.  However, there is a question of reproducability
of the order of 1 cm.  Further the agreement between the single axis
manipulator and the multiaxis manipulator is on the order of 1 cm.
Thus the manipulator measurements have an uncertainty of 1 cm.
  
    Discussion with Ken McFarlane gave a D2O AV bubbler water depth
reading of 131 in.  According to Dave Sinclair, the reading of that 
bubbler is 1% low and it is calibrated in inches of H2O.  The ratio
of H2O density to D2O density is 62.4/69.  Further, the bubbler is
one inch above the bottom of the AV.  Thus,
\[  
  1.01 * 131 * (62.4/69) + 1.0 = 120.65 in  * 2.54 cm/in = 306.46 cm
\]  
So the manipulator and bubbler are in excellent agreement with each other,
having a difference in the water position of 0.32 cm.


  


%------------------------------------------------------------------
\chapter{Deck Elevations}
\begin{verbatim} 

URL to original Doc:
http://manhattan.sno.laurentian.ca/sno/detector.nsf/URL/MANN-46VUSJ

SUBJECT:  Elevations
FROM:     Davis Earle
DATE:     4/12/99 6:50:25 PM
CATEGORY:      Commissioning,

_/_/_/_/_/_/_/_/_/_/_/_/_/_/_/_/_/_/_/_/_/_/_/_/_/_/_/_/_/_/_/_/_/



LOG ENTRY:
Relative Elevation of Various Detector Components
By Davis Earle
April 12, 1999


On April 7 & 8th we measured the elevation of a number of positions on the deck,
the elevation of the water levels and of the top of the AV. The Queen?s
theodolite was used. The deck positions were identified with yellow labels " BM
x" where X is some number from 1 to 15.

Table 1

Identify              Location
relative elevation

BM 1                 F1 ultra sound port, South side                   -11.90
BM 2                 PSUP cable support #4, NE location            -22.35
BM 3                 F2 cover gas port, North side                     -14.92
BM 4                 Edge of hatch to cavity
-24.60, -24.80
BM 5                 Lip of the bathtub, under the cover.              -2.5
BM 6                 AV 2 wipple tree, NE                               -0.40
BM 7                 AV 4 wipple tree, SE                                0.0
BM 8                 PSUP cable support #8, S                          -23.00
BM 9                 PSUP cable support #15, N                        -23.10
BM 10               Glove box flange to AV neck, S side            80.95
BM 11               Glove box gate valve to D2O, top               127.00
BM 12               Calibration Port #6                                    9.90
BM 13               SW glove box corner, top                           104.05
BM 14               SE glove box cornet, top                           104.00
BM 15               NE glove box corner, top.                          103.90

     The theodolite was set up in four different places and BM 6 & 7 were used
to cross reference the readings.  All readings were normalized to BM 7 and are
shown in Table 1 in cm.

     On April 8th the distance from BM 10 to the top of the AV neck NCD ring was
measured as 117.8 cms. At that time the CMA neck Z value was 1281.26 cm and the
average Equator monitor reading was ?5.83 cm. These readings may be compared
with the readings taken Sept 17, 1998 before the water fill at which time the
distance to the NCD ring was 111.4 cm, the neck Z value was 1288.02 cm and the
average equator monitor reading was 0.13 cm. The vessel has moved down due to
the water loading. The amount of downward movement is listed in Table 2.

Table 2

                                                                         Sept
17/98           Apr 8/99            movement

>From the measurements to the NCD ring                  111.4
117.8                   6.4 cm
>From the neck Z via calibration computer                1288.02
1281.26                6.76 cm
>From the average equator monitor                            +0.13
-5.83                   5.96 cm

During the intervening  seven months adjustments have been made to the
calibration program and to the equator monitor measuring equipment. Changes of
several mm have been seen from time to time and also now that the water is in
the chimney some movement of the AV neck with respect to the AV equator is not
unexpected.  The physical measurement of 6.4 cm must be considered the most
reliable of the three different measurements. The implied smaller movement of
the equator as compared to the neck may or may not be real since uncertainties
of several mms in both the neck Z and the equator monitors is expected.

     The distance to the NCD ring at the top of the AV neck is measured by
dropping a plump bob through one of the glove box ports. By noting its position
on the NCD ring one can also determine if there has been any horizontal movement
of the neck. Between Sept and Apr there appears to have been a 1 cm movement of
the chimney top in a NE direction.  This movement is not seen in a change  of
the neck X and neck Y which show only a 1mm movement. As with the neck Z
measurement uncertainties as large as 1 cm may be present in the data from the
calibration computer and equipment.

     The water group measured the distance from the gate valve (BM 11) to the
D2O and from the hatch (BM 4) to the H2O on April 7th. These measurements were
combine with the theodolite measurements to get the difference between the two
water levels which was compared with the bubbler tube measurement of the same
difference.  This comparison was originally reported in a memo by McFarlane.

"A comparison was made with physical measurements made by Davis, Tony, Hardy,
and David Bailey against the Data Logger measurements corrected for D2O density.
Physical measurements using combinations of a tape measure and a theodolite give
a physical distance of 47.09 inches between the H2O and D2O levels, as measured
at 15:13, April 7, 1999. The consistency check program (looking at columns not
normally printed) calculated an H2O  water level of 547.53 and a D2O level of
500.42, using bubbler set B,  at a recorded time of 15:32, for a calculated
water level difference of 47.11 inches. The discrepancy between these two
measurements is 0.02 inches. Neither measurement is this accurate, and the
closeness of the numbers can only be attributed to good fortune.
On an absolute scale, using the same measurements, the D2O level is 501.21 above
the AV invert at ~15:13. Data averaged between recorded points at 15:07 and
15:32 gives a D2O height above the AV invert of 500.33 inches. The discrepancy
between the two methods is 0.88 inches. This is much better agreement than
expected."

The distance from the top of the NCD plate to the actual acrylic of the top of
the chimney, from the drawings of the assembly, is 7/8" or 2.22 cm. This
thickness is made up of two metal plates and a gasket. The distance from the top
of the ACRYLIC chimney to the centre of the vessel was reported by R. Komar in
STR-98-003 as 506.16" or 1285.63 cm. Combining these numbers with the numbers
presented in Table 1, the PSUP plates are 1301.9 cm above the centre of the
vessel when the equator monitor average is ?5.83 cm This assumes no movement of
the neck down with respect to the equator. Such movement, which might be as
large as 0.5 cm, would decrease the height difference between the AV centre and
the PSUP plates to 1301.4 cm.

Tony Noble measured the distance from the top flange of the gate valve to the
D2O at 15:13 on Apr 7th as 25? 7 5/8" = 781.368 cm, when the D2O logger recorded
the D2O pressure as 551.73".
The position on the gate valve was 1.905 cm higher than BM 11
BM 11 is 46.05 cm higher than BM 10
BM 10 is 117.8 cm higher than the NCD ring on the neck
which is 2.22 cm higher than the AV acrylic neck top and
the distance to the bottom of the AV is 742.59" from Komar and assuming no
vessel shape change.
Taking all these measurements into consideration gives the depth of the D2O in
the vessel as 501.1" which is in excellent agreement with Ken McFarlane?s number
of 501.2 reported in his memo.
\end{verbatim}


