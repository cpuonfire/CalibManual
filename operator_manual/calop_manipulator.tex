
\markright{CalOp: Manupulator}

%--------------------------------------------------------------
%--------------------------------------------------------------
%--------------------------------------------------------------
\chapter{Manipulator}
\shwlabel{ChapterManipulator}
  


  The SNO calibration source manipulator is a positioning device
used to place calibration sources inside the Acrylic Vessel of the
SNO detector or down special calibration guide tubes in the region between
the AV and the PSUP.  By using a system of three ropes, a central and two
side ropes, the manipulator is able to position a source on either an east
west plane or north south plane inside the AV.  About 3/4 of the plane inside
the AV can be reached by the manipulator, the remaining quarter is off limits
due to the geometry of the manipulator system.  In addition to the manipulator
ropes (referred to as {\bf axes}) there is an {\bf umbilical} attached
to the manipulator that provides the necessary services for the source
(electrical signals, fibre optics, gas lines etc).
  
\section{Overview}
  Each calibration source is stored in an {\bf Umbilical Retrieval Mechanism}
or {\bf URM}.  A URM consists of a block and tackle mechanism for taking
up the source {\bf Umbilical} used to provide services to the source
and a {\bf central rope} used to support the weight of the source.  Below
the URM is the {\bf source tube} which is a 4' long stainless steel pipe
used to store sources when not deployed in the vessel.  Normally, the 
URM and source tube are mounted on a calibration port on the {\bf glovebox}
which is located on the {\bf universal interface} located directly over
the neck of the acrylic vessel.  When not in use, the source is stored in
the source tube and a gate valve on the glove box seals off the detector.
The central rope in the URM is instrumented with a {\bf shaft encoder}
which determines the length of rope played out and a {\bf load cell} used
to measure the tension in the rope.  The umbilical is similarly instrumented.
  
  The layout of the system is shown schematically in figure \ref{figmansystem}.

\begin{description}
\item[Anchor Blocks]~\\
  Each side rope can be thought of as attached at two points (not exactly 
  true).  At the feedthrough where it comes down from the roof of the DCR
  into the glovebox and at the {\bf anchor block} in the AV which is located
  just above the AV equator.  The end of the rope at the anchor block is fixed
  and by playing the rope in or out through the glovebox feedthrough, the 
  calibration source is moved about the AV.
\item[Calibration Guide Tubes]~\\
  In addition to deploying sources through the glovebox into the centre of the
  AV, it is possible to deploy sources through 6 calibration guide tubes into 
  the light water volume between the AV and the PSUP.  These guide tubes are 
  located on the floor of the DCR and are sealed with gate valves.
\item[Carriage and Weight]~\\
  Attached to each calibration source is a {\bf carriage} and a {\bf weight}.  
  The carriage provides attachment points for the central rope and umbilical 
  and has pullies that the side ropes go around.  The weight cylinder is a 
  stainless steel tube containing lead.  The manipulator requires a minimum 
  weight for each source to function properly  (in particular to give the 
  sources negative bouyancy) and the weight cylinder provides this.
\item[Deck Clean Room(DCR)]~\\
  Also known as the Dark Clean Room, the DCR is the room centred on the deck.
  Most of the calibration equipment is located inside the DCR.  The DCR is
  kept clean and has relatively few airborn particles compared to the rest of
  the lab.  The clean conditions are maintained to prevent introduction of 
  radioactive
  contamination into the detector during the deployment of sources.
\item[Glove Box]~\\
  The glove box is the rectangular box with many valves and flanges.  It is 
  mounted on the Universal Interface (UI) directly over the top of the AV.  In 
  addition to sensors used by the water group to monitor the heavy water 
  levels, the glove box
  has three ports on it that calibration sources can be mounted on.  The 
  glove box gets it's name from the four glove ports on its sides.  The 
  gloves are used to attach the side ropes to the manipulator carriage 
  which must be done in darkness
  (to protect the PMT's) and in the radon free cover gas that caps the AV.
\item[Rubbing Ring]~\\
  The rubbing ring is an acrylic ring located just below the neck of the AV 
  inside the heavy water volume.  When the manipulator positions a source 
  off the central axis, the manipulator ropes and the source umbilical are 
  pulled to the side of the AV neck. The rubbing ring provides a wearing 
  surface for the ropes.
\item[Side Rope Motor Mounts]~\\
  The spooling mechanisms for the side ropes are located above the DCR 
  (Deck Clean Room).  They consist of a motor driven spool system 
  instrumented with a loadcell to measure the rope tension and a shaft 
  encoder to measure the rope length.  There are four side ropes, North, 
  South, East and West which are operated in pairs to allow positioning
  of the source inside the AV on an East-West plane or a North-South plane.
\item[Source Tube]~\\
  The stainless steel tube connecting the URM to the calibration port.
  The calibration source is parked in the source tube when not deployed in
  the detector.
\item[Umbilical Retrieval Mechanism (URM)]
  The unit to which a calibration source is attached consisting of
  a rope to support the weight of the source and an umbilical which
  provides services (power, signals control, light etc.) to the
  source.
\item[Universal Interface (UI)]~\\
  The universal interface (UI) is the stainless steel circular platform located
  in the center of the DCR directly over the AV.  It has mounted on it the 
  glove box used to deploy calibration sources into the detector.
\end{description}
  
  The manipulator carriage and weight assembly is shown schematically
in figure \ref{figmancarriage}.
\begin{figure}[htbp]
\begin{center}
\leavevmode
\epsfxsize=3in
\epsfbox{figures/mancarriage.ps}
\caption[Manipulator Carriage]{
  \shwlabel{figmancarriage}}
  Manipulator Carriage and Weight Assembly (not to scale).
\end{center}
\end{figure}
It consists of the {\bf carriage} to which the central rope is
attached.  The umbilical passes through the {\bf carriage neck},
through the weight assembly into the source.  The side ropes are
not attached to the carriage, but rather pass around {\bf pullies}
mounted on the {\bf pully bar}.  the pully bar and carriage neck are
free to rotate about the {\bf pivot}.  At different positions in the
AV, the pully bar and carriage neck will be at different orientations
while the weight assembly and source will always hang vertically below
the pivot.
 
  The weight consists of a stainless steel torus filled with lead.  The
lead is potted into the {\bf weight cylinder} with silicone and 
then capped with the {\bf cylinder end plate} which is sealed with o-rings.
Sources are usually attached to the weight cylinder using the 
{\bf extension tube}.


%=====================================================================

\section{Manipulator Control System}

  The manipulator is controlled by the {\bf manip} computer
which is a DOS based PC running a C++ program also called {\bf manip}.
The manip program interacts with the manipulator hardware by
by controlling a stepper motor for each axis to change the length
of the rope or umbilical.  A shaft encoder on each axis measures
the length and a loadcell measures the tension.  The shaft encoders
and load cells are connected to {\em encoder boxes}, one per axis.
In the encoder box an up/down counter counts the  number of steps
taken by the shaft encoder.  The loadcell is connected to an amplifier
in the encoder box.  The encoder box is in turn read out
by the {\em Data Concentrator Box} which contains up to eight
{\em Data Concentrator Cards} Each data concentrator card can read
out up to 4 encoder boxes.  The shaft encoder up/down counters 
are read out through a digital bus through the data concentrator cards.
The amplified signals from the load cells are fed to a multiplexing
ADC located in the data concentrator card.  Each encoder box has a
unique digital address and one of four analog addresses (for the 
loadcell signals).  The analog addresses must be unique on a given
data concentrator card.  The data concentrator box is read out
from the {\bf manip} computer via a PLC750 card which contains a
parallel bus connection to the Data Concentrator Card.  Up to 32
axes (rope or umbilical) can be monitored by the Data Concentrator
Box (4 encoder boxes on each of the eight data concentrator cards).
  The stepper motors
are controlled by one of two TIO10 cards in {\bf manip}.  These
cards produce stepper motor control signals and pulse trains to 
step the motors.  Eight motors can be controlled by each TIO10 
card for a total of 16 motors.  The TIO10 cards are 
are interfaced to the stepper motors through a 
{\em Watchdog timer box}.  In addition to the motor control signals
from the TIO10 cards, the watchdog timer box takes an interlock
signal generated by the PLC750 card in the {\bf manip} computer.
If signal has a timeout that has to be reset by the {\bf manip}
program.  In the event of the manip program stopping the interlock
signal turns off.  The Watchdog Timer box then shuts off the motors.
A block diagram of the manipulator control system is shown in
figure \ref{figmanctrl}.
 
\begin{figure}[htbp]
\begin{center}
\leavevmode
\epsfxsize=6in
\epsfbox{figures/MANCTRL.eps}
\caption[SNO manipulator control system]
  {SNO manipulator control system.
  \shwlabel{figmanctrl}}
\end{center}
\end{figure}


  


%=====================================================================

\section{Modes of Source Deployment}

\subsection{Single Axis Deployment}
  
  Sources can be deployed in a {\bf single axis mode} which consists of
lowering a source straight down from the URM on just the central rope
and umbilical.  The horizontal position of the source is determined
by the location of the URM.  The vertical position of the source is
determined by the measured length of central rope played out.  The single
axis deployment mode is useful for operation along the central axis of
the detector and for deployment of sources down the guide tubes.
  
\begin{figure}[htb]
\begin{center}
\leavevmode
\epsfxsize=5in
\epsfbox{figures/mansingleaxis.ps}
~\\
\caption[Single Axis Source Deployment]
        {Single Axis Source Deployment
         \shwlabel{figsingle}
        }
\end{center}
\end{figure} 
  
  
\subsection{Three Axis Deployment}
  The main purpose of the manipulator however, is to deploy a source
{\em off} the central axis of the detector inside the acrylic vessel.
This is done attaching two {\bf side ropes} to the manipulator carriage
once it is deployed into the glovebox.  The side ropes are attached at 
one end to {\bf anchor blocks} in the AV are anchored at the other end
by feedthroughs on the glovebox.  The side ropes go over pullies on 
the manipulator carriage.  Once the source is lowered into the vessel, it
can be pulled off the central axis by shortening one side rope and lengthening
the other.  Because only two side ropes are attached at a time, the source
can only be moved in a plane.  There are two sets of side ropes allowing
motion in an east-west plane or a north-south plane.  The side ropes
are instrumented in the same fashion as the central rope with the 
side rope motor mounts located on the roof of the DCR.  The ropes pass
through the roof of the DCR into the glovebox through stainless steel tubes.
  
\begin{figure}[htb]
\begin{center}
\leavevmode
\epsfxsize=5in
\epsfbox{figures/manpolyaxis.ps}
~\\
\caption[Three Axis Source Deployment]
        {Three Axis Source Deployment
         \shwlabel{figpoly}
        }
\end{center}
\end{figure} 
  
  
  The manipulator is controlled by the manipulator computer, a DOS based
PC.  The manipulator computer runs a C++ based program called {\bf manip}
which monitors the instrumentation on the manipulator, calculates the positon
of the source and accepts commands to control the manipulator.  {\bf manip}
can be accessed both from the console in the DCR and remotely via TCP/IP.
When taking data control of the manipulator is nominally done through the
SNO DAQ.  The reason for this is that the DAQ then automatically incorporates
any change in the calibration source configuration into the data stream.
In addition there is a standalone unix utility called {\bf manmon} which
allows remote monitoring of the manipulator and is useful for diagnostics.
  
\begin{figure}[htb]
\begin{center}
\leavevmode
%\epsfysize=0.85\textheight
\epsfxsize=5in
\epsfbox{figures/rope_lengths.ps}
~\\
\caption[Rope Lengths]
        {Rope Lengths
         \shwlabel{figropelengths}
        }
\end{center}
\end{figure} 
  
\begin{figure}[htb]
\begin{center}
\leavevmode
%\epsfysize=0.85\textheight
\epsfxsize=5in
\epsfbox{figures/tension0_0.ps}
~\\
\caption[Rope Tension]
        {Rope Tension
         \shwlabel{figropetension00}
        }
\end{center}
\end{figure} 
   
\begin{figure}[htb]
\begin{center}
\leavevmode
%\epsfysize=0.85\textheight
\epsfxsize=5in
\epsfbox{figures/tension0_2.ps}
~\\
\caption[Rope Tension]
        {Rope Tension
         \shwlabel{figropetension02}
        }
\end{center}
\end{figure}
  
   

