
  
%--------------------------------------------------------------
%--------------------------------------------------------------
%--------------------------------------------------------------
\chapter{TroubleShooting}
\shwlabel{ChapterTroubleShooting}
  

    This chapter describes various problems that can arise from
the calibration system and suggests ways to diagnose and correct
them.

%-----------------------------------------------------------------------
\newpage
\section{Manipulator Trouble}
\shwlabel{secmanipulatortrouble}
  
\subsection{Manipulator Error Messages}
\begin{center}
\begin{tabular}{|l|l|}
\hline
 & \\
{\bf STOPPED\_NONE} &
\parbox[t]{3in}{
 Sometimes the manipulator will stop with a {\bf STOPPED\_NONE}
error message.  The cause of this could be a {\em Watchdog Timer Error}.   
} \\
 & \\
\hline
 & \\
{\bf STOPPED\_LOW\_TENSION} &
\parbox[t]{3in}{
 A rope or umbilical has too low of a tension.  This is potentually
a serious problem.   
} \\
 & \\
\hline
 & \\
{\bf STOPPED\_HIGH\_TENSION} &
\parbox[t]{3in}{
 A rope or umbilical tension is too high.  
} \\
 & \\
\hline
 & \\
{\bf STOPPED\_STUCK} & 
\parbox[t]{3in}{
 The manipulator is unable to get closer to the requested end point.
This sort of error can be caused by friction in the ropes and pullies
that cause the manipulator motion to be jerky and have hysteresis.
It can sometimes be corrected by requesting the desired position again.
} \\
 & \\
\hline
 & \\
{\bf STOPPED\_NET\_FORCE} &
\parbox[t]{3in}{
  The forces on the source do not sum to zero.  {\em Not used.}
} \\
 & \\
\hline
 & \\
{\bf STOPPED\_AXIS\_ERROR} &
\parbox[t]{3in}{
  The encoder error for an axis is greater than the allowed limit
(set to +/- 10 cm).  The encoder error is the difference in position
determined by the axis' encoder and the number of steps sent to the
axis stepper motor.
This is may be caused by a slow drift on the
encoder or it may indicate the failure of an encoder or a motor.
} \\
 & \\
\hline
 & \\
{\bf STOPPED\_CALC\_ERROR} &
\parbox[t]{3in}{
  A calculation error stopped the motion.
} \\
 & \\
\hline
 & \\
{\bf STOPPED\_AXIS\_FLAG} &
\parbox[t]{3in}{
  A digital flag stopped the motion
} \\
 & \\
\hline
 & \\
{\bf STOPPED\_AXIS\_ALARM} &
\parbox[t]{3in}{
  If an axis tension exceeds a preset limit, the manipulator will
shut down.  This limit is designed to prevent excessive loads on
the manipulator components.
} \\
 & \\
\hline
\end{tabular}
\end{center}

 
%--------------------------------------------------------------------------
\newpage 
\subsection{Specific Errors}
  
\subsubsection{STOPPED\_NONE --- Watchdog Timer Error}
  Sometimes the manipulator will stop for no obvious reason and
will report a {\bf STOPPED\_NONE} error.   This could be caused by
the watchdog timer timing out.  To see if this is the case look
at the {\bf max poll time} on the main display of the GUI or on the
manipulator console.  If this time is of order 500 (ms) or more, then
it is likely that the error was a watchdog timer error.  The max
poll time is displayed as a strip chart on the main GUI panel.  
  
  The cause of the watchdog timer error is intense ethernet trafic
that causes the manipulator computer to freeze up.  The watchdog timer
then times out shutting down the manipulator.  To prevent the motors
from becoming unsynced, the manipulator computer generates a {\bf STOP}
if the poll time goes above a few hundred ms.

  
\subsection{STOPPED\_LOW\_TENSION}
  This is potentially the most serious kind of error.  If a rope
goes to too low of tension it will unspool from the it's takeup drum
and it will be necessary to open the motor box to repair the damage.

  If a {\bf STOPPED\_LOW\_TENSION} error occurs, examine the situation
to determine if it is a reasonable error.  For instance if the source
is being driven down into a lower quadrant of the detector the side rope
on the opposite side must go to low tension.  
  
\begin{enumerate}
\item If the low tension alarm appears to be caused by the source in
 being in a low tension region, try moving the source away from the 
  low tension region.
  \begin{itemize}
  \item If the source will not move because the low tension alarm persists,
    {\bf CONTACT EXPERT}.
  \end{itemize}
\item If there is no obvious reason for the low tension alarm, 
    {\bf CONTACT EXPERT}.  
\end{enumerate}
  
\subsection{STOPPED\_HIGH\_TENSION}
  If the manipulator stopped with a high tension error, examine the 
situation.  A high tension alarm can occur if trying to drive the
source into an upper quadrant of the detector.  The rope on the side
of the detector that the source is moving towards goes to high tension.
  
\begin{enumerate}
\item If the alarm is due to driving into a high tension region,
  attempt to back out of the region.  If the rope is still in
  high tension and won't allow the  source to move,
  {\bf CONTACT EXPERT}.
\item If there is no obvious reason for the high tension alarm,
  {\bf CONTACT EXPERT}.
\end{enumerate}
  
\subsection{STOPPED\_STUCK}
  This error indicates that the manipulator is having trouble getting
to a desired location.  The likely cause is friction in the manipulator
and the discreteness of the encoders cause the manipulator to overshoot.
This is not an error that indicates potential risk to the manipulator.
The options in this situation are:
\begin{itemize}
\item Retry the move to the position desired.
\item Go somewhere else.
\end{itemize}

\subsection{STOPPED\_STUCK}
  This error indicates an inconsistancy in the position calculations.
It could indicate that one or more of the ropes is significantly miscalibrated.
\begin{itemize}
\item If the error persists, a recalibration of the system should be done.
  {\bf CONTACT EXPERT}.
\end{itemize}

\subsubsection{STOPPED\_AXIS\_ERROR due to Umbilical}
  
  A {\bf STOPPED\_AXIS\_ERROR} caused by the umbilical may or may not
be serious.  The calibration of the umbilicals is not ideal and so there
is a slow drift that accumulates.  On the other hand a rapid accumulation
of the encoder error indicates that either the encoder is  faulty or
the motor is not functioning properly.  The quantity derived in the 
GUI, {\bf dE/dL} is the rate of change of encoder error with umbilical
length (as determined by the encoder).  A large dE/dL indicates problems
with either the encoder or motor.  A small value indicates ``drift''.
The figure of merrit to use is:

\begin{description}
\item[ $|$dE/dL$|$ $<$ 0.1] This is a slow drift indicative of 
  miss calibration.\\
  {\bf Reset the umbilical encoder and proceed}
\item[ $|$dE/dL$|$ $>$ 0.1] This is a fast drift and may indicate
  encoder or motor problems.\\
  {\bf This is a potentially serious hardware fault.  Contact
  manipulator expert.}
\end{description}

  
\subsubsection{STOPPED\_AXIS\_ERROR due to a Rope}
  This is an encoder error due to a rope's encoder being out of 
agreement with it's motor.  This is an almost never unseen situation
since the calibration of both motor and encoder on the ropes is 
excellent.  Therefore this error should be considered an indication of
a serious problem.
  
\begin{center}
{\bf Contact a manipulator expert}
\end{center}

\subsection{STOPPED\_AXIS\_FLAG on Umbilical}
  Umbilicals (URM2 and URM3) have limit switches to prevent
playing out too much umbilical.  This is flag bit 0.  URM3 has
a limit switch to prevent taking up too much umbilical as well
(flag bit 1).  If there is a {\bf STOPPED\_AXIS\_FLAG} error, it
could be due to the limit switches.   
\begin{enumerate}
\item  Check to see which bit has
  failed.  In the GUI these bits are red if failed or green if the
  interlock is satisfied.  
\item If the interlock is failed, try moving the source away from
  the limit switch.  I.e. if too much umbilical is played out, 
  try moving the source up (shorter umbilical).
  
\item If moving the source away from the interlock condition does
  not help, inspect the manipulator wiring to ensure the limit switches
  are plugged in to the data concentrator (the interlocks default to
  failed if  the path to the limit switch is broken).
 
\item If these actions do not fix the problem, {\bf call a manipulator
  expert}.

\end{enumerate}


\subsection{STOPPED\_AXIS\_FLAG on a Rope}
  Ropes have an interlock that detects heavy loads (such as sudden
shocks) on the rope.  This interlock trips at approximately 100 lb.
If the manipulator stops due to a {\bf STOPPED\_AXIS\_FLAG} on
a rope, it is possible that the rope unit has undergone a large
load.  This could compromise the ability of the rope unit to hold
sources.  
\begin{enumerate}
\item {\bf CONTACT EXPERT}
\item Inspect the rope unit wiring to verify all cables are connected.
\item Using a DVM check if the shock load switch is open (open indicates
  a shock load).  
\item If the DVM indicates that the switch  is indeed open, then the 
  rope unit should be inspected visually.  If the detector is on and
  the rope unit opens into the detector volume, it may be necessary to
  turn off the detector to make the inspection.
\end{enumerate}


\subsection{STOPPED\_AXIS\_ALARM}
  
  If the manipulator halts due to a {\bf STOPPED\_AXIS\_ALARM},
check the current tension of the rope or umbilical that caused the
alarm to see if it exceeds the alarm limit.  The alarm function in
the manipulator code also stores the maximum value of the tension
that caused the alarm.  

{\bf CONTACT EXPERT}


 
  



