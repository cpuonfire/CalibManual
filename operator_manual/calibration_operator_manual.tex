%       
%        Manipulator User Manual
%
%

\documentclass[]{report}
\usepackage{epsfig,color,html}
\usepackage{times}

  
  
\Roman{table}
\setlength{\textwidth}{6.5in}
\setlength{\textheight}{9.5in}
\pagestyle{plain}
\setcounter{secnumdepth}{2}
\setcounter{tocdepth}{2}
\setlength{\parindent}{.4in}
\setlength{\oddsidemargin}{0 in}
\setlength{\topmargin}{-1.0in}
\setlength{\marginparwidth}{0in}
\setlength{\marginparsep}{0in}
\setlength{\headheight}{1.0in}
%\setlength{\footheight}{1.0in}
\setlength{\footskip}{.37in}      %to get numbers printed
\setlength{\topskip}{.333in}
  
%\newcommand{\shwlabel}[1]{{\bf #1}\label{#1}}

\newcommand{\shwlabel}[1]{\label{#1}}


\newcommand{\version}{Ver. 1.2}
  


\newcommand{\checkbox}{\framebox{$^{}$\hspace*{0.05in}}}
\newcommand{\checkitem}{\item\framebox{$^{}$\hspace*{0.05in}}~~}
\newcommand{\tfile} [1] {\texttt{#1}}


\newcommand{\procedure}[4]{
\shwlabel{#1}
~\\
\begin{tabular}{|l|l|}
\hline
Written/Revised By  &  #2 \\
\hline
Date        &  #3 \\
\hline
Version     &  #4 \\
\hline
\end{tabular}
~\\
\vspace*{0.25in}
}
 
 
\newcommand{\newprocedure}[5]{     % Use this procedure instead of the one above
\shwlabel{#1}
\markright{Intro: #2 Rev.#5}
~\\
\begin{tabular}{|l|l|}
\hline
Written/Revised By  &  #3 \\
\hline
Date        &  #4 \\
\hline
Revision     &  #5 \\
\hline
\end{tabular}
~\\
\vspace*{0.25in}
}

 

\newcommand{\operator}{
~\\
\begin{tabular}{|l|l|}
\hline
 & \\
Operator    & ~~~~~~~~~~~~~~~~~~~~~~~~~~~~~~~~~~~~~~~~~~~~~~~~~~~~~~~ \\
 & \\
\hline
 & \\
Date        & ~~~~~~~~~~~~~~~~~~~~~~~~ \\
 & \\
\hline
 & \\
Time        & ~~~~~~~~~~~~~~~~~~~~~~~~ \\
 & \\
\hline
\end{tabular}
~\\
\vspace*{0.25in}
}
 




\newcommand{\standardheader}{Detector Operator Manual \version}

\markright{\standardheader}



%--------------------------------------------------------------------------
%--------------------------------------------------------------------------
%--------------------------------------------------------------------------
%--------------------------------------------------------------------------
\begin{document}
\pagestyle{myheadings}
\thispagestyle{empty}


\noindent
\epsfxsize=0.5in
\epsfbox{./figures/snologo.eps}
{\Huge\bf SNO Calibration Operator Manual}
~\\
\vspace*{0.05in}
~\\
\noindent
\begin{picture}(0,0)
\linethickness{3pt}
%\put(-18,10){\line(1,0){470}}
\put(0,10){\line(1,0){410}}
\end{picture}
\begin{center}
{\Large\em Version \version}
\end{center}

%~\\
%\vspace*{0.25in}

\begin{center}
\epsfxsize=6in
\epsfbox{./figures/manipulator.ps}
\end{center}



%------------------------------------------------------------
\newpage
  

~\\
\vspace*{0.5in}
~\\
\noindent
{\large
Last Updated on \today
}\\
~\\
\vspace*{0.5in}
~\\
\noindent
{\large
Compiled by:
\begin{tabbing}
aaaa\=aaa\kill
  \>Fraser Duncan\\
  \>Phil Harvey\\
  \>Peter Skensved\\
  \>{\bf \ldots}\\
\end{tabbing}
}

%--------------------------------------------------------------------------
%--------------------------------------------------------------------------
%--------------------------------------------------------------------------


  
\tableofcontents
\listoftables
\listoffigures


  
  
%------------------------------------------------------------------------
%------------------------------------------------------------------------
%------------------------------------------------------------------------



\newpage


\begin{center}
  {\Large\bf Quick Start}
\end{center}  



\begin{itemize}


\item Whatever your level of familiarity with the various SNO calibration
systems is make sure you read and understand the rules of engagements 
( section 1.2 ). Make sure you know who the On Call Expert ( OCE ) is
and where he is {\em before} trying to operate any of the systems ( contact numbers are listed
in section 1.1 ). Think before you act and call the OCE if you have any doubts about what to do
or if anything appears to be out of the ordinary. 


\item The first chapters of this manual contains brief descriptions and
general procedures for each of the sub systems or components ( manipulator,
laser, DT generator, ... ) and Chapter \ref{ChapterProcedures} contains detailed step by step
procedures to perform some of the more routine calibration tasks ( PCA,
$^{16}$N Stability run etc. ) and procedures for  switching between the laserball and one
of the acrylic sources. 

\item The standard procedures are set up to function like a check list and you are
strongly encouraged to print out the relevant pages and check off each item
as you go through the steps.


\item If you are  unfamiliar with the SNO calibration source manipulator,
Chapters \ref{ChapterManipulator}, \ref{ChapterGeometry} and \ref{ChapterControls},
describes the manipulator system, its geometry and how to control it.
  
\item If you {\em are} familiar with the system and just want to start 
  operating it, go directly to Chapter\ref{ChapterControls}, {\bf Controls}.
 

\item Chapter \ref{ChapterManipulator}, {\bf Operating the Manipulator}, 
  describes basic manipulator operations.
  
  
\item Chapter \ref{ChapterCalGas} {\bf Operating the Gas System},
 describes the basic Gas Board operations.

\item Chapter \ref{ChapterLaser}, {\bf Operating the Laser},
  describes basic laser operations.

\item Chapter \ref{ChapterSources}, {\bf Sources }, describes the some of the sources.

\item Finally chapter \ref{ChapterTroubleShooting} lists some of the more common manipulator
error conditions and their possible resolutions.

\end{itemize}





%------------------------------------------------------------------------
%------------------------------------------------------------------------
%------------------------------------------------------------------------

\clearpage
\begin{center}
{\huge\bf ToDo List for Manipulator System}
\begin{enumerate}
\item Add generic source deployment section ( Peter )
\item Add AmBe procedure ( Peter )
\item Add acrylic stem procedures ( x 4 ) ( Peter )
\item Add some words on when and how to use the Wessington / High Pressure Dewar ( Peter )
\item Assembly/Disassembly of acrylic sources ( + AmBe )  ( Peter )
\item Add something on how to recover from caught umbilical to trobleshooting section ( Peter )
\item Label valves and add text ( Peter )
\end{enumerate}
\end{center}


%------------------------------------------------------------------------
%------------------------------------------------------------------------
%------------------------------------------------------------------------



%------------------------------------------------------------------------
%------------------------------------------------------------------------
%------------------------------------------------------------------------

\chapter{Introduction}
\shwlabel{ChapterIntroduction}

 The calibration of the SNO detector can be divided into several
aspects:
\begin{enumerate}
\item Calibration of the low level electronics channels (ECA).
\item Calibration of phototube timing (PCA).
\item Calibration of the detector's physics response.
\end{enumerate}
The low level electronics calibrations (ECA) are done via calibration
circuitry built into the electronics hardware and are not considered
further here.  The scope of this document is the equipment and procedures
used to do the phototube calibrations (PCA) and the multitude of 
physics calibrations.
  
  The phototube and physics calibration of the SNO detector consists
of placing standard sources of light or radioactivity (gamma rays and
electrons) at known locations within the detector.  These sources
are used to extract the calibration parameters for both the individual
phototubes (timing resolution, charge vs time, charge resolution,
efficiency) and bulk properties of the detector (wavelength
dependent light attenuation of materials, global detection efficiency).

  To achieve the calibrations there are special calibration {\em sources}
prepared for SNO.  These sources are light sources such as the 
{\bf laserball} or radioactive sources such as the 
{\bf $^{16}$N Gamma Ray Source},
the {\bf $^{8}$Li Electron Source}, 
or the {\bf $^{252}$Cf Neutron Source}.
Each source has unique physical properties but they all share the
requirements of being constructed of materials with little or no
radioactivity (accept for the source material in the case of active
sources).  These sources are positioned within the SNO detector
using the {\bf Calibration Manipulator System}.  The manipulator 
is a computer controlled system that is designed to place a source
within either the inner detector volume (the AV) or in the light
water between the AV and PSUP.  The manipulator's nominal positioning
accuracy is 5cm. 



%=====================================================================

\section{Overview of the Calibration Apparatus}
\shwlabel{SecOverview}

  Calibration sources are deployed into the SNO detector using
the {\em Calibration Source Manipulator} system shown schematically
in figure \ref{figmansystem}. 
\begin{figure}[htbp]
\begin{center}
\leavevmode
%\epsfysize=0.85\textheight
\epsfxsize=5.0in
\epsfbox{figures/mansystem.ps}
\caption[Schmatic layout of the SNO manipulator]{
  SNO manipulator System (Not to Scale).  The calibration
  source manipulator is a system of ropes the allow the positioning
  of calibration sources inside SNO detector.  Either within the AV
  or between the AV and the PSUP.
  \shwlabel{figmansystem}}
\end{center}
\end{figure}
  Sources
are mounted on {\em Umbilical Retreival Mechanisms} ({\em URMs})
and then mounted on either the {\em Glovebox} or {\em Calibration
Guide Tubes} and then lowered into the detector.  The apparatus is
designed to maintain the {\em Cover Gas} seal on the detector.  Prior
to opening valves to the detector volume, air is flushed out of the
URMs with pure N$_2$ gas derived from liquid nitrogen dewars located
in the Junction.  The apparatus also maintains the light seal on the
detector allowing the deployment of sources (with the acception of
the calibration camera) with the detector turned on.


  The manipulator and calibration sources are located in the 
{\em Deck Clean Room} which is located at the centre of the deck
over the detector.  It is shown in figure \ref{FigDcr}. 
\begin{figure}
\begin{center}
\leavevmode
\epsfxsize=7in
\epsfbox{figures/sno_dcr.ps}
~\\
\caption[Deck Clean Room]
        {The Deck Clean Room
         \shwlabel{FigureDcr}
        }
\end{center}
\end{figure} 
Inside the DCR are the URMs, the manipulator control computer,
the calibration laser, storage cabinets for the sources, lifting
mechanisms to move the URMs on and off the glovebox and carts for
moving the URMs about in the DCR. 

  There are presently three URMs in use for the calibration sources.
URM1 is kept for ``guest sources'' that are not used on a regular basis.
It is usually mounted on gatevalve 1 on the glove box (10'' gatevalve)
or on the calibration guide tubes.  URM2 is used for the Laserball
optical calibration source and is usually mounted on gatevalve 1 on
the glovebox.  The design of the Laserball hardware is such that
it can be quickly removed and acrylic encapsulated radiactive sources
mounted instead.  This is occasionally done on URM2 but it is usually
restored to the Laserball configuration.  URM3 is the permanent home
of the  $^{16}$N gamma-ray source and is usually mounted on gatevalve
3 (4'' valve) on the glovebox.  Because disassembly of the $^{16}$N
source is and involved process, this URM is rarely used for other
purposes.



%=====================================================================
\newpage
\section{Contacts}

Below are names and phone numbers for people familiar with different
parts of the calibration system.  However, if there is any question
about the state and safety of the system, contact the {\bf On Call Expert}
(OCE)
who is the person carrying the calibration pager.  Prior to any running
of the manipulator the head of the calibration group or his designate will
indicate who the On Call Expert is.

\begin{center}
\begin{tabular}{|l|l|rl|}
\hline
         &           &             &                             \\
Calibration Pagers   &             &  (705)669-8813 &  Sudbury  \\
         &                         &  (613)548-2739 &  Kingston   \\
         &             &               & \\
\hline
Peter Skensved & Manipulator & (613)533-2676 &  Queen's Office \\
               & Laser       & (613)376-3491 &  Home  \\
               & N16         & (705)692-5892 & House Lively \\
               &             & (613)888-9573 & Cell Phone \\
               &             &          x221 & Office on Site \\
\hline
Fraser Duncan & Manipulator &          x205 & Office on Site \\
              & Laser       & (613)549-3360 & Home Kingston\\
              & N16         & (705)522-4893 & Home Sudbury \\
              &             & (705)561-0453 & Cell phone \\
\hline
Phil Harvey    & Manipulator  & (613)533-6000 \  x77788     & Office Queen's   \\
               &                      & (613)389-9154 &  Home  \\
\hline
Aksel Hallin  & Manipulator & (613)533-6766 & Office Queen's \\
              & Laser       & (613)634-1571 & Home \\
              & N16         &               &  \\
\hline
\end{tabular}
\end{center}   


\newpage
\section{Rules of (Dis)Engagement}

  Chapter \ref{ChapterTroubleShooting} of these procedures describe 
possible problems
with the manipulator system and how to correct them.  Some problems are
straight forward and are of little consequence.  These problems can be
corrected by the  operator.  Some problems are significant and the
On Call Expert (OCE) should be contacted before addressing them.  
The OCE is in turn required to inform the Head of the Calibration Group
or dhis designate 
of any significant problem (here significant is defined as a problem
that could potentially put at risk the manipulator system, the detector
or the heavy water). 
  
{\bf
\begin{enumerate}
\item Contact the OCE for any problems that the trouble shooting 
  section indicate the OCE should be contacted.
\item Contact the OCE if you have doubts about a procedure.
\item Contact the OCE if you have had to execute any of the 
  emergency shutdown procedures.
\item Contact the OCE if any abnormal situation occurs.
\end{enumerate}
}



\renewcommand{\newprocedure}[5]{
\shwlabel{#1}
\markright{CalOp: #2 Rev.#5}
~\\
\begin{tabular}{|l|l|}
\hline
Written/Revised By  &  #3 \\
\hline
Date        &  #4 \\
\hline
Revision     &  #5 \\
\hline
\end{tabular}
~\\
\vspace*{0.25in}
}




\markright{CalOp: Controls}
  
%------------------------------------------------------------------------
%------------------------------------------------------------------------
%------------------------------------------------------------------------
\chapter{Controls}
\shwlabel{ChapterControls}

  Control of the calibration manipulator and the calibration laser
is done through the {\bf manip} computer running the {\bf manip} program
which is a C++ program running under DOS.  There are three ways for
the operator to send commands to {\bf manip}:
\begin{itemize}
\item From the {\bf manip} console in the DCR
\item From the SHaRC Detector user interface.
\item From the {\bf manmon} program runing on a number of
  Linux or Unix boxes.
\end{itemize}
When changing sources or adjusting equipment in the DCR, the console
is usually used.  Otherwise the calibration operator usually interacts
with the manipulator through {\bf manmon}.  The manipulator can be 
operated both from underground or from surface without any personnel in
the lab.  There is also a 
monitoring program running on {\bf crag1} (Surface control room workstation).
This program called {\bf manip\_logger} generates a web page where
the current manipulator status can be viewed and log files of the commands
and errors can be downloaded.  


\section{Controlling and Monitoring the Manipulator From manmon}

  Normal operation of the manipulator is through {\bf manmon} which
is a Tcl/Tk program which runs on various Linux workstations on site.
Specifically, the workstations
  \begin{description}
  \item[ crug1, crug2, alcor ] The workstations in the underground control room.
  \item[ crag1, crag2 ] The workstations in the surface control room.
  \end{description}
{\bf manmon} provides a GUI interface to display the manipulator data
and to allow the Calibration Operator to move sources or operate the
calibration laser.


\begin{figure}[htb]
\begin{center}
\leavevmode
%\epsfysize=0.85\textheight
\epsfxsize=5.0in
\epsfbox{figures/f_manmon.ps}
~\\
\end{center}
\caption[manmon main window] 
        {manmon main window
          \shwlabel{figmanmon}
        }
\end{figure}
  
\subsection{Starting manmon}
\begin{enumerate}
\item Log on to {\tt alcor} as yourself ( or as user {\tt operator} ).

\item Go to the current manmon directory,
  \begin{verbatim}
      cd ~manipulator/manmon
  \end{verbatim}
  The directory {\tt manmon} is a softlink to the current version
  of the manmon code.
  
\item start the {\bf manmon} program by typing
  \begin{verbatim}
       manmon
  \end{verbatim}
  A GUI user interface will pop up (figure \ref{figmanmon}).
  
\item On the control room computers ( {\tt crag}/{\tt crug } ) log on as
user {\tt operator} .

\item Go to the current manmon directory,
\begin{verbatim}
      cd ~calibrator/manmon
\end{verbatim}




\item Connect to the manipulator computer by clicking on the {\bf connect}
  button.  A window will pop up (see figure \ref{figconnection}) 
  asking if you wish to make this a
  control connection (to operate the manipulator and laser) or a monitor
  connection (to view but not affect the calibration system).
  \begin{figure}[h]
  \begin{center}
  \leavevmode
  %\epsfysize=0.85\textheight
  \epsfxsize=2in
  \epsfbox{figures/f_connection.ps}
  \end{center}
  \caption[manmon main window] 
        {manmon connection dialog
          \shwlabel{figconnection}
        }
  \end{figure}
  The default is to make a monitor only connection.  To make this a 
  control connection, click on the {\bf monitor only} button to deselect
  it.  The window will disappear and the connection will be attempted.
  If the connection is accepted properly, the message 
  \begin{verbatim}
         Connection Accepted
  \end{verbatim}
  will appear in the {\bf from server:} window and in the {\bf Connections}
  panel of the display (top right) a line indicating the connection will
  appear.  This line contains the {\em ID} of the process, the {\em Address}
  of the machine that {\bf manmon} was started from and the {\em Idle Time}.
  Because {\bf manmon} is constantly polling the manipulator the idle time
  should always be zero.  If a different process connected to the
  manipulator computer becomes hung, then that process's idle time
  will continually grow.
  
\end{enumerate}
 



%---------------------------------------------------
  
\newpage
\section{Controlling the Manipulator from the Console}
  
Change to directory
\begin{verbatim}
   c:\motors\manip\
\end{verbatim}
and run program 
\begin{verbatim}
   manrun
\end{verbatim}
(this runs a batch file which runs manip ).  Basic Commands are :
\begin{description}
\item[help]~
\item[list] lists all objects
\item[logout] disconnects tcp/ip connection
\item[quit] exits program
\end{description}
Commands for the {\tt prototype} object ( typically a source objcet like {\tt laserball}, {\tt acrylic}, {\tt n16} etc. ) 
\begin{description}
\item[{\tt prototype} by $<x>$ $<y>$ $<z>$] move manipulator amout x y z in 
  manipulator space.
\item[{\tt prototype} reset] resets encoders to agree with motor position and tension
\item[{\tt prototype} locate $<x>$ $<y>$ $<z>$] defines the position to be x y z.
\item[{\tt prototype} to $<x>$ $<y>$ $<z>$] moves to absolute position
\end{description}
A sample session:
\begin{verbatim}
motors> prototype reset              -- resets encoders to agree with motor position
                                        and tension

motors> prototype by  35 0 -40       -- moves differential amount in x,y,z

motors> prototype locate 0 0 92.8    -- defines the current position to be
                                        (0,0,92.8)
  
motors> prototype to  -180.0 0 32.3  -- moves to absolute position
\end{verbatim}
  

A session in which the central rope was originally fully spooled and is
unspooled, threaded through the pullies and attached to the carriage.
\begin{verbatim}
motors> centralrope reset              -- to reset the encoder on the central rope
  
motors> centralropemotor setcruise 2   -- change from maximum 4cm/c to 2cm/s
                                          so it feeds off the drum slower.
  
motors> centralrope down 10000         -- feed out string
  
motors> stop                           -- stop when you have enough
  
motors> centralropemotor setcruise 4   -- change speed back to max.
\end{verbatim}  
  
A session where the {\tt prototype} is disconnected and reconnected.
\begin{verbatim}
motors> prototype disconnect
motors> prototype connect eastrope centralrope westrope
motors> show prototype
\end{verbatim}
Note that the {\tt show} command has a different syntax.  This is because
it is a system wide command unlike the others which are object commands.
  

  
  
%----------------------------------------------------------------------
\newpage
\section{Running the Manipulator from SHaRC}
  
You can run the manipulator through the data acquisition program.  First
make sure that the manipulator program is running on the PC. Then start
the program by clicking on :
\begin{verbatim}
  Windows -> Configuration 
\end{verbatim}
Then `Splat-Click' on MAN-1 to get a menu :
\begin{verbatim}
  Configure
  Basic Ops
  Special Ops
  \end{verbatim}
  select {\tt Configure}

\begin{enumerate}
\item Select {\tt Basic Ops} and a window with the manipulator status
  will pop up.  This window has places to input commands to the manipulator
  and readback.
\item You can get a graphical display of the AV and the manipulator
  position from the {\tt Special Ops} option.

%\item To shut down the GUI interface to the manipulator,
%  \begin{enumerate}
%  \item shut down the DAQ.  MAC will wait forever trying to close the TCP/IP
%    connection.
%  \item close the TCP/IP connection from the PC by using either the 
%    {\tt logout} command or by quitting.
%  \end{enumerate}
\end{enumerate}

  



%--------------------------------------------------------------
%--------------------------------------------------------------
 %--------------------------------------------------------------
\section{Data Logging}
\shwlabel{SecDataLogging}
  
  There are several means by which data from the manipulator
system are logged.
\begin{itemize}
\item {\bf manip\_logger} web based logging tool
\item CMA logging of AV rope lengths
\item Command and Error logging on {\bf manip}
\item Explicit logging of source data through {\bf manmon}
\end{itemize}
   

\subsection{manip\_logger web based data logging}

 At preset the {\bf manip\_logger } is running on crag1. The output is stored in
~detector/public\_html/private/manip\_logger . The polling can be configured in
the file ``extra\_log.dat''  . A ``monitor'' argument should be added for frequently
polled commands to cut down the amount of noise in the log file(s).
An example file follows.

\begin{verbatim}

%Here is my documentation:

%--------------------------------------------------------------- 

This is the current "extra_log.dat" file:

salt {
state = off;
delay = 60; // delay time (secs)
entry = "TESTROPELOADCELL", "Tension:", 1;
entry = "LASERBALL", "Position:", 3;
}

You can add up to 20 entries similar to this one.

To enable the above polling, simply edit this file and change
the state from "off" to "on".

This entry produces a log file called "salt_YYMM.log2" with
the following format:

2001 7 17 12 49 43 94.12 9.00 -25.00 0.00
2001 7 17 12 50 41 94.12 9.00 -25.00 0.00
2001 7 17 12 51 41 94.12 9.00 -25.00 0.00
2001 7 17 12 52 41 94.12 9.00 -25.00 0.00

The first 6 lines are year, month, day, hour, minute, second,
as with the other log2 files.  The next lines are those
specified by the "entry" lines in "extra_log.dat".  Here,
it has logged the testropeloadcell tension and the laserball
position.

Here are more details about the format of "extra_log.dat":

state - "on" or "off" to enable or disable logging for this item.

delay - nominal delay between log entries in seconds (actual
delay implemented is the nearest multiple of 10 seconds).

entry - The first argument is the object to poll (case
insensitive).  The 2nd argument is the case-sensitive string
to match the specific parameter of the "monitor" command to
be logged.  This must be an exact match of all characters
up to and including the colon.  The 3rd argument is the number
of values to log for this parameter (ie.  x,y,z position is
3 values).

The manip_logger checks every logging cycle (10 seconds) to
see if this file has been modified.  If it has, the new
settings are loaded.

\end{verbatim}



\markright{CalOp: Manupulator}

%--------------------------------------------------------------
%--------------------------------------------------------------
%--------------------------------------------------------------
\chapter{Manipulator}
\shwlabel{ChapterManipulator}
  


  The SNO calibration source manipulator is a positioning device
used to place calibration sources inside the Acrylic Vessel of the
SNO detector or down special calibration guide tubes in the region between
the AV and the PSUP.  By using a system of three ropes, a central and two
side ropes, the manipulator is able to position a source on either an east
west plane or north south plane inside the AV.  About 3/4 of the plane inside
the AV can be reached by the manipulator, the remaining quarter is off limits
due to the geometry of the manipulator system.  In addition to the manipulator
ropes (referred to as {\bf axes}) there is an {\bf umbilical} attached
to the manipulator that provides the necessary services for the source
(electrical signals, fibre optics, gas lines etc).
  
\section{Overview}
  Each calibration source is stored in an {\bf Umbilical Retrieval Mechanism}
or {\bf URM}.  A URM consists of a block and tackle mechanism for taking
up the source {\bf Umbilical} used to provide services to the source
and a {\bf central rope} used to support the weight of the source.  Below
the URM is the {\bf source tube} which is a 4' long stainless steel pipe
used to store sources when not deployed in the vessel.  Normally, the 
URM and source tube are mounted on a calibration port on the {\bf glovebox}
which is located on the {\bf universal interface} located directly over
the neck of the acrylic vessel.  When not in use, the source is stored in
the source tube and a gate valve on the glove box seals off the detector.
The central rope in the URM is instrumented with a {\bf shaft encoder}
which determines the length of rope played out and a {\bf load cell} used
to measure the tension in the rope.  The umbilical is similarly instrumented.
  
  The layout of the system is shown schematically in figure \ref{figmansystem}.

\begin{description}
\item[Anchor Blocks]~\\
  Each side rope can be thought of as attached at two points (not exactly 
  true).  At the feedthrough where it comes down from the roof of the DCR
  into the glovebox and at the {\bf anchor block} in the AV which is located
  just above the AV equator.  The end of the rope at the anchor block is fixed
  and by playing the rope in or out through the glovebox feedthrough, the 
  calibration source is moved about the AV.
\item[Calibration Guide Tubes]~\\
  In addition to deploying sources through the glovebox into the centre of the
  AV, it is possible to deploy sources through 6 calibration guide tubes into 
  the light water volume between the AV and the PSUP.  These guide tubes are 
  located on the floor of the DCR and are sealed with gate valves.
\item[Carriage and Weight]~\\
  Attached to each calibration source is a {\bf carriage} and a {\bf weight}.  
  The carriage provides attachment points for the central rope and umbilical 
  and has pullies that the side ropes go around.  The weight cylinder is a 
  stainless steel tube containing lead.  The manipulator requires a minimum 
  weight for each source to function properly  (in particular to give the 
  sources negative bouyancy) and the weight cylinder provides this.
\item[Deck Clean Room(DCR)]~\\
  Also known as the Dark Clean Room, the DCR is the room centred on the deck.
  Most of the calibration equipment is located inside the DCR.  The DCR is
  kept clean and has relatively few airborn particles compared to the rest of
  the lab.  The clean conditions are maintained to prevent introduction of 
  radioactive
  contamination into the detector during the deployment of sources.
\item[Glove Box]~\\
  The glove box is the rectangular box with many valves and flanges.  It is 
  mounted on the Universal Interface (UI) directly over the top of the AV.  In 
  addition to sensors used by the water group to monitor the heavy water 
  levels, the glove box
  has three ports on it that calibration sources can be mounted on.  The 
  glove box gets it's name from the four glove ports on its sides.  The 
  gloves are used to attach the side ropes to the manipulator carriage 
  which must be done in darkness
  (to protect the PMT's) and in the radon free cover gas that caps the AV.
\item[Rubbing Ring]~\\
  The rubbing ring is an acrylic ring located just below the neck of the AV 
  inside the heavy water volume.  When the manipulator positions a source 
  off the central axis, the manipulator ropes and the source umbilical are 
  pulled to the side of the AV neck. The rubbing ring provides a wearing 
  surface for the ropes.
\item[Side Rope Motor Mounts]~\\
  The spooling mechanisms for the side ropes are located above the DCR 
  (Deck Clean Room).  They consist of a motor driven spool system 
  instrumented with a loadcell to measure the rope tension and a shaft 
  encoder to measure the rope length.  There are four side ropes, North, 
  South, East and West which are operated in pairs to allow positioning
  of the source inside the AV on an East-West plane or a North-South plane.
\item[Source Tube]~\\
  The stainless steel tube connecting the URM to the calibration port.
  The calibration source is parked in the source tube when not deployed in
  the detector.
\item[Umbilical Retrieval Mechanism (URM)]
  The unit to which a calibration source is attached consisting of
  a rope to support the weight of the source and an umbilical which
  provides services (power, signals control, light etc.) to the
  source.
\item[Universal Interface (UI)]~\\
  The universal interface (UI) is the stainless steel circular platform located
  in the center of the DCR directly over the AV.  It has mounted on it the 
  glove box used to deploy calibration sources into the detector.
\end{description}
  
  The manipulator carriage and weight assembly is shown schematically
in figure \ref{figmancarriage}.
\begin{figure}[htbp]
\begin{center}
\leavevmode
\epsfxsize=3in
\epsfbox{figures/mancarriage.ps}
\caption[Manipulator Carriage]{
  \shwlabel{figmancarriage}}
  Manipulator Carriage and Weight Assembly (not to scale).
\end{center}
\end{figure}
It consists of the {\bf carriage} to which the central rope is
attached.  The umbilical passes through the {\bf carriage neck},
through the weight assembly into the source.  The side ropes are
not attached to the carriage, but rather pass around {\bf pullies}
mounted on the {\bf pully bar}.  the pully bar and carriage neck are
free to rotate about the {\bf pivot}.  At different positions in the
AV, the pully bar and carriage neck will be at different orientations
while the weight assembly and source will always hang vertically below
the pivot.
 
  The weight consists of a stainless steel torus filled with lead.  The
lead is potted into the {\bf weight cylinder} with silicone and 
then capped with the {\bf cylinder end plate} which is sealed with o-rings.
Sources are usually attached to the weight cylinder using the 
{\bf extension tube}.


%=====================================================================

\section{Manipulator Control System}

  The manipulator is controlled by the {\bf manip} computer
which is a DOS based PC running a C++ program also called {\bf manip}.
The manip program interacts with the manipulator hardware by
by controlling a stepper motor for each axis to change the length
of the rope or umbilical.  A shaft encoder on each axis measures
the length and a loadcell measures the tension.  The shaft encoders
and load cells are connected to {\em encoder boxes}, one per axis.
In the encoder box an up/down counter counts the  number of steps
taken by the shaft encoder.  The loadcell is connected to an amplifier
in the encoder box.  The encoder box is in turn read out
by the {\em Data Concentrator Box} which contains up to eight
{\em Data Concentrator Cards} Each data concentrator card can read
out up to 4 encoder boxes.  The shaft encoder up/down counters 
are read out through a digital bus through the data concentrator cards.
The amplified signals from the load cells are fed to a multiplexing
ADC located in the data concentrator card.  Each encoder box has a
unique digital address and one of four analog addresses (for the 
loadcell signals).  The analog addresses must be unique on a given
data concentrator card.  The data concentrator box is read out
from the {\bf manip} computer via a PLC750 card which contains a
parallel bus connection to the Data Concentrator Card.  Up to 32
axes (rope or umbilical) can be monitored by the Data Concentrator
Box (4 encoder boxes on each of the eight data concentrator cards).
  The stepper motors
are controlled by one of two TIO10 cards in {\bf manip}.  These
cards produce stepper motor control signals and pulse trains to 
step the motors.  Eight motors can be controlled by each TIO10 
card for a total of 16 motors.  The TIO10 cards are 
are interfaced to the stepper motors through a 
{\em Watchdog timer box}.  In addition to the motor control signals
from the TIO10 cards, the watchdog timer box takes an interlock
signal generated by the PLC750 card in the {\bf manip} computer.
If signal has a timeout that has to be reset by the {\bf manip}
program.  In the event of the manip program stopping the interlock
signal turns off.  The Watchdog Timer box then shuts off the motors.
A block diagram of the manipulator control system is shown in
figure \ref{figmanctrl}.
 
\begin{figure}[htbp]
\begin{center}
\leavevmode
\epsfxsize=6in
\epsfbox{figures/MANCTRL.eps}
\caption[SNO manipulator control system]
  {SNO manipulator control system.
  \shwlabel{figmanctrl}}
\end{center}
\end{figure}


  


%=====================================================================

\section{Modes of Source Deployment}

\subsection{Single Axis Deployment}
  
  Sources can be deployed in a {\bf single axis mode} which consists of
lowering a source straight down from the URM on just the central rope
and umbilical.  The horizontal position of the source is determined
by the location of the URM.  The vertical position of the source is
determined by the measured length of central rope played out.  The single
axis deployment mode is useful for operation along the central axis of
the detector and for deployment of sources down the guide tubes.
  
\begin{figure}[htb]
\begin{center}
\leavevmode
\epsfxsize=5in
\epsfbox{figures/mansingleaxis.ps}
~\\
\caption[Single Axis Source Deployment]
        {Single Axis Source Deployment
         \shwlabel{figsingle}
        }
\end{center}
\end{figure} 
  
  
\subsection{Three Axis Deployment}
  The main purpose of the manipulator however, is to deploy a source
{\em off} the central axis of the detector inside the acrylic vessel.
This is done attaching two {\bf side ropes} to the manipulator carriage
once it is deployed into the glovebox.  The side ropes are attached at 
one end to {\bf anchor blocks} in the AV are anchored at the other end
by feedthroughs on the glovebox.  The side ropes go over pullies on 
the manipulator carriage.  Once the source is lowered into the vessel, it
can be pulled off the central axis by shortening one side rope and lengthening
the other.  Because only two side ropes are attached at a time, the source
can only be moved in a plane.  There are two sets of side ropes allowing
motion in an east-west plane or a north-south plane.  The side ropes
are instrumented in the same fashion as the central rope with the 
side rope motor mounts located on the roof of the DCR.  The ropes pass
through the roof of the DCR into the glovebox through stainless steel tubes.
  
\begin{figure}[htb]
\begin{center}
\leavevmode
\epsfxsize=5in
\epsfbox{figures/manpolyaxis.ps}
~\\
\caption[Three Axis Source Deployment]
        {Three Axis Source Deployment
         \shwlabel{figpoly}
        }
\end{center}
\end{figure} 
  
  
  The manipulator is controlled by the manipulator computer, a DOS based
PC.  The manipulator computer runs a C++ based program called {\bf manip}
which monitors the instrumentation on the manipulator, calculates the positon
of the source and accepts commands to control the manipulator.  {\bf manip}
can be accessed both from the console in the DCR and remotely via TCP/IP.
When taking data control of the manipulator is nominally done through the
SNO DAQ.  The reason for this is that the DAQ then automatically incorporates
any change in the calibration source configuration into the data stream.
In addition there is a standalone unix utility called {\bf manmon} which
allows remote monitoring of the manipulator and is useful for diagnostics.
  
\begin{figure}[htb]
\begin{center}
\leavevmode
%\epsfysize=0.85\textheight
\epsfxsize=5in
\epsfbox{figures/rope_lengths.ps}
~\\
\caption[Rope Lengths]
        {Rope Lengths
         \shwlabel{figropelengths}
        }
\end{center}
\end{figure} 
  
\begin{figure}[htb]
\begin{center}
\leavevmode
%\epsfysize=0.85\textheight
\epsfxsize=5in
\epsfbox{figures/tension0_0.ps}
~\\
\caption[Rope Tension]
        {Rope Tension
         \shwlabel{figropetension00}
        }
\end{center}
\end{figure} 
   
\begin{figure}[htb]
\begin{center}
\leavevmode
%\epsfysize=0.85\textheight
\epsfxsize=5in
\epsfbox{figures/tension0_2.ps}
~\\
\caption[Rope Tension]
        {Rope Tension
         \shwlabel{figropetension02}
        }
\end{center}
\end{figure}
  
   


\markright{CalOp: Geometry}
  
%--------------------------------------------------------------
%--------------------------------------------------------------
%--------------------------------------------------------------
\chapter{Manipulator Geometry}
\shwlabel{ChapterGeometry}
  
\section{Global Coordinate System}
  When positioning the manipulator the user is in fact positioning the
carraige pivot of the manipulator in the {\bf global coordinate system}
which is located at the designed centre of the PSUP and the AV.  Be
warned that neither the PSUP nor the AV are actually expected to be centred
on the global origin. In fact it is known that the PSUP has shifted by
at least 2'' and the AV shifts depending on the load.  Because the exact
location of the AV is needed to position the manipulator (need to know the
location of the anchor blocks in the AV) there are neck monitors used
to measure the position of the top of the AV neck and thus infer the position
of the centre of the AV.
  
  From the construction drawings the deck of the DCR
floor is located at
\begin{verbatim}
        100' = 1200 in.
\end{verbatim}
by definition.  The nominal centre of the AV 
and thus the origin of the global coordinate system is located at a height
of 
\begin{verbatim}
           56' 7 1/2 `` = 679.5 in.
\end{verbatim}
Therefore the distance from the global origin to the deck is
\[
        520.5 in. = 1322.07 cm
\]
by design.
  
\begin{table}[htbp]
\begin{center}
\begin{tabular}{|l|l|l|l|}
\hline
measurement                  & dim (in)     & dim (cm)        & from
\\ \hline
height of DCR floor          & 100'= 1200'' & 3048.00         & definition \\
height of global origin      & 56' 7 1/2'' = 679.5'' & 1725.93 & design \\
d(global origin to DCR floor)& 520.5''               & 1322.07 & calc \\
\hline
\end{tabular}
\caption[Global Coordinate System]
        {Global Coordinate System
         \shwlabel{tabglobaldim}
        }
\end{center}
\end{table}
  
  
\begin{table}
\begin{center}
\begin{tabular}{|l|l|l|l|} \hline
                               & coord system & location & source of measurement \\
DCR Floor                      & global &  z = 1318.47 cm &  \\
\hline
Bottom of Tube Flange on URM-1 & global & z= 1504.43 & \\
\hline
Height of side rope feedthroughs & global & z = 1424.22 & \\
on glovebox                      & & & \\
\hline
\end{tabular}
\caption[Manipulator Geometry]
        {Manipulator Geometry
         \shwlabel{tabmangeo}
         }
\end{center}
\end{table}
  
  
\section{Glovebox and Universal Interface}
  
  The dais of the universal inteface is located 17 and 1/8 `` above
the floor of the cleanroom.  Taking the nominal height of the DCR floor
as 1322.07 cm, the UI Dais is located at 1384.30 cm and the top of the
glovebox is located at 1427.80 cm in global coordinates.
\begin{table}[htbp]
\begin{center}
\begin{tabular}{|l|l|l|l|}
\hline
measurement                      & dim (in)   & dim (cm) & from \\ \hline
DCR floor to UI Dais             & 17 1/8 ``  & 43.50    & measurement \\
UI Dais to top plate of glovebox & 24 1/2 ``  & 62.23    & measurement \\
Nominal height of UI Dais        &            & 1384.30  & measurement \\
Nominal height of glovebox top   &            & 1427.80  & measurement \\
\hline
westrope feedthrough  x          & -21.000''  & -53.340  & Drawing \\
westrope feedthrough  y          &   0.750''  &   1.905  & Drawing \\
\hline
eastrope feedthrough  x          &  21.000''  &  53.340  & Drawing \\
eastrope feedthrough  y          &   0.750''  &   1.905  & Drawing \\
\hline
northrope feedthrough  x         &   0.000''  &   0.000  & Drawing \\
northrope feedthrough  y         &  15.750''  &  40.005  & Drawing \\
\hline
southrope feedthrough  x         &   0.000''  &   0.000  & Drawing \\
southrope feedthrough  y         & -20.250''  & -51.435  & Drawing \\
\hline
10'' gate valve        x         &   0.000''  &   0.000  & Drawing \\
10'' gate valve        y         &  -8.500''  & -21.590  & Drawing \\
\hline
 6'' gate valve        x         &   6.656''  &  16.906 & Drawing \\
 6'' gate valve        y         &   9.250''  &  23.495 & Drawing \\
\hline
 4'' gate valve        x         &  -6.313''  & -16.035 & Drawing \\
 4'' gate valve        y         &   9.250''  &  23.495 & Drawing \\
\hline
\end{tabular}
\caption[Glovebox and UI dimensions]
        {Glovebox and UI dimensions
         \shwlabel{tabgbdim}
        }
\end{center}
\end{table}
  
  
\section{Acrylic Vessel}
The thermal expansion coefficient for the acrylic is,
\[
        6 \times 10^{-5} C^{-1}
\]
The design specs for the AV give the 
distance from top of chimmney to centre of vessel at 23 C
\begin{verbatim}
        42' 2 3/8''
\end{verbatim}
which is 506.375 cm. and the
nominal outside radius
\begin{verbatim}
        236.6''
\end{verbatim} which is 600.964 cm.
with a nominal thickness of 2.15'' (5.461cm).

This can be compared to the results found in SNO-STR-98-003 (R. Komar) 
for actual measurements of the AV.
\begin{table}[htbp]
\begin{center}
\begin{tabular}{|l|l|l|}
\hline
measurement                    & design     & as built \\ \hline
Vessel Inner Radius            &  236.43''  & 236.38$\pm$0.23'' \\ 
Top of Chimney to AV centre    & 506.375''  & 506.16$\pm$0.12'' \\
Top of Chimney to bottom of AV & 742.59''   & 742.59$\pm$0.05'' \\
\hline
\end{tabular}
\end{center}
\end{table}
Using the Komar measurements, the nominal dimensions of the AV
are given in table \ref{tabavdim}.
\begin{table}[htbp]
\begin{center}
\begin{tabular}{|l|l|l|l|}
\hline
measurement                      & dim (in)   & dim (cm)& from \\ \hline
Average Vessel Inner Radius &  236.38$\pm$0.23'' & 600.41$\pm$0.58 
                            & measurement\\
Top of Chimney to AV bottom &  742.59$\pm$0.05''  & 1886.18$\pm$0.13
                            & measurement\\
Top of Chimney to AV centre &  506.16$\pm$0.12''  & 1285.65$\pm$0.30 
                            & measured?\\
Neck Ring gasket            &   1/8''             & 0.3175 & measured\\
Neck Ring plate             &   3/8''             & 0.9525 & measured \\
AV Centre to AV top plate   &  506.66''           & 1286.92 & calculated \\
DCR floor to AV top plate   &  12.4375       & 31.59 & measured/calculated\\
\hline
\end{tabular}
\caption[Acrylic Vessel Dimensions]
        {Acrylic Vessel Dimensions
         \shwlabel{tabavdim}
        }
\end{center}
\end{table}
On top of the AV neck flange is a gasket (1/8'') and a stainless steel
top plate (3/8'').  This gives a distance from the centre of the
AV to the top plate of,
\[
     506.16 + 1/8 + 3/8 = 506.66 in = 1286.92 cm
\]
The distance from the AV top plate to the UI flange on the DCR floor
was measured before the UI was installed.  (This flange is no longer
accessable since the UI has been installed.)  This was a measurement
after the final installation of the AV at nominal lab temperature before
any water in the AV.
The distance measured on 3 April 1998 was
28 9/16''.  The distance of the flange from the DCR floor was measured
to be 16 1/8''.  Therefore the distance of the AV topplate from the
DCR floor
\[
           28 9/16 - 16 1/8 = 12.4375'' = 31.59 cm
\]
The distance from the DCR floor to the centre of the AV from the 
measurement of the topplate location and the Komar measurements of the
AV are therefore,
\[
          1286.92 cm + 31.59 cm  = 1318.51 cm
\]
which corresponds to the AV being located above the nominal position
by 
\[
           1322.07 - 1318.51 = 3.56 cm
\]







\section {Calibration Guide Tubes}
\shwlabel{SecCalGuideTubes}
  The locations of the calibration guide tubes in the Deck Clean Room
are shown in figures \ref{FigGuideTubes} and \ref{FigGuideTubesElevation}.


\begin{table}
\begin{center}
\begin{tabular}{|l|r|r|r|r|r|}
\hline
Tube &   X       &   Y     & Enter & Z Leaves & Z Touches\\
     &    (cm)   &   (cm)  & PSUP  & PSUP     & AV(cm)   \\
\hline
 1   &  361.00   &  193.04 & 734.36 &         & 446.91 \\
\hline
 2   & -112.08   &  104.14 & 825.47 &         & 586.44 \\
\hline
 3   & -361.00   &  193.04 & 733.51 &         & 446.91 \\
\hline
 4   & -586.11   &  207.96 & 564.64 & -564.64  & \\
\hline
 5   & -586.11   & -252.41 & 546.23 & -546.23  & \\
\hline
 6   &  118.11   & -119.38 & 823.80 &          & 582.32 \\
\hline
\end{tabular}
\caption[Calibration Guide Tube Locations]{
  Calibration Guide Tube Locations.
  \shwlabel{TabGuideTubes}
  }
\end{center}
\end{table}

\begin{figure}
\begin{center}
\epsfxsize=7in
\epsfbox{./figures/guide_tubes.ps}
\end{center}
\caption[Calibration Guide Tubes]
        {
         Calibration Guide Tubes located in Deck Clean Room.
         \shwlabel{FigGuideTubes}
        }
\end{figure}



\begin{figure}
\begin{center}
\epsfxsize=7in
\epsfbox{./figures/guide_tubes_elevation.ps}
\end{center}
\caption[Calibration Guide Tube Elevation]
        {
         Elevations of Calibration Guide Tubes.
         \shwlabel{FigGuideTubesElevation}
        }
\end{figure}







\markright{CalOp: Calibration Gas Systems}

%========================================================================
%========================================================================
%========================================================================



\chapter{Calibration Gas Systems}
\shwlabel{ChapterCalGas}
  
  The SNO Detector calibration systems utilize several different
sources of gas in the DCR.  Shown in figure \ref{FigDcrGas},
these are:
\begin{description}

\item[Vacuum] There is a vacuum pump located in the Junction with
  a 2 inch (check) line terminating at a valve in the cable tray above
  the pipe box in the DCR.

\item[High Pressure N$_2$] gas derived from a 150 PSIG LN$_2$ dewar
  located in the Junction.  This N$_2$ gas has no significant amounts of
  oxygen or radon and is used to feed the Calibration N$_2$ laser
  and to supply (relatively) high pressure flushes of the 
  URMs.

\item[Low Pressure N$_2$] gas derived from the boil off of the
  Detector Cover Gas Wessington Dewar located in the Junction.  This
  gas has no significant amounts of oxygen or radon and is used
  to maintain a low rate flush of the URMs and the side rope units
  (located on the roof of the DCR).

\item[Instrument Air] compressed air taken from the laboratory 
  house air supply.  This is derived from the INCO compressed air
  with a booster compressor located outside the  car wash.  This 
  is compressed air containing the normal concentrations of oxygen
  and radon.  It is used to pressurize the tensioning cylinders in
  the URMs which provide the tension on the URM umbilicals.

\item[Radioactive Gas] There is a radioactive gas handling system
  located in the Junction.  This primarily used for the  $^{16}$N
  gamma ray source but is also used for the $^{8}$Li $\beta$
  source and the $^{17}$N neutron source.

\end{description}

 
\begin{figure}[htb]
\begin{center}
\leavevmode
\epsfxsize=7in
\epsfbox{figures/dcr_gas_systems.ps}
~\\
\caption[DCR Gas Systems]
        {DCR Gas Systems
         \shwlabel{FigDcrGas}
        }
\end{center}
\end{figure}


\section{(Proposed)Calibration Gas System Nomenclature}

  To distinguish the different gas sys handling systems for
the SNO Calibration equipment and to distinguish the calibration
gas systems from the other SNO systems a unique naming convention
is proposed for the calibration systems.  All Calibration 
gas components will start with the letter ``C''.


\begin{table}
\begin{center}
\begin{tabular}{|ll|}
\hline
  CMV  & Mechanical valve \\
  CSV  & Solenoid valve \\
  CRV  & Relief valve \\
  CPR  & Pressure Regulator \\
  CFM  & Flow meter \\
  CPG  & Pressure gauge (mechanical) \\
  CPT  & Pressure transducer (electronic readout) \\
  CP   & Pump\\
  CVP  & Vacuum Pump\\
  CCV  & Check valve\\
\hline
  100 series & Vacuum \\
  200 series & House Air \\
  300 series & Low Pressure N$_2$ \\
  400 series & High Pressure N$_2$ \\
  600 series & Calibration Laser \\
  800 series & Radioactive Gas \\
\hline
\end{tabular}
\caption[Proposed numbering scheme for gas system components]
  {Proposed numbering scheme for gas system components
   \shwlabel{TabGasNumbering}
  }
\end{center}
\end{table}

  
%========================================================================
\clearpage
\section{Nitrogen Flush System}
\shwlabel{SecN2Flush}
  
\subsection{Introduction}  

  The gas board is designed to allow the URMs and the side rope motor boxes
to be flushed with dry  radon free N$_2$. In normal mode  the gas flow to the
motorboxes and the URMs is restricted to a few liters per minute in order
not to perturb the D$_2$O covergas system and to conserve LN$_2$. For initial
flushing the flow may be increased to the URMs provided the gatevalve is closed.
This is referred to as `bypass mode' below.

 The gas enters the board on the left side and leaves through one or more exits
at the bottom. There are individual lines to each of the side rope motor boxes
located on the roof of the DCR and a common line to the URMs. This line is 
in series with a flow meter and a needle valve ( both located at the south 
east corner of the pipebox ). From there it branches out to the two URMs.


 
 {\em Note : The maximum allowed pressure for the flush system is 10 psi.
Do not under any circumstances exceed this pressure ! }

 


\begin{figure}[htb]
\begin{center}
\leavevmode
\epsfxsize=7in
\epsfbox{figures/flush_gas_panel.ps}
~\\
\caption[DCR Flush System]
        {N$_2$ Flush Gas Board
         \shwlabel{FigFlushGasBoard}
        }
\end{center}
\end{figure}

\subsection{Operation}

  N$_2$ to the board comes from one of two sources :

\begin{itemize}

\item Low pressure supply.

  In this mode the gas is supplied from the Wessington dewar located in 
the junction area.  The gas enters through the lower left input line and 
the pressure is controlled by a fixed regulator on the dewar.


\item High pressure supply.

 Here the gas supply is a 160 psi dewar located in the junction. This dewar
is also used for the laser. Gas enters the board through the top 
left line and flows through the regulator on the board which should be 
set at { \em no more than 10 psi}. Note that there is a checkvalve in  the 
low pressure line which prevents gas flowing  back to the Wessington. This 
particular valve appears to be `missing' on the board (~in case you were 
wondering ...~).


\end{itemize}



 As indicated above the gas board can be operated in one of two modes :

\begin{itemize}

\item ``Normal mode''

  In this mode the gas flows through small restrictions to the devices 
and the flow is limited to approximately 1 liter per minute. 


\item ``Bypass mode'' 

  This mode is used  for rapid flush of a URM.   Direct the gas to the bypass
line and set the three-way valve at the URM line appropriately.


\end{itemize}
  



  
%========================================================================
\clearpage
\section{Radioactive Gas System}
\shwlabel{SecRadioactiveGas}

 
\begin{figure}[htb]
\begin{center}
\leavevmode
\epsfxsize=7in
\epsfbox{figures/radioactive_gas_system.ps}
~\\
\caption[Radioactive Gas system]
        {Radioactive Gas Handling System
         \shwlabel{FigRadioactiveGas}
        }
\end{center}
\end{figure}




\markright{CalOp: Laser}
%--------------------------------------------------------------
%--------------------------------------------------------------
%--------------------------------------------------------------
\chapter{Laser}
\shwlabel{ChapterLaser}

\newprocedure{CalOpLaser}
       {Laser}
      {P. Skensved}{Sept. 2003}{2}

  
   
\begin{figure}[htb]
\begin{center}
\leavevmode
%\epsfysize=0.85\textheight
\epsfxsize=7in
\epsfbox{figures/lasergas.eps}
~\\
\caption[Laser Gas System]
        {Laser Gas System
         \shwlabel{figlasergas}
        }
\end{center}
\end{figure}
  
\section{Commands}
\begin{description}
\item[n2laser]~\\
  Lists commands for n2laser
\item[n2laser monitor]~\\
  displays status information on the n2laser.
\item[n2laser poweron]~\\
  Turns on the power and the N$_2$ gas to the laser
\item[n2laser poweroff]~\\
  Turns off the power and the N$_2$ gas to the laser
\item[dyelaser]~\\
  display dyelaser commands
\item[dyelaser init]~\\
  Initialize the dyelaser.  Must be followed by {\tt dyelaser findzero}
\item[dyelaser findzero]~\\
  Find the zero position of the dyelaser mirror.  The mirror
  travels down it's track till it hits a stop.
\item[dyelaser cell <0-4>]~\\
  Select the dyelaser cell between 0 and 4.
\item[filterwheela]~\\
  Show commands for filterwheela.
\item[filterwheela init]~\\
  Initialize filterwheela.
\item[filterwheela findtab]~\\
  Find the calibration tab on filterwheela.
\end{description}
  
\section{Procedures}

\subsection{Laser Emergency Shutdown}
\begin{enumerate}
\item Pull the powercord to the Laser.
\item If possible turn off the N$_2$ gas at the dewar ( valve MV1 ). 
\end{enumerate}


\subsection{Starting the Laser}
\begin{enumerate}
\item Open MV1 the manual valve on the LN$_2$ dewar.
\item Open MV2 the pressure builder valve on the LN$_2$ dewar.
\item Note pressure on PG1, the pressure gauge on the dewar.
      The pressure should be at least 120 psig.
\item Enter DCR and make sure the manual valve PRV8 is open.
\item Plug in the powercord to the laser.
\item On the laser set the POWER switch to {\bf remote}
\item On the laser set the CONTROL switch to {\bf remote}
\item Check that the manual shutoff valve MV9 is open.
\item Reset the `Kill Switch' by pushing the red reset button.
\item Do a 
 \begin{verbatim}
 n2laser poweron
 \end{verbatim}
from the manip computer. This will turn on
the low voltage control power to the laser and energize the solenoid valve
which controls  the N$_2$ gas flow to the laser head.
\item Note values on PG2 and PG3 and FG1.
  PG2 should be at least 120 psig, PG3 should be at least 100 psig and
  FG1 should be at least 40 psig.
If there is no flow consult an expert. {\bf Running the laser without sufficient N$_2$ flow
causes serious damage to the laser head !}


\item Note time.
\item At this point the Thyratron should be warming up ( controlled by an
internal timer in the laser ). Make sure that the head is thoroughly flushed
before turning on the high voltage (~wait $\approx$ 10 minutes~)
\item While waiting test the dyecell selection and check that the neutral density filter selection works.
\item Use the manip computer to read PT4 and PT5.
  PT4 should be at least 80 psi.
\item Make sure the light is blocked
\begin{verbatim}
 n2laser setnd -1
\end{verbatim}
\item Once the Thyratron is warmed up and the head is flushed 
the laser may be turned on with the 
\begin{verbatim}
n2laser start
\end{verbatim}
command.
\end{enumerate}

\subsection{Calibrating the Filter Wheels}
The control system can sometimes lose track
of where the filter wheels are positioned.
To fix this follow this procedure.  This example
is for {\tt filterwheela}, replace with 
{\tt filterwheelb} for the 2nd filterwheel.
\begin{enumerate}
\item Reinitialize the filterwheel
  \begin{verbatim}
  filterwheela init
  \end{verbatim}
\item Find the tab on the wheel
  \begin{verbatim}
  filterwheela findtab
  \end{verbatim}
\item select the desired filterwheel position using
  the {\tt n2laser setnd} command.
\end{enumerate}

\subsection{Calibrating the Dye Cell Mirror}
\begin{enumerate}
\item Reinitialize the mirror
  \begin{verbatim}
   dyelaser init
  \end{verbatim}
\item Find the zero of the mirror
  \begin{verbatim}
  dyelaser findzero
  \end{verbatim}
\item Select desired wavelength ( Note that the 5 {\bf physical} dyecell locations
are shared between 10 {\bf logical} ones ). This is to distinguish bewteen 
for example having the 620 nm or the 365 nm dye in cell position 1.
  \begin{verbatim}
  dyelaser cell <0-9>
  \end{verbatim}
\end{enumerate}



\subsection{Shutting down the Laser}  
\begin{enumerate}
\item Block the light by typing 
\begin{verbatim}
n2laser setnd -1
\end{verbatim}
on the manip console
\item At the manip console type 
 \begin{verbatim}
    n2laser stop
 \end{verbatim}
\item Then type
\begin{verbatim}
  n2laser poweroff
\end{verbatim}
\item Unplug the powercord to the laser 
\item Turn off the pressure builder ( MV2 ) and the turn off the N$_2$ gas ( MV1 )
\item Close the manual gas valve on the laser ( PRV8 )
\end{enumerate}
  


\markright{CalOp: Sources}
\chapter{Sources}
\shwlabel{ChapterSources}


\newprocedure{CalOpSources}
       {Sources}
       {F. Duncan}{2001}{1}


  There are many source designed for the SNO detector.  Several of
them will have dedicated URM's assigned to them.
\begin{description}
\item[laserball] is a spherical source containing a diffusing material
  that isotropically distributes light from a Nitrogen laser pumped
  dye laser system.  It is used for the optical calibration of the detector.
  The laser is controlled by the manipulator computer and has both adjustable
  wavelength and intensity.

\item[$^{16}$N]
  The $^{16}$N source is a radioactive gas source used to provide 
monoenergetic gamma rays to measure the detector's energy response.  The
radioactive $^{16}$N gas is created in a d-t generator located in the
junction outside the control room and then piped into the DCR and down
into the source through an umbilical.
  
\item[Rotating Source]
  The rotating source is a device that has a collimated flashing light
source that spins on two axes.  By sweeping out the detector, it can
be used to verify that the locations of PMT's is correctly incorporated
into the various databases.  It requires electrical connections from
the rotating source umbilical.
  
\item[Sonoball]
  The sonoball is a sonoluminence source operating at 
approximately 25kHz.  It uses four wires out of the rotating source
umbilical.
  

\begin{table}
\begin{center}
\begin{tabular}{lcccccc} 
\hline
Source               &  Weight
   & d(pivot to centre) & d(pivot to bottom) & Volume & Max Diameter \\
   & (N)   & (cm)& (cm) & (cm$^{3}$) & (cm) \\
\hline
Laserball (Mark III) &  61  &  64.6 & 69.5  &   & 10.16 (4'') \\
\hline
$^{16}$N Source      &      &  71.0 & 78.59 &   & 11.43 \\
\hline
Camera               & $\sim$100  &  64.77 & 69.2 &   &  \\
\hline
\end{tabular}
\caption[Calibration Source Dimensions]
  {Weights, volumes and dimensions of the calibration sources.
   \shwlabel{TabSourceDimensions}
  }
\end{center}
\end{table}



%======================================================================  
\clearpage
\section{Laserball ( Mark III )}
\shwlabel{SecLaserball}
  
\begin{table}[htb]
\begin{center}
\begin{tabular}{|l|l|}
\hline
weight(source and carriage) & 61 N  \\
d(pivot to centre)          & 64.4 cm  \\
d(pivot to bottom)          & 69.5 cm \\
maximum diameter            & 10.16 cm (4'')\\
\hline
\end{tabular}
\caption[Laserball Parameters]
        {Laserball Parameters
         \shwlabel{tablaserballpars}
        }
\end{center}
\end{table}
  
  



%======================================================================  
\clearpage
\section{$^{16}$N Source}
  
\begin{table}[htb]
\begin{center}
\begin{tabular}{|l|l|}
\hline
weight(source and carriage) & \\
d(pivot to centre)          &  71.0 cm\\
d(pivot to bottom)          & 78.59 cm  \\
maximum diameter            & 11.43 cm \\
\hline
\end{tabular}
\caption[$^{16}$N Parameters]
        {$^{16}$N Parameters
         \shwlabel{tab16Npars}
        }
\end{center}
\end{table}


\begin{figure}
\begin{center}
\leavevmode
%\epsfysize=0.85\textheight
\epsfxsize=7in
\epsfbox{./figures/n16_partial_exploded.ps}
~\\
\caption[Exploded View of N16 Source]
        {Partial Exploded View of N16 Source
         \shwlabel{fign16exploded}
        }

\end{center}
\end{figure}




%======================================================================  
\clearpage
\section{Acrylic Sources}





\begin{figure}
\begin{center}
\leavevmode
%\epsfysize=0.85\textheight
\epsfxsize=7in
\epsfbox{./figures/acrylic_assembly_bw.ps}
~\\
\caption[Acrylic Source]
        {Acrylic Source
         \shwlabel{figacrylic}
        }

\end{center}
\end{figure}



%======================================================================  
\clearpage
\section{AmBe Source}

%======================================================================  
\clearpage
\section{Manipulator Camera}

  The Manipulator Camera is an epoxy encapsulated CCD camera with LED
light source.  It has a special mounting hardware which mounts on a 
standard manipulator weight cylinder but requires it's own camera
umbilical.  The assembled camera is shown in figure \ref{FigCamera}.

\begin{figure}[htbp]
\begin{center}
\leavevmode
\epsfxsize=3in
\epsfbox{figures/Camerab.ps}
\caption[Manipulator mounted Camera]{
  \shwlabel{FigCamera}}
  Manipulator mounted Camera.
\end{center}
\end{figure}



%--------------------------------------------------------------
%--------------------------------------------------------------
%--------------------------------------------------------------


\markright{CalProc: Calibration Procedures}

\renewcommand{\newprocedure}[5]{
\shwlabel{#1}
\markright{CalProc: #2 Rev.#5}
~\\
\begin{tabular}{|l|l|}
\hline
Written/Revised By  &  #3 \\
\hline
Date        &  #4 \\
\hline
Revision     &  #5 \\
\hline
\end{tabular}
~\\
\vspace*{0.25in}
}



\chapter{Calibration Procedures}
\shwlabel{ChapterProcedures}
  
    
%------------------------------------------------------------------------
%------------------------------------------------------------------------
%------------------------------------------------------------------------
\section{Manipulator Operation Procedures}
\shwlabel{secprocman}
 
  This section contains procedures related to the normal operation
of the manipulator system.  Procedures for specific sources or
tasks may supercede some of the procedures here.



%------------------------------------------------------------
\newpage
\subsection{Manipulator System Shutdown}


\newprocedure{CalProcManipulatorShutdown}
             {Manipulator System Shutdown}
             {Fraser Duncan/ Peter S.}{Sept. 2004}{1}


  
  The purpose of this procedure is to shutdown the calibration manipulator
electronics in an orderly fashion.  The circumstances when this should be
done are when there is a scheduled power outage to the underground lab.
Except in an obvious emergency, the manipulator computer should only be
shutdown with the permission of the Calibration Group {\bf and} the AV group.
  
\noindent
{\bf Outline of Procedure:}
\begin{itemize}
  
\item Stop the manipulator control program.
  
\item Turn off the manipulator computer.
  
\item Turn off the manipulator computer monitor.
  
\item Turn off the data concentrator.
  
\item Turn off the watchdog timer box.

\end{itemize}  

\noindent  
{\bf Prior to Starting this procedure:}
\begin{itemize}
   
\item Obtain permission to shutdown the manipulator computer from 
    the Calibration Group and the AV Group.
  
\item Verify that access to the DCR can be obtained.
\end{itemize}
  
\noindent
{\bf Procedure:}
  
\begin{enumerate}
\checkitem Enter the DCR.  The Manipulator electronics are in "Aksel's Garage",
     the alcove immediately to the right of the entry way to the DCR.  
  
\checkitem If the monitor is off, turn it on.  
     The manip program should be running.  This can be seen by the
     {\small\begin{verbatim} 
         manip>
     \end{verbatim}}
     prompt at the bottom of the screen.
    
\checkitem At the prompt, type the command
     {\small\begin{verbatim}
         quit
     \end{verbatim}}
     The manip program should shutdown either returning to a DOS prompt,
     {\small\begin{verbatim}
         C:\MOTORS.3_8\MANIP>
     \end{verbatim}}
     or it will display a message stating that the program will restart
     in 5 seconds.  You wish to prevent the program from restarting so
     type the letter "n" at the prompt. 

\small
{\em
Note that the version number may not
be exactly what is indicated above - this is ok. Also note
that sometimes the program stalls while exiting ( it has hung network connections ).
  Regardless of this procede to step 4 after waiting some 10 seconds. }
  \normalsize
  
\checkitem Press RESET and then turn off the manip computer as soon as the BIOS screen appear.  The power button is located on 
    the front.\\
     {\small\em Hitting RESET interrupts the hardware ( disks especially ) cleanly thus minimizing any
potential damage.
         The computer lights may be covered with aluminum tape, you may have to lift
         the tape to see the lights. Alternatively, check the monitor screen.
     }
  
\checkitem Turn off the monitor.
  
\checkitem Turn off the data concentrator box.  This is the box on the top
     shelf of the equipment rack.  When standing in front of it (the
     side with the label and cables) the power switch is on the back at
     the top left.  Off is in the down position.
  
\checkitem Turn off the watchdog timer box.  This is the box on the 
  second shelf of the equipment rack beside the computer.  Looking at the front
  of the box (the side with the label and the cables) the power switch
  is located on the back left side at the bottom.   Off is in the
  down position.
 
\end{enumerate}


{\small
~\\
~\\
\noindent
{\bf Revision History:}\\
\begin{tabular}{llll}
Rev. & Date & Author & Comments\\

0             & 
?    & 
Fraser Duncan &
\parbox[t]{3.0in}{
  First Draft
}
\end{tabular}
}





%===========================================================================
%===========================================================================
%===========================================================================

\newpage
\subsection{Manipulator System Startup}

\newprocedure{CalProcManipulatorStartup}
             {Manipulator System Startup}
             {Fraser Duncan/ Peter S}{?}{1}

  
  The purpose of this procedure is to start the manipulator control computer
after it has been turned off.  This procedure should only be done with
the permission of the Calibration Group.
    
\noindent
{\bf Outline of Procedure:}
\begin{itemize}  
\item Turn on the manipulator computer monitor.
  
\item Turn on the data concentrator.
  
\item Turn on the watchdog timer.
  
\item Turn on the computer.
  
\item Verify that the manip program has started correctly.
  
\item Verify that the CMA system has connected to the manipulator
    computer. 
\end{itemize}  
  
\noindent
{\bf Prior to Starting this procedure:}
\begin{itemize}   
\item Obtain permission to start the manipulator computer from 
    the Calibration Group.
  
\item Verify that access to the DCR can be obtained.
\end{itemize}
  

\noindent
{\bf Procedure:}
\begin{enumerate}  
\checkitem Enter the DCR.  The Manipulator electronics are in "Aksel's Garage",
     the alcove immediately to the right of the entry way to the DCR.  
  
\checkitem If the monitor is off, turn it on.
  
\checkitem Turn on the Data Concentrator Box.  This is the box on the top
     shelf of the equipment rack.  When standing in front of it (the
     side with the label and cables) the power switch is on the back at
     the top left.  On is in the up position.
  
\checkitem Turn on the Watchdog Timer Box.   This is the box on the second shelf
     of the equipment rack beside the computer.  Looking at the front
     of the box (the side with the label and the cables) the power switch
     is located on the back left side at the bottom.   On is in the
     up position.
  
\checkitem Get ready to turn on the Manipulator Computer.  This is the computer on the
     second shelf of the equipment rack.  The power switch is on the
     front.  When turned on the computer will go through the normal BIOS and DOS startup procedure
     and then ask you to verify that the hardware ( Data Concentrator and Watchdog Timer )
    has been turned on. Answer appropriately. Note that you have to respond within about
10 seconds. It is {\bf always} safe to answer N ( for no ) to this question.

\small
{\em Turning on manip without turning on the hardware will usually lead to a loss of 
important and vital calibration  information ( like rpoe positions ). 
 }
\normalsize

\checkitem Turn on the Manipulator Computer and answer the question about the hardware state.
Once running the screen should
     display the manipulator status and have a 
     \begin{verbatim}
         MANIP>
\end{verbatim}
     prompt at the bottom.


\small

{\em If you for any reason answered ``N'' the computer will end up with a normal DOS prompt. 
Simply type {\tt MANRUN} to continue. }

\normalsize



\checkitem Watch for the CMA computer to connect to the manipulator 
     computer.  This will be indicated by a message at the bottom of
     the screen indicating that a connection has been made from the
     IP address,
                      142.51.70.153
     The connection should be made within 20 seconds of the start of
     the program.  If the connection is not made within a minute or
     so, inform the AV group.
   
\end{enumerate}




{\small
~\\
~\\
\noindent
{\bf Revision History:}\\
\begin{tabular}{llll}
Rev. & Date & Author & Comments\\

0             & 
?    & 
Fraser Duncan &
\parbox[t]{3.0in}{
  First Draft
}
\end{tabular}
}



%====================================================================
\newpage
\subsection{Remote Reboot of the Calibration Manipulator Computer}

\newprocedure{ProcManipReboot}
             {Remote Reboot of the Calibration Manipulator Computer}
             {P. Skensved}{2002/10/26}{0}


  This procedure is used to reboot the calibration manipulator computer,
{\em manip}.  It is used if the computer freezes up or most commonly
if it refuses a TCP connection.  Rebooting the {\em manip} computer
should only be done with the approval of the SIC, SAC or the Calibration
Lead Hand.

\begin{enumerate}
\checkitem Log on to either {\em alcor}, {\em crag1}, {\em crag2},
  {\em crug1} or  {\em crug2}.

\checkitem run the {\bf wakelan} program by typing
  \begin{verbatim}
          wakelan 0004e21cc5eb
  \end{verbatim}

\checkitem Wait for {\em manip} to reboot (takes approximately 30 seconds).
  
\end{enumerate}


{\small
~\\
~\\
\noindent
{\bf Revision History:}\\
\begin{tabular}{llll}
Rev. & Date & Author & Comments\\
0           & 
2002/10/26  & 
Peter Skensved &
\parbox[t]{3.0in}{
  First Draft
}\\
\end{tabular}
}






%=================================================================
\newpage
\subsection{URM Light Leak Check}
\newprocedure{CalProcLightLeakCheck}
             {URM Light Leak Check}
             {Fraser Duncan}{2002/11/10}{1}


  After a URM has been opened up (cover plate taken off or removed
from the  glovebox), it is necessary to do a check for light  leaks.
This is done using the 24 OWL tube light monitor on the detector
itself.  This monitor consists of doing singles rate reads of the
top 24 OWL tubes that look up towards the deck and the DCR and 
glovebox.  


\begin{enumerate}
\checkitem Contact detector operator, verify that either the detector
  is in a maintenance run or has the UC bit set.
\checkitem Turn on Owl Tube Light Monitor
\checkitem Turn off DCR lights.
\checkitem While Watching the Owl Tube Light Monitor:
  \begin{enumerate}
  \item Open the gate valve for the URM being lightleak checked.
  \item Shine flashlight around the gate valve, 
         and then around all seals on URM
  \end{enumerate}
  %----------------------
  \small
  {\em
    Note that the light monitor only updates once a second.  The
    flashlight must be moved at an appropriate speed.
  }
  \normalsize
  %----------------------
\end{enumerate}
 
  



%=================================================
{\small
~\\
~\\
\noindent
{\bf Revision History:}\\
\begin{tabular}{llll}
Rev. & Date & Author & Comments\\
0           & 
?  & 
Fraser Duncan &
\parbox[t]{3.0in}{
  First draft
}\\

1             & 
2002/11/10    & 
Fraser Duncan &
\parbox[t]{3.0in}{
  Made procedure more general.
}
\end{tabular}
}



%=================================================================
\newpage
\subsection{URM Central Rope Length Calibration}
\newprocedure{CalProcCentralRope}
             {URM Central Rope Length Calibration}
             {Fraser Duncan}{2002/11/10}{1}

  The length calibration of the central rope for each URM is determined
by sighting the pivot of the manipulator carriage 
(see figure \ref{figmancarriage}) against a fiducial line scribed on the
window of the source tube viewing port. 


\begin{center}
\begin{tabular}{|l|c|}
\hline
   & \\
 URM2  on GV 1 with wide 4'' Source Tube & 1559.9 \\
  & \\
\hline
  & \\
 URM3  on GV 3 with normal 4'' Source Tube & 1558.5 \\
  & \\
\hline
\end{tabular}
\end{center}    


 The height of the fiducial
mark is also indicated on the side of each source tube.  If the number on the source tube
differs from the one listed above use the number written on the tube.





\noindent
{\bf Prior to Procedure:}
\begin{enumerate}
\item Source is above gate valve.
\item Gate valve is {\bf closed}.
\item Gate valve is locked or handle is removed.
\end{enumerate}

\noindent
{\bf Procedure:}
\begin{enumerate}
\checkitem Verify gatevalve  on glovebox below source tube is
  locked in the {\bf CLOSED} position.

\checkitem Open view port on source tube.  Requires 7/16" wrench.

\checkitem Operate manipulator until the centre of the manipulator
      carriage is at the horizontal line marked on window.\\
      {\small\em Note: The example below 
       assumes the n16 source.  for the laserball or
        a different source replace the object name {\tt n16} below as
        appropriate.}\\
      From the manip console:
  \begin{center}
  \begin{tabular}{|l|l|}
  \hline
  console & {\tt manip$>$ n16 by $<$dx$>$ $<$dy$>$ $<$dz$>$ } \\
  \hline
  \end{tabular}
  \end{center}  
  For example,
  \begin{verbatim}
                    n16 by  0 0 2
  \end{verbatim}
        moves the n16 2 cm up, and
   \begin{verbatim}
                    n16 by 0 0 -0.5
   \end{verbatim}
   moves the n16 0.5 cm down.

\checkitem Set the calibration in the manip program:
  \begin{center}
  \begin{tabular}{|l|l|}
  \hline
  console & {\tt manip$>$ n16 locate 0 0 1558.5 } \\
  \hline
  \end{tabular}
  \end{center}  
  %--------------------------
  \small
  {\em
    The position 1558.5 is the location of the calibration mark on
    the view port window.  It was determined by measuring the height
    of the source tube and the location of the AV below deck.  
  }
  \normalsize
  %--------------------------

\checkitem Reseal view port window.

\checkitem When approriate (after the URM has been radon flushed) perform
  a light leak check (see procedure \ref{CalProcLightLeakCheck}.

\end{enumerate}


%=================================================
{\small
~\\
~\\
\noindent
{\bf Revision History:}\\
\begin{tabular}{llll}
Rev. & Date & Author & Comments\\
0           & 
?  & 
Fraser Duncan &
\parbox[t]{3.0in}{
  First draft
}\\

1             & 
2002/11/10    & 
Fraser Duncan &
\parbox[t]{3.0in}{
  Made procedure more general.  Corrected fiducial mark for
  URM2.
}
\end{tabular}
}






%==========================================================================
%==========================================================================
%==========================================================================
%==========================================================================



\newpage
\subsection{Calibrating East/West Side Ropes}
\shwlabel{secsideropes}
\vspace*{0.25in}
\noindent
\newprocedure{CalProcEastWestSideRopes}
             {Calibrating East/West Side Ropes}
             {Fraser Duncan/Peter S.}{Sept 2004}{2}

Before each use the side ropes require both a tension calibration
and a length calibration.

\subsubsection{Calibrating Side Rope Tension Offsets}
\shwlabel{secsidtension}
The load cells that measure rope tension are prone to have their
offsets drift.  I.e. although the slope of the loadcell calibration
does not change, the apparent zero tension point drifts.  This
is potentially very bad since when operating the manipulator with
side ropes on, it is necessary to go down to low tension (low is
on the order of 5N or less).\\
{\bf
  If the ropes are operated at zero tension they will unspool from
  the takeup reels in the motor units resulting in tangling and
  requiring major intervention.\\
}
Therefore this procedure to reset the zeros on the loadcells is
important.  Unfortunately it involves taking all the tension off
the rope units and thus risks the same problem it is trying to 
prevent.\\
{\bf
  Extreme care must be taken when performing this procedure. 
}

\vspace*{0.25in}
\noindent
{\bf Calibrating the ( East ) Side Rope Tension Offset}

 In the following we outline  the steps needed to calibrate the East Rope. The 
calibration of the other ropes is identical except for the obvious change of object name.
Note that the offset on the southrope is weird and that one may have to ``lie'' to
it to get the proper offset. Contact the OCE if you're trying to calibrate the South Rope
and do not understand what to do.


\begin{enumerate}


\checkitem Obtain permission from OCE to (re- ) calibrate the sideropes.

\checkitem Go to expert mode at the manipulator console

  \begin{center}
  \begin{tabular}{|l|l|}
  \hline
  console & {\tt manip$>$ expert room601} \\
  \hline
  \end{tabular}
  \end{center}
  {\small\em Note: Expert mode has a 30 minute time out.  If 
   you take longer than this it will be necessary to reenter expert mode.}

\checkitem  Drive out 30 to 40 cm of rope under constant tension.
\begin{itemize}
\item
     Have one person apply tension to the rope in question while another sets the
tope in constant tension mode. For eaxmple `:
  \begin{center}
  \begin{tabular}{|l|l|}
  \hline
  console & {\tt manip$>$ eastrope tension 15} \\
  \hline
  \end{tabular}
  \end{center}
\small
{\em  In tension mode the motor will attempt to keep the rope under constant tension. However,
motors have a maximum speed of 3 cm per second so whatever you do {\bf do it slowly  !}
 Note that a STOP command ( or really low tension ) causes MANIP to exit tension mode. }
\normalsize
\item The person at the glovebox can now pull out the desired amount of rope. {\bf Do it
slowly !}
\item Wait for the motor to stop. ( Listen )
\item The person at the console hits the ESC key or types STOP.
\item The person holding the rope may now slack it off. {\bf Do it gently !}
\end{itemize}
\small
{\em If the rope is not completely slack repeat the above steps }
\normalsize


%While one person maintains tension on the rope by hand 
%  an operator at the {\bf manip} console plays out
%  approximately 30-40cm of rope.
%  \begin{center}
%  \begin{tabular}{|l|l|}
%  \hline
%  console & {\tt manip$>$ eastrope down 30} \\
%%  \hline
%  \end{tabular}
%  \end{center}

%\checkitem {\bf Gently} release the tension from the rope. \\
%   {\em Do not push the rope back up into the motor mount.\\
%   If the the rope does not go slack, re-apply tension
%   by hand and  play out more rope.\\

\checkitem Check what the tension is by doing a   
  \begin{center}
  \begin{tabular}{|l|l|}
  \hline
  console & {\tt manip$>$ eastrope monitor} \\
  \hline
  \end{tabular}
  \end{center}
 If the tension is within 0.2 N of zero there is no need to do the next two steps


\checkitem Calibrate the loadcell offset. {\bf Make sure the rope really has zero tension !} \\
\small
{\em The {\tt calibrate} command is a ``toggle'' command}
\normalsize

  \begin{center}
  \begin{tabular}{|l|l|}
  \hline
  console & {\tt manip$>$ eastrope calibrate} \\
          & {\tt manip$>$ eastrope point 0 N} \\
          & {\tt manip$>$ eastrope calibrate} \\
  \hline
  \end{tabular}
  \end{center}
\small
  {\em It is important that only {\bf one} calibration point is used while the rope is in
{\tt calibrate} mode. Two or more points will change the slope of the calibration as well.
Thus, if you happen to mistype the {\tt point 0N} do {\bf not } under any circumstances
just retype the {\tt point}  command !  Instead, complete the calibration ( ie. exit {\tt calibration}  mode and re-do all three steps.}
\normalsize 

\checkitem Check that the rope tension now reads 0 by using the

  monitor command,
  \begin{center}
  \begin{tabular}{|l|l|}
  \hline
  console & {\tt manip$>$ eastrope monitor} \\
  \hline
  \end{tabular}
  \end{center}
  and reading the rope tension.

\checkitem Wind the rope back in under constant tension.
\begin{itemize}
\item The person at the glovebox applies tension to the rope.
\item Set the rope in constant tension mode
  \begin{center}
  \begin{tabular}{|l|l|}
  \hline
  console & {\tt manip$>$ eastrope tension 15} \\
  \hline
  \end{tabular}
  \end{center}
\item The person holding the rope may now {\bf gently} let the motor take in the excess
rope. {\bf Remember : slow movements only !} 
\item Once the excess rope has been taken up the person at the console types STOP ( or hits the ESC key )
\small
{\em Listen to the motor while you do this. Wait until it settles down before stopping}
\normalsize

\end{itemize}
%  \begin{center}
%  \begin{tabular}{|l|l|}
%  \hline
%  console & {\tt manip$>$ eastrope up 30} \\
%  \hline
%  \end{tabular}
%  \end{center}
%  The self tension of the rope should be about 10N.

  
\end{enumerate}


\vspace*{0.25in}
\noindent
{\bf Calibrating the  Side Rope Tension Offset}

The procedure for the west rope is identical to that for
the east rope.

  
%%\end{enumerate}




\subsubsection{Calibrating the Side Rope Lengths}
\shwlabel{secsidelength}
  Now that the side rope tension offsets have been
calibrated, it is necessary to calibrate the rope
lengths.  This is done by calculating the rope length
based on the positions of the rope attachment points.
Before the actual calibration of the length is done,
the side ropes are pulled tight to high tension and
then relaxed.  This is to prestretch the ropes.
 
\begin{enumerate}

\checkitem Go to expert mode at the manipulator console

  \begin{center}
  \begin{tabular}{|l|l|}
  \hline
  console & {\tt manip$>$ expert room601} \\
  \hline
  \end{tabular}
  \end{center}
  {\small\em Note: Expert mode has a 30 minute time out.  If 
   you take longer than this it will be necessary to reenter expert mode.}

\checkitem Run the siderope calibration command file from the 
  {\bf manip} console  
  \begin{center}
  \begin{tabular}{|l|l|}
  \hline
  console & {\tt manip$>$ calew} \\
  \hline
  \end{tabular}
  \end{center}
  %--------------------------
  \small
  {\em
    The command calew is actually a command file that executes
    a series of commands.  First the ropes are wound to 90N tension
    and held there for 30 seconds.  Then the ropes are relaxed to
    10N and then the rope lengths are set.
  }
  \normalsize
  %--------------------------

\checkitem At the end of the calibration process, the change in rope lengths
  are reported.  If either of the changes in rope lengths are greater
  than 0.5 cm, repeat the calibration process. Record the change in rope length
in the calibration logbook.


\end{enumerate}

%=================================================
{\small
~\\
~\\
\noindent
{\bf Revision History:}\\
\begin{tabular}{llll}
Rev. & Date & Author & Comments\\
0           &  ?           &  Fraser Duncan & \parbox[t]{3.0in}{   First draft }  \\

1             & 2002/11/10    & Fraser Duncan &  \parbox[t]{3.0in}
                                           {   Added steps to go to expert mode. }\\
2          &   2004/08/10 & Peter Skensved & \parbox[t]{3.0in}
     {  Use constant tension mode }\\

\end{tabular}
}






%=========================================================================
%=========================================================================
%=========================================================================

\newpage
\subsection{Attaching the East/West Side Ropes}
\shwlabel{secattachropes}
\newprocedure{CalProcAttachingEastWestRopes}
             {Attaching the East/West Side Ropes}
             {Fraser Duncan}{2002/11/10}{1}
 
  Connecting or disconnecting the side ropes to the source is the most
delicate part of the calibration procedure.  A mistake
in the procedure could easily destroy the laserball and
drop fragments of it into the detector.  \\
{\bf
This procedure should only be done under the supervision of
an experienced operator. Before embarking ensure that everybody involved
is aware of what is about to happen. 
}

\noindent
{\em Note:  This procedure assumes you are connecting the sideropes
  to the laserball.  If a different source such as the {\tt N16} source
  is being used replace {\tt laserball} with the appropriate source
  name in  the  following procedure.
}
  
\vspace*{0.25in}
\noindent
{\bf Prior to this Procedure}\\
The Source has been deployed into the glovebox from the source
tube with the pivot located at approximately $z_{pivot} = 1380$.

  The N16 source is further away from the center of the glovebox
than the laserball. Some people find it easier to attach the side ropes
to this source if is is at a slightly lower pivot position like
1370. Also, for the N16 source the primary operator sits at the west
gloveports.

  If at some point in the procedure you find that any of
the operators cannot reach the side ropes or are unable to safely
pass the ropes undo all the steps completed in reverse order and contact
the OCE before regrouping and trying again.

\begin{enumerate}


\checkitem Go to expert mode at the manipulator console

  \begin{center}
  \begin{tabular}{|l|l|}
  \hline
  console & {\tt manip$>$ expert room601} \\
  \hline
  \end{tabular}
  \end{center}
  {\small\em Note: Expert mode has a 30 minute time out.  If 
   you take longer than this it will be necessary to reenter expert mode.}

\checkitem Open the glove ports on the glovebox.


\checkitem Operator at {\bf manip} console puts the
  side ropes in constant tension mode,
  \begin{center}
  \begin{tabular}{|l|l|}
  \hline
  console & {\tt manip$>$ moveew} \\
  \hline
  \end{tabular}
  \end{center}
  %-----------------------
  \small
  {\em
    The command {\tt moveew} is a command file that puts the east and
    west side ropes into a constant tension mode.  This mode
    causes the manipulator to try and maintain a constant ( 12 N )
    on each rope.  If an operator pulls on the rope and increases the
    tension, the manipulator plays out more rope to decrease the tension
    back to 12 N.  This allows the operator to pull the ropes about
    and have the manipulator ``follow''. Note that the ropes cannot move
    faster than 3 cm per second. So - wheneever you move the ropes { \bf do
    it slowly ! } 
  }
  \normalsize
  %-----------------------
\checkitem Primary operator at south  glove ports reaches in and grasps source
  at the lower part of the carriage.  The rope slot   on the source carriage should face south unless
  the OCE has given different instructions. Make sure your hand is low enough that the pulleys will pivot.

\small
{\em 
    During this procedure the source will be pulled away from its normal
    vertical position under the gatevalve. This means that the source will swing
if the operator lets go of it causing damage to both the detector and the source.
 {\bf Be extremely careful !}
}
\normalsize

\checkitem Operator at the west glove port reaches in and checks that
  the side rope is in constant tension mode by pulling on it.  The
  west rope motor unit on the rope should activate to play out more 
  rope.


\checkitem  The south port operator holds the source with his or her right hand while 
the  west port operator moves the west side rope  to a position within
 easy reach of the south port operator.

  \small
  {\em
    A good way to move the side ropes is to think of them as
    bow strings as in a bow and arrow.  The way to move the rope
    is to hook it with a finger and slowly pull it sideways.  What
    the operator should try avoiding is pulling down on the rope
    such that it goes slack down in the vessel.
  }
  \normalsize
  %---------------------
  
\checkitem The west port operator hands the west rope to the south port operator
{\bf but does not let go of the rope until the south port operator confirms
that he or she has hold of it.}

\small
{\em The handover must be done in a controlled manner with tension on the siderope at
all times. Make sure the other person
is aware of what is about to happen. Ask and recieve confirmation before proceeding
with each step of the handover. }
\normalsize

\checkitem The south port operator attaches the west rope to the source.

\small
{\em The easiest way to do this is to hold the rope above the carriage
with a tiny amount of slack in the rope below. Gently work the slack line into
the slot and slowly let the motor unit take up the slack before
letting go of the rope. Make sure the rope runs correctly over the pulley.
}
\normalsize


\checkitem The south port operator switches the source to his or her left
hand. Make sure the source doen't swing and make sure the west rope
stays put.

\checkitem Either a third operator goes to the east ports or the west
  port operator moves over to the east ports.


\checkitem The east port operator checks that the east rope is in
  constant tension mode by gently pulling on the rope and checking
  that the rope motor unit plays out more rope.

\checkitem The east port operator
  moves the east rope to a position within easy reach of the south port
 operator.

\checkitem The east port operator hands the east rope to the south port
operator {\bf but does not let go of the rope until the south port operator confirms
that he or she has hold of it.}

\small
{\em The handover must be done in a controlled manner with tension on the siderope at
all times. Make sure the other person
is aware of what is about to happen. Ask and recieve confirmation before proceeding
with each step of the handover. }
\normalsize


\checkitem The south port operator attaches the east rope to the source.

\small
{\em Hold the source with your left hand on  the lower part of the
carriage. Make sure there is room for the pulleys to pivot. Hold the tensioned east
rope with your right hand above the pulley with a tiny amount of slack below and work
the slack part of the rope into the slot. Then let the east motor unit take up
the slack ( gently ! ). 
}
\normalsize

  
\checkitem The south port operator checks that the side ropes are sitting
  securely on their pullies.
  
\checkitem The south port operator finally moves the source back towards
the center of the glovebox. {\bf Do it slowly and don't let the source swing !}

\small
{\em Hold the source with the palm of your hand {\bf behind} the source as you
move it towards the center of the glovebox. This way you will not pull the source
too far to the other side.
}
\normalsize

\checkitem Close all glove ports on glovebox.
  
\checkitem Console operator logically connects the side ropes to the laserball object
  in the manipulator code.
  \begin{center}
  \begin{tabular}{|l|l|}
  \hline
  console & {\tt manip$>$ laserball connect eastrope westrope} \\
  \hline
  \end{tabular}
  \end{center}
\small
{\em Once the ( logical ) connection is made the side rope in question will
show up on the display.
  Note that the ropes can be connected one at a time if so desired.

}
\normalsize
\checkitem Set the source orientation ( laserball only ). The orientaion is
a number between 0 and 4 . To list the possible orientation codes do a :

\begin{center}
\begin{tabular}{|l|l|}
\hline
console & {  \tt manip$>$ laserball orientation }\\

\hline
\end{tabular}
\end{center}

 To set the orientation to EAST do a 

\begin{center}
\begin{tabular}{|l|l|}
\hline
console & {  \tt manip$>$ laserball orientation 2 }\\

\hline
\end{tabular}

\end{center}


\end{enumerate}




%=================================================
{\small
~\\
~\\
\noindent
{\bf Revision History:}\\
\begin{tabular}{llll}
Rev. & Date & Author & Comments\\
0           &  ?  & Fraser Duncan &
\parbox[t]{3.0in}{
  First draft
}\\

1             & 2002/11/10    & Fraser Duncan &
\parbox[t]{3.0in}{
  Added steps to go to expert mode.
}\\

2                & Oct. 2004 & P. Skensved &
\parbox[t]{3.0in}{
  Add more detailed instructions. 
} \\

\end{tabular}
}

 



%========================================================================== 
%========================================================================== 
%========================================================================== 
%========================================================================== 

\newpage
\subsection{Detaching Side Ropes}
\shwlabel{secdetachside}
\newprocedure{CalProcDettachingEastWestRopes}
             {Dettaching the East/West Side Ropes}
             {Fraser Duncan}{2002/11/10}{1}

 
  Connecting or disconnecting the side ropes to the source is the most
delicate part of the calibration procedure.  A mistake
in the procedure could easily destroy the laserball and
drop fragments of it into the detector.  
{\bf
This procedure should only be done under the supervision of
an experienced operator. Before embarking ensure that everybody involved
is aware of what is about to happen. 
}
 
\noindent
{\em Note:  This procedure assumes that  the sideropes are connected 
  to the laserball.  If a different source such as the {\tt N16} source
  is being used replace {\tt laserball} with the appropriate source
  name in  the  following procedure.
}
  
\vspace*{0.25in}
\noindent
{\bf Prior to this Procedure}\\
The Source has been deployed into the glovebox from the source
tube with the pivot located at approximately $z_{pivot} = 1380$.

  The N16 source is further away from the center of the glovebox
than the laserball. Some people find it easier to detach the side ropes
from this source if is is at a slightly lower pivot position like
1370. Also, for the N16 source the primary operator sits at the west
gloveports.

  If at some point in the procedure you find that any of
the operators cannot reach the side ropes or are unable to safely
pass the ropes undo all the steps completed in reverse order and contact
the OCE before regrouping and trying again.


  
\begin{enumerate}

\checkitem Go to expert mode at the manipulator console

  \begin{center}
  \begin{tabular}{|l|l|}
  \hline
  console & {\tt manip$>$ expert room601} \\
  \hline
  \end{tabular}
  \end{center}
  {\small\em Note: Expert mode has a 30 minute time out.  If 
   you take longer than this it will be necessary to reenter expert mode.}

\checkitem Open glove ports on the glovebox.

\checkitem Verify source pivot is located at approximately $z_{pivot}=1380$.

\checkitem Locate primary operator at south ports, hands in gloves.

\checkitem Operator at {\bf manip} console logically detaches the side ropes
  from the laserball object.
  \begin{center}
  \begin{tabular}{|l|l|}
  \hline
  console & {\tt manip$>$ laserball disconnects eastrope westrope} \\
  \hline
  \end{tabular}
  \end{center}
  %-------------------
  \small
  {\em 
     The east and west ropes should disappear from the display.
    The laserball at this point becomes a single axis source
    (but the side ropes are still physically attached).
  }
  \normalsize
  %--------------------


  
\checkitem Operator at the console puts the east and west side ropes
  in constant tension mode to allow the physical detachment.
  \begin{center}
  \begin{tabular}{|l|l|}
  \hline
  console & {\tt manip$>$ moveew} \\
  \hline
  \end{tabular}
  \end{center}




\checkitem The primary operator at the south gloveports reaches in and
grasps the source at the lower part of the  carriage. Make sure your
hand is low enough that the pulleys will pivot.

\small
{em 
    During this procedure the source will be pulled away from its normal
    vertical position under the gatevalve. This means that the source will swing
if the operator lets go of it causing damage to both the detector and the source.
 {\bf Be extremely careful !}
}
\normalsize

\checkitem Another operator at the east gloveports hold his or her hand 
near the source  ready to receive  the east rope .



\checkitem The southrope operator holds the source with his or her left hand and
detaches the east rope with his or her right hand.

\small
{\em The easiest way to do this is to hold the rope above the carriage
with a tiny amount of slack in the rope below. Gently work the slack line out of
the slot.
}
\normalsize



\checkitem The south prot operator now passes the east rope  to the east port operator
{\bf but does not let go of the rope until the east
 port operator confirms that he or she has hold of it.}

\small
{\em The handover must be done in a controlled manner with tension on the siderope at
all times. Make sure the other person
is aware of what is about to happen. Ask and recieve confirmation before proceeding
with each step of the handover. }
\normalsize


\checkitem The east port operator gently moves the east rope back to its vertical
resting position while keeping tension on the rope at all times.

  \small
  {\em
    A good way to move the side ropes is to think of them as
    bow strings as in a bow and arrow.  The way to move the rope
    is to hook it with a finger and slowly pull it sideways.  What
    the operator should try avoiding is pulling down on the rope
    such that it goes slack down in the vessel.
  }
  \normalsize



\checkitem The south port operator switches the source to his or her right hand.

\checkitem The east port operator or a third operator gets ready to receive
the west rope from the west side ports.



\checkitem South port operator detaches west rope 

\small
{\em Hold the source with your right hand on  the lower part of the
carriage. Make sure there is room for the pulleys to pivot. Hold the tensioned west
rope with your left hand above the pulley with a tiny amount of slack below and work
the slack part of the rope out of the slot.
}
\normalsize

\checkitem Hand the west rope  to the
  west port operator who allows the rope to slowly relax to
  its resting position.

\small
{\em The handover must be done in a controlled manner with tension on the siderope at
all times. Make sure the other person
is aware of what is about to happen. Ask and recieve confirmation before proceeding
with each step of the handover. }
\normalsize


  
\checkitem South port operator moves the source back to its resting position
under the gatevalve.
{\bf Do it slowly and don't let the source swing !}

\small
{\em Hold the source with the palm of your hand {\bf behind} the source as you
move it towards the gatevalvex. This way you will not pull the source
too far to the other side.
}
\normalsize


\checkitem  Console operator takes the side ropes out of constant tension
  mode by pressing the STOP button (the ESC key) or by typing the
  command:
  \begin{center}
  \begin{tabular}{|l|l|}
  \hline
  console & {\tt manip$>$ stop} \\
  \hline
  \end{tabular}
  \end{center}

\checkitem Close all glove ports on glovebox.


\end{enumerate}  


%=================================================
{\small
~\\
~\\
\noindent
{\bf Revision History:}\\
\begin{tabular}{llll}
Rev. & Date & Author & Comments\\
0           &  ?  &Fraser Duncan &
\parbox[t]{3.0in}{
  First draft
}\\

1             & 2002/11/10    & Fraser Duncan &
\parbox[t]{3.0in}{
  Added steps to go to expert mode.
}\\

2   & Oct. 2004 & P. Skensved &
\parbox[t]{3.0in}{
 Added more detail to the procedure
}
\end{tabular}
}

 
  




  \include{calproc_move}
  \include{calproc_laser}
  

  
%------------------------------------------------------------------------
%------------------------------------------------------------------------
%------------------------------------------------------------------------
\section{N16 Source Procedures}
\shwlabel{secprocN16}

\newprocedure{CalProcN16SourceProc}
             {N16 Source Procedures}
             {F. Duncan/P. Skensved}{Oct. 2004}{1}



  These procedures describe the operation of the N16 calibration
source.


\newpage

\subsection{DT Generator Emergency Shutdown Procedure}

\shwlabel{procn16start}

\newprocedure{CalProcDTEmShut}
             {DT Generator Emergency Shutdown Procedure}
             {F. DUuncan/P. Skensved}{Jun. 2002}{2}



  This procedure describes how to turn off the DT generator and
gas board in case of an emergency such as fire, severe gas leak,
radiation problem etc.



\begin{itemize}
\item {\bf Turn off HV. }
\item Turn off neutron pulser.
\item Turn key to  ``Off''
\item Close main valve on CO$_2$ bottle.


\end{itemize}




%------------------------------------------------------------
\newpage
\subsection{DT Generator Turn On Procedure}
\shwlabel{procdton}


\newprocedure{CalProcDTTurnOn}
             {DT Generator Turn On Procedure}
             {F. Duncan/P. Skensved}{Sept. 2004}{2}


 
  This procedure describes how to turn on the DT generator
prior to operating either the N16 or LI8 sources.  Only authorized
DT operators are allowed to operate the generator.
  
\subsubsection{State Prior To This Procedure}
\begin{itemize}
\item DT generator is OFF.
\end{itemize}

\subsubsection{Summary of Procedure}
\begin{itemize}
\item {\bf Make sure you understand the DT Generator shutdown and emergency shutdown  procedures.}
\item Secure DT generator area.
\item Turn on spare NCD counter.
\item Turn on  DT generator.
\item Check operation of DT generator at full NOC setting.
\item Record maximum and minimum neutron flux.

\end{itemize}

\newpage
\begin{tabular}{|l|l|}
\hline
\multicolumn{2}{|l|}{\large\bf DT Generator Turn On Procedure}\\
\hline
 & \\
Operator(s):~~~~~~~~~~~~~~~~~~~~~~~~~~~~~~~~~~~~ & Date: ~~~~~~~~~~~~~~~~~~~~\\
 & \\
\hline
\end{tabular} 



\begin{enumerate}

\checkitem Read and  make sure you { \bf understand } the DT Generator
Turn Off and Emergency Turn Off procedures (also referred to as the DT Generator Shutdown
procedures).


\checkitem If it is not already on, turn on the gas computer.

{\em The use of the Gas Computer is optional }

\checkitem If it is not already running, start the Gas/DT monitoring
  program on the gas computer by:
  \begin{enumerate}
  \item start labview from the programs menu.
  \item from the labview file menu item select {\tt open}.
  \item go to {\tt c:/labview/develop/}
  \item select {\bf febmain.vi}.
  \item After the VI has loaded, click on the run (arrow) button.
  \end{enumerate}
\checkitem If it is not already running and selected, on the labview
  monitoring program, click on the {\bf fraser monitor} check box.

\item\checkbox
 Secure the DT area : ensure that the shielding blocks ( the ``donuts'' )
 are in place and that the grey interlock boxes are in place and connected.

\checkitem
  Turn on the main instument panel, the flow controller and the stepper
motor controller.

\item\checkbox
 Set the Interlock Override on the main instrument panel
 to ``On'' for manual control (normal setting) or ``Computer'' for
 Labview control.


\item\checkbox
 Check that the High Voltage (HV) and neutron pulser
 switches on the DT control panel are in the OFF position,

\item\checkbox
   Set the Neutron Pulse Control (NPC) to 5 and the Neutron
   Output Control (NOC) to 0. 



\item\checkbox Make sure the HV supply for the spare NCD counter is turned
off and the dial setting is at 0.

\item\checkbox Turn on the NIM bin with the Fast Neutron Flux Meter (F.N.F.M)

\item\checkbox Turn on the low voltage powersupply for the spare NCD counter.


\item\checkbox Turn on the F.N.F.M high voltage supply and set the
voltage to 800 V.

\item\checkbox Connect an oscilloscope to the spare NCD counter amplifier. Set
timebase to approximately 1 us/div and vertical scale to a small fraction of a
volt per division.


\item\checkbox Turn on the HV for the spare NCD counter. While watching the scope
for pulses slowly ( about 1 minute ) increase the voltage to 1850 V.  

\item\checkbox
   Connect an oscilloscope to the SYNC output. Set timebase to 
approximately 2 ms full sweep.

\item\checkbox
  Insert the DT generator key and turn on the power.

\item\checkbox
   At this time, the main power light on the DT control panel, and
   the {\bf D.T. GEN ON} light located behind the DT Pit should be on.
   {\bf If either of these lights does not come on, proceed to}
   {\bf step five in the shutdown procedure. }


\item\checkbox
  Switch the neutron pulser on.  Make sure that the neutron pulser light
comes on.  {\bf If the neutron pulser light does not come on, \
proceed to step number four of the shutdown procedure.} 


\item\checkbox
 Make sure that the neutron pulser light comes on. 
 {\bf If the neutron pulser light does not come on, }
 {\bf proceed to step number four of the shutdown procedure
 immediately.} 


\item\checkbox
 Verify that the DT generator is pulsing at  1.8 ms and that the pulse width
is 180 ${\mu}$s.  

\checkitem Leave the pulser running for approximately one minutes.

\item\checkbox
  When the HV is turned on the ``normal''  (green) and the ``high current'' (red)
 will likely flicker for a few  seconds after which the green light will be off for
a while before it comes on steady. The red light should be off .
 { \bf There is now a built in timedelay in the DT generator source circuit - it may
take up to 1 minute for the green light to be steady. As long as the red light does
not  come on don't panic. If the green light continues to flicker past 20 seconds or the 
  red light comes on steady proceed to step number two in the
 shutdown procedure immediately } 


\item\checkbox
 Record the high voltage turn on time below,
 as well as in the DT Generator Log Book. 
     \begin{center}
     \begin{tabular}{|l|}
     \hline
      \\
     HV Turn on Time:~~~~~~~~~~~~~~~~~~~~~~~~\\
      \\
     \hline
     \end{tabular}
     \end{center}
 

\item\checkbox
 Record Initial value of the F.N.F. Meter, 
     \begin{center}
     \begin{tabular}{|l|}
     \hline
      \\
     FNFM Meter Reading:~~~~~~~~~~~~~~~~~~~~~~~~\\
      \\
     \hline
       \\
     FNFM Computer Reading:~~~~~~~~~~~~~~~~~~~~~~~~\\
      \\
     \hline
     \end{tabular}
     \end{center}
\item\checkbox
 Increase the NOC to 10 (slowly over a period of 20 sec) 
 

\item\checkbox The NCD counter should be counting at a few Hz


\item\checkbox
 Record the final (max) value of the F.N.F. Meter, 
     \begin{center}
     \begin{tabular}{|l|}
     \hline
      \\
     FNFM Meter Reading:~~~~~~~~~~~~~~~~~~~~~~~~\\
      \\
     \hline
       \\
     FNFM Computer Reading:~~~~~~~~~~~~~~~~~~~~~~~~\\
      \\
     \hline
     \hline
     \end{tabular}
     \end{center}


\item\checkbox
 Reset the NCD counter scaler.

\checkitem Record start time and initial flux on DT Generator Log Sheet.

\checkitem Turn the NOC setting back down to zero until the source
 is deployed.


\end{enumerate}


%------------------------------------------------------------
\newpage
\subsection{DT Generator Turn Off Procedure}
\shwlabel{procn16start}


\newprocedure{CalProcDTTurnOn}
             {DT Generator Turn Off Procedure}
         {F. Duncan/P. Skensved}
         {Jun. 2002}{3}


  This procedure describes how to turn off the DT generator.
A trained and authorized DT generator operator is required to
perform this procedure except {\bf in an emergency}.


\vspace*{0.25in}
~\\
\begin{tabular}{|l|l|}
\hline
\multicolumn{2}{|l|}{\large\bf DT Generator Turn Off Procedure}\\
\hline
 & \\
Operator:~~~~~~~~~~~~~~~~~~~~~~~~~~~~~~~~~~~~~ & Date: ~~~~~~~~~~~~~~~~~~~~\\
 & \\
\hline
\end{tabular} \\
\begin{enumerate}

\checkitem Decrease the NOC from 10 to 0 (Slowly over a period of 20 sec.)
   Leave the Neutron Pulse Control at 5.
  
\checkitem Turn the HV Off.  Record this time in the DT Logbook.
     \begin{center}
     \begin{tabular}{|l|}
     \hline
      \\
     HV Turn off Time:~~~~~~~~~~~~~~~~~~~~~~~~\\
      \\
     \hline
     \end{tabular}
     \end{center}
 
\checkitem Wait 1 minute.
  
\checkitem Switch the neutron pulser off.
  
\checkitem Turn the DT Generator off and reurn 
  keys to their proper location in the N$^{16}$ cabinet.

\checkitem Record NCD counter total :
     \begin{center}
     \begin{tabular}{|l|}
     \hline
      \\
     Dose:~~~~~~~~~~~~~~~~~~~~~~~~\\
      \\
     \hline
     \end{tabular}
     \end{center}
  
\checkitem Turn down the HV for the NCD counter ( slowly ! )

\checkitem Turn off the HV for the NCD counter

\checkitem Turn off the low voltage powersupply for the NCD counter

\item\checkbox Turn off HV for FNFM (just turn it off, don't ramp down).

\checkitem Turn off the NIM bin

\checkitem Fill Out DT Generator Log Sheet.


\end{enumerate}

\newpage

%------------------------------------------------------------
\newpage
\begin{center}
{\bf DT Generator Log Sheet}
\end{center}
\begin{tabular}{|l|}
\hline
      \\
Previous Total Minutes:~~~~~~~~~~~~~~~~~~~~~~~~\\
      \\
Previous Total Hours:~~~~~~~~~~~~~~~~~~~~~~~~\\
      \\
\hline
\end{tabular}

\begin{center}
\begin{tabular}{|c|c|c|c|c|c|c|c|c|c|}
\hline
\multicolumn{3}{|c|}{start} &      &\multicolumn{4}{|c|}{stop} & duration & total\\
\hline
yy/mm/dd & hh:mm & FNFM & operator  & yy/mm/dd & hh:mm & FNFM & NCD counter   &time &total\\
         &       & (hz) &           &         &       & (hz) & (counts) & (min)    &~~~minutes~~~~\\
\hline
  &  &  &  & & & & & &\\
\hline
  &  &  &  & & & & & &\\
\hline
  &  &  &  & & & & & &\\
\hline
  &  &  &  & & & & & &\\
\hline
  &  &  &  & & & & & &\\
\hline
  &  &  &  & & & & & &\\
\hline
  &  &  &  & & & & & &\\
\hline
  &  &  &  & & & & & &\\
\hline
  &  &  &  & & & & & &\\
\hline
  &  &  &  & & & & & &\\
\hline
  &  &  &  & & & & & &\\
\hline
  &  &  &  & & & & & &\\
\hline
  &  &  &  & & & & & &\\
\hline
  &  &  &  & & & & & &\\
\hline
  &  &  &  & & & & & &\\
\hline
  &  &  &  & & & & & &\\
\hline
  &  &  &  & & & & & &\\
\hline
  &  &  &  & & & & & &\\
\hline
  &  &  &  & & & & & &\\
\hline
  &  &  &  & & & & & &\\
\hline
  &  &  &  & & & & & &\\
\hline
  &  &  &  & & & & & &\\
\hline
  &  &  &  & & & & & &\\
\hline
  &  &  &  & & & & & &\\
\hline
  &  &  &  & & & & & &\\
\hline
  &  &  &  & & & & & &\\
\hline
  &  &  &  & & & & & &\\
\hline
  &  &  &  & & & & & &\\
\hline
  &  &  &  & & & & & &\\
\hline
  &  &  &  & & & & & &\\
\hline
  &  &  &  & & & & & &\\
\hline
  &  &  &  & & & & & &\\
\hline
  &  &  &  & & & & & &\\
\hline
  &  &  &  & & & & & &\\
\hline
  &  &  &  & & & & & &\\
\hline
  &  &  &  & & & & & &\\
\hline
  &  &  &  & & & & & &\\
\hline
  &  &  &  & & & & & &\\
\hline
\end{tabular}
\end{center}

\noindent
\begin{tabular}{|l|}
\hline
      \\
Total Minutes:~~~~~~~~~~~~~~~~~~~~~~~~\\
      \\
Total Hours:~~~~~~~~~~~~~~~~~~~~~~~~~~~~~~~~\\
      \\
\hline
\end{tabular}



%------------------------------------------------------------
\newpage
\subsection{N16 Source Startup Procedure}
\shwlabel{procn16start}~\\

\newprocedure{CalProcN16SourceStart}
          {N16 Source Startup Procedure}
          {F. Duncan/P. Skensved}{Sept. 2004}{2}

 
  This procedure describes how to turn on the N16 source.
An authorized DT and N16 operator is required to perform this 
procedure.
  
\subsubsection{State Prior To This Procedure}
\begin{itemize}
\item DT generator is OFF.
\item Gas flow to N16 gas board and chamber is OFF.
\item N16 chamber is assembled and connected to gas board.
%%\item N16 PMT is wired in to SNO trigger system.
\end{itemize}

\subsubsection{Summary of Procedure}
\begin{itemize}
\item Turn on DT generator.
\item Set up the N16 gas board.
\item Turn on the CO$_2$ gas flow to the gas board.
\item Adjust to desired flow rate.
\item Check pressure in N16 source chamber.
\item Turn on N16 PMT.
\item Adjust NOC and target position to desired $^{16}$N rate.
\end{itemize}


\begin{figure}
\begin{center}
\leavevmode
%\epsfysize=0.85\textheight
\epsfxsize=7in
\epsfbox{./figures/n16_gas_panel.ps}
~\\
\caption[N16 Gas Panel]
        {N16 Gas Panel
         \shwlabel{fign16gaspanel}
        } 
        
\end{center}
\end{figure}


\newpage
\subsubsection{Procedure}
~\\
\begin{tabular}{|l|l|}
\hline
\multicolumn{2}{|l|}{\bf N16 Source Startup Procedure}\\
\hline
 & \\
Operator:~~~~~~~~~~~~~~~~~~~~~~~~~~~~~~~~~~~~~ & Date: ~~~~~~~~~~~~~~~~~~~~\\
 & \\
\hline
\end{tabular} \\

\begin{itemize}
\checkitem
  Complete DT Generator startup procedure.

\checkitem
  Complete Gas Board startup procedure.

\checkitem
  Complete N16 PMT Turn On procedure.

\checkitem
  Complete $^{16}$N Detector Setup procedure.

\end{itemize}







\newpage
\subsubsection{Procedure}
~\\
\begin{tabular}{|l|l|}
\hline
\multicolumn{2}{|l|}{\bf Gas Board Startup Procedure}\\
\hline
 & \\
Operator:~~~~~~~~~~~~~~~~~~~~~~~~~~~~~~~~~~~~~ & Date: ~~~~~~~~~~~~~~~~~~~~\\
 & \\
\hline
\end{tabular} \\


\begin{enumerate}

\checkitem Verify all valves on the gas board are closed.
\checkitem Verify the N$_2$ / CO$_2$ flow control (side B) is off (fully CCW).
\checkitem Verify the Solenoid Valve II switch is in the OFF position.

\checkitem Enter the DCR using standard entry procedure and 
verify that 
\begin {itemize}
 \checkitem the gas input line is connected to the ``dry'' end of the umbilical
 \checkitem the ``blue'' valve is in the open position. 
 \checkitem the gas return line is connected to the ``dry'' end of the umbilical.
  \checkitem the pressure transducer is powered up and that a voltmeter is connected to it.
  \checkitem the voltmeter is reading the correct ( 18 psi )  ``pressure''.
\end{itemize}
\small
{ \em The pressure transducer is hooked up in a temporary way for the time being. Consult
OCE for details.}

\normalsize

\item \checkbox Turn on the power on the main instrument panel
 and the dual flow controller box. 
  
\item\checkbox Flip the switch on flow controller box to ``B'' (for
 N$_2$ and/or CO$_2$).  The reading takes
  a few minutes to equilibriate to zero or near zero.

\item \checkbox Ensure that the CO$_2$ bottle is hooked up to the input
input line at valve VA2 . 

\item\checkbox Check that VA2 is closewd.

\checkitem Close the needle valve on the CO$_2$ bottle. 

\checkitem Open the main CO$_2$ bottle valve.

\checkitem Set the regulator to 80 PSIG. Note that once the gas starts
 flowing the regulator and bottle starts to cool off and you will have 
to readjust the setting.

\checkitem  Record the bottle pressure:
  \begin{center}
  \begin{tabular}{|c|c|}
  \hline
  Transducer & Reading\\
  \hline
    CO$_2$ Bottle Pressure & \\
  (PSIG) & \\
  \hline 
  \end{tabular}
  \end{center}
  {\em The pressure of a full bottle is 850 to 900 psiG. The CO$_2$ in
  the bottle is in liquid or solid form and the pressure will stay
  relatively high until gas only and will drop rapidly thereafter.
  The CO$_2$ regulator has a small white plastic insert ( disc with small
hole ) in order to seal to the bottle. Do't damage or lose it.  }

\checkitem Open the inline needle valve after the CO$_2$ regulator.

\checkitem Open SVII by setting the switch on the main control panel
  to {\bf MAN}.

\item\checkbox Open the manual shut-off valve VB2.

\item\checkbox Direct flow to the CO$_2$/N$_2$ flow meter using valve VC1.
  

\item\checkbox Set the CO$_2$/N$_2$ flow meter to manual and turn the
control knob one turn so that some gas will flow.

\item \checkbox Direct flow towards the CO$_2$/N$_2$ flow meter using valve VD1.

\item\checkbox Direct flow to the N16 target chamber using valve VF1. Make
 sure you are   not sending it to the oven !

\item\checkbox Open valve VF2 to accept the returning CO$_2$.

\item\checkbox Open valve VC2.

\item\checkbox Direct the flow towards the exhaust using valve VB4.

\item\checkbox Open the exhaust needle valve VA4 fully.

\item\checkbox {\bf Slowly } open valve VA2 and look to see if gas is flowing.

\item\checkbox {\bf Slowly} increase the flow until you reach around a 200-280
  reading. Make sure P1 never goes above 100 PSIA. Remember that
  the time constants are long.


\checkitem 
  Record Gas Presures and Flow.
  \begin{center}
  \begin{tabular}{|c|c|}
  \hline
  Transducer & Reading\\
  \hline
    P1 & \\
  (PSIA) & \\
  \hline 
    P2 & \\
     (PSIA)  & \\
  \hline 
    P3 & \\
     (PSIA)  & \\
  \hline 
    P4 & \\
     (PSIA)  & \\
  \hline 
    Flow & \\
    (cc/s x(1/1000)) & \\
  \hline 
    Target   & \\
    Position & \\
  \hline 
  \end{tabular}
  \end{center}


\checkitem Go back to the DCR and verify that the pressure in the N16 source chamber is within
the acceptable upper limit.
\small
{\em This is done by checking the voltage reading from the pressure transducer. The transducer gets
its power from the monitor PMT supply. Contact the OCE for details }


\normalsize


\end{enumerate}




%------------------------------------------------------------
\newpage
\subsection{N16 Source Shutdown Procedure}
\shwlabel{procn16start}

\newprocedure{CalProcN16SourceShut}
        {N16 Source Shutdown Procedure}
             {F. Duncan/P. Skensved}{Jul. 2003}{2}


  This procedure describes how to shut down the
N16 source.  An authorize DT operator and an authorized N16 operator
is required to perform this procedure.
  
\subsubsection{State Prior To This Procedure}
\begin{itemize}
\item DT generator is ON.
\item Gas is flowing through the N16 board to the Chamber.
\item N16 PMT is on.
\end{itemize}

\subsubsection{Summary of Procedure}
\begin{itemize}
\item Complete DT Generator Shutdown procedure.

\item Complete Gas Board Shutdown procedure.

\item Complete N16 PMT turn off procedure.

\end{itemize}


\newpage
\subsubsection{Procedure}
~\\
\begin{tabular}{|l|l|}
\hline
\multicolumn{2}{|l|}{\bf N16 Source Shutdown Procedure}\\
\hline
 & \\
Operator:~~~~~~~~~~~~~~~~~~~~~~~~~~~~~~~~~~~~~ & Date: ~~~~~~~~~~~~~~~~~~~~\\
 & \\
\hline
\end{tabular} \\

\begin{enumerate}

\item\checkbox Close the main CO$_2$ gas bottle
  valve.
  
\item\checkbox Wait for pressure on P1, P2, P3, P4 to reach atmosphere
  (approximately 18 PSIA).
  
\item\checkbox Turn off regulator on CO$_2$ bottle.
  
\item\checkbox Close inline valve near CO$_2$ regulator.
  
\item\checkbox Close VA2.

\checkitem Close SVII by flipping the switch on the main controller
panel to {\bf Off }.

\checkitem Close VB2.

\checkitem Close VC1.

\checkitem Turen the flow control knob to zero.

\checkitem VD1.

\checkitem VF1.

\checkitem VF2.

\checkitem VC2.

\checkitem VB4.

\checkitem VA4.


  
\item\checkbox Turn OFF power to Flowmeter, Stepper Motor panel and
main instrument panel.
  


\end{enumerate}




%------------------------------------------------------------
\newpage
\subsection{Adjusting Neutron Generator Target Position}
\shwlabel{proctargetladder}


\newprocedure{CalProcTargPos}
     {Adjusting Neutron Generator Target Position}
             {F. Duncan}
             {Aug. 2000}{1}

 
  This procedure describes how to change the position of
the neutron generator target ladder position.  This procedure
is used to move from the N16 target to the Li8 target or to
adjust the neutron capture efficiency on a target (for example if
it is desired to decrease the N16 rate).  The nominal target
positions for the N16 and Li8 are listed in table \ref{tabladder}.

\begin{table}[htbn]
\begin{center}
\begin{tabular}{|l|c|}
\hline
target & Position \\
\hline
N16    & 36.4 \\
\hline
Li8    & 101 \\
\hline
\end{tabular}
\caption[Neutron Generator Target Positions]
        {Neutron Generator Target Positions
        \shwlabel{tabladder}
        }
\end{center}
\end{table}

\subsubsection{Procedure}
 
\begin{center}
\begin{tabular} {|l|l|l|l|}
\hline
\multicolumn{4}{|c|}{\bf Adjusting Neutron Generator Target Position}\\
\hline
     &         &           &                   \\
Date & Initial & Procedure ~~~~~~~~~~~~~~~~~~~~~~~~~~~~~~~~~~~~~~~~~~~~&
 Data and Comments ~~~~~~~~~~~~~~~~~\\
     &         &           &                   \\
\hline
&& & \\
&& Turn on Power & \\
&& & \\
\hline
&& Run switch in down position & \\
&& & \\
\hline
&& & \\
&& Stop switch in down position & \\
\hline
&& & \\
&& in1 in up position & \\
&& & \\
\hline
&& & \\
&& in2 in up position & \\
&& & \\
\hline
&& & \\
&& in3 in up position & \\
&& & \\
\hline
&& & \\
&& speed set to HIGH & \\
&& & \\
\hline
&& Use jog forward/reverse to change position & \\
&& to desired position& \\
&& & \\
\hline
&& & \\
&& Turn off controller box & \\
&& & \\
\hline
\end{tabular}
\end{center}


\newpage


\subsection{N16 PMT Turn On Procedure}

\newprocedure{CalProcPMTTurnOn}
             {N16 PMT Turn On Procedure}
        {P. Skensved}{Jul. 2003}{1}



 This procedure describes how to turn on the N16 monitoring PMT.

\begin{enumerate}

\checkitem Verify that the that the HV controller is turned off and that the dial is turned fully CCW.

\checkitem Verify that the controller cable is plugged into the HV controller.
\checkitem Verify that the controller cable is connected to the ``dry'' end of the umbilical.

\checkitem Connect a fast oscilloscope to the linear PMT output at the ``dry'' end
of the umbilical. Terminate with ( real ) 50 ohm resistor. 
\checkitem Set scope to something like 
10 mV per division vertical, 10-20 ns per division horizontal, trigger on negative edge. Use
AUTO trigger at first to verify presence of baseline/noise.

\checkitem Turn on the NIM bin.

\checkitem Turn on HV switch.

\checkitem Start dialling ( slowly ! ) up the HV while watching the scope. You should see the glitches and changes in
baseline on the scope as you do so. If not stop immediately and check that the scope is setup correctly.

\checkitem Real PMT signals should be evident at a voltage setting at or above 2.00

\checkitem Slowly increase the setting to 2.37 while watching for signs of breakdown. If you see any 
{\bf stop immediately}  and reduce the voltage. Contact OCE.
\small
{\em There may be a tag attached to the power supply with a different voltage than the one 
listed above. Always use the value on the tag if there is one}
\normalsize 


\checkitem Disconnect the scope.


  The next steps need only be done if this procedure is executed as part of a general N16 procedure.

\checkitem Verify that the FECD input at 17/15/4 is disconnected. Leave the `tee'' and 50 ohm in place.


\checkitem Use the N16 custom cable to connect the N16 PMT to the correct spigot near the door.
Cable tie this cable  to the URM. Don't cable tie to the ``dry'' end part of the N16 cable !


\small
{\em It is important that this cable be as short as possible to get the correct trigger timing. Don't
just use any random cable you find. Use the correct cable.

  Also, the ``dry'' end of the N16 PMT cable is fragile. Do {\bf not} cable tie to it. Cable tie to the 
long cable instead}

\normalsize


\checkitem Do not connect 17/15/4 until permission has been obtained from the detector operator.


\end{enumerate}


\newpage

\subsection{N16 PMT Turn Off Procedure}

\newprocedure{CalProcPMTTurnOff}
             {N16 PMT Turn Off Procedure}
        {P. Skensved}{Jul. 2003}{1}


 This procedure describes how to turn off the N16 monitoring PMT.

\begin{enumerate}

\checkitem Disconnect the cable at 17/15/4. Leave the ``tee'' and the 50 ohm in place.


\checkitem Disconnect the cable going from the ``dry'' end to the patch  panel near the door.
Coil it up and hang it on the cable rack.

\checkitem Turn down ( slowly ! ) the HV control to a dial setting of zero,

\checkitem Turn off the HV supply

\checkitem Turn off the NIM bin
\small
{\em This also turns off the power to the pressure transducer. You may have to delay the last
two steps until  a later time.}

\normalsize

\end{enumerate}


  
%------------------------------------------------------------------------
%------------------------------------------------------------------------
%------------------------------------------------------------------------
\section{Standard Calibration Procedures}
\shwlabel{secprocstandard}
 
  This chapter contains ``start to finish'' procedures for
standard calibrations.
  
%------------------------------------------------------------------------
%------------------------------------------------------------------------
%------------------------------------------------------------------------
\newpage
\subsection{PCA Calibration}

\newprocedure{CalProcPCA}
             {PCA Calibration Procedure}
             {Peter Skensved}{2003/01/06}{2}

 

\subsubsection{Introduction}
  
  This procedure describes step by step the process of performing a
laserball PCA calibration of the SNO detector.  It is intended as
a guideline for operators who have been trained on the manipulator
and laser.  It assumes that the laserball is mounted on URM2 which
is located on the 10'' gatevalve on the glovebox.
  

  Supplementing this procedure are the documents available
from the SNO calibration home page,
\begin{verbatim}
  http://www.sno.phy.queensu.ca/private/calibration/index.html
\end{verbatim}
In particular look at:
\begin{description}
\item[Online Manipulator Documentation] contains online manual
  for all commands and operations with the manip program running
  on the manip computer.  This is the reference source for commands
  done from the manip console.
\item[Manipulator User Manual] contains an overview of the manipulator
  system and descriptions of sources and some procedures.  In particular
  it describes how to start the manmon program for controling and monitoring
  the manipulator.
\item[Manipulator Reference Manual] contains technical information on
  the manipulator.
\end{description}
 
\noindent 
  The outline of the procedure is:
\begin{enumerate}
\item Prepare the laser and URM for use (turn on N$_2$ supply etc).
\item Flush URM2 with N$_2$ gas to remove O$_2$ and Rn.
\item Calibrate URM2 central rope.
\item Lower source into glovebox.
\item Connect side ropes to source. (if not single axis mode)
\item Deploy source into detector.
\item Take PCA data
\item Retract source to glovebox.
\item Remove side ropes. (if not single axis mode)
\item Retract source into source tube above gatevalve.
\item Shutdown laser and gas flow to laser and URM.
\end{enumerate}
  



%----------------------------------------------------------------------
\newpage
\subsubsection{Procedure}
~\\
\begin{tabular}{|l|l|}
\hline
 & \\
Operator(s):~~~~~~~~~~~~~~~~~~~~~~~~~~~~~~~~~~~~~~~~~~~~~ 
 & Date: ~~~~~~~~~~~~~~~~~~~~~~~~~~~~~~~~\\
 & \\
\hline
\end{tabular} 
~\\
~\\
  The procedures in this section are intended to be followed
sequentually for the PCA calibration except where it is noted
that a following procedure can be skipped.  Specifically,
if the PCA is to be done in {\em single axis} mode, the side
ropes do not need to be attached or detached from the source.
Procedures supplementary to the main PCA calibration are found
in section \ref{secproclaserball}.  

\begin{center}
                     {\bf Prior to PCA}
\end{center}
\begin{enumerate}
\item\checkbox Permission for procedure and confirmation of equipment readiness
  has been received from Head of Calibration Group.

\item\checkbox Laserball is mounted in URM2 which is
  mounted on 10'' valve on glovebox.

\item\checkbox 10'' gatevalve is closed and locked.

  

\begin{center}
                  {\bf Readying Laser and URM for Operation}
\end{center}

\item\checkbox Verify that the LN$_2$ dewar in the junction is
  at least 1/4 full.  If not, swap it out with another dewar.
  Record liquid level of Dewar,
     \begin{center}
     \begin{tabular}{|l|}
     \hline
      \\
     LN$_2$ Level:~~~~~~~~~~~~~~~~~~~~~~~~\\
      \\
     \hline
     \end{tabular}
     \end{center}

\item\checkbox Verify that the dewar gas pressure is approximately
  130 to 150 psig. If not, swap it out with another dewar.

\item\checkbox Turn on N$_2$ Flow to laser from dewar at junction 
  (Marked {\bf Gas Use} on dewar).
     \begin{center}
     \begin{tabular}{|l|}
     \hline
      \\
     Note Time:~~~~~~~~~~~~~~~~~~~~~~~~\\
      \\
     \hline
     \end{tabular}
     \end{center}

\item\checkbox Turn on pressure builder valve (Marked {\bf Pressure Builder} 
  on dewar).\\
  %------------------------
  \small
  {\em The pressure builder valve opens a controlled leak on the dewar
       to maintain the 150 psi pressure head.  If the valve is not
       opened, the gas pressure to the laser will eventually
       drop below the operating level.}
  \normalsize
  %------------------------


\item Once the URM is flushed the N$_2$ supply may be switched from the high
pressure dewar to the Wessington dewar. Check with the Operations Group first
before switching. Do not use the Wessington if a transfer is in progress.
Consult the gasboard section for details on how to switch.


\item\checkbox Contact Detector Operator and get permission to enter DCR. Make sure
that the DCR activity bit is set.
  

\item\checkbox Turn on lights in DCR following standard procedure. ( See Detector Operator 
Manual )

\item\checkbox Remove the flush return line on the URM.
  %------------------------
  \small
  {\em The presence of the buffer line makes it difficult to measure the O$_2$ from
the URM.
  }
  \normalsize
  %------------------------


  
\item\checkbox Check that flush inlet line is connected to URM2.  If not
  connect it. Open the valve on the source tube.\\
  %------------------------
  \small
  {\em It may be necessary to valve off other URMs to get sufficient flow.
  }
  \normalsize
  %------------------------

\item\checkbox Set up Gas Board in `bypass mode' for `URM flush' only. If you are
using the high pressure feed {\bf do not exceed } 10 psi on the regulator.
  %------------------------
  \small
  {\em Bypass mode maximizes the flow to the URM.
  }
  \normalsize
  %------------------------

\item\checkbox Check that flow meter ( located at South-East corner of
pipe box ) is railed. If not, open needle valve near the flowmeter fully.

   {\bf Flush should continue until O$_2$ reading at the rear of the URM is less than 0.8\%.}
   %--------------------------
   \small
   {\em
     This may take up to an hour depending on when the URM was last
     flushed.
   }
   \normalsize



\item\checkbox Check that the source clamps are in the OUT position.  
{\bf Both } knobs have to be in the extreme {\bf OUT} position. 
{\bf 
     WARNING:  If the source is moved with the clamps in the {\bf IN} position,
       the source, umbilical,
     and manipulator may be severely damaged ! 
   }
  %--------------------------------
  \small
  {\em
   The clamps are used to secure the source while the URM is being moved
   on and off the glovebox.
  }
  \normalsize
  %--------------------------------


\item\checkbox Check the pressure on the air cylinder for the umbilical
takeup mechanism. It should be between 45 and 55 psig. 
   {\bf Do not operate the URM if the pressure is below 40 psig. }
 If the pressure falls below 10 psig at any point ( even momentarily ) call the OCE. 
An internal inspection of the URM is mandatory before operating the unit again.
   %-------------------------------
   \small
   {\em
     The pressure cylinder on the URM maintains tension on the umbilical
     takeup reel.  A low gas pressure can result in the umbilical falling
     off the takeup reel and getting caught or jammed leading to destruction
     of the umbilical.
   }
   \normalsize
   %--------------------------------
  
   
\item\checkbox Verify that the 10'' gatevalve  is locked in the  closed position.\\
   %-------------------------------
   \small
   {\em
     The valve is CLOSED when the handle points towards the pipebox and the slot
      on the handle stem points AWAY from the source tube.
   }
   \normalsize
   %--------------------------------
 
\item \checkbox Calibrate Central Rope Length\\
      (see procedure  \ref{seccalcentre} 
       {\em Central Rope Position Calibration}).
      Record changes in length of central rope and umbilical,
      The current fiducial mark for URM2  on the 10'' gatevalve
      is
      \[
               z_{mark} = 1559.9
      \]
       Note : the fiducial mark is written on the source tube. If it
       differs from the above number use it instead. 

     \begin{center}
     \begin{tabular}{|l|}
     \hline
      \\
     $\Delta$l rope:~~~~~~~~~~~~~~~~~~~~~~~~\\
      \\
     \hline
      \\
     $\Delta$l umbilical:~~~~~~~~~~~~~~~~~~~~~~~~\\
      \\
     \hline
     \end{tabular}
     \end{center}


\item\checkbox Check that all seals are in place on URM.  Including:
   \begin{itemize}
      \item\checkbox flush inlet line
      \item\checkbox window on front of URM  motorbox
      \item\checkbox window on rear of URM motorbox
      \item\checkbox umbilical feedthrough on rear of motorbox
      \item\checkbox view port window cover on source tube
			\item\checkbox window on rear of stretcher box
   \end{itemize}

\item\checkbox Check that you are familiar with the section on operating the laser 
( section \ref{ChapterLaser} ) especially the {\bf Emergency Shutdown Procedure}. Also, be
aware that UV absorbing safety glasses {\bf MUST} be worn while
the laser is running unless {\bf ALL } covers on the laser are in place.


\item\checkbox Plug in the powercord to the laser.

\item\checkbox Check that the POWER  switch on the laser is to {\bf remote}

\item\checkbox Check that the CONTROL switch on the laser is set to {\bf remote}

\item\checkbox Check that the manual shutoff valve on the right of MV5 is open

\item\checkbox Check that the manual shutoff valve MV9 is open.

\item\checkbox  Reset the `Kill Switch' by pushing the red reset button.

\item\checkbox Type {\tt n2laser poweron } on the manip computer.
   \small
   {\em
 This turns on the low voltage power and  energizes the N$_2$ gas valve.

   }
   \normalsize
   %--------------------------------


  
\item\checkbox Verify that N$_2$ is flowing through flow gauge FG5 to the laser head.
If there is no flow consult an expert. {\bf Running the laser without sufficient N$_2$ flow
causes serious damage to the laser head !}

\item\checkbox Record observed laser gas pressure  and
  flow values.

\small
{\em Note that the expected values listed below may be superseded by ones
listed on one or more tags attached to the valves or flow meters. Always use the values found on the tags.}
\normalsize

  \begin{center}
  \begin{tabular}{|l|c|c|}
  \hline
  Transducer & Expect & Observed \\
  \hline
         & & \\
     PG2 & 110--150 psig &\\
         & & \\
  \hline
         & & \\
     PG3 & 90--110 psig & \\
         & & \\
  \hline
         & & \\
     PT4 & 100--110 psig & \\
         & & \\
  \hline
         & & \\
     FG5 & $\approx$ 50 (bottom of ball) & \\
         & & \\
  \hline
         & & \\
     PT6 & $\approx$ 90 psig & \\
         & & \\
  \hline
  \end{tabular}
  \end{center}
  PG2, PG3 and FG5 are read off the gas panel on the end of the
  laser cabinet.  PT4 and PT6 are read from the manipulator computer
  either from the {\tt manmon} laser window or using the commands,
  \begin{tabbing}
   aaaaaaaaaaaaaaaaaaa\=aaaaaaaaaaaaaaaaaaaaaaaaaaaaaaaaaaa\=aaaa\kill
         \>{\tt n2laser hipressure}  \> (for PT4) \\
         \>{\tt n2laser lowpressure}  \> (for PT6) \\
   \end{tabbing}

  
   Gas must flow through the laser for $\approx$ 10 min before turning
  the laser high voltage on.
  
\item\checkbox Block the light by setting the neutral density to 20 
\begin{verbatim}
 n2laser setd 20
on the manip computer.
\end{verbatim}

\item\checkbox Select  dyecell 4 ( 500 nm ) by typing
\begin{verbatim}
  dyelaser cell 4
\end{verbatim}

\item\checkbox Check the state of the laser by issuing a 
\begin{verbatim}
n2laser monitor 
\end{verbatim}
command on the manip console. It will tell you what the general state of the laser is.
\begin{itemize}
\item\checkbox Check that all 4 stirmotors are on.
\item\checkbox Check that there is 120V to the laser
\item\checkbox Check that the filterwheels do not report any problems.
\end{itemize}

\item \checkbox Wait until the O$_2$ level in the URM is at or below 0.8\%



%--------------------------------------------------------------
\begin{center}
            {\bf Deploying Source from Source Tube Into Glovebox}
\end{center}

 \item\checkbox Verify that the URM is below 0.8\% O$_2$.


\item\checkbox Check that flush return line is connected to
  URM2.  If not, connect it.\\
  %------------------------
  \small
  {\em It may be necessary to move it from another URM.
  }
  \normalsize
  %------------------------
  

 \item\checkbox Turn off DCR lights.

 \checkitem Record the Cover Gas O$_2$ level
     \begin{center}
     \begin{tabular}{|l|}
     \hline
      \\
     Cover Gas O$_2$ Reading:~~~~~~~~~~~~~~~~~~~~~~~~\\
      \\
     \hline
     \end{tabular}
     \end{center}

 \item\checkbox Verify OWL light monitor is on.  Establish communications
  with person watching light monitor.
  %-------------------------
  \small
  {\em
    Suggestion:  Station the person watching the OWL monitor at
    the Deck Mac.  Then he/she can shout through the  wall of the
    DCR and you don't need to use the phones which slow communications
    down.
  }
  \normalsize
  %-------------------------
 

\item\checkbox Open gate valve ( {\bf Slowly !} ).\\
  Record the time the valve is opened.
     \begin{center}
     \begin{tabular}{|l|}
     \hline
      \\
     Time Gate Valve Opened:~~~~~~~~~~~~~~~~~~~~~~~~\\
      \\
     \hline
     \end{tabular}
     \end{center}

 \item\checkbox Lock gate valve open.

 \item\checkbox With flashlight perform light leak check on URM.  In particular
   check the seal of the source tube window and around the base of the source tube.
   Also, check around any inspection panel which may have been removed in the recent past.

 \item\checkbox Using the dimmer switch, { \bf slowly } bring up breaker 9 lights in
   the DCR.  Person still watching owl monitor.


 \item\checkbox DAQ is connected to the {\bf manip} computer.

 \item \checkbox In DAQ, source type is set to {\bf LASERBALL}.

 \item\checkbox DAQ is in a {\bf source transitional run}.

 \item \checkbox Verify that {\bf manip\_logger} on {\bf crag1}
                 is running and logging the {\bf Laserball}  source.

 \item\checkbox Check movement of laserball down:
  \begin{center}
  \begin{tabular}{|l|l|}
  \hline
  console & {\tt manip$>$ laserball by 0 0 -5} \\
  \hline
  manmon  & in laserball window: \\
          & set x = 0, y = 0, z = -5\\
          & click on {\bf move by} \\
  \hline
  \end{tabular}
  \end{center}
  %--------------------
  \small
  {\em 
    The laserball should move down 5 cm.  The tension on the rope
    should be 40-60 N.  The tension on the umbilical should be
    10-30N.
  }
  \normalsize
  %--------------------

 \item\checkbox Check that the source offset and orientation is set correctly.
   At the console type 
 {\tt laserball sourceoffset } \\
  The current laserball has an offset of -64.4 cm. If the reported number is
different contact the OCE. For single axis deployment mode the orientation should
be 0

  {\tt laserball orientation 0 } \\
 If deployed with sideropes the orientation depends on what direction the slot
faces. If in doubt contact the OCE. 

 \item\checkbox Deploy source into the glovebox:
  \begin{center}
  \begin{tabular}{|l|l|}
  \hline
  console & {\tt manip$>$ laserball to 0 0 1380} \\
  \hline
  manmon  & in laserball window: \\
          & set x = 0, y = 0, z = 1380\\
          & click on {\bf move to} \\
  \hline
  \end{tabular}
  \end{center}



%-------------------------------------------------------------
\begin{center}
  {\bf Deploying Manipulator into Centre of 
            Detector from Glovebox}
\end{center}
\shwlabel{sectocentre}
 
 \item\checkbox Contact Water Supervisor and advise him/her that the source is
   being lowered into the D2O.  \\
   %--------------------
   \small
   {\em
     The water group maintains a very small differential pressure
     between the light and heavy water.  The volume of the source
     is enough to disrupt this differential pressure.
   }
   \normalsize
   %---------------------

 \item\checkbox Check tensions on urm2rope and urm2umbilical.  Rope tension
   should be approximately 30-50 N.  Umbilical tension should
   be between 15-40 N. Note that the tensions are reduced once the
source is submerged.
  
 \item\checkbox Move laserball to centre of detector.
  \begin{center}
  \begin{tabular}{|l|l|}
  \hline
  console & {\tt manip$>$ laserball to 0 0 66.4} \\
  \hline
  manmon  & in laserball window: \\
          & click on {\bf Position the source}\\
          & set x = 0, y = 0, z = 0\\
          & click on {\bf move to} \\
  \hline
  \end{tabular}
  \end{center}

%-------------------------------------------------------------
\begin{center}
                 {\bf Turning On the Laser}
\end{center}

 \item\checkbox Verify on the console that the control power on the laser is on :
\begin{verbatim}
 n2laser monitor
\end{verbatim} 
  All voltages should be on, all stir motors should should be ON,
 both filterwheels should be IDLE, the dye cell motor should be IDLE and
 gas should be flowing. If not contact OCE.

     If using manmon, the 
  power lights for 120VAC and 40 VDC should turn green
       the lights next to each dye cell indicating the stir motor
       status should turn green and the status boxes should indicate
  IDLE.



 \item\checkbox Verify that the N2 gas has been flowing through the laser for
    $\approx$ 10 min. Note that the gas flow is turned on and off with the 
    n2laser poweron/poweroff commands.



%   It will take about 5 minutes while the laser warms up.  A status
%   message should indicate this waiting period.
%   After the laser is warmed up, it will indicate that it is ready.

 \item\checkbox  Select desired wavelength or dye cell.
  \begin{center}
  \begin{tabular}{|l|l|}
  \hline
  console & {\tt manip$>$ dyelaser cell $<$0-9$>$} \\
          & or \\
          & {\tt manip$>$ dyelaser wavelength $<$wavelength$>$}\\
  \hline
  manmon  & in laserwindow \\
          & click on button above desired dye cell\\
  \hline
  \end{tabular}
  \end{center}
 \item\checkbox Set ND = 6.0 or higher.
  \begin{center}
  \begin{tabular}{|l|l|}
  \hline
  console & {\tt manip$>$ n2laser setnd 6.0} \\
  \hline
  manmon  & laserwindow-$>$Windows-$>$Neutral Density Settings \\
          & click on desired neutral density\\
  \hline
  \end{tabular}
  \end{center}
  %--------------------
  \small
  {\em
    The filter wheels are set to a large attenuation when first turning
    on the laser to prevent a large amount of light being introduced 
    into the detector and overwhelming the data aquisition.  Once
    the laser is on, the rate and intensity can be adjusted to the
    desired level.
  }
  \normalsize
  %--------------------


  \item\checkbox Wait for the laser to return status READY

 \item\checkbox Turn on laser light.
  \begin{center}
  \begin{tabular}{|l|l|}
  \hline
  console & {\tt manip$>$ n2laser start} \\
  \hline
  manmon  & in laserwindow \\
          & click on {\bf light on}\\
  \hline
  \end{tabular}
  \end{center}
   Now wait 90 seconds while the trigger is delayed.
   Laser will come on at 10 Hz trigger rate.
  
\checkitem Plug the Laserball Trigger signal into the EXTA input on the
  MTCD.
  
%-------------------------------------------------------------

\begin{center}
                 {\bf Taking PCA Data}
\end{center}
\item {\bf PCA Runs}
  The exact nature of the PCA runs will vary.  The ``caonical''
  run tends to be 
  \begin{enumerate}
  \checkitem Long Low Occupancy Run.
    \begin{itemize}
    \item 500nm
    \item The centroid for the raw TAC is usually at 1800.
    \item 5-8\% occupancy (ND setting at 500nm is approximately 5.5).
    \item 40 Hz laser trigger rate (or best you can do without buffer
          overflow).
    \item Run Type: PCA 
    \item 10 subruns (2-3 hours).
    \end{itemize}
  \checkitem Short Medium Occupancy Run.
    \begin{itemize}
    \item 500nm
    \item The centroid for the raw TAC is usually at 1800.
    \item 20-25\% occupancy (ND setting at 500nm is approximately 5.0).
    \item 5-10 Hz laser trigger rate (or best you can do without buffer
          overflow).
    \item Run Type: PCA 
    \item 3 subruns (20 minutes).
    \end{itemize}
  \end{enumerate}
%-------------------
\small
{\em
  The laser can be run up to 45Hz but the current DAQ looks like
it sometimes has trouble keeping up.  Generally you want to keep
the data rate to no more than 300kB/s.
}
\normalsize
%------------------

%-------------------------------------------------------------

\begin{center}
                 {\bf Turning Off Laser}
\end{center}

\checkitem Unplug the  EXTA at the MTCD.

 \item\checkbox Turn off laser light
  \begin{center}
  \begin{tabular}{|l|l|}
  \hline
  console & {\tt manip$>$ n2laser stop} \\
  \hline
  manmon  & in laserwindow \\
          & click on {\bf light off}\\
  \hline
  \end{tabular}
  \end{center}
 \item\checkbox Turn off laser power
  \begin{center}
  \begin{tabular}{|l|l|}
  \hline
  console & {\tt manip$>$ n2laser poweroff} \\
  \hline
  manmon  & in laserwindow \\
          & click on {\bf power off}\\
  \hline
  \end{tabular}
  \end{center}
 \item\checkbox Unplug laser power cord from wall outlet\\
  %-------------------
  \small
  {\em
    The laser is unplugged when it is not intended to be used for
    extended periods.  This is because it has been observed that
    on several occasions after an unscheduled power outage that the
    laser has come up in a funny state.
  }
  \normalsize
  %-------------------
  


%-------------------------------------------------------------
\begin{center}
                   {\bf Retracting Manipulator to glovebox}
\end{center}

\item\checkbox Contact Water Supervisor.  Inform him/her that the source is
   about to be removed from the D$_2$O. 

\item\checkbox Retract laserball from AV into glovebox.
  \begin{center}
  \begin{tabular}{|l|l|}
  \hline
  console & {\tt manip$>$ laserball to 0 0 1300} \\
  \hline
  manmon  & in laserball window: \\
          & click on {\bf Position the pivot}\\
          & set x = 0, y = 0, z = 1300\\
          & click on {\bf move to} \\
  \hline
  \end{tabular}
  \end{center}
\item\checkbox Retract laserball to position to disconnect side ropes.
  \begin{center}
  \begin{tabular}{|l|l|}
  \hline
  console & {\tt manip$>$ laserball to 0 0 1380} \\
  \hline
  manmon  & in laserball window: \\
          & click on {\bf Position the pivot}\\
          & set x = 0, y = 0, z = 1380\\
          & click on {\bf move to} \\
  \hline
  \end{tabular}
  \end{center}
  %------------------------
  \small
  {\em
    When moving the laserball to 1380, it is important to make sure
    you are moving with respect to the { \bf pivot } and  { \bf not } the
   centre of the source which is
    approximately 64 cm  below the pivot.  This is especially important if the sideropes
are attached ! 

  }
  \normalsize
  %-----------------------

  


  
%-------------------------------------------------------------
\begin{center}
           {\bf Retracting source above gate valve.  Side ropes NOT attached.}
\end{center}
\shwlabel{secabovegv}
\item\checkbox move laserball to 1530
  \begin{center}
  \begin{tabular}{|l|l|}
  \hline
  console & {\tt manip$>$ laserball to 0 0 1530} \\
  \hline
  \end{tabular}
  \end{center}  
\item\checkbox move laserball to 1540
  \begin{center}
  \begin{tabular}{|l|l|}
  \hline
  console & {\tt manip$>$ laserball to 0 0 1540} \\
  \hline
  \end{tabular}
  \end{center}  
\item\checkbox move laserball to 1550
  \begin{center}
  \begin{tabular}{|l|l|}
  \hline
  console & {\tt manip$>$ laserball to 0 0 1550} \\
  \hline
  \end{tabular}
  \end{center}
{\bf
 NOTE:\\
   MINIMUM SAFE HEIGHT TO CLOSE GATEVALVE IS 1535cm.\\
   If unable to get above this height, contact expert.
}
\item\checkbox Retrieve the gatevalve key from the DCR lock box.
\item\checkbox Unlock the gatevalve.
\item\checkbox Carefully close the gate valve by rotating the handle {\em clockwise}.
  {\em Expect resistance when the handle is about 3/4 of the way to
  the closed position.  This is the normal overcentering of the
  valve mechanism.} {\bf If resistance is felt before this or 
  if any sounds are heard that might be caused by valve hitting the source,
  STOP and contact an expert.}
  Record the time the valve is closed.
     \begin{center}
     \begin{tabular}{|l|}
     \hline
      \\
     Time Gate Valve Closed:~~~~~~~~~~~~~~~~~~~~~~~~\\
      \\
     \hline
     \end{tabular}
     \end{center}
 
\item\checkbox Lock the gatevalve in the {\bf CLOSED} position.
\item\checkbox Return the gatevalve key to the DCR lock box.
 
 \checkitem Record the Cover Gas O$_2$ level
     \begin{center}
     \begin{tabular}{|l|}
     \hline
      \\
     Cover Gas O$_2$ Reading:~~~~~~~~~~~~~~~~~~~~~~~~\\
      \\
     \hline
     \end{tabular}
     \end{center}

\item\checkbox Close the URM flush valve if the soure does not need drying out.
\small
{\em It is desirable to leave a minute flow of N$_2$ through the URM in order to dry
out the source and the umbilical. Contact OCE for instructions.}

\normalsize
\item\checkbox Turn off the URM flush regulator ( if the source does not need drying out ).

\item\checkbox IF the laser is off,
   turn off gas flow at the LN$_2$ dewar in the junction:
   \begin{enumerate}
   \item close {\bf Gas Use} valve
   \item close {\bf Pressure Building} valve
   \end{enumerate}


 \item\checkbox If the source is retracted, and gate valve closed,
   turn off gas flow at the high pressure LN$_2$ dewar in the junction:
   \begin{enumerate}
   \item close {\bf Gas Use} valve
   \item close {\bf Pressure Building} valve
   \end{enumerate}






%-------------------------------------------------------------
\begin{center}
           {\bf After Calibration}
\end{center}
\item\checkbox Source is above gate valve.
\item\checkbox Gate valve is closed and locked.
\item\checkbox Laser is off.
\item\checkbox Manual shutoff valve to the right of MV5 is closed
\item\checkbox Laser power cord is unplugged from wall outlet.
\item\checkbox High pressure LN$_2$ dewar is turned off (both {\bf Gas Use} valve and 
  {\bf Pressure Building} valve).
\item\checkbox Flush return line is disconnected from rear of URM2
\item\checkbox Gas board is set up to provide sufficient flow to dry out the inside
of the URM.

\end{enumerate}


   

{\small
~\\
~\\
\noindent
{\bf Revision History:}\\
\begin{tabular}{llll}
Rev. & Date & Author & Comments\\

0             & 
?    & 
Fraser Duncan &
\parbox[t]{3.0in}{
  First Draft
}\\

1             & 
?    & 
Fraser Duncan &
\parbox[t]{3.0in}{
  Many earlier drafts
}\\


2             & 
2003/01/06 & 
Peter Skensved &
\parbox[t]{3.0in}{
  Many updates
}\\


\end{tabular}
}


%------------------------------------------------------------------------
%------------------------------------------------------------------------
%------------------------------------------------------------------------
\newpage
\subsection{N16 Calibration}

\newprocedure{CalProcN16}
             {N16 Calibration Procedure}
             {Fraser Duncan/Peter Skensved}{2003/09/20}{2}

  
  This procedure describes step by step the process of performing an
N16  calibration at the centre of the detector in single axis mode.  It is intended as
a guideline for operators who have been trained on the manipulator
and N16.  It assumes that the N16 source is mounted on URM3,
located on gatevalve 3 on the glovebox (gatevalve 3 is the 4''
valve located at the northwest corner.
  


\noindent 
  The outline of the procedure is:
\begin{enumerate}
\item Prepare the URM for use (turn on N$_2$ supply etc).
\item Flush URM3 with N$_2$ gas to remove room air and radon.
\item Calibrate URM3 central rope.
\item Lower source into glovebox.
\item Deploy source into detector.
\item Take N16 data
\item Retract source to glovebox.
\item Retract source into source tube above gatevalve.
\item Shutdown N16 gas board and gas flow to laser and URM.
\end{enumerate}
  
\vspace*{0.25in}
\noindent 
You will need the following procedures to complete an N16
calibration.
\begin{itemize}
\item {\bf N16 Calibration} (this procedure) 
\item {\bf Central Rope Calibration Procedure}
\item {\bf DT Generator Turn On Procedure}
\item {\bf N16 Source Startup Procedure}
\item {\bf N16 Source Shutdown Procedure}
\item {\bf DT Generator Turn Off Procedure}

\end{itemize}

%----------------------------------------------------------------------
\newpage
\subsubsection{Procedure}
~\\

\noindent
\begin{tabular}{|l|l|}
\hline
 & \\
Operator(s):~~~~~~~~~~~~~~~~~~~~~~~~~~~~~~~~~~~~~~~~~~~~~ 
 & Date: ~~~~~~~~~~~~~~~~~~~~~~~~~~~~~~~~\\
 & \\
\hline
\end{tabular} 
~\\
~\\

  The procedures in this section are intended to be followed
sequentually for the N16 calibration except where it is noted
that a following step can be skipped.  Specifically,
if the N16 is to be done in {\em single axis} mode, the side
ropes do not need to be attached or detached from the source.
%%Procedures supplementary to the main N16 calibration are found
%%in section \ref{secsupp}.  

\begin{center}
                    {\bf Prior to N16}
\end{center}
\begin{enumerate}
\item \checkbox Permission for procedure and confirmation of equipment readiness
  has been received from Head of Calibration Group.

\item \checkbox N16 is mounted in URM3 which is
  mounted on glovebox gatevalve 3.

\item \checkbox Gatevalve 3 is closed and locked ( or handle is removed ).

  

\begin{center}
                  {\bf Readying N16 Source and URM for Operation}
\end{center}
 

\item \checkbox Contact Detector Operator and verify that the DCR activity bit is set.
  

\item \checkbox Turn on lights in DCR following standard procedure.
  

\item \checkbox Verify that the high pressure N$_2$ to the laser is valved
off ( end of laser, upper right hand corner )


\item \checkbox Verify that all valves on the gasboard are closed and the regulator
is set to zero.



\item\checkbox Remove the flush return line on the URM.
  %------------------------
  \small
  {\em The presence of the buffer line makes it difficult to measure the O$_2$ from
the URM.
  }
  \normalsize
  %------------------------



\item\checkbox Check that flush inlet line is connected to URM3.  If not
  connect it. Open the valve at the source tube.\\
  %------------------------
	\small
  {\em It may be necessary to valve off the other URMs to get sufficient flow.
  }
  %------------------------
  \normalsize



\item \checkbox Verify that the LN$_2$ dewar in the junction is
  at least 1/4 full.  If not, swap it out with another dewar.
  Record liquid level of Dewar,
     \begin{center}
     \begin{tabular}{|l|}
     \hline
      \\
     LN$_2$ Level:~~~~~~~~~~~~~~~~~~~~~~~~\\
      \\
     \hline
     \end{tabular}
     \end{center}

\item \checkbox Verify that the dewar gas pressure is above
  20 psig. If not, swap it out with another dewar. Unlike the laser the 16N
does not require a minimum of 100 psig N$_2$. The N$_2$ is only used for flushing.
Note that the dewar won't last long if the pressure is below 100 psig

\item \checkbox Turn on the high pressure N$_2$ flow from the dewar at junction 
  (Marked {\bf Gas Use} on dewar).
     \begin{center}
     \begin{tabular}{|l|}
     \hline
      \\
     Note Time:~~~~~~~~~~~~~~~~~~~~~~~~\\
      \\
     \hline
     \end{tabular}
     \end{center}

\item \checkbox Turn on pressure builder valve (Marked {\bf Pressure Builder} 
  on dewar).\\
  %------------------------
  \small
  {\em The pressure builder valve opens a controlled leak on the dewar
       to maintain the 150 psig pressure head.  If the valve is not
       opened, the gas pressure to the laser and URM will eventually
       drop below the operating level ( 10psig ). }
  \normalsize
  %------------------------



\item\checkbox Set up Gas Board in `bypass mode' for `URM flush' only. If you are
using the high pressure feed {\bf do not exceed } 10 psi on the regulator.
  %------------------------
  \small
  {\em Bypass mode maximizes the flow to the URM. ( Gas enters at top left hand corner,
flows through regulator along outside right hand side line
down to the URM with all other valves closed - see GasBoard Section for details ).

  }
  \normalsize
  %------------------------

\item\checkbox Check that flow meter ( located at east side of
pipe box ) is railed. If not, open needle valve near the flowmeter fully.

   {\bf Flush should continue until O$_2$ reading at the rear of the URM is less than 0.8\%.}
   %--------------------------
   \small
   {\em
     This may take up to an hour depending on when the URM was last
     flushed.
   }
   \normalsize



\item Once the URM is flushed the N$_2$ supply may be switched from the high
pressure dewar to the Wessington dewar. Check with the Operations Group first
before switching. Do not use the Wessington if a transfer is in progress.
Consult the gasboard section for details on how to switch. If the flush is continued
from the high pressure dewar reduce the flow using the needle valve. A setting of
2 full turns above the point where the flowmeter rails is sufficient.



\item\checkbox Verify ( blue ) valve on N16 Umbilical gas feed line (the translucent
               line) is OPEN.  If not, open it.


\item\checkbox Execute the N16 PMT Turn On Procedure



\item\checkbox Do not connect the PMT trigger cable to channel 4, slot 15, crate 17
before obtaining permission from the detector operator.

\item \checkbox Check that the source clamps are in the RELEASE position.  
   There are two clamps.  Check both.\\
  {\bf
    WARNING:  The clamp positions RELEASE and HOLD are 90 deg apart.
      Rotate the clamp in the SHORT direction (90deg) from the HOLD
      to RELEASE position.       The clamp must {\bf not}  be rotated the 
other way !

  }
  %--------------------------------
  \small
  {\em
   The clamps are used to secure the source while the URM is being moved
   on and off the glovebox.  If the source is moved with the clamps in
   the HOLD position, it may severely damage the manipulator and the umbilical.
  }
  \normalsize
  %--------------------------------
 
\item \checkbox Check the pressure on the air cylinder for the umbilical
takeup mechanism. It should be between around 55 psig.
   {\bf Do not operate the URM if the pressure is below 40 psig. }
 If the pressure falls below 10 psig at any point ( even momentarily ) call the OCE. 
An internal inspection of the URM is mandatory before operating the unit again.
   %-------------------------------
   \small
   {\em
     The pressure cylinder on the URM maintains tension on the umbilical
     takeup reel.  A low gas pressure can result in the umbilical falling
     off the takeup reel and getting caught or jammed leading to destruction
     of the umbilical.
   }
   \normalsize
   %--------------------------------


  
\item\checkbox Verify that Gate Valve 3 is closed and locked ( or handle is removed ).
   %-------------------------------
   \small
   {\em
     The valve is CLOSED when the handle points towards the pipebox and the slot
      on the handle stem points AWAY from the source tube.
   }
   \normalsize
   %--------------------------------


\item \checkbox Calibrate Central Rope Length\\
      (see procedure  
       {\em Central Rope Position Calibration}).
      Record changes in length of central rope and umbilical.
       The current fiducial mark for URM3 on gatevalve 3 is 
   \[
         z_{mark} = 1558.5 
       \]
       Note : the fiducial mark is written on the source tube. If it
       differs from the above number use what is written on the source tube.

     \begin{center}
     \begin{tabular}{|l|}
     \hline
      \\
     $\Delta$l rope:~~~~~~~~~~~~~~~~~~~~~~~~\\
      \\
     \hline
      \\
     $\Delta$l umbilical:~~~~~~~~~~~~~~~~~~~~~~~~\\
      \\
     \hline
     \end{tabular}
     \end{center}

 
\item \checkbox Check that all seals are in place on URM.  Including:
   \begin{itemize}
      \item\checkbox flush inlet line
      \item\checkbox window on front of URM  motorbox
      \item\checkbox window on rear of URM motorbox
      \item\checkbox umbilical feedthrough on rear of motorbox
      \item\checkbox view port window cover on source tube
       \item\checkbox window on rear of stretcher box
   \end{itemize}
\small
{ \em
 Note that the flush outlet is small enough and is shadowed well enough that it does not
pose a threat to the detector unless light is shone directly at into it. 


} 

\normalsize

\end{enumerate}



%--------------------------------------------------------------
\begin{center}
            {\bf Deploying Source from Source Tube Into Glovebox}
\end{center}


\begin{enumerate}

 \item\checkbox Verify that the URM is below approximately 0.8\% O$_2$.


\item\checkbox Check that flush return line is connected to
  URM3.  If not, connect it.\\
  %------------------------
  \small
  {\em It may be necessary to move it from another URM.
  }
  \normalsize
  %------------------------
  

 \item\checkbox Turn off DCR lights.


 \checkitem Record the Cover Gas O$_2$ level
     \begin{center}
     \begin{tabular}{|l|}
     \hline
      \\
     Cover Gas O$_2$ Reading:~~~~~~~~~~~~~~~~~~~~~~~~\\
      \\
     \hline
     \end{tabular}
     \end{center}

 \item\checkbox Verify OWL light monitor is on.  Establish communications
  with person watching light monitor.
  %-------------------------
  \small
  {\em
    Suggestion:  Station the person watching the OWL monitor at
    the Deck Mac.  Then he/she can shout through the  wall of the
    DCR and you don't need to use the phones which slow communications
    down. Or contact the detector operator by phone.
  }
  \normalsize
  %-------------------------
 

\item\checkbox Open gate valve ( {\bf Slowly !} ).\\
  Record the time the valve is opened.
     \begin{center}
     \begin{tabular}{|l|}
     \hline
      \\
     Time Gate Valve Opened:~~~~~~~~~~~~~~~~~~~~~~~~\\
      \\
     \hline
     \end{tabular}
     \end{center}

 \item\checkbox Remove handle and key for gatevalve 3 and place {\bf Gatevalve Open}
sign on valve.

 \item\checkbox With a flashlight perform light leak check on URM.  In particular
   check the seal of the source tube window and around the base of the source tube.
   Also, check around any inspection panel which may have been removed in the recent past.

 \item\checkbox Using the dimmer switch, { \bf slowly } bring up breaker 9 lights in
   the DCR.  Person still watching owl monitor. If everything is ok bring up the other
    breakers. {\bf If there is any sign of a lightleak turn off the DCR lights
immediately, close the gatevalve and contact the OCE.}




 \item\checkbox DAQ is connected to the {\bf manip} computer.

 \item \checkbox In DAQ, source type is set to {\bf N16}.

 \item\checkbox DAQ is in a {\bf source transitional run}.

 \item \checkbox Verify that {\bf manip\_logger} on {\bf crag1}
                 is running and logging the {\bf N16}  source.





 \item \checkbox Check movement of N16 down:
  \begin{center}
  \begin{tabular}{|l|l|}
  \hline
  console & {\tt manip$>$ n16 by 0 0 -5} \\
  \hline
  manmon  & in n16 window: \\
          & set x = 0, y = 0, z= -5\\
          & click on {\bf move by} \\
  \hline
  \end{tabular}
  \end{center}
  %--------------------
  \small
  {\em 
    The N16 should move down 5 cm.  The tension on the rope
    should be 90-110 N.  The tension on the umbilical should be
    20-40N.
  }
  \normalsize
  %--------------------

 \item \checkbox Deploy source into the glovebox:
  \begin{center}
  \begin{tabular}{|l|l|}
  \hline
  console & {\tt manip$>$ n16 to 0 0 1370} \\
  \hline
  manmon  & in n16 window: \\
          & set x = 0, y = 0, z= 1370\\
          & click on {\bf move to} \\
  \hline
  \end{tabular}
  \end{center}

\end{enumerate}


%-------------------------------------------------------------
\begin{center}
              {\bf Deploying Manipulator into Centre of 
            Detector from Glovebox}
\end{center}
\shwlabel{sectocentre}
 

\begin{enumerate}



 \item \checkbox Contact Water Supervisor and advise him/her that the source is
   being lowered into the D$_2$O.  \\
   %--------------------
   \small
   {\em
     The water group maintains a very small differential pressure
     between the light and heavy water.  The volume of the source
     is enough to disrupt this differential pressure and could potentially
     result in an SDS trip.
   }
   \normalsize
   %---------------------

 \item \checkbox Check tensions on URM3ROPE and URM3UMBILICAL.  Rope tension
   should be approximately 90-110 N.  Umbilical tension should
   be between 20-40 N.
  
 \item \checkbox Move N16 to centre of detector.
  \begin{center}
  \begin{tabular}{|l|l|}
  \hline
  console & {\tt manip$>$ n16 to 0 0 71} \\
  \hline
  manmon  & in n16 window: \\
          & click on {\bf Position the source}\\
          & set x = 0, y = 0, z= 0\\
          & click on {\bf move to} \\
  \hline
  \end{tabular}
  \end{center}
  As the source goes into the water, the rope tension will decrease to
  approximately 60 N and the umbilical to approximately 20 N.

\end{enumerate}
%-------------------------------------------------------------
\begin{center}
                {\bf Turning On N16 Source}
\end{center}
  

\begin{enumerate}

\item\checkbox Check that you are familiar with the section on operating the DT generator
and the associated gasboard especially the {\bf Emergency Shutdown Procedure}.


\item \checkbox Execute procedure \ref{procdton}, {\em Turning on DT Generator}.

\item \checkbox Execute procedure \ref{procn16start}.
  
%-------------------------------------------------------------

\begin{center}
             {\bf Taking N16 Data}
\end{center}
\item\checkbox Taking N16 Data.
  The exact nature of the N16 runs will vary.  The ``canonical''
  run tends to be 
  \begin{itemize}
  \item NOC setting of 10
  \item Target setting of 36.4
  \item Flow rate 280-300
  \end{itemize}

\end{enumerate}

%-------------------------------------------------------------

\begin{center}
             {\bf Turning Off N16 Source}
\end{center}

\begin{enumerate}

\item \checkbox Execute procedure {\em Shutting Down N16 Gas System}
\item \checkbox Execute procedure {\em Turning off DT Generator}

\end{enumerate}


%-------------------------------------------------------------
\begin{center}
              {\bf Retracting Manipulator to glovebox}
\end{center}

\begin{enumerate}

\item \checkbox Contact Water Supervisor.  Inform him/her that the source is
   about to be removed from the Heavy water. 

\item \checkbox Retract N16 from AV into glovebox.
  \begin{center}
  \begin{tabular}{|l|l|}
  \hline
  console & {\tt manip$>$ n16 to 0 0 1300} \\
  \hline
  manmon  & in N16 window: \\
          & click on {\bf Position the pivot}\\
          & set x = 0, y = 0, z= 1300\\
          & click on {\bf move to} \\
  \hline
  \end{tabular}
  \end{center}

\small
{\em If sideropes are attached move the source to approximately z = 1370 and
disconnect the sideropes before retracting the source any further. See the
siderope procedures.}


  {\em
    When moving the N16 to 1370, it is important to make sure
    you are moving the {\bf pivot}, not the centre of the source which is
    approximately 71 cm below the pivot.
    This is especially important if the sideropes
    are attached ! 


  }
  \normalsize
  %-----------------------

  
\end{enumerate}
  
%-------------------------------------------------------------
\begin{center}
            {\bf Retracting source above gate valve.  Side ropes NOT attached.}
\end{center}
\shwlabel{secabovegv}

\begin{enumerate}

\item \checkbox move N16 to 1530
  \begin{center}
  \begin{tabular}{|l|l|}
  \hline
  console & {\tt manip$>$ n16 to 0 0 1530} \\
  \hline
  \end{tabular}
  \end{center}  
\item \checkbox move n16 to 1540
  \begin{center}
  \begin{tabular}{|l|l|}
  \hline
  console & {\tt manip$>$ n16 to 0 0 1540} \\
  \hline
  \end{tabular}
  \end{center}  
\item \checkbox move N16 to 1550
  \begin{center}
  \begin{tabular}{|l|l|}
  \hline
  console & {\tt manip$>$ n16 to 0 0 1550} \\
  \hline
  \end{tabular}
  \end{center}
{\bf
 NOTE:\\
   MINIMUM SAFE HEIGHT TO CLOSE GATEVALVE IS 1530cm.\\
   If unable to get above this height ( due for example to high tension ), contact OCE
immediately.
}
%%\item \checkbox Retrieve the gatevalve key from the DCR lock box.

%%\item \checkbox Unlock the gatevalve.

\item Remove `Gatevalve open' sign.

\item \checkbox Carefully close the gate valve by rotating the handle {\em clockwise}.
  {\em Expect resistance when the handle is about 3/4 of the way to
  the closed position.  This is the normal overcentering of the
  valve mechanism.} {\bf If resistance is felt before this or 
  if any sounds are heard that might be caused by valve hitting the source,
  STOP and contact the OCE.}
  Record the time the valve is closed.
     \begin{center}
     \begin{tabular}{|l|}
     \hline
      \\
     Time Gate Valve Closed:~~~~~~~~~~~~~~~~~~~~~~~~\\
      \\
     \hline
     \end{tabular}
     \end{center}


\item \checkbox Remove handle and key.

 
 \checkitem Record the Cover Gas O$_2$ level
     \begin{center}
     \begin{tabular}{|l|}
     \hline
      \\
     Cover Gas O$_2$ Reading:~~~~~~~~~~~~~~~~~~~~~~~~\\
      \\
     \hline
     \end{tabular}
     \end{center}



\item\checkbox Execute N16 PMT Turn Off Procedure

\item  It is advisable to leave a slow N$_2$ flush of the URM in order to dry out
the source and the inside of the URM. Contact the OCE to organize who will turn
off the flow and when it should be done. If it is decided not to leave the
flush on follow the shutoff procedure listed below :



\item  Turn off gas flow at the LN$_2$ dewar in the junction:
   \begin{enumerate}
   \item \checkbox close {\bf Gas Use} valve
   \item \checkbox close {\bf Pressure Building} valve
   \end{enumerate}

\item \checkbox Turn off the flush on the gasboard.
\item \checkbox Close the URM flush valve.


\end{enumerate}

%-------------------------------------------------------------
\begin{center}
                 {\bf After Calibration}
\end{center}

\begin{enumerate}

\item \checkbox Source is above gate valve.
\item \checkbox Gate valve is closed and handle plus key removed.
\item \checkbox N16 PMT Power Supply set to  0V, Turned off, NIM BIN turned off.
\item \checkbox N16 gas board is OFF.
\item \checkbox DT Generator is OFF.


%\item \checkbox LN$_2$ dewar is turned off (both {\bf Gas Use} valve and 
%  {\bf Pressure Building} valve).

\end{enumerate}








{\small
~\\
~\\
\noindent
{\bf Revision History:}\\
\begin{tabular}{llll}
Rev. & Date & Author & Comments\\

0             & 
?    & 
Fraser Duncan &
\parbox[t]{3.0in}{
  First Draft
}\\

1             & 
2000/07/30 & 
Fraser Duncan &
\parbox[t]{3.0in}{
  Many earlier drafts
}\\

2     &
2003/08/20 &
Peter Skensved & 
Revisions to reflect new realities  \\
2004/08/10 &
Peter Skensved & 
More revisions to reflect even newer realities ...  \\

\end{tabular}
}



%------------------------------------------------------------------------
%------------------------------------------------------------------------
%------------------------------------------------------------------------
\newpage
\subsection{Acrylic Source Calibration Procedure}
\shwlabel{procpca}~\\


\newprocedure{CalProcAcryl}
             {Acrylic Source Calibration Procedure}
             {Fraser Duncan/Peter Skensved}{Oct. 2004}{3}




\subsubsection{Introduction}
   
  This procedure describes step by step the deployment process
for a acrylic source into the AV.  It is basically identical
to the Laserball or N16 deployment procedure.
  It is intended as
a guideline for operators who have been trained on the manipulator
and laser.  It assumes that the laserball is mounted on URM2 which
is located on the 10'' gatevalve on the glovebox.


  Supplementing this procedure are the documents available
from the SNO calibration home page,
\begin{verbatim}
  http://www.sno.phy.queensu.ca/private/calibration/index.html
\end{verbatim}
In particular look at:
\begin{description}
\item[Online Manipulator Documentation] contains online manual
  for all commands and operations with the manip program running
  on the manip computer.  This is the reference source for commands
  done from the manip console.
\item[Manipulator User Manual] contains an overview of the manipulator
  system and descriptions of sources and some procedures.  In particular
  it describes how to start the manmon program for controling and monitoring
  the manipulator.
\item[Manipulator Reference Manual] contains technical information on
  the manipulator.
\end{description}

\noindent
  The outline of the procedure is:
\begin{enumerate}
\item Prepare the URM for use (turn on N$_2$ supply etc).
\item Flush URM2 with N$_2$ gas to remove O$_2$ and Rn.
\item Calibrate URM2 central rope.
\item Lower source into glovebox.
\item Connect side ropes to source. (if not single axis mode)
\item Deploy source into detector.
\item Take data
\item Retract source to glovebox.
\item Remove side ropes. (if not single axis mode)
 \item Retract source into source tube above gatevalve.
\item Shutdown gasflow to  URM.
\end{enumerate}






%----------------------------------------------------------------------
\newpage
\subsubsection{Procedure}
~\\
\begin{tabular}{|l|l|}
\hline
 & \\
Operator(s):~~~~~~~~~~~~~~~~~~~~~~~~~~~~~~~~~~~~~~~~~~~~~
 & Date: ~~~~~~~~~~~~~~~~~~~~~~~~~~~~~~~~\\
 & \\
\hline
\end{tabular}
~\\
~\\
  The procedures in this section are intended to be followed
sequentually for the source calibration except where it is noted
that a following procedure can be skipped.  Specifically,
if the source run is to be done in {\em single axis} mode, the side
ropes do not need to be attached or detached from the source.

\begin{center}
                     {\bf Prior to Calibration Run}
\end{center}

\begin{enumerate}
\item\checkbox Permission for procedure and confirmation of equipment readiness
  has been received from Head of Calibration Group.

\item\checkbox Source is mounted in URM2 which is
  mounted on 10'' valve on glovebox.

\item\checkbox 10'' gatevalve is closed and locked.



\begin{center}
                  {\bf Readying URM for Operation}
\end{center}

\item\checkbox Verify that the LN$_2$ dewar in the junction is
  at least 1/4 full.  If not, swap it out with another dewar.
  Record liquid level of Dewar,
     \begin{center}
     \begin{tabular}{|l|}
     \hline
      \\
     LN$_2$ Level:~~~~~~~~~~~~~~~~~~~~~~~~\\
      \\
     \hline
     \end{tabular}
     \end{center}

\item\checkbox Verify that the dewar gas pressure is approximately
  130 to 150 psig. If not, swap it out with another dewar.

\item\checkbox Turn on N2 Flow to laser from dewar at junction
  (Marked {\bf Gas Use} on dewar).
     \begin{center}
     \begin{tabular}{|l|}
     \hline
      \\
     Note Time:~~~~~~~~~~~~~~~~~~~~~~~~\\
      \\
     \hline
     \end{tabular}
     \end{center}

\item\checkbox Turn on pressure builder valve (Marked {\bf Pressure Builder}
  on dewar).\\
  %------------------------
  \small
  {\em The pressure builder valve opens a controlled leak on the dewar
       to maintain the 150 psi pressure head.  If the valve is not
       opened, the gas pressure to the laser will eventually
       drop below the operating level.}
  \normalsize
  %------------------------


\item Once the URM is flushed the N$_2$ supply may be switched from the high
pressure dewar to the Wessington dewar. Check with the Operations Group first
before switching. Do not use the Wessington if a transfer is in progress.
Consult the gasboard section for details on how to switch.


\item\checkbox Contact Detector Operator and get permission to enter DCR. Make sure
that the DCR activity bit is set.


\item\checkbox Turn on lights in DCR following standard procedure. ( See Detector Operator
Manual )

\item\checkbox Remove the flush return line on the URM.
  %------------------------
  \small
  {\em The presence of the buffer line makes it difficult to measure the O$_2$ from
the URM.
  }
  \normalsize
  %------------------------


\item\checkbox Check that flush inlet line is connected to URM2.  If not
  connect it. Open the valve on the source tube.\\
  %------------------------
  \small
  {\em It may be necessary to valve off other URMs to get sufficient flow.
  }
  \normalsize
  %------------------------

\item\checkbox Set up Gas Board in `bypass mode' for `URM flush' only. If you are
using the high pressure feed {\bf do not exceed } 10 psi on the regulator.
  %------------------------
  \small
  {\em Bypass mode maximizes the flow to the URM.
  }
  \normalsize
  %------------------------


\item\checkbox Check that flow meter ( located at South-East corner of
pipe box ) is railed. If not, open needle valve near the flowmeter fully.

   {\bf Flush should continue until O$_2$ reading at the rear of the URM is less than 0.8\%.}
   %--------------------------
   \small
   {\em
     This may take up to an hour depending on when the URM was last
     flushed.
   }
   \normalsize



\item\checkbox Check that the source clamps are in the OUT position.
{\bf Both } knobs have to be in the extreme {\bf OUT} position.
{\bf
     WARNING:  If the source is moved with the clamps in the {\bf IN} position,
       the source, umbilical,
     and manipulator may be severely damaged !
   }
  %--------------------------------
  \small
  {\em
   The clamps are used to secure the source while the URM is being moved
   on and off the glovebox.
  }
  \normalsize
  %--------------------------------


\item\checkbox Check the pressure on the air cylinder for the umbilical
takeup mechanism. It should be between 45 and 55 psig.
   {\bf Do not operate the URM if the pressure is below 40 psig. }
 If the pressure falls below 10 psig at any point ( even momentarily ) call the OCE.
An internal inspection of the URM is mandatory before operating the unit again.
   %-------------------------------
   \small
   {\em
     The pressure cylinder on the URM maintains tension on the umbilical
     takeup reel.  A low gas pressure can result in the umbilical falling
     off the takeup reel and getting caught or jammed leading to destruction
     of the umbilical.
   }
   \normalsize
   %--------------------------------


\item\checkbox Verify that the 10'' gatevalve  is locked in the  closed position.\\
   %-------------------------------
   \small
   {\em
     The valve is CLOSED when the handle points towards the pipebox and the slot
      on the handle stem points AWAY from the source tube.
   }
   \normalsize
   %--------------------------------

\item \checkbox Calibrate Central Rope Length\\
      (see procedure  \ref{seccalcentre}
       {\em Central Rope Position Calibration}).
      Record changes in length of central rope and umbilical,
      The current fiducial mark for URM2  on the 10'' gatevalve
      is
      \[
               z_{mark} = 1559.9
      \]
       Note : the fiducial mark is written on the source tube. If it
       differs from the above number use it instead.

     \begin{center}
     \begin{tabular}{|l|}
     \hline
      \\
     $\Delta$l rope:~~~~~~~~~~~~~~~~~~~~~~~~\\
      \\
     \hline
      \\
     $\Delta$l umbilical:~~~~~~~~~~~~~~~~~~~~~~~~\\
      \\
     \hline
     \end{tabular}
     \end{center}


\item\checkbox Check that all seals are in place on URM.  Including:
   \begin{itemize}
      \item\checkbox flush inlet line
      \item\checkbox window on front of URM  motorbox
      \item\checkbox window on rear of URM motorbox
      \item\checkbox umbilical feedthrough on rear of motorbox
      \item\checkbox view port window cover on source tube
      \item\checkbox window on rear of stretcher box
   \end{itemize}


\item \checkbox Wait until the O$_2$ level in the URM is at or below 0.8\%


%--------------------------------------------------------------
\begin{center}
            {\bf Deploying Source from Source Tube Into Glovebox}
\end{center}

 \item\checkbox Verify that the URM is below 0.8\% O$_2$.


\item\checkbox Check that flush return line is connected to
  URM2.  If not, connect it.\\
  %------------------------
  \small
  {\em It may be necessary to move it from another URM.
  }
  \normalsize
  %------------------------


 \item\checkbox Turn off DCR lights.

 \checkitem Record the Cover Gas O$_2$ level
     \begin{center}
     \begin{tabular}{|l|}
     \hline
      \\
      Cover Gas O$_2$ Reading:~~~~~~~~~~~~~~~~~~~~~~~~\\
      \\
     \hline
     \end{tabular}
     \end{center}

 \item\checkbox Verify OWL light monitor is on.  Establish communications
  with person watching light monitor.
  %-------------------------
  \small
  {\em
    Suggestion:  Station the person watching the OWL monitor at
    the Deck Mac.  Then he/she can shout through the  wall of the
    DCR and you don't need to use the phones which slow communications
    down.
  }
  \normalsize
  %-------------------------


\item\checkbox Open gate valve ( {\bf Slowly !} ).\\
  Record the time the valve is opened.
     \begin{center}
     \begin{tabular}{|l|}
     \hline
      \\
     Time Gate Valve Opened:~~~~~~~~~~~~~~~~~~~~~~~~\\
      \\
     \hline
     \end{tabular}
     \end{center}

 \item\checkbox Lock gate valve open.

 \item\checkbox With flashlight perform light leak check on URM.  In particular
   check the seal of the source tube window and around the base of the source tube.
   Also, check around any inspection panel which may have been removed in the recent past.

 \item\checkbox Using the dimmer switch, { \bf slowly } bring up breaker 9 lights in
   the DCR.  Person still watching owl monitor.


 \item\checkbox DAQ is connected to the {\bf manip} computer.

 \item \checkbox In DAQ, source type is set to {\bf ACRYLIC}.

 \item\checkbox DAQ is in a {\bf source transitional run}.

 \item \checkbox Verify that {\bf manip\_logger} on {\bf crag1}
                 is running and logging the {\bf Acrylic}  source.

 \item\checkbox Check movement of acrylic source down:
  \begin{center}
  \begin{tabular}{|l|l|}
  \hline
  console & {\tt manip$>$ acrylic by 0 0 -5} \\
  \hline
  manmon  & in acrylic window: \\
          & set x = 0, y = 0, z = -5\\
          & click on {\bf move by} \\
  \hline
  \end{tabular}
  \end{center}

  %--------------------
  \small
  {\em
    The source should move down 5 cm.  The tension on the rope
    should be 40-60 N.  The tension on the umbilical should be
    10-30N.
  }
  \normalsize
  %--------------------


 \item\checkbox Check that the source offset is set correctly.
   At the console type
 {\tt acrylic sourceoffset } \\
  The actual offset depends on the exact configuration ( canned / un-canned, spacer present etc. )  

 \item\checkbox Deploy source into the glovebox:
  \begin{center}
  \begin{tabular}{|l|l|}
  \hline
  console & {\tt manip$>$ acrylic to 0 0 1380} \\
  \hline
  manmon  & in acrylic window: \\

          & set x = 0, y = 0, z = 1380\\
          & click on {\bf move to} \\
  \hline
  \end{tabular}
  \end{center}



%-------------------------------------------------------------
\begin{center}
  {\bf Deploying Manipulator into Centre of
            Detector from Glovebox}
\end{center}
%%%%\shwlabel{sectocentre}


 \item\checkbox Contact Water Supervisor and advise him/her that the source is
   being lowered into the D$_2$O.  \\
   %--------------------
   \small
   {\em
     The water group maintains a very small differential pressure
     between the light and heavy water.  The volume of the source
     is enough to disrupt this differential pressure.
   }
   \normalsize
   %---------------------

 \item\checkbox Check tensions on urm2rope and urm2umbilical.  Rope tension
   should be approximately 30-50 N.  Umbilical tension should
   be between 15-40 N. Note that the tensions are reduced once the
source is submerged.


 \item\checkbox Move acrylic source to centre of detector ( assuming source offset is 70 cm ).
  \begin{center}
  \begin{tabular}{|l|l|}
  \hline
  console & {\tt manip$>$ acrylic to 0 0 70} \\
  \hline
  manmon  & in acrylic window: \\
          & click on {\bf Position the source}\\
          & set x = 0, y = 0, z = 0\\
          & click on {\bf move to} \\
  \hline
  \end{tabular}
  \end{center}

%-------------------------------------------------------------

\checkitem Take data. The exact configuration will vary. 


%-------------------------------------------------------------
\begin{center}
                   {\bf Retracting Manipulator to glovebox}
\end{center}

\item\checkbox Contact Water Supervisor.  Inform him/her that the source is
   about to be removed from the D$_2$O.

\item\checkbox Retract acrylic source from AV into glovebox.
  \begin{center}
  \begin{tabular}{|l|l|}
  \hline
  console & {\tt manip$>$ acrylic to 0 0 1300} \\
  \hline
  manmon  & in acrylic window: \\
           & click on {\bf Position the pivot}\\
          & set x = 0, y = 0, z = 1300\\
          & click on {\bf move to} \\
  \hline
  \end{tabular}
  \end{center}
\item\checkbox Retract acrylic source to position to disconnect side ropes.
  \begin{center}
  \begin{tabular}{|l|l|}
  \hline
  console & {\tt manip$>$ acrylic to 0 0 1380} \\
  \hline
  manmon  & in acrylic window: \\
          & click on {\bf Position the pivot}\\
          & set x = 0, y = 0, z = 1380\\
          & click on {\bf move to} \\
  \hline
  \end{tabular}
  \end{center}
  %------------------------
  \small
  {\em
    When moving the acrylic source to 1380, it is important to make sure
    you are moving with respect to the { \bf pivot } and  { \bf not } the
   centre of the source which is
    approximately 64 cm  below the pivot.  This is especially important if the sideropes
are attached !
}
  \normalsize
  %-----------------------





%-------------------------------------------------------------
\begin{center}
           {\bf Retracting source above gate valve.  Side ropes NOT attached.}
\end{center}
\shwlabel{secabovegv}
\item\checkbox move acrylic to 1530
  \begin{center}
  \begin{tabular}{|l|l|}
  \hline
  console & {\tt manip$>$ acrylic to 0 0 1530} \\
  \hline
  \end{tabular}
  \end{center}
\item\checkbox move acrylic to 1540
  \begin{center}
  \begin{tabular}{|l|l|}
  \hline
  console & {\tt manip$>$ acrylic to 0 0 1540} \\
  \hline
  \end{tabular}
  \end{center}
\item\checkbox move acrylic to 1550
  \begin{center}
  \begin{tabular}{|l|l|}
  \hline
  console & {\tt manip$>$ acrylic to 0 0 1550} \\
  \hline
  \end{tabular}
  \end{center}
{\bf
 NOTE:\\
   MINIMUM SAFE HEIGHT TO CLOSE GATEVALVE IS 1535cm.\\
   If unable to get above this height, contact expert.
}
\item\checkbox Retrieve the gatevalve key from the DCR lock box.
\item\checkbox Unlock the gatevalve.
\item\checkbox Carefully close the gate valve by rotating the handle {\em clockwise}.
  {\em Expect resistance when the handle is about 3/4 of the way to
  the closed position.  This is the normal overcentering of the
  valve mechanism.} {\bf If resistance is felt before this or
  if any sounds are heard that might be caused by valve hitting the source,
  STOP and contact an expert.}
  Record the time the valve is closed.
     \begin{center}
     \begin{tabular}{|l|}
     \hline
      \\
     Time Gate Valve Closed:~~~~~~~~~~~~~~~~~~~~~~~~\\
      \\
     \hline
     \end{tabular}
     \end{center}

\item\checkbox Lock the gatevalve in the {\bf CLOSED} position.
\item\checkbox Return the gatevalve key to the DCR lock box.

 \checkitem Record the Cover Gas O$_2$ level
     \begin{center}
     \begin{tabular}{|l|}
     \hline
      \\
     Cover Gas O$_2$ Reading:~~~~~~~~~~~~~~~~~~~~~~~~\\
      \\
     \hline
     \end{tabular}
     \end{center}

\item\checkbox Close the URM flush valve if the soure does not need drying out.
\small
{\em It is desirable to leave a minute flow of N$_2$ through the URM in order to dry
out the source and the umbilical. Contact OCE for instructions.}

\normalsize
\item\checkbox Turn off the URM flush regulator ( if the source does not need drying out ).

\item\checkbox IF the laser is off,
   turn off gas flow at the LN$_2$ dewar in the junction:
   \begin{enumerate}
   \item close {\bf Gas Use} valve
   \item close {\bf Pressure Building} valve
   \end{enumerate}


 \item\checkbox If the source is retracted, and gate valve closed,
   turn off gas flow at the high pressure LN$_2$ dewar in the junction:
   \begin{enumerate}
   \item close {\bf Gas Use} valve
   \item close {\bf Pressure Building} valve
   \end{enumerate}






%-------------------------------------------------------------
\begin{center}
           {\bf After Calibration}
\end{center}
\item\checkbox Source is above gate valve.
\item\checkbox Gate valve is closed and locked.
\item\checkbox High pressure LN$_2$ dewar is turned off (both {\bf Gas Use} valve and
  {\bf Pressure Building} valve) if source is not to be dryed out.
\item\checkbox Flush return line is disconnected from rear of URM2
\item\checkbox Gas board is set up to provide sufficient flow to dry out the inside
of the URM.

\end{enumerate}



%------------------------------------------------------------

\newpage
\section{Laserball to/from Acrylic Procedures}

\newprocedure{CalProcSwitch}
   {Laserball to/from Acrylic Procedures}
   {P. Skensved}{Oct. 2004}{3}


\subsection{Procedure for Changing from Laserball to Acrylic Source.}


\subsubsection{Procedure}
~\\
\begin{tabular}{|l|l|}
\hline
 & \\
Operator(s):~~~~~~~~~~~~~~~~~~~~~~~~~~~~~~~~~~~~~~~~~~~~~
 & Date: ~~~~~~~~~~~~~~~~~~~~~~~~~~~~~~~~\\
 & \\
\hline
\end{tabular}
~\\
~\\

This procedure describes going from the Mark III laserball to a
generic 
acrylic source.  The acrylic source stem replaces the laserball and
it's stem where it attaches to the laserball ``can''.



\begin{enumerate}
      
\checkitem Obtain permission to proceed from calibration group. 
\checkitem Remove salt probe / plug clamp. 
\checkitem While supporting both laserball and can remove screws which hold the laserball to the can.
\checkitem Gently slide the laserball away from the can until CAJON fitting is exposed. Be sure to support both
can and laserball and make sure they're aligned. 
\checkitem Carefully unscrew top end of CAJON fitting and pull out the fiber end. Do NOT let it retract into the can.
\checkitem Store laserball in toolbox
\checkitem Ensure o-ring is in place on stem.
\checkitem Ensure dummy CAJON fitting is pin place on stem.
\small
{\em  
  Note : The teflon uses a steel spacer between the can and the stem ( the teflon stem is too soft to
seal properly. Make sure the o-ring is present on the top of the spacer. It has a pin which holds
the CAJON fitting. The o-ring between the stem and the plate is optional.
}
\normalsize
\checkitem Carefully insert fiber end into stem CAJON fitting and tighten lightly.
\checkitem Verify the saltprobe / plug o-ring is still in place.
\checkitem Push the fiber in to the cajon fitting on the stem. Tighten the fitting ligthly.
\checkitem Push the stem up against the can. Make sure the fiber is not bent or pinched
in any way.
\checkitem Slide the steel plate into place and bolt plate plus stem to can. Use correct screws.
\checkitem Slide clamp up around saltprobe or plug. Make sure it is oriented correctly ( so that
 it fits ) and
tighten screw appropriately. ( {\bf Do not overtigthen  } ! )

  The next few steps depend on the desired source configuration ( canned, un-canned, 
un-canned with spacer, sealed can ).


\checkitem Canned source
 \begin{enumerate}
 \checkitem Retrieve the appropriate source from the source cabinet. Make an entry in the log.
 \checkitem Place source in delrin can
 \checkitem Ensure o-ring is in place on lid. Ensure all three captured nuts are in place.
 \checkitem Gently press lid into place. Make sure it is oriented correctly.
 \small
  {\em Use something soft like a cable tie to check that the holes line up.
  Delrin is very soft and it is extremely easy to strip the threads. { \bf  Make absolutely sure that the
 holes are line up properly before inserting
  the screws ! } Use something soft like a cable tie to check the alignment. }
 \normalsize

 \checkitem Put in the three screws.  Do not use any tools - use your fingers. If there is {\bf any }
 friction at all chances are that the holes aren't lined up. Stop immediately, remove the screws and
 check the alingment again.

 \checkitem Tighten the screws infinitesimally with and allen key. {\bf Do not overtighten ! }

 \checkitem Secure srcews with stainless steel wire.

 \checkitem Ensure small o-ring is in place on top of lid. 

 \checkitem Attach can to stem with three screws. Use proper length screws. 

 \small
 { \em It is very easy to cross thread the screws in the captured nuts. Make {\bf absolutely} sure
 they are in correctly and { \bf do not overtighten ! } }

 \checkitem Secure screws with wire.
 \normalsize

 \end{enumerate}




\checkitem Uncanned source
 \begin{enumerate}
 \checkitem Mount source with three screws. Use spacer if so desired. Use proper length screws.
 \small
 {\em Acrylic is brittle and easy to damage. Use extreme care when threading the
 screws into the source. {\bf Do not overtighten !}  And do not cross thread the screws.  }
 \normalsize
 \checkitem Secure with wire.
 \end{enumerate}

\checkitem Sealed can

   Note : A sealed can is { \bf not } to be opened anywhere in the lab. The can is an essential
part onf the containment system for the source. At present we have two sealed sources ( both AmBe sources ).
One is in a teflon can the other is in a stainless can. The latter differs from the standard steel can
in that it has a small top cover. {\bf If in doubt contact the OCE !}.


 \begin{enumerate}
  
  \checkitem 
 \checkitem Attach can to stem with three screws. Use proper length screws. Do {\bf NOT } open can.
    The sealed stainless steel can has a small top cover plate and requires a ``donut''
    to fit the stem properly. If in doubt contact the OCE. 



 \small
 { \em It is very easy to cross thread the screws in the captured nuts. Make {\bf absolutely} sure
 they are in correctly and { \bf do not overtighten ! } }
 \normalsize
 \checkitem Secure screws with wire.
 \end{enumerate}




\end{enumerate}




%------------------------------------------------------------


%------------------------------------------------------------

\newpage


\subsection{Procedure for Changing from Acrylic Source to Laserball.}


\subsubsection{Procedure}
~\\
\begin{tabular}{|l|l|}
\hline
 & \\
Operator(s):~~~~~~~~~~~~~~~~~~~~~~~~~~~~~~~~~~~~~~~~~~~~~
 & Date: ~~~~~~~~~~~~~~~~~~~~~~~~~~~~~~~~\\
 & \\
\hline
\end{tabular}
~\\
~\\

This procedure describes going from the generic Acrylic Source to the Mark III Laserball.
  The laserball replaces the acrylic source and stem.



\begin{enumerate}

\checkitem Obtain permission to proceed from the calibration group.
\checkitem Remove salt probe / plug clamp.
\checkitem Remove the 3 screws holding the can containing the source ( or the source if uncanned configuration ).

\checkitem Contact OCE for instructions to verify what steps are to be followed regarding the can and source.
   {\bf Do not proceed unless authorized to do so !}

\small
{\em  The can may be part of the containment system for the source. {\bf Do NOT open the can unless directed to do so
by the OCE !   }    }
\normalsize 

  \checkitem Uncanned source

\begin{enumerate}

 \checkitem Place source in the proper bag and box in the source cabinet. Fill out logbook.

\end{enumerate}

\checkitem Canned source  ( {\bf NOT SEALED source }

 \begin{enumerate}

 \checkitem Remove 3 screws on the side.

  \checkitem Gently pry off the top.

  \checkitem Remove acrylic source from container. Fill out logbook and place source in proper bag and box in the source cabinet.

\end{enumerate}


\checkitem Sealed source.

\small
{\em  For the sealed source the can is an essential part of the containment system for the source.
 {\bf Do NOT open the can  !   }    }
\normalsize 

\begin{enumerate}

 \checkitem Place sealed source and ``donut'' ( if present ) in a bag and fill out the source logbook.

\end{enumerate}


\checkitem While supporting both source stem and the rest of the assembly remove screws which hold the stem to the can.
\checkitem Remove pressure plate.
\checkitem Gently slide the stem away from the can until CAJON fitting is exposed. Be sure to support both
can and stem and make sure they're aligned. 
\checkitem Carefully unscrew top end of CAJON fitting and pull out the fiber end. Do not let it retract into the can.
\checkitem Make sure the o-rings  are in place ( one for the saltprobe / plug
and another for the center ).
\checkitem Push the fiber in to the CAJON  fitting on the laserball. Tighten the fitting ligthly.
Make sure you don't twist or otherwise disturb the solid fiber on the laserball itself and make
 sure the fiber end is through
the o-ring.
\checkitem Mount the laserball on the can. Make sure the fiber is not bent or pinched
in any way. Use correct screws.
\checkitem Slide clamp up around saltprobe or plug. Make sure it is oriented correctly ( so that it fits ) and
tighten screw appropriately. ( {\bf Do not overtigthen  } ! )


\end{enumerate}


\newpage

\section{Other Source  Procedures}

\newprocedure{CalProcOther}
   {Other Source  Procedures}
   {P. Skensved}{Sept. 2004}{1}


   The following sections describe procedues for assembling and disassembling various sources.

 In order to ensure that there are no leaks and
that the source is securely attached to the umbilical
it is important that correct size o-rings are used everywhere. The
assembly uses small screws and nuts in many places. These are easily
stripped and damaged. {\bf Do not overtighten ! And do not re-tighten
just to make ``sure'' !}  If you do not know what the proper torque
should be contect an expert.
  If there are holes drilled through the screws secrure them with wire.
Do not twist or flex the wire more than necessary and make sure you do not
leave any weakened pieces of wire  as
they may end up in the detector. Re-do the wire instead. Bend the ends so
that they will not poke holes in the gloves.


  \include{calproc_laserball_assembly}
  \include{calproc_n16_assembly}
  
  

%------------------------------------------------------------
%\newpage
\subsection{Acrylic Source Assembly}


\newprocedure{CalProcAcrAssem}
{Acrylic Source Assembly}
             {F. Duncan/P. Skensved}
             {Oct. 2004}{4}

  This procedure describes the assembly of the acrylic source.  

  This procedure is very similar to the assembly of the laserball and uses much
of the same hardware. The can may be different in that it may be a blind can ( ie. no
LED mount and no hole for the saltprobe ). There are two choices for the stem which
holds the source : a polypropylene one or a teflon one.

\subsubsection{Procedure}
~\\
\begin{tabular}{|l|l|}
\hline
\multicolumn{2}{|l|}{\bf Acrylic Source Assembly Procedure}\\
\hline
 & \\
Operator:~~~~~~~~~~~~~~~~~~~~~~~~~~~~~~~~~~~~~ & Date: ~~~~~~~~~~~~~~~~~~~~\\
 & \\
\hline
\end{tabular} \\







\subsubsection{Prior To This Procedure}
  \begin{itemize}
  \item The rope may or may not be attached to the rotating bearing.
  \item The carriage may or may not be attached to weight cylinder.
  \item The spool piece may or may not be attached to weight cylinder.
  \item The can may or may not be attached to the spool piece.
  \item The can may or may not be assembled.
  \item The stem may or may not be attached to the can
  \item The salt probe may or may not be installed and connected.
  \end{itemize}



\newpage

\subsection{Assembly of Acrylic Source}

\begin{enumerate}
\checkitem Tie the rope to the rotating bearing. {\bf This is to be done by an expert only !}
\checkitem Slide the rotating bearing on the umbilical.
\checkitem Slide the o-ring on the umbilical. Use correct size o-ring.
\checkitem Attach the carriage to the weight cylinder. Secure the nuts with wire.
\checkitem Attach the spool piece to the weight cylinder. Secure the nuts with wire.
\checkitem Slide the carriage, weight cylinder and spool piece  on the umbilical.
\checkitem Slide a spacer with groove facing down on the umbilical.
\checkitem Slide the o-ring on the umbilical.
\checkitem Slide the stainless steel lid on the umbilical.
\checkitem Slide the o-ring on the umbilical.
\checkitem Slide the acrylic lid on the umbilical.
\checkitem Tie the wires in a  `Hallinian' knot and secure with cable ties. {\bf This is to
be done by an expert only !}
\checkitem Pull the knot up against the lid ( Gently !!! )
\checkitem Make sure the acrylic plate o-ring is in place.

  If the can is a ``blind'' can the next few steps do not apply. If the can
is the one used for the laseball then you will have to install either the saltprobe
or a dummy plug. Use the
appropriate section for the next few steps. The can may also hold a blue LED which
is used to index the laserball in the H$_2$O. Note that only one electrical
device can be connected at any given time ( ie. saltprobe or LED ).

\checkitem Assemble saltprobe.
\begin{enumerate}

\item Mount the inner clamp around the salt probe.
\item Put the salt probe into the can.
\item Slide the o-ring up atround the saltprobe.
\item Connect the wires to the saltprobe. The colour codes are listed in the log book.
item Make an entry in the logbook stating that the saltprobe is now connected to the umbilical.
\end{enumerate}

\checkitem Assemble plug.
\begin{enumerate}
\item Put the plug in the bottom of the can.
\item Slide the o-ring up around the plug.
\end{enumerate}

\checkitem LED assembly.
\begin{enumerate}
\item Press the LED into place. Be careful not to break anything.
\item Connect the wires to the LED. The colour codes are listed in the logbook.
\item Make an entry in the logbook stating that the LED is now connected to the umbilical
\end{enumerate}

  Laserball can only :

\checkitem Feed the fibers through the can and through the hole in the bottom of the can.
\checkitem Make sure the wires and the fibers are placed correctly inside the can. The
preferred routing of the fiber is in a helix which is free to move vertically

\checkitem Attach the acrylic lid to the can with 4 screws. Ensure that the o-ring is in place.
Make sure you're using the correct holes in the plate and make sure you're using the correct
length  screws.

\checkitem Check that everything looks ok inside the can
\checkitem Slide the umbilical o-ring up against the acrylic lid.
\checkitem Slide the stainless steel lid into place. Attach with screws. Use correct screws.
\checkitem Slide the second o-ring into place.
\checkitem Slide the pressure plate into place.
\checkitem Attach the can to the spool piece using 5 screws. Use correct screws.

\checkitem Make sure the o-rings for the are in place ( one for the saltprobe / plug
and another for the center ).

    If the polypropylene stem is being used do the following :

\checkitem Push the fiber in to the CAJON  fitting on the stem. Tighten the fitting ligthly. 
\checkitem Mount the stem on the can. Make sure the fiber is not bent or pinched
in any way. Use correct screws.


    If the teflon stem is being used do the following :

\checkitem Push a CAJON fitting onto the peg on the stainless seal plate and tighten the
appropriate part of the fitting

\checkitem Push the fiber into the other end of the CAJON fitting. Tighten lightly.
\checkitem Hold the plate up against the can while attaching the teflon stem.
Make sure the fiber is not bent or pinched
in any way. Use correct screws. Make sure the o-rings are in place.

     The following applies to both stems.

\checkitem Slide clamp up around saltprobe or plug. Make sure it is oriented correctly ( so that
 it fits ) and
tighten screw appropriately. ( {\bf Do not overtigthen  } ! )

   If the ``blind'' can is being used do the following :

\checkitem Place the wires and the fiber in the can.
\checkitem Attach the acrylic lid to the can with 4 screws. Ensure that the o-ring is in place.
Make sure you're using the correct holes in the plate and make sure you're using the correct
length  screws.
\checkitem Slide the umbilical o-ring up against the acrylic lid.
\checkitem Check that everything looks ok inside the can

\checkitem Slide the stainless steel lid into place. Attach with screws. Use correct screws.
\checkitem Slide the second o-ring into place.
\checkitem Slide the pressure plate into place.
\checkitem Attach the can to the spool piece using 5 screws. Use correct screws.
\checkitem Mount the stem on the blind can.


  The rest of the procedure applies to all cans



\checkitem Slide the o-ring for the rotating bearing into place.
\checkitem Attach the rotating bearing to the carriage with 4 screws and secure them
with wire. Do not overtighten !!!  Do not pinch or damage the o-ring.

\checkitem Attach the correct source to the stem. The flat-topped stainless steel
cans, the standard steel can and the teflon can all attach to the stem without
any other hardware. The sealed stainless steel AmBe  can requires an extar `donut'
to make it flat-topped.
\small
{\em If something looks like it does not go together or appears to be
wrong call the OCE. Do not under any circumstances open a can or remove
any part of it. Some of the cans are to remain sealed at all times.
}

\normalsize


\end{enumerate}


  
%------------------------------------------------------------------------
%------------------------------------------------------------------------
%------------------------------------------------------------------------
\newpage
\subsection{Disassembly of Acrylic Source}
\shwlabel{procpca}~\\
\noindent
\begin{tabular}{|l|l|}
\hline
Version              & 2 \\
\hline
Written/Revised by   & F. Duncan \\
\hline
Date Written/Revised & 2000/10/05\\
\hline
\end{tabular}
 

%----------------------------------------------------------------------
\subsubsection{Procedure}
~\\
\begin{tabular}{|l|l|}
\hline
\multicolumn{2}{|l|}{\bf Disassembly of Acrylic Source Ver 2}\\
\hline
 & \\
Operator(s):~~~~~~~~~~~~~~~~~~~~~~~~~~~~~~~~~~~~ & Date: ~~~~~~~~~~~~~~~~~~~~\\
 & \\
\hline
\end{tabular} 
~\\
\begin{enumerate}
\item\checkbox To be written - see laserball disassembly for now


\end{enumerate}



  

  
%------------------------------------------------------------------------
%------------------------------------------------------------------------
%------------------------------------------------------------------------
\subsection{AmBe Source Procedures}
\shwlabel{secprocAmBe}


 The AmBe Neutron Source is an Americium-Berylium encapsulated source
of neutrons used to calibrate the detector neutron capture efficiency.
It consists of a comercial Amercium Source on a small wafer that produces
$\alpha$ particles which capture on a Berylium disk producin neutrons
in a ($\alpha$,n) interaction on the Be.  


\begin{table}[htb]
\begin{center}
\begin{tabular}{|l|c|}
\hline
Assembled Weight & ~~~~~~~~ 64N ~~~~~~~~\\
\hline
Volume           & ~~~~~~~~~~~~~~~~~~~~~\\
(including weight and carriage) & \\
\hline
Pivot Centre Offset & 76.6 cm\\
\hline
Pivot Bottom Offset & 77.3 cm\\
\hline
\end{tabular}
\caption[AmBe Neutron Source]
  {AmBe Neutron Source
   \shwlabel{TabSourceAmBe}
  }
\end{center}
\end{table}



\clearpage
\begin{figure}
\begin{center}
\epsfxsize=7in
\epsfbox{./figures/ambe_exploded.ps}
~\\
\caption[Assembly drawing of the manipulator mounted AmBe Source]
        {Assembly drawing of the manipulator mounted AmBe Source 
         showing from the Laserball Canister down.
         \shwlabel{figAmBe}
        } 
        
\end{center}
\end{figure}




\clearpage

\begin{figure}[t]
\begin{center}
\leavevmode
\epsfxsize=5.0in
\epsfbox{figures/Photo_AmBe_Assembled.eps}
\caption[AmBe Source Assembled]{
  \shwlabel{PhotoAmBeAssembled}}
  The Assembled AmBe Source
\end{center}
\end{figure}
\begin{figure}[b]
\begin{center}
\leavevmode
\epsfxsize=5.0in
\epsfbox{figures/Photo_AmBe_SealPlate.eps}
\caption[AmBe Source Sealplate]{
  \shwlabel{PhotoAmBeSealPlate}}
  The AmBe Sealplate.  The Cajon fitting, the fibre and
  the o-rings (around salt probe and around base of the stud)
  can be seen.
\end{center}
\end{figure}



%------------------------------------------------------------
\clearpage

\subsection{AmBe Source Assembly Procedure}
\newprocedure{CalProcAmBeAssembly}
             {AmBe Source Assembly Procedure}
             {Fraser Duncan}{2002/11/08}{1}

  The AmBe is similar to the acrylic encapsulated sources in that
it is mounted on a stem that mounts on the Laserball source's canister.
instead of the Laserball itself.  Thus there are two paths to
the assembly of the AmBe source.  
\begin{enumerate}
\item The Laserball is mounted on a URM.  The Laserball is removed and
  the AmBe source is mounted in it's place.
\item Full assembly of all the source components for the AmBe on the
  URM.
\end{enumerate}
While the AmBe source mounting is similar to the acrylic source mounts,
there are two important differences:
\begin{itemize}
\item The mounting stem for the AmBe source is made from Teflon to
  (low hydrogen content) to minimize neutron capture on the mounting
   hardware.
\item A stainless steel disk is sandwitched between the stem and the
  Laserball canister to form the water seal on the canister. This is
  because the teflon stem does not make a good water seal.  It is
  very important to ensure the canister is in place.
\end{itemize}



{\em Temporary Note:  This procedure describes the assembly of the
  source starting with the assembled Laserball and ending with the
  assembled AmBe source.}






\noindent
{\bf State Prior To This Procedure:}
\begin{enumerate}
\item URM2 has been unmounted from the glovebox and rolled back
\item Source has been lowered out of the URM source tube.
\item The laserball canister is mounted on URM2 but the
  Laserball stem and ball have been removed.
\item The fibre for the Laserball is protruding from the bottom
  of the laserball cansister.
\end{enumerate}




\noindent
{\bf Procedure:}
~\\
\begin{tabular}{|l|l|}
\hline
 & \\
Operator(s):~~~~~~~~~~~~~~~~~~~~~~~~~~~~~~~~~~~~~~~~~~~~~ 
 & Date: ~~~~~~~~~~~~~~~~~~~~~~~~~~~~~~~~\\
 & \\
\hline
\end{tabular} 
~\\
~\\
\begin{enumerate}

\checkitem Put o-ring in the {\em Seal Plate} around the stud.

\checkitem Verify that the o-ring around the salt probe is present
  in the bottom of the Laserball Cansiter.

\checkitem Put the {\em cajun fitting}  
  on to the {\em stud} on the {\em seal plate}.

\checkitem Tighten the {\em cajun fitting} onto the {\em stud}.

\checkitem Loosen the top of the {\em cajun fitting}

\checkitem Push the {\em Laserball fibre} into the top of the
  {\em cajun fitting}.\\
  {\em There will be resistance as the fibre passed through the
    o-ring inside the  cajun fitting.}

\checkitem Tighten the {\em cajun fitting} finger tight.\\
  {\small\em At this point the assembly should look like the
   photograph in figure \ref{PhotoAmBeSealPlate}.}

\checkitem Slide the {\em seal plate} up against the 
   {\em Laserball canister}.\\
  {\em As you do this make sure that both the o-ring on the seal plate
    around the stud and the oring on the canister around the salt probe
    remain seated.}

\checkitem Slide AmBe {\em stem} over the salt probe and ``snug'' it up
  against the {\em seal plate}.

\checkitem Slide the {\em compression plate} over the {\em stem} and
  the {\em salt probe} and ``snug'' it up against the base of the {\em stem}.

\checkitem Put the 4 1'' screws through the {\em compression plate},
   base of the {\em stem} and {\em seal plate} and thread them into
   the {\em Laserball cansiter}.

\checkitem Align the  hole in the salt probe to be tangential to
  the {\em stem}.

\checkitem Tighten the 4 1'' screws holding on the {\em stem}.

\checkitem Slide the salt probe {\em clamp} down the salt probe as ``snug''
  it up against the {\em compression plate}. \\
  {\em There is a flat spot on the clamp that fits against the stem.}

\checkitem Tighten the set screw on the {\em clamp}.

\checkitem Remove the AmBe {\em source container} from the Radioactive Source
  cabinet.

\checkitem Sign out the AmBe source in the {\em Radioactive Source Log Book}.

\checkitem  Inspect the three screws on the side of the
  AmBe {\em source container}.  All three should be present and
  fully threaded in.

\checkitem One at a time install the three mounting screws that hold the
  AmBe source to the stem.
  \begin{enumerate}
  \item Place the nut in the slot on the lid of the source.
  \item Push the cap screw with washer through the stem and source lid
    into the nut.
  \item partially tighten the cap screw into the nut.\\
    {\em It is probably necessary to use a small screw driver to hold
      the nut in place while tightening the screw.}
  \end{enumerate}
  
\checkitem Tighten all three mounting screws.

\checkitem Inspect all screws on the source assembly:
  \begin{itemize}
  \item The horizontal sealing screws in the source container.
  \item The 3 mounting screws in the holding the source to the stem.
  \item The 4 screws holding the stem to the Laserball Canister.
  \item The screws on the Laserball Canister, Weight Cylinder and
     Carriage.
  \end{itemize}

\checkitem Inspect the knot tying the URM rope to the carriage.

\checkitem Check the zero tension in the URM rope.  If it is not zero,
  recalibrate the zero offset.

\checkitem Retract the rope and umbilical until the source is hanging
  from the URM.
 
\checkitem Note the weight of the source.  It should agree with the
  value in table \ref{TabSourceAmBe}.

\checkitem Retract the source back into the URM.

\checkitem Calibrate the central rope length.




\end{enumerate}



{\small
~\\
~\\
\noindent
{\bf Revision History:}\\
\begin{tabular}{llll}
Rev. & Date & Author & Comments\\

0             & 
2002/11/08    & 
Fraser Duncan &
\parbox[t]{3.0in}{
  First Draft
}
\end{tabular}
}





%========================================================================
%========================================================================
%========================================================================

\newpage
\markright{\standardheader}




  
%--------------------------------------------------------------
%--------------------------------------------------------------
%--------------------------------------------------------------


\section{Misc Procedures}
  
\subsection{Backing Up the Manip program}
Suppose you are backing up the version 2.2.
\begin{enumerate}
\item Go to the motors.2\_2 directory,
\begin{verbatim}
  cd c:\motors.2_2
\end{verbatim}
\item To to a full backup of the manipulator code type
  \begin{verbatim}
    backup all
  \end{verbatim}
  To only backup the code and not the data files type,
  \begin{verbatim}
    backup
  \end{verbatim}
  Either command creates the zip file,
  \begin{verbatim}
     manip.zip
  \end{verbatim}
\item rename the file
  \begin{verbatim}
    rename manip.zip manip2_2.zip
  \end{verbatim}
\item Copy the file to surf for archive using ftp
\end{enumerate}

   


\subsection{Calibrating Load Cells}
  The loadcells on the manipulator will require calibration periodically.
In particular the zero's of the loadcells are known to drift and have a
significant temperature variation.
  






%--------------------------------------------------------------
%--------------------------------------------------------------
%--------------------------------------------------------------


  
%--------------------------------------------------------------
%--------------------------------------------------------------
%--------------------------------------------------------------
\chapter{TroubleShooting}
\shwlabel{ChapterTroubleShooting}
  

    This chapter describes various problems that can arise from
the calibration system and suggests ways to diagnose and correct
them.

%-----------------------------------------------------------------------
\newpage
\section{Manipulator Trouble}
\shwlabel{secmanipulatortrouble}
  
\subsection{Manipulator Error Messages}
\begin{center}
\begin{tabular}{|l|l|}
\hline
 & \\
{\bf STOPPED\_NONE} &
\parbox[t]{3in}{
 Sometimes the manipulator will stop with a {\bf STOPPED\_NONE}
error message.  The cause of this could be a {\em Watchdog Timer Error}.   
} \\
 & \\
\hline
 & \\
{\bf STOPPED\_LOW\_TENSION} &
\parbox[t]{3in}{
 A rope or umbilical has too low of a tension.  This is potentually
a serious problem.   
} \\
 & \\
\hline
 & \\
{\bf STOPPED\_HIGH\_TENSION} &
\parbox[t]{3in}{
 A rope or umbilical tension is too high.  
} \\
 & \\
\hline
 & \\
{\bf STOPPED\_STUCK} & 
\parbox[t]{3in}{
 The manipulator is unable to get closer to the requested end point.
This sort of error can be caused by friction in the ropes and pullies
that cause the manipulator motion to be jerky and have hysteresis.
It can sometimes be corrected by requesting the desired position again.
} \\
 & \\
\hline
 & \\
{\bf STOPPED\_NET\_FORCE} &
\parbox[t]{3in}{
  The forces on the source do not sum to zero.  {\em Not used.}
} \\
 & \\
\hline
 & \\
{\bf STOPPED\_AXIS\_ERROR} &
\parbox[t]{3in}{
  The encoder error for an axis is greater than the allowed limit
(set to +/- 10 cm).  The encoder error is the difference in position
determined by the axis' encoder and the number of steps sent to the
axis stepper motor.
This is may be caused by a slow drift on the
encoder or it may indicate the failure of an encoder or a motor.
} \\
 & \\
\hline
 & \\
{\bf STOPPED\_CALC\_ERROR} &
\parbox[t]{3in}{
  A calculation error stopped the motion.
} \\
 & \\
\hline
 & \\
{\bf STOPPED\_AXIS\_FLAG} &
\parbox[t]{3in}{
  A digital flag stopped the motion
} \\
 & \\
\hline
 & \\
{\bf STOPPED\_AXIS\_ALARM} &
\parbox[t]{3in}{
  If an axis tension exceeds a preset limit, the manipulator will
shut down.  This limit is designed to prevent excessive loads on
the manipulator components.
} \\
 & \\
\hline
\end{tabular}
\end{center}

 
%--------------------------------------------------------------------------
\newpage 
\subsection{Specific Errors}
  
\subsubsection{STOPPED\_NONE --- Watchdog Timer Error}
  Sometimes the manipulator will stop for no obvious reason and
will report a {\bf STOPPED\_NONE} error.   This could be caused by
the watchdog timer timing out.  To see if this is the case look
at the {\bf max poll time} on the main display of the GUI or on the
manipulator console.  If this time is of order 500 (ms) or more, then
it is likely that the error was a watchdog timer error.  The max
poll time is displayed as a strip chart on the main GUI panel.  
  
  The cause of the watchdog timer error is intense ethernet trafic
that causes the manipulator computer to freeze up.  The watchdog timer
then times out shutting down the manipulator.  To prevent the motors
from becoming unsynced, the manipulator computer generates a {\bf STOP}
if the poll time goes above a few hundred ms.

  
\subsection{STOPPED\_LOW\_TENSION}
  This is potentially the most serious kind of error.  If a rope
goes to too low of tension it will unspool from the it's takeup drum
and it will be necessary to open the motor box to repair the damage.

  If a {\bf STOPPED\_LOW\_TENSION} error occurs, examine the situation
to determine if it is a reasonable error.  For instance if the source
is being driven down into a lower quadrant of the detector the side rope
on the opposite side must go to low tension.  
  
\begin{enumerate}
\item If the low tension alarm appears to be caused by the source in
 being in a low tension region, try moving the source away from the 
  low tension region.
  \begin{itemize}
  \item If the source will not move because the low tension alarm persists,
    {\bf CONTACT EXPERT}.
  \end{itemize}
\item If there is no obvious reason for the low tension alarm, 
    {\bf CONTACT EXPERT}.  
\end{enumerate}
  
\subsection{STOPPED\_HIGH\_TENSION}
  If the manipulator stopped with a high tension error, examine the 
situation.  A high tension alarm can occur if trying to drive the
source into an upper quadrant of the detector.  The rope on the side
of the detector that the source is moving towards goes to high tension.
  
\begin{enumerate}
\item If the alarm is due to driving into a high tension region,
  attempt to back out of the region.  If the rope is still in
  high tension and won't allow the  source to move,
  {\bf CONTACT EXPERT}.
\item If there is no obvious reason for the high tension alarm,
  {\bf CONTACT EXPERT}.
\end{enumerate}
  
\subsection{STOPPED\_STUCK}
  This error indicates that the manipulator is having trouble getting
to a desired location.  The likely cause is friction in the manipulator
and the discreteness of the encoders cause the manipulator to overshoot.
This is not an error that indicates potential risk to the manipulator.
The options in this situation are:
\begin{itemize}
\item Retry the move to the position desired.
\item Go somewhere else.
\end{itemize}

\subsection{STOPPED\_STUCK}
  This error indicates an inconsistancy in the position calculations.
It could indicate that one or more of the ropes is significantly miscalibrated.
\begin{itemize}
\item If the error persists, a recalibration of the system should be done.
  {\bf CONTACT EXPERT}.
\end{itemize}

\subsubsection{STOPPED\_AXIS\_ERROR due to Umbilical}
  
  A {\bf STOPPED\_AXIS\_ERROR} caused by the umbilical may or may not
be serious.  The calibration of the umbilicals is not ideal and so there
is a slow drift that accumulates.  On the other hand a rapid accumulation
of the encoder error indicates that either the encoder is  faulty or
the motor is not functioning properly.  The quantity derived in the 
GUI, {\bf dE/dL} is the rate of change of encoder error with umbilical
length (as determined by the encoder).  A large dE/dL indicates problems
with either the encoder or motor.  A small value indicates ``drift''.
The figure of merrit to use is:

\begin{description}
\item[ $|$dE/dL$|$ $<$ 0.1] This is a slow drift indicative of 
  miss calibration.\\
  {\bf Reset the umbilical encoder and proceed}
\item[ $|$dE/dL$|$ $>$ 0.1] This is a fast drift and may indicate
  encoder or motor problems.\\
  {\bf This is a potentially serious hardware fault.  Contact
  manipulator expert.}
\end{description}

  
\subsubsection{STOPPED\_AXIS\_ERROR due to a Rope}
  This is an encoder error due to a rope's encoder being out of 
agreement with it's motor.  This is an almost never unseen situation
since the calibration of both motor and encoder on the ropes is 
excellent.  Therefore this error should be considered an indication of
a serious problem.
  
\begin{center}
{\bf Contact a manipulator expert}
\end{center}

\subsection{STOPPED\_AXIS\_FLAG on Umbilical}
  Umbilicals (URM2 and URM3) have limit switches to prevent
playing out too much umbilical.  This is flag bit 0.  URM3 has
a limit switch to prevent taking up too much umbilical as well
(flag bit 1).  If there is a {\bf STOPPED\_AXIS\_FLAG} error, it
could be due to the limit switches.   
\begin{enumerate}
\item  Check to see which bit has
  failed.  In the GUI these bits are red if failed or green if the
  interlock is satisfied.  
\item If the interlock is failed, try moving the source away from
  the limit switch.  I.e. if too much umbilical is played out, 
  try moving the source up (shorter umbilical).
  
\item If moving the source away from the interlock condition does
  not help, inspect the manipulator wiring to ensure the limit switches
  are plugged in to the data concentrator (the interlocks default to
  failed if  the path to the limit switch is broken).
 
\item If these actions do not fix the problem, {\bf call a manipulator
  expert}.

\end{enumerate}


\subsection{STOPPED\_AXIS\_FLAG on a Rope}
  Ropes have an interlock that detects heavy loads (such as sudden
shocks) on the rope.  This interlock trips at approximately 100 lb.
If the manipulator stops due to a {\bf STOPPED\_AXIS\_FLAG} on
a rope, it is possible that the rope unit has undergone a large
load.  This could compromise the ability of the rope unit to hold
sources.  
\begin{enumerate}
\item {\bf CONTACT EXPERT}
\item Inspect the rope unit wiring to verify all cables are connected.
\item Using a DVM check if the shock load switch is open (open indicates
  a shock load).  
\item If the DVM indicates that the switch  is indeed open, then the 
  rope unit should be inspected visually.  If the detector is on and
  the rope unit opens into the detector volume, it may be necessary to
  turn off the detector to make the inspection.
\end{enumerate}


\subsection{STOPPED\_AXIS\_ALARM}
  
  If the manipulator halts due to a {\bf STOPPED\_AXIS\_ALARM},
check the current tension of the rope or umbilical that caused the
alarm to see if it exceeds the alarm limit.  The alarm function in
the manipulator code also stores the maximum value of the tension
that caused the alarm.  

{\bf CONTACT EXPERT}


 
  





  
%------------------------------------------------------------------------
%------------------------------------------------------------------------
%------------------------------------------------------------------------

 

  
  

  
%-------------------------------------------------------------------------
%-------------------------------------------------------------------------
%-------------------------------------------------------------------------
\appendix
 


\chapter{Calibration Glossary}
\shwlabel{ChapterGlossary}

\begin{description}

\item[Axis]
  Each rope or umbilical in the manipulator system is referred to
  as an axis.

\item[Data Concentrator]
  Is the box that reads out the encoder boards that measure the 
  manipulator rope lengths and tensions.


\item[DCR]
  Deck Clean Room.

\item[Encoder Box]
  Each manipulator {\em Axis} has an encoder box to which is attached
  a shaft encoder to measure the axis length and a load cell to measure
  the tension of the axis.  The shaft encoder and load cell are
  readout through the encoder box by the {\em Data Concentrator}.

\item[Glovebox]
  The box mounted on the Universal Interface over top of the 
neck of the detector.  Calibration sources that are deployed into
the detector are mounted in URMs which are attached to the glovebox
at either gate valve 1 (10 inch) or gate valve 3 (4 inch).  There
is also a 6 inch gate valve which is not used.

\item[Manipulator] The systems used to deploy calibration sources
  into the SNO detector.

\item[Polyaxis] is the name used in the manipulator system
  for multiple axes (ropes or umbilicals) attached to a calibration
  source.

\item[Umbilical]
  The umbilical is a 1/2 inch diameter silicone tube that which
  carries the services to the source.

\item[URM]
  Umbilical Retreival Mechanism.  The sources are mounted in
  {\em Source Tubes} on URMs.  The URM contains a rope which
  supports the weight of the source and an {\em Umbilical} which
  carries the services to the source.

\item[Watchdog Timer Box]
  The manipulator stepper motors are controlled by a National Instruments
  TIO10 card.  The Watchdog Timer Box is actually a breakout box for
  the TIO10 card.  The manipulator computer periodically sends a 
  {\em motor enable} signal to the box.  If this signal is not present
  the manipulator motors are shut down.



\end{description}


  
%-------------------------------------------------------------------------
%-------------------------------------------------------------------------
%-------------------------------------------------------------------------
  
\chapter{Guide Tube Operations}
  
\begin{table}
\begin{center}
\begin{tabular}{lrrrrrr}
  & CT4 \\
\hline
x &  230.75 in \\
  & -586.11 cm \\

y & 81.875 in \\
  & 207.96 cm \\
  
r & 586.12 cm \\
  
Enters PSUP & 564.64 cm \\

\end{tabular}
\caption[Guide Tubes]
        {Guide Tubes
         \shwlabel{tabguidetubes}
        }
\end{center}
\end{table}
   
\begin{figure}
\begin{center}
\leavevmode
%\epsfysize=0.85\textheight
\epsfxsize=6.5in
\epsfbox{figures/guide_tubes.ps}
~\\\
\caption[Guide Tube Layout In DCR]
        {Layout of the Guide Tubes in the DCR
         \shwlabel{figguidetubeplan}
        }
\end{center}
\end{figure} 
 
  
   
\begin{figure}
\begin{center}
\leavevmode
%\epsfysize=0.85\textheight
\epsfxsize=6.5in
\epsfbox{figures/guide_tubes_elevation.ps}
~\\\
\caption[Guide Tube Elevation View]
        {Guide Tube Elevation View
         \shwlabel{figguidetubeelevation}
        }
\end{center}
\end{figure} 
 
  
  
\end{description}





\chapter{Obsolete Procedures}






%-------------------------------------------------------------
%-------------------------------------------------------------
\newpage
\section{Calibration Tube \# 4 Deployment of N16 Source}
  
\begin{center}
\begin{tabular}{|l|l|}
\hline
Version    & 1.1 \\
\hline
Date       & 7 Dec 1999\\
\hline
Written by & F. Duncan\\
\hline
\end{tabular}
\end{center}
 
\noindent
{\bf Overview}

 
  This is a temporary procedure for the deployment of the N16
source down guide tube \# 4.  In is intended for the first deployment
of the source down this tube into the light water volume between
the AV and the PSUP.  Since this has not been done before, this procedure
will proceed cautiously.
  

  The only deployment of a source in a calibration guide tube
has been the lowering of the laserball into CT4 during air fill.
That deployment was done, guiding the source by hand with the
detector off and while visually verifying that the source was
centred in the tube.  That deployment showed that the source tube
was in fact not vertically aligned with the bottom of the tube
being approximately 2'' out of true wrt the top.  After water fill,
all source tubes were realigned with the condition that a 5 1/8''
source could pass through the tube without touching the sides.

  
  Ideally, a ``dry run'' of the deployment down the guide tube would
consist of lowering a dummy source with tapered ends down the guide 
tube without the source tube (the 4' tube that contains the
source when it is retracted) attached.  This would allow observation
of the dummy source and would help diagnose any unexpected problems.  
  Deployment of the N16 source without a dry run complicates the
the situation if there is a problem but should not pose additional
risks.
  
  The procedure will be to deploy the N16 source with a polypropolyne
cone attached above the source.  This will help guide the source
back into the guide tube in case of misalignment.  In the event that
the source gets stuck during the deployment, it will be necessary to
turn off the detector, unbolt the source tube from the gate valve
and raise the URM and source tube up approximately 2 ft to allow
an inspection of the source within the guide tube.

\begin{figure}
\begin{center}
\leavevmode
%\epsfysize=0.85\textheight
\epsfxsize=7in
\epsfbox{./figures/n16_guide_tube.ps}
~\\
\caption[N16 source in Source Tube with Cone]
        {N16 Source in Source Tube with Cone
         \shwlabel{fign16guidetube}
        } 
        
\end{center}
\end{figure}
  
 
\noindent
{\bf Risks}

  The largest risk with the deployment down the guide tube is
that the source gets stuck.  In the worst case we would not be
able to get it out and it would remain in the tube, possibly blocking
part of the PSUP indefinately.  This is probably a relatively low
risk.  The likely problems are:
\begin{description}
\item[Source hangs up on way down]~\\
  The source does not hang vertically from the rope since it is off centre.
  Since the source is not square in the guide tube and since the guide
  tubes are not exactly aligned, there is a chance that the source will
  catch on the way down. An indication that the source has hung up is
  if the tensions in either rope or umbilical drop.
  
  If this happens we will try putting most of the source load on the
  umbilical to straighten it in the tube.  If this does not get the source
  past the obstructions, then we will abort the run.

\item[Source gets stuck while being retracted]~\\
  While retracting source it hangs in the guide tube well below the
  deck.  
  
  The intent of the umbilical cone is to prevent such an event.  However,
  if it does happen, we will have to shut down the detector, open the
  guide tube and try to dislodge the source by hand.
  
\item[Source does not retract completely into source tube]~\\
  The source is deployed single axis and is prone to having the
  rope and umbilical twist.  This is prevented from being a problem
  in the detector by having a rotating collar for the rope attachment.
  However, the umbilical cone negates this, locking the rope and umbilical
  together.  If there is significant twist, we will not be able to lift
  the source all the way up and close the gate valve.
  
  If this happens, we will have to shut the detector off, open up
  and lift the source out by hand.  This should be straight forward
  but does require detector off.
\end{description}



%------------------------------------------
\newpage
~
\vspace*{0.5in}
\begin{center}
\begin{tabular}{|l|l|}
\hline
 & \\
DCR Floor Z position     & 1322.07 cm\\
 & \\
\hline
 & \\
Fiducial Mark to DCR floor & 140.7 cm \\
 & \\
\hline
& \\
Top of Cone to Pivot & 18.5 cm\\
& \\
\hline
Fiducial Calibration & \\
at Cone top          & 1466.27 cm\\
 & \\
\hline
 & \\
Bottom of Source entering PSUP & $z_{pivot}$=643.44 cm \\
 & \\
\hline
 & \\
Minimum Z position &  $z_{pivot}$= -465 cm \\
 & \\
\hline
 & \\
source centre at PSUP &  $z_{pivot}$= 635.82 cm \\
 & \\
\hline
 & \\
cone just below PSUP &  $z_{pivot}$= 583.14 cm \\
 & \\
\hline
 & \\
{\bf Minimum Safe Height to Close Gate Valve} &  $z_{pivot}$= 1451.6 cm \\
 & \\
\hline
\end{tabular}
\end{center}
  
%--------------------------------------
\newpage
%\vspace*{0.2in}
\noindent
{\bf State Prior To This Procedure}
 
\begin{center}
\begin{tabular} {|l|l|l|l|}
\hline
\multicolumn{4}{|c|}{\bf Prior to CT4 N16 Source Deployment}\\
\hline
     &         &           &                   \\
Date & Initial & State ~~~~~~~~~~~~~~~~~~~~~~~~~~~~~~~~~~~~~~~~~~~~&
 Data and Comments ~~~~~~~~~~~~~~~~~\\
     &         &           &                   \\
\hline
&& DCR Lights are on.  Owl tube light monitor is on and being & \\
&& monitored by detector operator.& \\
&& & \\
\hline
&& & \\
&& DCR has been cleaned.  Area around CT4 has been cleared. & \\
&& & \\
\hline
&& URM2 unmounted from glovebox and suspended over DCR floor & \\
&& using the lifting straps and cart.& \\
&& & \\
\hline
&& & \\
&& N16 source mounted on URM & \\
&& & \\
\hline
&& & \\
&& URM rope and umbilical calibrations ok. & \\
&& & \\
\hline
\end{tabular}
\end{center}
 
  
%--------------------------------------
\vspace*{0.2in}
\noindent
{\bf Summary of Procedure}
\begin{itemize}
\item Determine lowest allowed point to put source and
  the minimum height to raise source to close gate valve.
  Put tube coordinates in the manip program.
\item Lower N16 source out of source tube.
\item Attach guide cone to umbilical above manipulator carriage.
\item Retract N16 source into source tube
\item Disassemble lifting rails and support URM-2 with lifting cart.
\item Move URM2 over to guide tube \# 4.
\item Clean gate valve.
\item Attach source tube to gate valve.
\item Turn off DCR lights.  Open gate valve do a light check.
\item Lower source 10 cm, raise again.
\item Deploy source into detector.
\item Take data runs.
\item Retract source
\item Close gate valve.



\end{itemize}

 
%--------------------------------------
\newpage
\begin{center}
\begin{tabular} {|l|l|l|l|}
\hline
\multicolumn{4}{|c|}{\bf CT4 Deployment of N16 Source}\\
\hline
     &         &           &                   \\
Date & Initial & Procedure ~~~~~~~~~~~~~~~~~~~~~~~~~~~~~~~~~~~~~~~~~~~~&
 Data and Comments ~~~~~~~~~~~~~~~~~\\
     &         &           &                   \\
\hline
&& & \\
&& Determine Fiducial mark calibration of Source Tube 
 & $z_{fiducial}=$\\
&& & \\
\hline
&& & \\
&& Determine Minimum Safe height to close gate valve
 & $z_{safe}=$\\
&& & \\
\hline
&& Enter CT4 coordinates into {\bf axis.dat} & \\
&&  x= ????  y = ???? z = ???? &\\
&& & \\
\hline
&& & \\
&& Lower N16 Source from Source Tube. & \\
&& & \\
\hline
&& & \\
&& Attach guide cone to umbilical. & \\
&& & \\
\hline
&& & \\
&& Verify source and cone attachments & \\
&& & \\
\hline
&& & \\
&& Retract Source into Source tube. & \\
&& & \\
\hline
&& & \\
&& Set Source Clamps on Source tube into the {\bf HOLD} position & \\
&& & \\
\hline
&& & \\
&& Detatch URM from lifting straps, supporting it from lifting cart. & \\
&& & \\
\hline
&& & \\
&& Position URM and source tube over CT4. & \\
&& & \\
\hline
&& & \\
&& Clean CT4 gate valve with U/P water and lint free rag. &\\
&& & \\
\hline
&& Attach Source Tube to CT4 gate valve. & \\
&& Verify Tube is vertical in both N/S and E/W planes. & \\
&& & \\
\hline
&& {\bf Set Source Clamps on source tube into the} & \\
&& {\bf RELEASE position.} &\\
&& & \\
\hline
&& & \\
&& Verify all light seals on URM2 and source tube. & \\
&& & \\

\hline
\end{tabular}
\end{center}

 
%----------------------------------------------------------
\newpage
\begin{center}
\begin{tabular} {|l|l|l|l|}
\hline
\multicolumn{4}{|c|}{\bf CT4 Deployment of N16 Source (cont'd)}\\
\hline
     &         &           &                   \\
Date & Initial & Procedure ~~~~~~~~~~~~~~~~~~~~~~~~~~~~~~~~~~~~~~~~~~~~&
 Data and Comments ~~~~~~~~~~~~~~~~~\\
     &         &           &                   \\
\hline
&& & \\
&& Turn off DCR lights. &\\
&& & \\
\hline
&& & \\
&& Open CT4 gate valve. & \\
&& & \\
\hline
&& & \\
&& Perform light check on URM2 and source tube. & \\
&& & \\
\hline
&& & \\
&& Turn DCR lights back on. & \\
&& & \\
\hline
&& Lower Source into CT4 in 10 cm steps& \\
&& until the source is completely below the deck. & \\
&& (See  {\em Possible Problems}.) & \\
\hline
&& Lower source into the detector volume. & \\
&& $z_{pivot}$= 643.44& \\
&& & \\
\hline

\end{tabular}
\end{center}
 
 
%----------------------------------------------------------
\newpage
\begin{center}
\begin{tabular} {|l|l|l|l|}
\hline
\multicolumn{4}{|c|}{\bf CT4 Retraction of N16 Source (cont'd)}\\
\hline
     &         &           &                   \\
Date & Initial & Procedure ~~~~~~~~~~~~~~~~~~~~~~~~~~~~~~~~~~~~~~~~~~~~&
 Data and Comments ~~~~~~~~~~~~~~~~~\\
     &         &           &                   \\
\hline
&& Raise Source to beginning of PSUP. &\\
&&   $z_{pivot}$=540.0  & \\
&& & \\
\hline
&& Set URM2ROPE Speed to 0.5 cm/s &\\
&&   {\tt urm2rope maxspeed 0.5}  & \\
&& & \\
\hline
&& Raise Source in 10 cm steps to  & \\
&&   $z_{pivot}$=725cm & \\
&& & \\
\hline
&& Set URM2ROPE Speed to 3 cm/s &\\
&&   {\tt urm2rope maxspeed 3.0}  & \\
&& & \\
\hline
&& Raise Source to just below deck & \\
&&   $z_{pivot}$=1300.0 & \\
&& & \\
\hline
&& In 10cm or smaller steps raise source & \\
&& into source tube.& \\
&& {\bf Minimum safe height  $z_{pivot}$= 1451.6} \\
\hline
&& & \\
&& Close Gate valve & \\
&& & \\
\hline
&& & \\
&& Remove handle from gate valve & \\
&& & \\
\hline

\end{tabular}
\end{center}
 
\vspace*{0.2in}
\noindent
{\bf State At Completion Of This Procedure}\\
The N16 source is retracted from detector, gate valve closed.
  
%-----------------------------------------------------------------
\newpage
\vspace*{0.2in}
\noindent
{\bf Possible Problems}
\begin{description}
\item[Source does not enter CT4 from the source tube.]~\\
  The last deployment of the N16 source on the glove box found
it difficult to get the source out of the source tube into the
glovebox.  It is believed that the problem was caused by two
things:
\begin{itemize}
\item The source is suspended with most of its load on a rope
  that is not centred above the source's centre of mass.
\item The source tube was not put on the gate valve square.
\end{itemize}
These two problems resulted in the source being off centre and
the flat bottom of the source hitting the protruding edge of the
gate valve.  
  The solution was to transfer most of the load of the source to
the umbilical which {\em is} in line with the centre of mass.
This let the source hang vertically and it then passed through
the gate valve.
  
If this problem is encountered during the CT4 deployment:
\begin{enumerate}
\item transfer load from the rope to the umbilical.  Try 90 N on
  the umbilical, 40N on the rope.  Do this by putting the rope
  in tension mode with the command:
  \begin{verbatim}
       manip> urm2rope tension 40
  \end{verbatim}
\item Lower the source 10 cm into the CT by using the command:
  \begin{verbatim}
       manip> urm2umbilical down 10
  \end{verbatim}
\end{enumerate}  

\item[Source cannot be retracted into source tube above gate valve]~\\
  This is most likely caused by the rope twisting around the umbilical.
  Attempts can be made to remove the source by lowering it and raising
  it again --- perhaps at a lower speed.  If these attempts fail, the
  only option is:
  \begin{enumerate}
  \item Turn off the detector.
  \item Disconnect all HV cables.
  \item Lower source 1m.
  \item unbolt URM from CT4.
  \item Raise URM with lifting cart approximately 1m.
  \item raise source until it is in DCR.  Hand feed it into
        source tube.
  \item When source is clear of gate valve.  Close gate valve.
  \end{enumerate}
  



\end{description}





  

  
%------------------------------------------------------------------------
%------------------------------------------------------------------------
%------------------------------------------------------------------------
\section{Proportional Counter Source Procedures}
\shwlabel{SecProcPCS}

  These procedures relate to the  assembly and operation of
the Proportional Counter Source.
  

%-------------------------------------------------------------
\subsection{Description of the PCS}




%------------------------------------------------------------
\newpage
\subsection{PCS Assembly}

 
\procedure{ProcPcsAssembly}{F. Dalnoki-Veress}{2001/09/04}{0.9}


 
This procedure describes the assembly and mounting of the
PCS on the URM for deployment into the detector.

\begin{enumerate}
\checkitem Step 1.
\checkitem Step 2.
\checkitem Step 3.
\checkitem Step 4.
\end{enumerate}


%------------------------------------------------------------
\newpage
\subsection{PCS Deployment}

 
\procedure{ProcPcsAssembly}{F. Dalnoki-Veress}{2001/09/04}{0.9}


 
This procedure describes the deployment of the PCS into the detector
for data taking.


\begin{enumerate}
\checkitem Step 1.
\checkitem Step 2.
\checkitem Step 3.
\checkitem Step 4.
\end{enumerate}


  
%------------------------------------------------------------------------
%------------------------------------------------------------------------
%------------------------------------------------------------------------
\section{CHIME Procedures}
\shwlabel{SecProcChime}

  These are the procedures related to the CHIME.  They supplement
the specific procedures written by the University of Washington for
the handling and assembly of the CHIME.


\begin{figure}
\begin{center}
\leavevmode
%\epsfysize=0.85\textheight
\epsfxsize=7in
\epsfbox{./figures/chime_gas.ps}
~\\
\caption[CHIME Gas and Vacuum Connections]
        {CHIME Gas and Vacuum Connections
         \shwlabel{figchimegas}
        } 
        
\end{center}
\end{figure}


\subsection{CHIME Deployment Procedure}
\noindent
\begin{tabular}{|l|l|}
\hline
Version              & 0.91 \\
\hline
Written/Revised by   & F. Duncan \\
\hline
Date Written/Revised & 2000/08/04\\
\hline
\end{tabular}
This procedure describes the final pump/purge of the CHIME and the
lowering of it into the detector.
\noindent 
  The outline of the procedure is:
\begin{enumerate}
\item Perform final pump/purge and leave at atmospheric pressure.
\item Tag out all valves on CHIME.
\item Open gatevalve, light check.
\item Deploy to centre of vessel.
\item Start run
\end{enumerate}
  


\newpage
{\Large\bf CHIME Deployment Procedure Ver 0.91}
~\\
\begin{tabular}{|l|l|}
\hline
\multicolumn{2}{|l|}{\bf CHIME Depolyment 4 August 2000}\\
\hline
 & \\
Operator(s):~~~~~~~~~~~~~~~~~~~~~~~~~~~~~~~~~~~~ & Date: ~~~~~~~~~~~~~~~~~~~~\\
 & \\
\hline
\end{tabular} 
~\\
\begin{enumerate}
\item\checkbox Note Pressure on CHIME.  It was +3'' Hg last night.
     \begin{center}
     \begin{tabular}{|l|}
     \hline
      \\
     CHIME Pressure:~~~~~~~~~~~~~~~~~~~~~~~~\\
      \\
     \hline
     \end{tabular}
     \end{center}

\item\checkbox If Pressure is at Atmosphere, Pump and Purge CHIME Twice.
  
\item\checkbox Pump out CHIME tube to vacuum.  PG202 should read 30 inches Hg Vac.
  
\item\checkbox Close MV101, MV102, MV201, MV203.
  
\item\checkbox Open MV201.
  
\item\checkbox Verify flush regulator is at 5 PSIG.
 
\item\checkbox Verify flush needle valve is open.
  
\item\checkbox Open MV203.  Wait for PG202 to reach {\bf positive} 2'' Hg.
  
\item\checkbox Close MV203.
  
\item\checkbox Slowly Open MV102.  Wait for PG202 to reach atmosphere (0''Hg).
  
\item\checkbox Close MV201.
  
\item\checkbox Unplug Vacuum Pump (Orange pump in Junction).
  
\item\checkbox Open MV101.  Vent vacuum line to atmosphere.
  
\item\checkbox Close MV101.
  
\item\checkbox Close MV102.

\item\checkbox Close N2 Flush needle valve (in line with flow meter).
  
\item\checkbox Shut off N2 Flush regulator.
  
\item\checkbox Shut off Gas and pressure builder valves on LN2 dewar in junction.
  
\item\checkbox At this point the state of the system should be:
  \begin{itemize}
  \item All valves closed.
  \item CHIME filled with Nitrogen at atmospheric pressure.
  \item Vacuum line backfilled with air.
  \item nitrogen dewar shut off.
  \item vacuum pump is unplugged.
  \end{itemize}

\item\checkbox Tag out:
  \begin{itemize}
  \item\checkbox MV203 N$_2$ flush valve (CLOSED).
  \item\checkbox MV201 vacuum valve (CLOSED).
  \item\checkbox MV101 vacuum line (CLOSED).
  \item\checkbox MV102 vacuum line vent valve (CLOSED).
  \item\checkbox VP100 Orange vacuum pump in junction UNPLUGGED.
  \end{itemize}

\item\checkbox Record value of counter in the home position. 
  Note if home contacts
   are closed.
     \begin{center}
     \begin{tabular}{|l|}
     \hline
      \\
     Contacts Closed?:~~~~~~~~~~~~~~~~~~~~~~~~\\
      \\
     \hline
      \\
     Home Position:~~~~~~~~~~~~~~~~~~~~~~~~\\
      \\
     \hline
     \end{tabular}
     \end{center}

\item\checkbox Turn off DCR Lights.  
  
\item\checkbox Open Gate Valve, remove handle, put "OPEN" tag on gate valve.
 
\item\checkbox Tag out 10'' gate valve as OPEN.
 
\item\checkbox Perform Light Check.
  
\item\checkbox Call detector operator.  Inform him/her of intent to  lower CHIME.
 
\item\checkbox Call Water Group.  Inform that the 4.5 l volume CHIME is going in
               water.
  
\item\checkbox Lower CHIME down approximately 3 m
  
\item\checkbox Open glove port on southsided of glove box.

\item\checkbox Check rope tensions by feeling with glove on glovebox.
  
\item\checkbox Lower till CHIME enters water and  load changes.
     Note position indicator where CHIME first enters the water and
     where it is fully submerged.
     \begin{center}
     \begin{tabular}{|l|}
     \hline
      \\
     First Enters Water:~~~~~~~~~~~~~~~~~~~~~~~~\\
      \\
     \hline
      \\
     Fully Submerged:~~~~~~~~~~~~~~~~~~~~~~~~\\
      \\
     \hline
     \end{tabular}
     \end{center}

  
\item\checkbox Move in/out of water to check that change in load can be felt.
  
\item\checkbox Deploy CHIME to centre of detector.  Deploy until
  the rope is fully out.  Note position.
     \begin{center}
     \begin{tabular}{|l|}
     \hline
      \\
     Lowest Position:~~~~~~~~~~~~~~~~~~~~~~~~\\
      \\
     \hline
     \end{tabular}
     \end{center}

\item\checkbox Back up 2 counts on the encoder. Record position.
     \begin{center}
     \begin{tabular}{|l|}
     \hline
      \\
     Final position:~~~~~~~~~~~~~~~~~~~~~~~~\\
      \\
     \hline
     \end{tabular}
     \end{center}
  
\item\checkbox Tag out CHIME winch.
  
\item\checkbox Close glove port.
  
\item\checkbox Place sign on door of DCR stating:
  \begin{verbatim}
                     CAUTION
            10'' GATE VALVE OPEN TO DETECTOR
                  CHIME DEPLOYED.
  \end{verbatim}
\end{enumerate}
  

%-------------------------------------------------------------------------
\newpage
~\\
\vspace*{2in}
\begin{center}
{\bf\Huge CAUTION!}\\
~\\
{\bf\Huge 10'' GATE VALVE OPEN TO}\\
~\\
{\bf\Huge       DETECTOR}\\
~\\
~\\
{\bf\Huge   CHIME DEPLOYED}
\end{center}
~\\
\vspace*{1in}
\begin{tabbing}
aaa\=aaaaaaa\kill
   \>{\bf\huge Operator:} \\
\>  \\
\> \\
    \>{\bf\huge Date:} \\
\end{tabbing}






  

  
%------------------------------------------------------------------------
%------------------------------------------------------------------------
%------------------------------------------------------------------------
\section{Manipulator Camera Procedures}
\shwlabel{secprocCamera}

  These procedures describe assembly and operation of the manipulator
mounted camera.

\begin{table}[htb]
\begin{center}
\begin{tabular}{|l|r|}
\hline
Assembled Weight & ~~~~~~~~~~~~~~~~~~~~~\\
\hline
Volume           & ~~~~~~~~~~~~~~~~~~~~~\\
(including weight and carriage) & \\
\hline
Pivot Centre Offset & \\
\hline
Pivot Bottom Offset & \\
\hline
\end{tabular}
\caption[Calibration Device:Manipulator Camera]
  {Calibration Device: Manipulator Camera
   \shwlabel{CalDevCamera}
  }
\end{center}
\end{table}



\clearpage
\begin{figure}
\begin{center}
\framebox{\Huge\bf Drawing of camera}
%\leavevmode
%\epsfxsize=7in
%\epsfbox{../figures/n16_partial_exploded.ps}
~\\
\caption[Assembly drawing of the manipulator mounted camera and mount]
        {Assembly drawing of the manipulator mounted camera and
         it's mount.
         \shwlabel{figcamera}
        } 
        
\end{center}
\end{figure}




%------------------------------------------------------------
\clearpage

\subsection{Manipulator Camera Assembly Procedure}
\newprocedure{CalProcCameraAssembly}
             {Manipulator Camera Assembly Procedure}
             {Fraser Duncan}{2002/09/26}{1}

\begin{enumerate}

\checkitem If it does not already exist, create a {\bf camera}
  polyaxis object in {\tt polyaxis.dat}.

\checkitem Mount spool piece on weight cylinder

\checkitem Mount carriage on weight cylinder

\checkitem Slide rotating bearing, o-ring and weight cylinder onto umbilical

\checkitem Slide top pressure plate, o-ring, bottom pressure 
           plate and o-ring onto umbilical

\checkitem Slide top cover onto umbilical

\checkitem Tie `Hallinian' knot on umbilical

\checkitem Slide o-rings, pressure plates and spool piece into place

\checkitem Do up associated nuts

\checkitem Slide o-ring and rotating bearing into place

\checkitem Do up screws

\checkitem Secure them with wire

\checkitem Tie central rope to rotating bearing ( standard figure 8 knot )

\checkitem Push camera into housing ( make sure both o-rings are in place )

\checkitem Slide o-ring and top cover over wires

\checkitem Fasten top cover

\checkitem Slide tube over wires

\checkitem Feed wires though bottom hole on can

\checkitem Mount camera housing on can

\checkitem Secure knurled nuts with locking wire

\checkitem Secure tubing with cable ties

\checkitem Do up connection inside can ( make sure o-ring is in place )

\checkitem Test camera:
  \begin{enumerate}
  \checkitem Connect umbilical to camera controller.
  \checkitem Connect camera controller to monitor and VCR.
  \checkitem Turn on camera and attempt to image appropriate object
     below URM.
  \checkitem Record image on video tape.
  \checkitem Play back video tape to verify recording.
  \checkitem Disconnect umbilical from camera.
  \end{enumerate}

\checkitem {\bf Do Not Operate}
  Status tag put dry end of umbilical to prevent connection
  of power to camera lights.

\checkitem Fasten lid on can

\checkitem Measure:
     \begin{center}
     \begin{tabular}{|l|}
     \hline
      \\
     carriage pivot to bottom of camera:~~~~~~~~~~~~~~~~~~~~~~~~\\
      \\
     \hline
      \\
     carriage pivot to camera pivot:~~~~~~~~~~~~~~~~~~~~~~~~\\
      \\
     \hline
     \end{tabular}
     \end{center}
\checkitem Record above distances in log book.

\checkitem Set the pivot-pivot offset in the {\bf camera} polyaxis
  object in {\tt polyaxis.dat}.

\checkitem Set the pivot-bottom distance in the {\bf camera} polyaxis
  object in {\tt polyaxis.dat} to 2cm greater than the above
  pivot-bottom distance.

\checkitem Clean and Inspect everything

\checkitem Mount guide tube cone

\checkitem Measure
     \begin{center}
     \begin{tabular}{|l|}
     \hline
      \\
     top of cone to bottom of camera:~~~~~~~~~~~~~~~~~~~~~~~~\\
      \\
      \\
     \hline
     \end{tabular}
     \end{center}
\checkitem Record above measurement in log book.

\checkitem Clean and inspect everything

\checkitem Adjust the camera to the desired orientation.\\
  {\em This is probably straight down for the first deployment
   to look for the Berkely Blob.}

\checkitem Retract the camera into the URM and calibrate the central
  rope.


\end{enumerate}



{\small
~\\
~\\
\noindent
{\bf Revision History:}\\
\begin{tabular}{llll}
Rev. & Date & Author & Comments\\
0           & 
2002/09/26  & 
Peter Skensved &
\parbox[t]{3.0in}{
  First draft
}\\

1             & 
2002/09/26    & 
Fraser Duncan &
\parbox[t]{3.0in}{
  Slight format changes.  Fleshed out camera test procedure.
}
\end{tabular}
}




%------------------------------------------------------------
\clearpage

\subsection{Manipulator Camera Deployment Down Guidetube \# 1}
\newprocedure{CalProcCameraDeployment}
             {Manipulator Camera Deployment Down Guidetube \# 1}
             {Fraser Duncan}{2002/09/26}{1}


\noindent
  The deployment is like any other source in the guide tube :

Since the clearance bewteen the bottom of the camera and the gate valve
special care must be exercised so as not to decapitate the source ie.
distances in z has to be double checked before the valve is closed.

Special care must be while driving the source in the guide tube since there
is a non-zero chance of getting the source caught especially at the entrance
to the guide tube, at the `knee' and at the valve. An experienced operator
has to be present at all times.


Since we will be driving the camera close to the acrylic vessel we will have to
override the minimum distances set in MANIP. It is important that the 
defaults be re-established after this deployment. This is an `expert only' 
task.


\begin{table}
\begin{center}
\begin{tabular}{|l|r|}
\hline
                  &   Z (cm) \\
\hline
  & \\
Entrance to PSUP  &  734.4  \\
  & \\
\hline
  & \\
Top of AV         &  446.9  \\
  & \\
\hline
\end{tabular}
\caption[Camera deployment down GT1]
  {Z positions necessary for deployment down guidetube \# 1.
   \shwlabel{TabCameraGuidetube}
  }
\end{center}
\end{table}

\noindent
{\bf Prior to deployment:}
\begin{enumerate}
\item Camera was assembled according to procedure \ref{CalProcCameraAssembly}
  in URM.
\item URM is mounted on guide tube.

\item Umbilical is disconnected from the camera controller.  This should
  be indicated by a status tag.

\item Sufficient blank video tapes are available to record the 
  entire deployment.

\end{enumerate}

{\bf 
  This device emitts light and could damage either phototubes
  or the trigger if used in the detector while HV is on.
}

\noindent
{\bf Procedure:}

\begin{enumerate}

\checkitem Verify guidetube gate valve is closed.

\checkitem Flush the URM with nitrogen until O$_2$ level is below
  0.5\%.

\checkitem Calibrate central rope using cross hairs or bottom flange
           ( there should be enough clearance to see the top of the
           cone through the window - if not use `high tension point'
           instead ).

\checkitem Remove {\bf Do Not Operate} status tag from umbilical end.

\checkitem Connect umbilical to camera controller.

\checkitem Turn on camera and attempt to image top of gate valve.  If unable
  to image the gatevalve, investigate before proceeding.

\checkitem Stop SNO detector data run.

\checkitem Ramp down detector HV and turn off HV supplies on all 
  crates.

\checkitem Do HV status from {\em HV Master} and verify all HV supplies
  are off.

\checkitem Do CMOS read on all crates to verify HV is off.

\checkitem Disconnect HV cables from all crates including OWL tubes.

\checkitem Verify that the Analog Light Monitor has HV off.

\checkitem Verify that the Analog Light Monitor has HV power supply
  unplugged.

\checkitem Verify that the Analog Light Monitor has the HV cables unplugged
  from the power supply.

\checkitem Start recording on the VCR.  Use best quality recording
  level.

\checkitem Open the guide tube gate valve.

\checkitem Deploy the source slowly down to just above the water
  surface.  attempt to image the surface of the water, looking
  for any surface film or debri.

\checkitem Deploy camera to 1 m above bottom of guide tube.\\
  ( Z = 834 cm)

\checkitem Attempt to image the opening of the guide tube into the
  the PSUP.  

\checkitem Deploy the camera assembly into the PSUP.
  
\checkitem Deploy camera assembly to 1 m above AV ( Z = 550).\\
  Attempt to image AV.

\checkitem Deploy camera to 50 cm above AV ( Z = 497).\\
  Attempt to image AV.

\checkitem Deploy camera to 30 cm above AV (Z = 477).\\
  Attempt to image AV.

\checkitem Deploy camera to 20 cm above AV (Z = 467).\\
  Attempt to image AV.

\checkitem  Deploy camera to 10 cm above AV (z = 457).\\
   Watch rope and umbilical tensions for indications of 
   contact.

\checkitem If a closer deployment to the AV is required,
  set the maximum speed of the rope and umbilical to 0.5 cm/s.

\checkitem After the imaging of the AV, retract the  camera from the
  detector.

\checkitem Verify the source is above gate valve visually with camera.

\checkitem Close the gate valve.

\checkitem Turn off the camera lights

\checkitem Disconnect the umbilical from the camera controller.

\checkitem Put {\bf Do Not Operate} status tag on the dry end of the
  umbilical.

\checkitem Reconnect HV cables to PMT crates and ramp up the detector.

\end{enumerate}



{\small
~\\
~\\
\noindent
{\bf Revision History:}\\
\begin{tabular}{llll}
Rev. & Date & Author & Comments\\
0           & 
2002/09/26  & 
Peter Skensved &
\parbox[t]{3.0in}{
  First Draft
}\\

1             & 
2002/09/26    & 
Fraser Duncan &
\parbox[t]{3.0in}{
 More detailed procedure.
}
\end{tabular}
}



\newpage
\markright{\standardheader}




  

  
%------------------------------------------------------------------------
%------------------------------------------------------------------------
%------------------------------------------------------------------------
\section{N17 Source Procedures}
\shwlabel{secprocN17}

  These procedures describe the operation of the N17 calibration
source.


%------------------------------------------------------------
\newpage
\subsection{N17 Engineering Run Procedure}
\shwlabel{procn17}~\\
\noindent
\begin{tabular}{|l|l|}
\hline
Version              & 0.9 \\
\hline
Written/Revised by   & Eric B. Norman and Yuen-Dat Chan \\
                     & D. Earle, F. Duncan 2001/04/05 \\
\hline
Date Written/Revised & 2001/03/28\\
\hline
\end{tabular}


%--------------------------------------------------------------------------
\newpage
\subsubsection{Background and ${16}$N}
~\\
\begin{tabular}{|l|l|}
\hline
\multicolumn{2}{|l|}{\bf Background and $^{16}$N Procedure Ver 0.9}\\
\hline
 & \\
Operator:~~~~~~~~~~~~~~~~~~~~~~~~~~~~~~~~~~~~~ & Date: ~~~~~~~~~~~~~~~~~~~~\\
 & \\
\hline
\end{tabular} \\
\begin{enumerate}

\checkitem Set pressure relief valves to 100 PSIA.
 

\checkitem Open and close appropriate valves on gas board so
  that  recirculation pump is in the gas loop and that we really have
  a closed loop for the gas to flow from the gas board to the dt
  generator, out to SNO and then back to the gas board.
  \begin{enumerate}
  \checkitem CO2 bottle valve CLOSED, CO2 regulator OPEN, CO2 line valve OPEN.
  \checkitem He bottle valve CLOSED, He regulator OPEN, He line valve OPEN.
  \checkitem VE3, VD2, VC3 CLOSED.
  \checkitem VC2 OPEN
  \checkitem VB5 CLOSED
  \checkitem VB4 to pump
  \checkitem VB3 to loop
  \checkitem VB2 closed
  \checkitem VC1 to CO2
  \checkitem VD1 to CO2
  \checkitem VF1 to N16
  \checkitem VE2 OFF
  \checkitem VF2 to Retrieval port.
  \checkitem Set CO2 flow controller to full OPEN (should it be fully open?).
  \end{enumerate}

\checkitem Use vacuum pump connected to valve VE3 on gas board
  to pump out the entire gas loop (gas board + transfer lines + decay
  chamber).  To be sure you have really pumped things out well, let pump
  run for 10 minutes with bypass valve VE2 open.  Pump out He and 
  C$^{nat}$O$_{2}$ supply lines by opening valve VB2, solenoid valve
  SV2 and valve VA2 (turn both ways)
  that leads to Helium and the C$^{nat}$O$_{2}$
  bottles.
  
\checkitem Close valve VB2 and valve VE3 and observe pressure on PG1 and
  PG3 to make sure that there are no significant leaks in the system.
  Note the ``zero'' pressure offsets on the gauges (expect of order 0.3 psi).
     \begin{center}
     \begin{tabular}{|l|}
     \hline
      \\
     PG1:~~~~~~~~~~~~~~~~~~~~~~~~\\
      \\
     \hline
      \\
     PG3:~~~~~~~~~~~~~~~~~~~~~~~~\\
      \\
     \hline
     \end{tabular}
     \end{center}

  
\checkitem Open valve VB2.
  
\checkitem Carefully open valve VA2 on  C$^{nat}$O$_{2}$ supply bottle
  to let in approximately 2 psi of  C$^{nat}$O$_{2}$ (above the
  base pressure observed in step 3) into system.
  Note:  you need to have the recirculation pump on and valve VB2
  closed to be sure that you fill the entire loop with gas.  Turn
  pump off to measure static gas pressure.  You can also open valve
  VE2 once or twice to help equalize the pressure in the system,
  but make sure VE2 is closed once you have put in the desired amount
  of gas.  This step may take a while to settle down (as much as 10 minutes).

  {\em Alternate method:  Allow line to fill with recirculation pump
   usually off.}
  
\checkitem Verify VE2 is CLOSED.

\checkitem Record static pressures PG1 and PG3.
     \begin{center}
     \begin{tabular}{|l|}
     \hline
      \\
     PG1:~~~~~~~~~~~~~~~~~~~~~~~~\\
      \\
     \hline
      \\
     PG3:~~~~~~~~~~~~~~~~~~~~~~~~\\
      \\
     \hline
     \end{tabular}
     \end{center}
  
\checkitem Close valve on  C$^{nat}$O$_{2}$ supply bottle.

\checkitem Open valve VB2 and then carefully open valve VA2 to let in
 enough He gas to make total pressure in the system equal approximately
  50 psi static pressure.  Again use recirculation pump with valve VE2
  closed to make sure the gas is really distributed everywhere in loop.
  The pressure reading on PG1 will lag that read on PG3 until gas has
  distributed itself uniformly throughout the system.  So, take your time.
  
  {\em Alternate method:  Leave recirc pump mostly off.  Turn it on
   occasionally to fill the pump gas volume.}
  
\checkitem Record static pressures PG1 and PG3.
     \begin{center}
     \begin{tabular}{|l|}
     \hline
      \\
     PG1:~~~~~~~~~~~~~~~~~~~~~~~~\\
      \\
     \hline
      \\
     PG3:~~~~~~~~~~~~~~~~~~~~~~~~\\
      \\
     \hline
     \end{tabular}
     \end{center}
  
\checkitem Close valve VB2, solenoid valve SV2, and valve VA2.
 
\checkitem Turn on recirculation pump.  Note PG1 and PG3.  
  The inbalance is due to the pump pressure head.
     \begin{center}
     \begin{tabular}{|l|}
     \hline
      \\
     PG1:~~~~~~~~~~~~~~~~~~~~~~~~\\
      \\
     \hline
      \\
     PG3:~~~~~~~~~~~~~~~~~~~~~~~~\\
      \\
     \hline
     \end{tabular}
     \end{center}

\checkitem  During the running period, continue to log the
  PG1 and PG2 pressures.  A drop in these pressures indicate
  gas loss.

\end{enumerate}

Once these steps are done, you are ready to turn on the dt generator
and run $^{16}$N.  With this set of gas pressures, one should get a total
rate of approximately 10 Hz in the decay chamber.  To reduce the rate,
turn down the neutron flux from the dt generator.





%--------------------------------------------------------------------------
\newpage
\subsubsection{Charging Source and Loop with $^{17}$N for the First Time}
~\\
\begin{tabular}{|l|l|}
\hline
\multicolumn{2}{|l|}{\bf $^{17}$N Procedure Ver 0.9}\\
\hline
 & \\
Operator:~~~~~~~~~~~~~~~~~~~~~~~~~~~~~~~~~~~~~ & Date: ~~~~~~~~~~~~~~~~~~~~\\
 & \\
\hline
\end{tabular} \\
\begin{enumerate}

\checkitem Set pressure relief valves to 100 PSIA.
 

\checkitem Open and close appropriate valves on gas board so that
  recirculation pump is in the gas loop and that we really have a closed
  loop for  the gas to flow from the gas board to the dt generator, out
  to SNO, and then back to the gas board.
  \begin{enumerate}
  \checkitem CO2 bottle valve CLOSED, CO2 regulator OPEN, CO2 line valve OPEN.
  \checkitem He bottle valve CLOSED, He regulator OPEN, He line valve OPEN.
  \checkitem VE3, VD2, VC3 all closed
  \checkitem VC2 open
  \checkitem VB5 closed
  \checkitem VB4 to pump
  \checkitem VB3 to loop
  \checkitem VB2 closed
  \checkitem VC1 to CO2
  \checkitem VD1 to CO2
  \checkitem VF1 to N16
  \checkitem VE2 off
  \checkitem VF2 to retreival port
  \checkitem Set CO2 flow controller to full OPEN (should it be fully open?).
  \end{enumerate}
 
\checkitem Use vacuum pump connected to valve VE3 on gas board to pump out
  entire gas loop (gas board + transfer lines + decay chamber).  To
  be sure you have really pumped things out well, let pump run for 10 minutes
  with bypass valve VE2 open.  Pump out He supply line from helium bottle
  by opening valve VB2, solenoid valve SV2, and valve VA2 that leads to
  Helium bottle.
  
\checkitem Close valve VB2 and valve VE3 and observe pressure on PG1 and
  PG3 to make sure that there are no significant leaks in the system.
  
\checkitem Hook up C$^{17}$O$_{2}$ supply bottle to the open tube that
  has a nut and a ferrule on the right-hand side of cross attached to 
  plastic line from VD2.  
  {\bf Note: Make SURE that the valve on the supply bottle is closed
    before you do anything!}
  
\checkitem With valves on supply bottle and recovery bottle closed, pump
  out the short section of plastic line from VD2 to the  C$^{17}$O$_{2}$
  cross.
  
\checkitem Close valve VE3 to vacuum pump.
  
\checkitem Carefully open valve on  C$^{17}$O$_{2}$ supply bottle to
  let in approximately 2 psi of  C$^{17}$O$_{2}$ (above the base pressure
  observed in step 3) into the system.  Note: you need to have
  the recirculation pump on to be sure that you fill the entire loop with
  gas.  Turn pump off to measure static gas pressure.  You can also open
  valve VE2 once or twice to help equalize the pressure in the system,
  but make sure VE2 is closed once you have put in the desired amount of
  gas.   This step may take a while to settle down (as much as 10 minutes).

\checkitem Verify VE2 is CLOSED.
  
\checkitem Record static pressure PG1 and PG3.
     \begin{center}
     \begin{tabular}{|l|}
     \hline
      \\
     PG1:~~~~~~~~~~~~~~~~~~~~~~~~\\
      \\
     \hline
      \\
     PG3:~~~~~~~~~~~~~~~~~~~~~~~~\\
      \\
     \hline
     \end{tabular}
     \end{center}

\checkitem Close valve on  C$^{17}$O$_{2}$ supply bottle.  Make sure
  it is tightly closed!
  
\checkitem Close valve VD2
  
\checkitem Open valve VB2 and then carefully open valve VA2 to let in
  enough He gas to make total pressure in the system equal approximately
  50 psi static pressure.  Again use recirculation pump with valve VE2
  closed to make sure the gas is really distributed everywhere in loop.
  The pressure reading on PG1 will lag that read on PG3 until gas has
  distributed itself uniformly throughout the system.  So, take your time.
  
\checkitem Record static pressures PG1 and PG3.
     \begin{center}
     \begin{tabular}{|l|}
     \hline
      \\
     PG1:~~~~~~~~~~~~~~~~~~~~~~~~\\
      \\
     \hline
      \\
     PG3:~~~~~~~~~~~~~~~~~~~~~~~~\\
      \\
     \hline
     \end{tabular}
     \end{center}
  
\checkitem Close valve VB2, solenoid valve SV2, and valve VA2.
  

\checkitem  During the running period, continue to log the
  PG1 and PG2 pressures.  A drop in these pressures indicate
  gas loss.

\end{enumerate}

Once these steps are done, you are ready to turn on the dt generator 
and run $^{17}$N.  With this set of gas pressures, one should get a
total rate of approximately 10 Hz in the decay chamber.  We estimate
7 Hz will be from $^{16}$N decaysa nd 3 Hz from $^{17}$N decays.  To
reduce the rate, turn down the neutron flux from the dt generator.




%--------------------------------------------------------------------------
\newpage
\subsubsection{Recovery of $^{17}$N}
~\\
\begin{tabular}{|l|l|}
\hline
\multicolumn{2}{|l|}{\bf $^{17}$N Recovery Procedure Ver 0.9}\\
\hline
 & \\
Operator:~~~~~~~~~~~~~~~~~~~~~~~~~~~~~~~~~~~~~ & Date: ~~~~~~~~~~~~~~~~~~~~\\
 & \\
\hline
\end{tabular} \\

\begin{enumerate}

\checkitem Turn off the dt generator.
  
\checkitem Place dewar filled with liquid nitrogen (LN2) around 
  recovery bottle (cover only lower half of recovery bottle with LN2).
 
\checkitem Open valve VD2.
 
\checkitem Open valve on the  C$^{17}$O$_{2}$ recovery bottle.  Look
  for about a 10\% drop in the pressures read on PG1 and PG3 as gas
  flows into recovery bottle.
  
\checkitem Wait approximately 1 minute.
  
\checkitem Isolate recovery bottle by closing valves VF2 and VC2 on gas
  board.
  
\checkitem Open valve VE3 to vacuum pump for approximately 10 seconds.
 This will pump He gas out of recovery bottle but will leave
  frozen  C$^{17}$O$_{2}$ there.
  
\checkitem Close valve VE3.
  
\checkitem Open valves VF2 and VC2.
  
\checkitem Repeat steps 5 through 9 {\bf ten} times.  Note:  each time
  you do this sequence of steps, you should se that the pressure on PG1
  and PG3 go down as you are removing about 1/6 of the helium each time
  you do this.  You will probably have to top off the LN2
  dewar in order to keep lower half of recovery bottle covered with LN2.
  
\checkitem  Close valve on recovery bottle and close valve VD2.
  
\checkitem  Remove LN2 to allow recovery bottle to warm up.
  To speed up the warming, one can place a dewar filled with warm
  water around recovery bottle.  Note: there is a pressure relief
  valve on the recovery bottle that will open if pressure inside
  recovery bottle reaches 150 psi.  This should never happen unless
  gas loop leaked and we trapped O$_{2}$ in recovery bottle.
  
\checkitem If desired, you can then remove the  C$^{17}$O$_{2}$ 
  supply/recovery system from gas board.



\end{enumerate}





  
%------------------------------------------------------------------------
%------------------------------------------------------------------------
%------------------------------------------------------------------------
\subsection{Activation of $^{24}$Na in 
            Salt Phase using the Super Hot Th Source}
\shwlabel{secprocstandard}
 
 
\procedure{ProcSaltActivate}{Fraser Duncan}{2001/08/29}{0.9}

 

 This procedure describes the use of the superhot acrylic encapsulated
Th source to activate the $^{24}$Na in the D2O during salt phase.  The method
of deployment is to place the encapsulated source inside a metal can
attached to a polypropelene stem mounted on the laserball bucket.
The deployment will be with the laserball URM.
The dangers
inherent in this procedure are:
\begin{enumerate}  
\item Failure of the superhot source encapsulation.
\item Loss of the encapsulated superhot source in the D2O volume
\item Damage to the detector triggers due to the excessive rate
     (500kHz) from the superhot source.
\end{enumerate}
The prevention of eventuality (1) is the triple encapsulation of
the source in acrylic.  Eventually (2) is to be prevented by
the intrinsic mechanical redundancy of the source deployment
mechanism.  This redundancy includes:
\begin{itemize}
\item Two independent support mechanisms (the rope and umbilical).
\item Multiple connections between the encapsulated source and the
  source mechanism (multiple screws, each able to take the weight of
  the source).
\end{itemize}
Item (3) is of concern because of the extremely high rate of the
source.  The means of protecting the detector will be to start with
the NHIT triggers off while deploying the source into the detector.
If the rate still becomes too large, the detector will be ramped
down.
 
\noindent
{\bf Personnel:}
\begin{itemize}
\item Calibration Expert:  Aksel Hallin (u/g)
\item Detector Expert: Noel Gagnon (u/g)
\item Trigger Expert: Josh Klein (phone)
\end{itemize}

\noindent
{\bf Required Detector Conditions}
\begin{itemize}
\item No D$_2$O recirculation being done.
\end{itemize}

\noindent
{\bf Notes:}
\begin{itemize}
\item Run type during activation not specified yet.  Will 
  have to be determined  when we see what rates are tolerable.
\item Duration of Activation run to be specified.
\item Type of running after the activation to be specified.  Presumably
  this should be a normal neutrino run but with  UC bit set to
  remind people that this is not for regular analysis.  Or perhaps
  it should still be a calibration source run.
\end{itemize}


\operator

\begin{center}
{\bf Preparation of the source for deployment.}
\end{center}

\begin{enumerate}

  
\checkitem Install the super-hot acrylic source on the peg:\\
  {\em Use rubber or latex gloves for all work.}
  \begin{enumerate}
  \checkitem  Ensure that the source, can, and all parts have been 
    cleaned with the standard cleaning procedure.
  \checkitem  Place the source in the stainless steel can
  \checkitem  With the approved teflon grease, very lightly grease the 
    inside of the can.  Wipe it as clean as possible with a wipe.  Insert 
    the o-ring into the groove in the black delrin universal adapter plug.
  \checkitem Tighten the three stainless steel bolts on the diameter 
      of the can, attaching the delrin piece to the can.
  \checkitem  Ensure that the small o-ring is installed in the ring 
      groove on the top
      of the delrin spacer.  This o-ring should not be greased.
  \checkitem  Ensure that the three stainless steel nuts are in the slots 
    of the delrin piece and insert and tighten the screws that attach this 
    to the bottom of the peg.
  \end{enumerate}
  Before deployment, re-inspect the peg, source, and carriage, looking for any
  missing or loose fasteners or any sign of contamination.
 
\checkitem The source can is secured and wired to the acrylic source peg.
  
\checkitem Inspect the assembled source.  All screws and bolts should be
    tight.  The laserball assembly should be verified.  If there
    is any doubt about the presence of the required o-rings, it should
    be opened for inspection.
  
\checkitem Verify the proper pivot to source position in the manipulator code
    (in polyaxis.dat the acrylic object).
    This will have to be measured.
     \begin{center}
     \begin{tabular}{|l|}
     \hline
      \\
     d(pivot-source):~~~~~~~~~~~~~~~~~~~~~~~~\\
      \\
     \hline
     \end{tabular}
     \end{center}
 \checkitem Measure the weight of the source using the URM rope.  Enter
  this value in {\tt polyaxis.dat}
     \begin{center}
     \begin{tabular}{|l|}
     \hline
      \\
     Acrylic Source Weight:~~~~~~~~~~~~~~~~~~~~~~~~\\
      \\
     \hline
     \end{tabular}
     \end{center}
 \checkitem Verify the CAST bank information on {\bf manip} is up to date
  for the source.
  
\checkitem Run the source up/down in the DCR several times to verify the
    manipulator is functioning correctly.

\checkitem Retract the  source into the source tube.  The pivot should be 
    visible from the window on the source tube.
  
\checkitem Insert the source clamps to secure the source in the source tube.
 
\checkitem Mount the URM on the glovebox.
  
\checkitem Retract the source clamps.
 

\item\checkbox Verify that the LN$_2$ dewar in the junction is
  at least 1/4 full.  If not, swap it out with another dewar.
  Record liquid level of Dewar,
     \begin{center}
     \begin{tabular}{|l|}
     \hline
      \\
     LN$_2$ Level:~~~~~~~~~~~~~~~~~~~~~~~~\\
      \\
     \hline
     \end{tabular}
     \end{center}

\item\checkbox Verify that the dewar gas pressure is approximately
  130 to 150 psig. If not, swap it out with another dewar.

\item\checkbox Turn on N2 Flow to DCR from dewar at junction 
  (Marked {\bf Gas Use} on dewar).
     \begin{center}
     \begin{tabular}{|l|}
     \hline
      \\
     Note Time:~~~~~~~~~~~~~~~~~~~~~~~~\\
      \\
     \hline
     \end{tabular}
     \end{center}

\item\checkbox Turn on pressure builder valve (Marked {\bf Pressure Builder} 
  on dewar).\\
  %------------------------
  \small
  {\em The pressure builder valve opens a controlled leak on the dewar
       to maintain the 150 psi pressure head.  If the valve is not
       opened, the gas pressure to the laser and URM will eventually
       drop below the operating level.}
  \normalsize
  %------------------------

  
\item\checkbox Check that flush return line is connected to
  URM1.  If not, connect it.\\
  %------------------------
  \small
  {\em It may be necessary to move it over to URM1 from URM3 (the laserball).
  }
  \normalsize
  %------------------------

\item\checkbox Set URM1 flush regulator at 40 psig.

\item\checkbox Open URM1 flush valve.  Flow meter should be railed.
     The sound of the gas flowing into the URM should be apparent within
     a half metre of the URM source tube.
     \begin{center}
     \begin{tabular}{|l|}
     \hline
      \\
     Note Time:~~~~~~~~~~~~~~~~~~~~~~~~\\
      \\
     \hline
     \end{tabular}
     \end{center}

   Flush should continue for at least 1 hour.
   {\bf Do NOT use the water group's O$_2$ meter.}

  
\item\checkbox Check that the source clamps are in the RELEASE position.  
   The RELEASE position for URM2 is for the knobs on the side of the source
   tube to be fully OUT.  {\bf There are two knobs.  Check BOTH.}\\
   {\bf 
     WARNING:  If the source is moved with the clamps in, the source
     may be damaged!
   }
  %--------------------------------
  \small
  {\em
   The clamps are used to secure the source while the URM is being moved
   on and off the glovebox.  If the source is moved with the clamps in
   the hold position, it will most likely foul in the clamps and
   require disassembly of the URM to extract.
  }
  \normalsize
  %--------------------------------
 
\item\checkbox Check gas pressure on URM pressure cylinder = 45 psig.\\
   {\bf 
   IF PRESSURE IS LESS THAN 10 PSIG DO NOT OPERATE MANIPULATOR
   AND CONTACT EXPERT.
   }\\
   %-------------------------------
   \small
   {\em
     The pressure cylinder on the URM maintains tension on the umbilical
     takeup reel.  A low gas pressure can result in the umbilical falling
     off the takeup reel and resulting in tangling and damage of the
     umbilical.
   }
   \normalsize
   %--------------------------------
  
   
\item\checkbox Verify that Gate Valve 1 is locked in the  closed position.\\
   %-------------------------------
   \small
   {\em
     If the handle is on the gate valve, CLOSED is when the  handle points to 
     the left when facing the source tube.  If the handle is not on the valve
     then the slot on the handle stem points AWAY from the source tube when
     the valve is closed.
   }
   \normalsize
   %--------------------------------
 
\item \checkbox Calibrate Central Rope Length\\
      (see procedure  \ref{seccalcentre} 
       {\em Central Rope Position Calibration}).
      Record changes in length of central rope and umbilical,
      The fiducial mark for the wide bottomed 4'' source tube on Gate Valve 1
      is
      \[
               z_{mark} = 1559.9
      \]
     \begin{center}
     \begin{tabular}{|l|}
     \hline
      \\
     $\Delta$l rope:~~~~~~~~~~~~~~~~~~~~~~~~\\
      \\
     \hline
      \\
     $\Delta$l umbilical:~~~~~~~~~~~~~~~~~~~~~~~~\\
      \\
     \hline
     \end{tabular}
     \end{center}


\item\checkbox Check that all seals are in place on URM.  Including:
   \begin{itemize}
      \item\checkbox view port window cover on source tube
      \item\checkbox window on front URM hood
      \item\checkbox window on back URM hood
      \item\checkbox umbilical feedthrough on URM
      \item\checkbox flush inlet line.
      \item\checkbox flush outlet line.
   \end{itemize}


\item\checkbox At end of URM flush, turn regulator down to 5 psig.
   Check that the flow meter is railed at 50.
   %--------------------------
   \small
   {\em
     The regulator only is marked down to 10 psig.  To set it to 5 psig,
     set the clear plastic indicator to half way between 10 psig and 0.
   }
   \normalsize
   %--------------------------
  

   
%--------------------------------------------------------------
\begin{center}
            {\bf Deploying Source from Source Tube Into Glovebox}
\end{center}

 \item\checkbox Verify the 40psi flush of the URM has been at least 1 hour.

 \item\checkbox Verify that the URM flush has been turned down to 5 psig.

 \item\checkbox Turn off DCR lights.

 \item\checkbox Verify Owl light monitor is on.  Establish communications
  with person watching light monitor.
  %-------------------------
  \small
  {\em
    Suggestion:  Station the person watching the OWL monitor at
    the Deck Mac.  Then he/she can shout through the  wall of the
    DCR and you don't need to use the phones which slow communications
    down.
  }
  \normalsize
  %-------------------------

 \item\checkbox Open gate valve.

 \item\checkbox Lock gate valve open.

 \item\checkbox With flashlight perform light leak check on URM.  In particular
   check the seal of the source tube window.

 \item\checkbox Using the dimmer switch, slowly bring up breaker 9 lights in
   the DCR.  Person still watching owl monitor.

 \item\checkbox DAQ is connected to the {\bf manip} computer.

 \item \checkbox In DAQ, source type is set to {\bf ACRYLIC}.

 \item\checkbox DAQ is in a {\bf source transitional run}.

 \item \checkbox Verify that {\bf manip\_logger} on {\bf polaris}
                 is running and logging the acrylic source.

 \item\checkbox Check movement of acrylic source down:
  \begin{center}
  \begin{tabular}{|l|l|}
  \hline
  console & {\tt manip$>$ acrylic by 0 0 -5} \\
  \hline
  manmon  & in acrylic window: \\
          & set x = 0, y = 0, z= -5\\
          & click on {\bf move by} \\
  \hline
  \end{tabular}
  \end{center}
  %--------------------
  \small
  {\em 
    The acrylic source should move down 5 cm.  The tension on the rope
    should be 60-90 N.  The tension on the umbilical should be
    5-30N.
  }
  \normalsize
  %--------------------

 \item\checkbox Deploy source into the glovebox:
  \begin{center}
  \begin{tabular}{|l|l|}
  \hline
  console & {\tt manip$>$ acrylic to 0 0 1370} \\
  \hline
  manmon  & in acrylic window: \\
          & set x = 0, y = 0, z= 1370\\
          & click on {\bf move to} \\
  \hline
  \end{tabular}
  \end{center}



%-------------------------------------------------------------
\begin{center}
  {\bf Deploying Source to Centre of 
            Detector from Glovebox}
\end{center}
\shwlabel{sectocentre}
 
 \item\checkbox Contact Water Supervisor and advise him/her that the source is
   being lowered into the D2O.  \\
   %--------------------
   \small
   {\em
     The water group maintains a very small differential pressure
     between the light and heavy water.  The volume of the source
     is enough to disrupt this differential pressure.
   }
   \normalsize
   %---------------------

 \item\checkbox Check tensions on urm1rope and urm1umbilical.  Rope tension
   should be approximately 60-80 N.  Umbilical tension should
   be between 10-30 N.

 \checkitem {\bf Contact Josh Klein by phone at Penn}
 \checkitem {\bf Turn off the NHIT Triggers}
  
 \item\checkbox Move acrylic source base of Neck.
  \begin{center}
  \begin{tabular}{|l|l|}
  \hline
  console & {\tt manip$>$ acryic to 0 0 600} \\
  \hline
  manmon  & in acrylic window: \\
          & click on {\bf Position the source}\\
          & set x = 0, y = 0, z= 600\\
          & click on {\bf move to} \\
  \hline
  \end{tabular}
  \end{center}
  {\bf While moving the source, monitor the ESUM triggers on the detector
  looking for excessive rates.}  {\em This will be complicated by
  the presence of manipulite.  Therefore it may be wise to stop every
  couple of meters so that a measure of the ESUM rate without the
  background manipulite can be made.} 

  {\bf At this point decide if the detector is acceptably stable to
  run or if it is necessary to shut it down.}
  
 \item\checkbox Move acrylic source to centre of detector.
  \begin{center}
  \begin{tabular}{|l|l|}
  \hline
  console & {\tt manip$>$ acryic to 0 0 0} \\
  \hline
  manmon  & in acrylic window: \\
          & click on {\bf Position the source}\\
          & set x = 0, y = 0, z= 0\\
          & click on {\bf move to} \\
  \hline
  \end{tabular}
  \end{center}
  {\bf While moving the source, monitor the ESUM triggers on the detector
  looking for excessive rates.}  {\em This will be complicated by
  the presence of manipulite.  Therefore it may be wise to stop every
  couple of meters so that a measure of the ESUM rate without the
  background manipulite can be made.} 

\checkitem Attempt to turn on the NHIT triggers:
  \begin{itemize}
  \item Set NHIT threshold to 100.
  \item Do an {\bf Enable Triggers} from the Standard Runs window while
    watching the trigger rates.  If they are not excessive, 
    lower the trigger threshold till it is no more than 100 Hz.
  \end{itemize}


%-------------------------------------------------------------
\begin{center}
  {\bf Retracting Source from Detector}
\end{center}

\checkitem Go to {\bf source transitional run}
\checkitem Turn off NHIT triggers (using the {\bf Disable PMTs} button).
\checkitem Retract the source to the glovebox.
  \begin{center}
  \begin{tabular}{|l|l|}
  \hline
  console & {\tt manip$>$ acryic to 0 0 1300} \\
  \hline
  manmon  & in acrylic window: \\
          & click on {\bf Position the pivot}\\
          & set x = 0, y = 0, z= 1300\\
          & click on {\bf move to} \\
  \hline
  \end{tabular}
  \end{center}

\checkitem Retract source into source tube:
  \begin{center}
  \begin{tabular}{|l|l|}
  \hline
  console & {\tt manip$>$ acryic to 0 0 1530} \\
  \hline
  \end{tabular}
  \end{center}

\checkitem Retract source to home position:
  \begin{center}
  \begin{tabular}{|l|l|}
  \hline
  console & {\tt manip$>$ acryic to 0 0 1550} \\
  \hline
  \end{tabular}
  \end{center}

\checkitem Reach into one of the south glove ports and verify that
  the source is above the gate valve.

\checkitem Close and lock gate valve.
 \item\checkbox Close the URM flush valve.
\item\checkbox Turn off the URM flush regulator.
\item\checkbox IF the laser is off,
   turn off gas flow at the LN$_2$ dewar in the junction:
   \begin{enumerate}
   \item close {\bf Gas Use} valve
   \item close {\bf Pressure Building} valve
   \end{enumerate}

%-------------------------------------------------------------
\begin{center}
           {\bf After Calibration}
\end{center}
\item\checkbox Source is above gate valve.
\item\checkbox Gate valve is closed and locked.
\item\checkbox LN$_2$ dewar is turned off (both {\bf Gas Use} valve and 
  {\bf Pressure Building} valve).
\end{enumerate}

 


  \include{calproc_saltprobe}



%============================================================================
%============================================================================
%============================================================================
%============================================================================
 
\end{document}





