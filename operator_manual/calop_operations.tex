
\markright{CalOp: Controls}
  
%------------------------------------------------------------------------
%------------------------------------------------------------------------
%------------------------------------------------------------------------
\chapter{Controls}
\shwlabel{ChapterControls}

  Control of the calibration manipulator and the calibration laser
is done through the {\bf manip} computer running the {\bf manip} program
which is a C++ program running under DOS.  There are three ways for
the operator to send commands to {\bf manip}:
\begin{itemize}
\item From the {\bf manip} console in the DCR
\item From the SHaRC Detector user interface.
\item From the {\bf manmon} program runing on a number of
  Linux or Unix boxes.
\end{itemize}
When changing sources or adjusting equipment in the DCR, the console
is usually used.  Otherwise the calibration operator usually interacts
with the manipulator through {\bf manmon}.  The manipulator can be 
operated both from underground or from surface without any personnel in
the lab.  There is also a 
monitoring program running on {\bf crag1} (Surface control room workstation).
This program called {\bf manip\_logger} generates a web page where
the current manipulator status can be viewed and log files of the commands
and errors can be downloaded.  


\section{Controlling and Monitoring the Manipulator From manmon}

  Normal operation of the manipulator is through {\bf manmon} which
is a Tcl/Tk program which runs on various Linux workstations on site.
Specifically, the workstations
  \begin{description}
  \item[ crug1, crug2, alcor ] The workstations in the underground control room.
  \item[ crag1, crag2 ] The workstations in the surface control room.
  \end{description}
{\bf manmon} provides a GUI interface to display the manipulator data
and to allow the Calibration Operator to move sources or operate the
calibration laser.


\begin{figure}[htb]
\begin{center}
\leavevmode
%\epsfysize=0.85\textheight
\epsfxsize=5.0in
\epsfbox{figures/f_manmon.ps}
~\\
\end{center}
\caption[manmon main window] 
        {manmon main window
          \shwlabel{figmanmon}
        }
\end{figure}
  
\subsection{Starting manmon}
\begin{enumerate}
\item Log on to {\tt alcor} as yourself ( or as user {\tt operator} ).

\item Go to the current manmon directory,
  \begin{verbatim}
      cd ~manipulator/manmon
  \end{verbatim}
  The directory {\tt manmon} is a softlink to the current version
  of the manmon code.
  
\item start the {\bf manmon} program by typing
  \begin{verbatim}
       manmon
  \end{verbatim}
  A GUI user interface will pop up (figure \ref{figmanmon}).
  
\item On the control room computers ( {\tt crag}/{\tt crug } ) log on as
user {\tt operator} .

\item Go to the current manmon directory,
\begin{verbatim}
      cd ~calibrator/manmon
\end{verbatim}




\item Connect to the manipulator computer by clicking on the {\bf connect}
  button.  A window will pop up (see figure \ref{figconnection}) 
  asking if you wish to make this a
  control connection (to operate the manipulator and laser) or a monitor
  connection (to view but not affect the calibration system).
  \begin{figure}[h]
  \begin{center}
  \leavevmode
  %\epsfysize=0.85\textheight
  \epsfxsize=2in
  \epsfbox{figures/f_connection.ps}
  \end{center}
  \caption[manmon main window] 
        {manmon connection dialog
          \shwlabel{figconnection}
        }
  \end{figure}
  The default is to make a monitor only connection.  To make this a 
  control connection, click on the {\bf monitor only} button to deselect
  it.  The window will disappear and the connection will be attempted.
  If the connection is accepted properly, the message 
  \begin{verbatim}
         Connection Accepted
  \end{verbatim}
  will appear in the {\bf from server:} window and in the {\bf Connections}
  panel of the display (top right) a line indicating the connection will
  appear.  This line contains the {\em ID} of the process, the {\em Address}
  of the machine that {\bf manmon} was started from and the {\em Idle Time}.
  Because {\bf manmon} is constantly polling the manipulator the idle time
  should always be zero.  If a different process connected to the
  manipulator computer becomes hung, then that process's idle time
  will continually grow.
  
\end{enumerate}
 



%---------------------------------------------------
  
\newpage
\section{Controlling the Manipulator from the Console}
  
Change to directory
\begin{verbatim}
   c:\motors\manip\
\end{verbatim}
and run program 
\begin{verbatim}
   manrun
\end{verbatim}
(this runs a batch file which runs manip ).  Basic Commands are :
\begin{description}
\item[help]~
\item[list] lists all objects
\item[logout] disconnects tcp/ip connection
\item[quit] exits program
\end{description}
Commands for the {\tt prototype} object ( typically a source objcet like {\tt laserball}, {\tt acrylic}, {\tt n16} etc. ) 
\begin{description}
\item[{\tt prototype} by $<x>$ $<y>$ $<z>$] move manipulator amout x y z in 
  manipulator space.
\item[{\tt prototype} reset] resets encoders to agree with motor position and tension
\item[{\tt prototype} locate $<x>$ $<y>$ $<z>$] defines the position to be x y z.
\item[{\tt prototype} to $<x>$ $<y>$ $<z>$] moves to absolute position
\end{description}
A sample session:
\begin{verbatim}
motors> prototype reset              -- resets encoders to agree with motor position
                                        and tension

motors> prototype by  35 0 -40       -- moves differential amount in x,y,z

motors> prototype locate 0 0 92.8    -- defines the current position to be
                                        (0,0,92.8)
  
motors> prototype to  -180.0 0 32.3  -- moves to absolute position
\end{verbatim}
  

A session in which the central rope was originally fully spooled and is
unspooled, threaded through the pullies and attached to the carriage.
\begin{verbatim}
motors> centralrope reset              -- to reset the encoder on the central rope
  
motors> centralropemotor setcruise 2   -- change from maximum 4cm/c to 2cm/s
                                          so it feeds off the drum slower.
  
motors> centralrope down 10000         -- feed out string
  
motors> stop                           -- stop when you have enough
  
motors> centralropemotor setcruise 4   -- change speed back to max.
\end{verbatim}  
  
A session where the {\tt prototype} is disconnected and reconnected.
\begin{verbatim}
motors> prototype disconnect
motors> prototype connect eastrope centralrope westrope
motors> show prototype
\end{verbatim}
Note that the {\tt show} command has a different syntax.  This is because
it is a system wide command unlike the others which are object commands.
  

  
  
%----------------------------------------------------------------------
\newpage
\section{Running the Manipulator from SHaRC}
  
You can run the manipulator through the data acquisition program.  First
make sure that the manipulator program is running on the PC. Then start
the program by clicking on :
\begin{verbatim}
  Windows -> Configuration 
\end{verbatim}
Then `Splat-Click' on MAN-1 to get a menu :
\begin{verbatim}
  Configure
  Basic Ops
  Special Ops
  \end{verbatim}
  select {\tt Configure}

\begin{enumerate}
\item Select {\tt Basic Ops} and a window with the manipulator status
  will pop up.  This window has places to input commands to the manipulator
  and readback.
\item You can get a graphical display of the AV and the manipulator
  position from the {\tt Special Ops} option.

%\item To shut down the GUI interface to the manipulator,
%  \begin{enumerate}
%  \item shut down the DAQ.  MAC will wait forever trying to close the TCP/IP
%    connection.
%  \item close the TCP/IP connection from the PC by using either the 
%    {\tt logout} command or by quitting.
%  \end{enumerate}
\end{enumerate}

  



%--------------------------------------------------------------
%--------------------------------------------------------------
 %--------------------------------------------------------------
\section{Data Logging}
\shwlabel{SecDataLogging}
  
  There are several means by which data from the manipulator
system are logged.
\begin{itemize}
\item {\bf manip\_logger} web based logging tool
\item CMA logging of AV rope lengths
\item Command and Error logging on {\bf manip}
\item Explicit logging of source data through {\bf manmon}
\end{itemize}
   

\subsection{manip\_logger web based data logging}

 At preset the {\bf manip\_logger } is running on crag1. The output is stored in
~detector/public\_html/private/manip\_logger . The polling can be configured in
the file ``extra\_log.dat''  . A ``monitor'' argument should be added for frequently
polled commands to cut down the amount of noise in the log file(s).
An example file follows.

\begin{verbatim}

%Here is my documentation:

%--------------------------------------------------------------- 

This is the current "extra_log.dat" file:

salt {
state = off;
delay = 60; // delay time (secs)
entry = "TESTROPELOADCELL", "Tension:", 1;
entry = "LASERBALL", "Position:", 3;
}

You can add up to 20 entries similar to this one.

To enable the above polling, simply edit this file and change
the state from "off" to "on".

This entry produces a log file called "salt_YYMM.log2" with
the following format:

2001 7 17 12 49 43 94.12 9.00 -25.00 0.00
2001 7 17 12 50 41 94.12 9.00 -25.00 0.00
2001 7 17 12 51 41 94.12 9.00 -25.00 0.00
2001 7 17 12 52 41 94.12 9.00 -25.00 0.00

The first 6 lines are year, month, day, hour, minute, second,
as with the other log2 files.  The next lines are those
specified by the "entry" lines in "extra_log.dat".  Here,
it has logged the testropeloadcell tension and the laserball
position.

Here are more details about the format of "extra_log.dat":

state - "on" or "off" to enable or disable logging for this item.

delay - nominal delay between log entries in seconds (actual
delay implemented is the nearest multiple of 10 seconds).

entry - The first argument is the object to poll (case
insensitive).  The 2nd argument is the case-sensitive string
to match the specific parameter of the "monitor" command to
be logged.  This must be an exact match of all characters
up to and including the colon.  The 3rd argument is the number
of values to log for this parameter (ie.  x,y,z position is
3 values).

The manip_logger checks every logging cycle (10 seconds) to
see if this file has been modified.  If it has, the new
settings are loaded.

\end{verbatim}

