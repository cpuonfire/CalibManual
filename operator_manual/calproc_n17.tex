

  
%------------------------------------------------------------------------
%------------------------------------------------------------------------
%------------------------------------------------------------------------
\section{N17 Source Procedures}
\shwlabel{secprocN17}

  These procedures describe the operation of the N17 calibration
source.


%------------------------------------------------------------
\newpage
\subsection{N17 Engineering Run Procedure}
\shwlabel{procn17}~\\
\noindent
\begin{tabular}{|l|l|}
\hline
Version              & 0.9 \\
\hline
Written/Revised by   & Eric B. Norman and Yuen-Dat Chan \\
                     & D. Earle, F. Duncan 2001/04/05 \\
\hline
Date Written/Revised & 2001/03/28\\
\hline
\end{tabular}


%--------------------------------------------------------------------------
\newpage
\subsubsection{Background and ${16}$N}
~\\
\begin{tabular}{|l|l|}
\hline
\multicolumn{2}{|l|}{\bf Background and $^{16}$N Procedure Ver 0.9}\\
\hline
 & \\
Operator:~~~~~~~~~~~~~~~~~~~~~~~~~~~~~~~~~~~~~ & Date: ~~~~~~~~~~~~~~~~~~~~\\
 & \\
\hline
\end{tabular} \\
\begin{enumerate}

\checkitem Set pressure relief valves to 100 PSIA.
 

\checkitem Open and close appropriate valves on gas board so
  that  recirculation pump is in the gas loop and that we really have
  a closed loop for the gas to flow from the gas board to the dt
  generator, out to SNO and then back to the gas board.
  \begin{enumerate}
  \checkitem CO2 bottle valve CLOSED, CO2 regulator OPEN, CO2 line valve OPEN.
  \checkitem He bottle valve CLOSED, He regulator OPEN, He line valve OPEN.
  \checkitem VE3, VD2, VC3 CLOSED.
  \checkitem VC2 OPEN
  \checkitem VB5 CLOSED
  \checkitem VB4 to pump
  \checkitem VB3 to loop
  \checkitem VB2 closed
  \checkitem VC1 to CO2
  \checkitem VD1 to CO2
  \checkitem VF1 to N16
  \checkitem VE2 OFF
  \checkitem VF2 to Retrieval port.
  \checkitem Set CO2 flow controller to full OPEN (should it be fully open?).
  \end{enumerate}

\checkitem Use vacuum pump connected to valve VE3 on gas board
  to pump out the entire gas loop (gas board + transfer lines + decay
  chamber).  To be sure you have really pumped things out well, let pump
  run for 10 minutes with bypass valve VE2 open.  Pump out He and 
  C$^{nat}$O$_{2}$ supply lines by opening valve VB2, solenoid valve
  SV2 and valve VA2 (turn both ways)
  that leads to Helium and the C$^{nat}$O$_{2}$
  bottles.
  
\checkitem Close valve VB2 and valve VE3 and observe pressure on PG1 and
  PG3 to make sure that there are no significant leaks in the system.
  Note the ``zero'' pressure offsets on the gauges (expect of order 0.3 psi).
     \begin{center}
     \begin{tabular}{|l|}
     \hline
      \\
     PG1:~~~~~~~~~~~~~~~~~~~~~~~~\\
      \\
     \hline
      \\
     PG3:~~~~~~~~~~~~~~~~~~~~~~~~\\
      \\
     \hline
     \end{tabular}
     \end{center}

  
\checkitem Open valve VB2.
  
\checkitem Carefully open valve VA2 on  C$^{nat}$O$_{2}$ supply bottle
  to let in approximately 2 psi of  C$^{nat}$O$_{2}$ (above the
  base pressure observed in step 3) into system.
  Note:  you need to have the recirculation pump on and valve VB2
  closed to be sure that you fill the entire loop with gas.  Turn
  pump off to measure static gas pressure.  You can also open valve
  VE2 once or twice to help equalize the pressure in the system,
  but make sure VE2 is closed once you have put in the desired amount
  of gas.  This step may take a while to settle down (as much as 10 minutes).

  {\em Alternate method:  Allow line to fill with recirculation pump
   usually off.}
  
\checkitem Verify VE2 is CLOSED.

\checkitem Record static pressures PG1 and PG3.
     \begin{center}
     \begin{tabular}{|l|}
     \hline
      \\
     PG1:~~~~~~~~~~~~~~~~~~~~~~~~\\
      \\
     \hline
      \\
     PG3:~~~~~~~~~~~~~~~~~~~~~~~~\\
      \\
     \hline
     \end{tabular}
     \end{center}
  
\checkitem Close valve on  C$^{nat}$O$_{2}$ supply bottle.

\checkitem Open valve VB2 and then carefully open valve VA2 to let in
 enough He gas to make total pressure in the system equal approximately
  50 psi static pressure.  Again use recirculation pump with valve VE2
  closed to make sure the gas is really distributed everywhere in loop.
  The pressure reading on PG1 will lag that read on PG3 until gas has
  distributed itself uniformly throughout the system.  So, take your time.
  
  {\em Alternate method:  Leave recirc pump mostly off.  Turn it on
   occasionally to fill the pump gas volume.}
  
\checkitem Record static pressures PG1 and PG3.
     \begin{center}
     \begin{tabular}{|l|}
     \hline
      \\
     PG1:~~~~~~~~~~~~~~~~~~~~~~~~\\
      \\
     \hline
      \\
     PG3:~~~~~~~~~~~~~~~~~~~~~~~~\\
      \\
     \hline
     \end{tabular}
     \end{center}
  
\checkitem Close valve VB2, solenoid valve SV2, and valve VA2.
 
\checkitem Turn on recirculation pump.  Note PG1 and PG3.  
  The inbalance is due to the pump pressure head.
     \begin{center}
     \begin{tabular}{|l|}
     \hline
      \\
     PG1:~~~~~~~~~~~~~~~~~~~~~~~~\\
      \\
     \hline
      \\
     PG3:~~~~~~~~~~~~~~~~~~~~~~~~\\
      \\
     \hline
     \end{tabular}
     \end{center}

\checkitem  During the running period, continue to log the
  PG1 and PG2 pressures.  A drop in these pressures indicate
  gas loss.

\end{enumerate}

Once these steps are done, you are ready to turn on the dt generator
and run $^{16}$N.  With this set of gas pressures, one should get a total
rate of approximately 10 Hz in the decay chamber.  To reduce the rate,
turn down the neutron flux from the dt generator.





%--------------------------------------------------------------------------
\newpage
\subsubsection{Charging Source and Loop with $^{17}$N for the First Time}
~\\
\begin{tabular}{|l|l|}
\hline
\multicolumn{2}{|l|}{\bf $^{17}$N Procedure Ver 0.9}\\
\hline
 & \\
Operator:~~~~~~~~~~~~~~~~~~~~~~~~~~~~~~~~~~~~~ & Date: ~~~~~~~~~~~~~~~~~~~~\\
 & \\
\hline
\end{tabular} \\
\begin{enumerate}

\checkitem Set pressure relief valves to 100 PSIA.
 

\checkitem Open and close appropriate valves on gas board so that
  recirculation pump is in the gas loop and that we really have a closed
  loop for  the gas to flow from the gas board to the dt generator, out
  to SNO, and then back to the gas board.
  \begin{enumerate}
  \checkitem CO2 bottle valve CLOSED, CO2 regulator OPEN, CO2 line valve OPEN.
  \checkitem He bottle valve CLOSED, He regulator OPEN, He line valve OPEN.
  \checkitem VE3, VD2, VC3 all closed
  \checkitem VC2 open
  \checkitem VB5 closed
  \checkitem VB4 to pump
  \checkitem VB3 to loop
  \checkitem VB2 closed
  \checkitem VC1 to CO2
  \checkitem VD1 to CO2
  \checkitem VF1 to N16
  \checkitem VE2 off
  \checkitem VF2 to retreival port
  \checkitem Set CO2 flow controller to full OPEN (should it be fully open?).
  \end{enumerate}
 
\checkitem Use vacuum pump connected to valve VE3 on gas board to pump out
  entire gas loop (gas board + transfer lines + decay chamber).  To
  be sure you have really pumped things out well, let pump run for 10 minutes
  with bypass valve VE2 open.  Pump out He supply line from helium bottle
  by opening valve VB2, solenoid valve SV2, and valve VA2 that leads to
  Helium bottle.
  
\checkitem Close valve VB2 and valve VE3 and observe pressure on PG1 and
  PG3 to make sure that there are no significant leaks in the system.
  
\checkitem Hook up C$^{17}$O$_{2}$ supply bottle to the open tube that
  has a nut and a ferrule on the right-hand side of cross attached to 
  plastic line from VD2.  
  {\bf Note: Make SURE that the valve on the supply bottle is closed
    before you do anything!}
  
\checkitem With valves on supply bottle and recovery bottle closed, pump
  out the short section of plastic line from VD2 to the  C$^{17}$O$_{2}$
  cross.
  
\checkitem Close valve VE3 to vacuum pump.
  
\checkitem Carefully open valve on  C$^{17}$O$_{2}$ supply bottle to
  let in approximately 2 psi of  C$^{17}$O$_{2}$ (above the base pressure
  observed in step 3) into the system.  Note: you need to have
  the recirculation pump on to be sure that you fill the entire loop with
  gas.  Turn pump off to measure static gas pressure.  You can also open
  valve VE2 once or twice to help equalize the pressure in the system,
  but make sure VE2 is closed once you have put in the desired amount of
  gas.   This step may take a while to settle down (as much as 10 minutes).

\checkitem Verify VE2 is CLOSED.
  
\checkitem Record static pressure PG1 and PG3.
     \begin{center}
     \begin{tabular}{|l|}
     \hline
      \\
     PG1:~~~~~~~~~~~~~~~~~~~~~~~~\\
      \\
     \hline
      \\
     PG3:~~~~~~~~~~~~~~~~~~~~~~~~\\
      \\
     \hline
     \end{tabular}
     \end{center}

\checkitem Close valve on  C$^{17}$O$_{2}$ supply bottle.  Make sure
  it is tightly closed!
  
\checkitem Close valve VD2
  
\checkitem Open valve VB2 and then carefully open valve VA2 to let in
  enough He gas to make total pressure in the system equal approximately
  50 psi static pressure.  Again use recirculation pump with valve VE2
  closed to make sure the gas is really distributed everywhere in loop.
  The pressure reading on PG1 will lag that read on PG3 until gas has
  distributed itself uniformly throughout the system.  So, take your time.
  
\checkitem Record static pressures PG1 and PG3.
     \begin{center}
     \begin{tabular}{|l|}
     \hline
      \\
     PG1:~~~~~~~~~~~~~~~~~~~~~~~~\\
      \\
     \hline
      \\
     PG3:~~~~~~~~~~~~~~~~~~~~~~~~\\
      \\
     \hline
     \end{tabular}
     \end{center}
  
\checkitem Close valve VB2, solenoid valve SV2, and valve VA2.
  

\checkitem  During the running period, continue to log the
  PG1 and PG2 pressures.  A drop in these pressures indicate
  gas loss.

\end{enumerate}

Once these steps are done, you are ready to turn on the dt generator 
and run $^{17}$N.  With this set of gas pressures, one should get a
total rate of approximately 10 Hz in the decay chamber.  We estimate
7 Hz will be from $^{16}$N decaysa nd 3 Hz from $^{17}$N decays.  To
reduce the rate, turn down the neutron flux from the dt generator.




%--------------------------------------------------------------------------
\newpage
\subsubsection{Recovery of $^{17}$N}
~\\
\begin{tabular}{|l|l|}
\hline
\multicolumn{2}{|l|}{\bf $^{17}$N Recovery Procedure Ver 0.9}\\
\hline
 & \\
Operator:~~~~~~~~~~~~~~~~~~~~~~~~~~~~~~~~~~~~~ & Date: ~~~~~~~~~~~~~~~~~~~~\\
 & \\
\hline
\end{tabular} \\

\begin{enumerate}

\checkitem Turn off the dt generator.
  
\checkitem Place dewar filled with liquid nitrogen (LN2) around 
  recovery bottle (cover only lower half of recovery bottle with LN2).
 
\checkitem Open valve VD2.
 
\checkitem Open valve on the  C$^{17}$O$_{2}$ recovery bottle.  Look
  for about a 10\% drop in the pressures read on PG1 and PG3 as gas
  flows into recovery bottle.
  
\checkitem Wait approximately 1 minute.
  
\checkitem Isolate recovery bottle by closing valves VF2 and VC2 on gas
  board.
  
\checkitem Open valve VE3 to vacuum pump for approximately 10 seconds.
 This will pump He gas out of recovery bottle but will leave
  frozen  C$^{17}$O$_{2}$ there.
  
\checkitem Close valve VE3.
  
\checkitem Open valves VF2 and VC2.
  
\checkitem Repeat steps 5 through 9 {\bf ten} times.  Note:  each time
  you do this sequence of steps, you should se that the pressure on PG1
  and PG3 go down as you are removing about 1/6 of the helium each time
  you do this.  You will probably have to top off the LN2
  dewar in order to keep lower half of recovery bottle covered with LN2.
  
\checkitem  Close valve on recovery bottle and close valve VD2.
  
\checkitem  Remove LN2 to allow recovery bottle to warm up.
  To speed up the warming, one can place a dewar filled with warm
  water around recovery bottle.  Note: there is a pressure relief
  valve on the recovery bottle that will open if pressure inside
  recovery bottle reaches 150 psi.  This should never happen unless
  gas loop leaked and we trapped O$_{2}$ in recovery bottle.
  
\checkitem If desired, you can then remove the  C$^{17}$O$_{2}$ 
  supply/recovery system from gas board.



\end{enumerate}




