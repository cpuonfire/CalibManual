
  

%------------------------------------------------------------
%\newpage
\subsection{Acrylic Source Assembly}


\newprocedure{CalProcAcrAssem}
{Acrylic Source Assembly}
             {F. Duncan/P. Skensved}
             {Oct. 2004}{4}

  This procedure describes the assembly of the acrylic source.  

  This procedure is very similar to the assembly of the laserball and uses much
of the same hardware. The can may be different in that it may be a blind can ( ie. no
LED mount and no hole for the saltprobe ). There are two choices for the stem which
holds the source : a polypropylene one or a teflon one.

\subsubsection{Procedure}
~\\
\begin{tabular}{|l|l|}
\hline
\multicolumn{2}{|l|}{\bf Acrylic Source Assembly Procedure}\\
\hline
 & \\
Operator:~~~~~~~~~~~~~~~~~~~~~~~~~~~~~~~~~~~~~ & Date: ~~~~~~~~~~~~~~~~~~~~\\
 & \\
\hline
\end{tabular} \\







\subsubsection{Prior To This Procedure}
  \begin{itemize}
  \item The rope may or may not be attached to the rotating bearing.
  \item The carriage may or may not be attached to weight cylinder.
  \item The spool piece may or may not be attached to weight cylinder.
  \item The can may or may not be attached to the spool piece.
  \item The can may or may not be assembled.
  \item The stem may or may not be attached to the can
  \item The salt probe may or may not be installed and connected.
  \end{itemize}



\newpage

\subsection{Assembly of Acrylic Source}

\begin{enumerate}
\checkitem Tie the rope to the rotating bearing. {\bf This is to be done by an expert only !}
\checkitem Slide the rotating bearing on the umbilical.
\checkitem Slide the o-ring on the umbilical. Use correct size o-ring.
\checkitem Attach the carriage to the weight cylinder. Secure the nuts with wire.
\checkitem Attach the spool piece to the weight cylinder. Secure the nuts with wire.
\checkitem Slide the carriage, weight cylinder and spool piece  on the umbilical.
\checkitem Slide a spacer with groove facing down on the umbilical.
\checkitem Slide the o-ring on the umbilical.
\checkitem Slide the stainless steel lid on the umbilical.
\checkitem Slide the o-ring on the umbilical.
\checkitem Slide the acrylic lid on the umbilical.
\checkitem Tie the wires in a  `Hallinian' knot and secure with cable ties. {\bf This is to
be done by an expert only !}
\checkitem Pull the knot up against the lid ( Gently !!! )
\checkitem Make sure the acrylic plate o-ring is in place.

  If the can is a ``blind'' can the next few steps do not apply. If the can
is the one used for the laseball then you will have to install either the saltprobe
or a dummy plug. Use the
appropriate section for the next few steps. The can may also hold a blue LED which
is used to index the laserball in the H$_2$O. Note that only one electrical
device can be connected at any given time ( ie. saltprobe or LED ).

\checkitem Assemble saltprobe.
\begin{enumerate}

\item Mount the inner clamp around the salt probe.
\item Put the salt probe into the can.
\item Slide the o-ring up atround the saltprobe.
\item Connect the wires to the saltprobe. The colour codes are listed in the log book.
item Make an entry in the logbook stating that the saltprobe is now connected to the umbilical.
\end{enumerate}

\checkitem Assemble plug.
\begin{enumerate}
\item Put the plug in the bottom of the can.
\item Slide the o-ring up around the plug.
\end{enumerate}

\checkitem LED assembly.
\begin{enumerate}
\item Press the LED into place. Be careful not to break anything.
\item Connect the wires to the LED. The colour codes are listed in the logbook.
\item Make an entry in the logbook stating that the LED is now connected to the umbilical
\end{enumerate}

  Laserball can only :

\checkitem Feed the fibers through the can and through the hole in the bottom of the can.
\checkitem Make sure the wires and the fibers are placed correctly inside the can. The
preferred routing of the fiber is in a helix which is free to move vertically

\checkitem Attach the acrylic lid to the can with 4 screws. Ensure that the o-ring is in place.
Make sure you're using the correct holes in the plate and make sure you're using the correct
length  screws.

\checkitem Check that everything looks ok inside the can
\checkitem Slide the umbilical o-ring up against the acrylic lid.
\checkitem Slide the stainless steel lid into place. Attach with screws. Use correct screws.
\checkitem Slide the second o-ring into place.
\checkitem Slide the pressure plate into place.
\checkitem Attach the can to the spool piece using 5 screws. Use correct screws.

\checkitem Make sure the o-rings for the are in place ( one for the saltprobe / plug
and another for the center ).

    If the polypropylene stem is being used do the following :

\checkitem Push the fiber in to the CAJON  fitting on the stem. Tighten the fitting ligthly. 
\checkitem Mount the stem on the can. Make sure the fiber is not bent or pinched
in any way. Use correct screws.


    If the teflon stem is being used do the following :

\checkitem Push a CAJON fitting onto the peg on the stainless seal plate and tighten the
appropriate part of the fitting

\checkitem Push the fiber into the other end of the CAJON fitting. Tighten lightly.
\checkitem Hold the plate up against the can while attaching the teflon stem.
Make sure the fiber is not bent or pinched
in any way. Use correct screws. Make sure the o-rings are in place.

     The following applies to both stems.

\checkitem Slide clamp up around saltprobe or plug. Make sure it is oriented correctly ( so that
 it fits ) and
tighten screw appropriately. ( {\bf Do not overtigthen  } ! )

   If the ``blind'' can is being used do the following :

\checkitem Place the wires and the fiber in the can.
\checkitem Attach the acrylic lid to the can with 4 screws. Ensure that the o-ring is in place.
Make sure you're using the correct holes in the plate and make sure you're using the correct
length  screws.
\checkitem Slide the umbilical o-ring up against the acrylic lid.
\checkitem Check that everything looks ok inside the can

\checkitem Slide the stainless steel lid into place. Attach with screws. Use correct screws.
\checkitem Slide the second o-ring into place.
\checkitem Slide the pressure plate into place.
\checkitem Attach the can to the spool piece using 5 screws. Use correct screws.
\checkitem Mount the stem on the blind can.


  The rest of the procedure applies to all cans



\checkitem Slide the o-ring for the rotating bearing into place.
\checkitem Attach the rotating bearing to the carriage with 4 screws and secure them
with wire. Do not overtighten !!!  Do not pinch or damage the o-ring.

\checkitem Attach the correct source to the stem. The flat-topped stainless steel
cans, the standard steel can and the teflon can all attach to the stem without
any other hardware. The sealed stainless steel AmBe  can requires an extar `donut'
to make it flat-topped.
\small
{\em If something looks like it does not go together or appears to be
wrong call the OCE. Do not under any circumstances open a can or remove
any part of it. Some of the cans are to remain sealed at all times.
}

\normalsize


\end{enumerate}


  
%------------------------------------------------------------------------
%------------------------------------------------------------------------
%------------------------------------------------------------------------
\newpage
\subsection{Disassembly of Acrylic Source}
\shwlabel{procpca}~\\
\noindent
\begin{tabular}{|l|l|}
\hline
Version              & 2 \\
\hline
Written/Revised by   & F. Duncan \\
\hline
Date Written/Revised & 2000/10/05\\
\hline
\end{tabular}
 

%----------------------------------------------------------------------
\subsubsection{Procedure}
~\\
\begin{tabular}{|l|l|}
\hline
\multicolumn{2}{|l|}{\bf Disassembly of Acrylic Source Ver 2}\\
\hline
 & \\
Operator(s):~~~~~~~~~~~~~~~~~~~~~~~~~~~~~~~~~~~~ & Date: ~~~~~~~~~~~~~~~~~~~~\\
 & \\
\hline
\end{tabular} 
~\\
\begin{enumerate}
\item\checkbox To be written - see laserball disassembly for now


\end{enumerate}


