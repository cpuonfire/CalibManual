
%--------------------------------------------------------------
%--------------------------------------------------------------
%--------------------------------------------------------------
\chapter{Gas Board ( Nitrogen Flush System )}
\shwlabel{ChapterGasBoard}
  
   
  
\section{Introduction}  

  The gas board is designed to allow the URMs and the side rope motor boxes
to be flushed with dry  radon free N$_2$. In normal mode  the gas flow to the
motorboxes and the URMs is restricted to a few liters per minute so as
not to perturb the D$_2$O covergas system and conserve LN$_2$. For initial
flushing the flow may be increased to the URMs provided the gatevalve is closed.
This is referred to as `bypass mode' below.

 The gas enters the board on the left side and leaves through one or more exits
 on the bottom. There are individual lines to each of the side rope motor boxes
located on the roof of the DCR and a common line to the URMs. This line is in series with
a flow meter and a needle valve ( both located at the south east corner of
the pipebox ). From there it branches out to the two URMs.


 
 {\em Note : The maximum allowed pressure for the flush system is 10 psi.
Do not under any circumstances exceed this pressure ! }

 


\begin{figure}[htb]
\begin{center}
\leavevmode
%\epsfysize=0.85\textheight
\epsfxsize=7in
%\epsfbox{figures/gasboard.eps}
~\\
\caption[Gas Board]
        {Gas Board
         \shwlabel{figgasboard}
        }
\end{center}
\end{figure}

\section{Operation}

  N$_2$ to the board comes from one of two sources :

\begin{itemize}

\item Low pressure supply.

  In this mode the gas is supplied from the Wessington dewar located in the junction area.
The gas enters through the lower left input line and the pressure is controlled by a
fixed regulator on the dewar.


\item High pressure supply.

 Here the gas supply is the 160 psi dewar again located in the junction. This
is the same dewar used for the laser. Gas enters the board through the top left line
and flows through the regulator on the board which should be set { \em no higher than 10 psi}. 
Note that there is a checkvalve in  the low pressure line which prevents gas 
flowing  back to the Wessington ( in case you were wondering ...~).


\end{itemize}



 As indicated above the gas board can be operated in one of two modes :

\begin{itemize}

\item ``Normal mode''

  In this mode the gas flows through small restrictions to the devices and the flow
is limited to approximately 1 liter per minute. 


\item ``Bypass mode'' 

  This mode is used  for rapid flush of a URM.   Direct the gas to the bypass
line and set the three-way valve at the URM line appropriately. Note that for
a three-way valve the arrow points to the `supply gas from' side rather than the
`flow direction'.


\end{itemize}
  

