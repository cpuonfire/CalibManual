\markright{CalOp: Sources}
\chapter{Sources}
\shwlabel{ChapterSources}


\newprocedure{CalOpSources}
       {Sources}
       {F. Duncan}{2001}{1}


  There are many source designed for the SNO detector.  Several of
them will have dedicated URM's assigned to them.
\begin{description}
\item[laserball] is a spherical source containing a diffusing material
  that isotropically distributes light from a Nitrogen laser pumped
  dye laser system.  It is used for the optical calibration of the detector.
  The laser is controlled by the manipulator computer and has both adjustable
  wavelength and intensity.

\item[$^{16}$N]
  The $^{16}$N source is a radioactive gas source used to provide 
monoenergetic gamma rays to measure the detector's energy response.  The
radioactive $^{16}$N gas is created in a d-t generator located in the
junction outside the control room and then piped into the DCR and down
into the source through an umbilical.
  
\item[Rotating Source]
  The rotating source is a device that has a collimated flashing light
source that spins on two axes.  By sweeping out the detector, it can
be used to verify that the locations of PMT's is correctly incorporated
into the various databases.  It requires electrical connections from
the rotating source umbilical.
  
\item[Sonoball]
  The sonoball is a sonoluminence source operating at 
approximately 25kHz.  It uses four wires out of the rotating source
umbilical.
  

\begin{table}
\begin{center}
\begin{tabular}{lcccccc} 
\hline
Source               &  Weight
   & d(pivot to centre) & d(pivot to bottom) & Volume & Max Diameter \\
   & (N)   & (cm)& (cm) & (cm$^{3}$) & (cm) \\
\hline
Laserball (Mark III) &  61  &  64.6 & 69.5  &   & 10.16 (4'') \\
\hline
$^{16}$N Source      &      &  71.0 & 78.59 &   & 11.43 \\
\hline
Camera               & $\sim$100  &  64.77 & 69.2 &   &  \\
\hline
\end{tabular}
\caption[Calibration Source Dimensions]
  {Weights, volumes and dimensions of the calibration sources.
   \shwlabel{TabSourceDimensions}
  }
\end{center}
\end{table}



%======================================================================  
\clearpage
\section{Laserball ( Mark III )}
\shwlabel{SecLaserball}
  
\begin{table}[htb]
\begin{center}
\begin{tabular}{|l|l|}
\hline
weight(source and carriage) & 61 N  \\
d(pivot to centre)          & 64.4 cm  \\
d(pivot to bottom)          & 69.5 cm \\
maximum diameter            & 10.16 cm (4'')\\
\hline
\end{tabular}
\caption[Laserball Parameters]
        {Laserball Parameters
         \shwlabel{tablaserballpars}
        }
\end{center}
\end{table}
  
  



%======================================================================  
\clearpage
\section{$^{16}$N Source}
  
\begin{table}[htb]
\begin{center}
\begin{tabular}{|l|l|}
\hline
weight(source and carriage) & \\
d(pivot to centre)          &  71.0 cm\\
d(pivot to bottom)          & 78.59 cm  \\
maximum diameter            & 11.43 cm \\
\hline
\end{tabular}
\caption[$^{16}$N Parameters]
        {$^{16}$N Parameters
         \shwlabel{tab16Npars}
        }
\end{center}
\end{table}


\begin{figure}
\begin{center}
\leavevmode
%\epsfysize=0.85\textheight
\epsfxsize=7in
\epsfbox{./figures/n16_partial_exploded.ps}
~\\
\caption[Exploded View of N16 Source]
        {Partial Exploded View of N16 Source
         \shwlabel{fign16exploded}
        }

\end{center}
\end{figure}




%======================================================================  
\clearpage
\section{Acrylic Sources}





\begin{figure}
\begin{center}
\leavevmode
%\epsfysize=0.85\textheight
\epsfxsize=7in
\epsfbox{./figures/acrylic_assembly_bw.ps}
~\\
\caption[Acrylic Source]
        {Acrylic Source
         \shwlabel{figacrylic}
        }

\end{center}
\end{figure}



%======================================================================  
\clearpage
\section{AmBe Source}

%======================================================================  
\clearpage
\section{Manipulator Camera}

  The Manipulator Camera is an epoxy encapsulated CCD camera with LED
light source.  It has a special mounting hardware which mounts on a 
standard manipulator weight cylinder but requires it's own camera
umbilical.  The assembled camera is shown in figure \ref{FigCamera}.

\begin{figure}[htbp]
\begin{center}
\leavevmode
\epsfxsize=3in
\epsfbox{figures/Camerab.ps}
\caption[Manipulator mounted Camera]{
  \shwlabel{FigCamera}}
  Manipulator mounted Camera.
\end{center}
\end{figure}
