
%------------------------------------------------------------------------
%------------------------------------------------------------------------
%------------------------------------------------------------------------
\subsection{Activation of $^{24}$Na in 
            Salt Phase using the Super Hot Th Source}
\shwlabel{secprocstandard}
 
 
\procedure{ProcSaltActivate}{Fraser Duncan}{2001/08/29}{0.9}

 

 This procedure describes the use of the superhot acrylic encapsulated
Th source to activate the $^{24}$Na in the D2O during salt phase.  The method
of deployment is to place the encapsulated source inside a metal can
attached to a polypropelene stem mounted on the laserball bucket.
The deployment will be with the laserball URM.
The dangers
inherent in this procedure are:
\begin{enumerate}  
\item Failure of the superhot source encapsulation.
\item Loss of the encapsulated superhot source in the D2O volume
\item Damage to the detector triggers due to the excessive rate
     (500kHz) from the superhot source.
\end{enumerate}
The prevention of eventuality (1) is the triple encapsulation of
the source in acrylic.  Eventually (2) is to be prevented by
the intrinsic mechanical redundancy of the source deployment
mechanism.  This redundancy includes:
\begin{itemize}
\item Two independent support mechanisms (the rope and umbilical).
\item Multiple connections between the encapsulated source and the
  source mechanism (multiple screws, each able to take the weight of
  the source).
\end{itemize}
Item (3) is of concern because of the extremely high rate of the
source.  The means of protecting the detector will be to start with
the NHIT triggers off while deploying the source into the detector.
If the rate still becomes too large, the detector will be ramped
down.
 
\noindent
{\bf Personnel:}
\begin{itemize}
\item Calibration Expert:  Aksel Hallin (u/g)
\item Detector Expert: Noel Gagnon (u/g)
\item Trigger Expert: Josh Klein (phone)
\end{itemize}

\noindent
{\bf Required Detector Conditions}
\begin{itemize}
\item No D$_2$O recirculation being done.
\end{itemize}

\noindent
{\bf Notes:}
\begin{itemize}
\item Run type during activation not specified yet.  Will 
  have to be determined  when we see what rates are tolerable.
\item Duration of Activation run to be specified.
\item Type of running after the activation to be specified.  Presumably
  this should be a normal neutrino run but with  UC bit set to
  remind people that this is not for regular analysis.  Or perhaps
  it should still be a calibration source run.
\end{itemize}


\operator

\begin{center}
{\bf Preparation of the source for deployment.}
\end{center}

\begin{enumerate}

  
\checkitem Install the super-hot acrylic source on the peg:\\
  {\em Use rubber or latex gloves for all work.}
  \begin{enumerate}
  \checkitem  Ensure that the source, can, and all parts have been 
    cleaned with the standard cleaning procedure.
  \checkitem  Place the source in the stainless steel can
  \checkitem  With the approved teflon grease, very lightly grease the 
    inside of the can.  Wipe it as clean as possible with a wipe.  Insert 
    the o-ring into the groove in the black delrin universal adapter plug.
  \checkitem Tighten the three stainless steel bolts on the diameter 
      of the can, attaching the delrin piece to the can.
  \checkitem  Ensure that the small o-ring is installed in the ring 
      groove on the top
      of the delrin spacer.  This o-ring should not be greased.
  \checkitem  Ensure that the three stainless steel nuts are in the slots 
    of the delrin piece and insert and tighten the screws that attach this 
    to the bottom of the peg.
  \end{enumerate}
  Before deployment, re-inspect the peg, source, and carriage, looking for any
  missing or loose fasteners or any sign of contamination.
 
\checkitem The source can is secured and wired to the acrylic source peg.
  
\checkitem Inspect the assembled source.  All screws and bolts should be
    tight.  The laserball assembly should be verified.  If there
    is any doubt about the presence of the required o-rings, it should
    be opened for inspection.
  
\checkitem Verify the proper pivot to source position in the manipulator code
    (in polyaxis.dat the acrylic object).
    This will have to be measured.
     \begin{center}
     \begin{tabular}{|l|}
     \hline
      \\
     d(pivot-source):~~~~~~~~~~~~~~~~~~~~~~~~\\
      \\
     \hline
     \end{tabular}
     \end{center}
 \checkitem Measure the weight of the source using the URM rope.  Enter
  this value in {\tt polyaxis.dat}
     \begin{center}
     \begin{tabular}{|l|}
     \hline
      \\
     Acrylic Source Weight:~~~~~~~~~~~~~~~~~~~~~~~~\\
      \\
     \hline
     \end{tabular}
     \end{center}
 \checkitem Verify the CAST bank information on {\bf manip} is up to date
  for the source.
  
\checkitem Run the source up/down in the DCR several times to verify the
    manipulator is functioning correctly.

\checkitem Retract the  source into the source tube.  The pivot should be 
    visible from the window on the source tube.
  
\checkitem Insert the source clamps to secure the source in the source tube.
 
\checkitem Mount the URM on the glovebox.
  
\checkitem Retract the source clamps.
 

\item\checkbox Verify that the LN$_2$ dewar in the junction is
  at least 1/4 full.  If not, swap it out with another dewar.
  Record liquid level of Dewar,
     \begin{center}
     \begin{tabular}{|l|}
     \hline
      \\
     LN$_2$ Level:~~~~~~~~~~~~~~~~~~~~~~~~\\
      \\
     \hline
     \end{tabular}
     \end{center}

\item\checkbox Verify that the dewar gas pressure is approximately
  130 to 150 psig. If not, swap it out with another dewar.

\item\checkbox Turn on N2 Flow to DCR from dewar at junction 
  (Marked {\bf Gas Use} on dewar).
     \begin{center}
     \begin{tabular}{|l|}
     \hline
      \\
     Note Time:~~~~~~~~~~~~~~~~~~~~~~~~\\
      \\
     \hline
     \end{tabular}
     \end{center}

\item\checkbox Turn on pressure builder valve (Marked {\bf Pressure Builder} 
  on dewar).\\
  %------------------------
  \small
  {\em The pressure builder valve opens a controlled leak on the dewar
       to maintain the 150 psi pressure head.  If the valve is not
       opened, the gas pressure to the laser and URM will eventually
       drop below the operating level.}
  \normalsize
  %------------------------

  
\item\checkbox Check that flush return line is connected to
  URM1.  If not, connect it.\\
  %------------------------
  \small
  {\em It may be necessary to move it over to URM1 from URM3 (the laserball).
  }
  \normalsize
  %------------------------

\item\checkbox Set URM1 flush regulator at 40 psig.

\item\checkbox Open URM1 flush valve.  Flow meter should be railed.
     The sound of the gas flowing into the URM should be apparent within
     a half metre of the URM source tube.
     \begin{center}
     \begin{tabular}{|l|}
     \hline
      \\
     Note Time:~~~~~~~~~~~~~~~~~~~~~~~~\\
      \\
     \hline
     \end{tabular}
     \end{center}

   Flush should continue for at least 1 hour.
   {\bf Do NOT use the water group's O$_2$ meter.}

  
\item\checkbox Check that the source clamps are in the RELEASE position.  
   The RELEASE position for URM2 is for the knobs on the side of the source
   tube to be fully OUT.  {\bf There are two knobs.  Check BOTH.}\\
   {\bf 
     WARNING:  If the source is moved with the clamps in, the source
     may be damaged!
   }
  %--------------------------------
  \small
  {\em
   The clamps are used to secure the source while the URM is being moved
   on and off the glovebox.  If the source is moved with the clamps in
   the hold position, it will most likely foul in the clamps and
   require disassembly of the URM to extract.
  }
  \normalsize
  %--------------------------------
 
\item\checkbox Check gas pressure on URM pressure cylinder = 45 psig.\\
   {\bf 
   IF PRESSURE IS LESS THAN 10 PSIG DO NOT OPERATE MANIPULATOR
   AND CONTACT EXPERT.
   }\\
   %-------------------------------
   \small
   {\em
     The pressure cylinder on the URM maintains tension on the umbilical
     takeup reel.  A low gas pressure can result in the umbilical falling
     off the takeup reel and resulting in tangling and damage of the
     umbilical.
   }
   \normalsize
   %--------------------------------
  
   
\item\checkbox Verify that Gate Valve 1 is locked in the  closed position.\\
   %-------------------------------
   \small
   {\em
     If the handle is on the gate valve, CLOSED is when the  handle points to 
     the left when facing the source tube.  If the handle is not on the valve
     then the slot on the handle stem points AWAY from the source tube when
     the valve is closed.
   }
   \normalsize
   %--------------------------------
 
\item \checkbox Calibrate Central Rope Length\\
      (see procedure  \ref{seccalcentre} 
       {\em Central Rope Position Calibration}).
      Record changes in length of central rope and umbilical,
      The fiducial mark for the wide bottomed 4'' source tube on Gate Valve 1
      is
      \[
               z_{mark} = 1559.9
      \]
     \begin{center}
     \begin{tabular}{|l|}
     \hline
      \\
     $\Delta$l rope:~~~~~~~~~~~~~~~~~~~~~~~~\\
      \\
     \hline
      \\
     $\Delta$l umbilical:~~~~~~~~~~~~~~~~~~~~~~~~\\
      \\
     \hline
     \end{tabular}
     \end{center}


\item\checkbox Check that all seals are in place on URM.  Including:
   \begin{itemize}
      \item\checkbox view port window cover on source tube
      \item\checkbox window on front URM hood
      \item\checkbox window on back URM hood
      \item\checkbox umbilical feedthrough on URM
      \item\checkbox flush inlet line.
      \item\checkbox flush outlet line.
   \end{itemize}


\item\checkbox At end of URM flush, turn regulator down to 5 psig.
   Check that the flow meter is railed at 50.
   %--------------------------
   \small
   {\em
     The regulator only is marked down to 10 psig.  To set it to 5 psig,
     set the clear plastic indicator to half way between 10 psig and 0.
   }
   \normalsize
   %--------------------------
  

   
%--------------------------------------------------------------
\begin{center}
            {\bf Deploying Source from Source Tube Into Glovebox}
\end{center}

 \item\checkbox Verify the 40psi flush of the URM has been at least 1 hour.

 \item\checkbox Verify that the URM flush has been turned down to 5 psig.

 \item\checkbox Turn off DCR lights.

 \item\checkbox Verify Owl light monitor is on.  Establish communications
  with person watching light monitor.
  %-------------------------
  \small
  {\em
    Suggestion:  Station the person watching the OWL monitor at
    the Deck Mac.  Then he/she can shout through the  wall of the
    DCR and you don't need to use the phones which slow communications
    down.
  }
  \normalsize
  %-------------------------

 \item\checkbox Open gate valve.

 \item\checkbox Lock gate valve open.

 \item\checkbox With flashlight perform light leak check on URM.  In particular
   check the seal of the source tube window.

 \item\checkbox Using the dimmer switch, slowly bring up breaker 9 lights in
   the DCR.  Person still watching owl monitor.

 \item\checkbox DAQ is connected to the {\bf manip} computer.

 \item \checkbox In DAQ, source type is set to {\bf ACRYLIC}.

 \item\checkbox DAQ is in a {\bf source transitional run}.

 \item \checkbox Verify that {\bf manip\_logger} on {\bf polaris}
                 is running and logging the acrylic source.

 \item\checkbox Check movement of acrylic source down:
  \begin{center}
  \begin{tabular}{|l|l|}
  \hline
  console & {\tt manip$>$ acrylic by 0 0 -5} \\
  \hline
  manmon  & in acrylic window: \\
          & set x = 0, y = 0, z= -5\\
          & click on {\bf move by} \\
  \hline
  \end{tabular}
  \end{center}
  %--------------------
  \small
  {\em 
    The acrylic source should move down 5 cm.  The tension on the rope
    should be 60-90 N.  The tension on the umbilical should be
    5-30N.
  }
  \normalsize
  %--------------------

 \item\checkbox Deploy source into the glovebox:
  \begin{center}
  \begin{tabular}{|l|l|}
  \hline
  console & {\tt manip$>$ acrylic to 0 0 1370} \\
  \hline
  manmon  & in acrylic window: \\
          & set x = 0, y = 0, z= 1370\\
          & click on {\bf move to} \\
  \hline
  \end{tabular}
  \end{center}



%-------------------------------------------------------------
\begin{center}
  {\bf Deploying Source to Centre of 
            Detector from Glovebox}
\end{center}
\shwlabel{sectocentre}
 
 \item\checkbox Contact Water Supervisor and advise him/her that the source is
   being lowered into the D2O.  \\
   %--------------------
   \small
   {\em
     The water group maintains a very small differential pressure
     between the light and heavy water.  The volume of the source
     is enough to disrupt this differential pressure.
   }
   \normalsize
   %---------------------

 \item\checkbox Check tensions on urm1rope and urm1umbilical.  Rope tension
   should be approximately 60-80 N.  Umbilical tension should
   be between 10-30 N.

 \checkitem {\bf Contact Josh Klein by phone at Penn}
 \checkitem {\bf Turn off the NHIT Triggers}
  
 \item\checkbox Move acrylic source base of Neck.
  \begin{center}
  \begin{tabular}{|l|l|}
  \hline
  console & {\tt manip$>$ acryic to 0 0 600} \\
  \hline
  manmon  & in acrylic window: \\
          & click on {\bf Position the source}\\
          & set x = 0, y = 0, z= 600\\
          & click on {\bf move to} \\
  \hline
  \end{tabular}
  \end{center}
  {\bf While moving the source, monitor the ESUM triggers on the detector
  looking for excessive rates.}  {\em This will be complicated by
  the presence of manipulite.  Therefore it may be wise to stop every
  couple of meters so that a measure of the ESUM rate without the
  background manipulite can be made.} 

  {\bf At this point decide if the detector is acceptably stable to
  run or if it is necessary to shut it down.}
  
 \item\checkbox Move acrylic source to centre of detector.
  \begin{center}
  \begin{tabular}{|l|l|}
  \hline
  console & {\tt manip$>$ acryic to 0 0 0} \\
  \hline
  manmon  & in acrylic window: \\
          & click on {\bf Position the source}\\
          & set x = 0, y = 0, z= 0\\
          & click on {\bf move to} \\
  \hline
  \end{tabular}
  \end{center}
  {\bf While moving the source, monitor the ESUM triggers on the detector
  looking for excessive rates.}  {\em This will be complicated by
  the presence of manipulite.  Therefore it may be wise to stop every
  couple of meters so that a measure of the ESUM rate without the
  background manipulite can be made.} 

\checkitem Attempt to turn on the NHIT triggers:
  \begin{itemize}
  \item Set NHIT threshold to 100.
  \item Do an {\bf Enable Triggers} from the Standard Runs window while
    watching the trigger rates.  If they are not excessive, 
    lower the trigger threshold till it is no more than 100 Hz.
  \end{itemize}


%-------------------------------------------------------------
\begin{center}
  {\bf Retracting Source from Detector}
\end{center}

\checkitem Go to {\bf source transitional run}
\checkitem Turn off NHIT triggers (using the {\bf Disable PMTs} button).
\checkitem Retract the source to the glovebox.
  \begin{center}
  \begin{tabular}{|l|l|}
  \hline
  console & {\tt manip$>$ acryic to 0 0 1300} \\
  \hline
  manmon  & in acrylic window: \\
          & click on {\bf Position the pivot}\\
          & set x = 0, y = 0, z= 1300\\
          & click on {\bf move to} \\
  \hline
  \end{tabular}
  \end{center}

\checkitem Retract source into source tube:
  \begin{center}
  \begin{tabular}{|l|l|}
  \hline
  console & {\tt manip$>$ acryic to 0 0 1530} \\
  \hline
  \end{tabular}
  \end{center}

\checkitem Retract source to home position:
  \begin{center}
  \begin{tabular}{|l|l|}
  \hline
  console & {\tt manip$>$ acryic to 0 0 1550} \\
  \hline
  \end{tabular}
  \end{center}

\checkitem Reach into one of the south glove ports and verify that
  the source is above the gate valve.

\checkitem Close and lock gate valve.
 \item\checkbox Close the URM flush valve.
\item\checkbox Turn off the URM flush regulator.
\item\checkbox IF the laser is off,
   turn off gas flow at the LN$_2$ dewar in the junction:
   \begin{enumerate}
   \item close {\bf Gas Use} valve
   \item close {\bf Pressure Building} valve
   \end{enumerate}

%-------------------------------------------------------------
\begin{center}
           {\bf After Calibration}
\end{center}
\item\checkbox Source is above gate valve.
\item\checkbox Gate valve is closed and locked.
\item\checkbox LN$_2$ dewar is turned off (both {\bf Gas Use} valve and 
  {\bf Pressure Building} valve).
\end{enumerate}

 

