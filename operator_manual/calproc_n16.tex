

  
%------------------------------------------------------------------------
%------------------------------------------------------------------------
%------------------------------------------------------------------------
\section{N16 Source Procedures}
\shwlabel{secprocN16}

\newprocedure{CalProcN16SourceProc}
             {N16 Source Procedures}
             {F. Duncan/P. Skensved}{Oct. 2004}{1}



  These procedures describe the operation of the N16 calibration
source.


\newpage

\subsection{DT Generator Emergency Shutdown Procedure}

\shwlabel{procn16start}

\newprocedure{CalProcDTEmShut}
             {DT Generator Emergency Shutdown Procedure}
             {F. DUuncan/P. Skensved}{Jun. 2002}{2}



  This procedure describes how to turn off the DT generator and
gas board in case of an emergency such as fire, severe gas leak,
radiation problem etc.



\begin{itemize}
\item {\bf Turn off HV. }
\item Turn off neutron pulser.
\item Turn key to  ``Off''
\item Close main valve on CO$_2$ bottle.


\end{itemize}




%------------------------------------------------------------
\newpage
\subsection{DT Generator Turn On Procedure}
\shwlabel{procdton}


\newprocedure{CalProcDTTurnOn}
             {DT Generator Turn On Procedure}
             {F. Duncan/P. Skensved}{Sept. 2004}{2}


 
  This procedure describes how to turn on the DT generator
prior to operating either the N16 or LI8 sources.  Only authorized
DT operators are allowed to operate the generator.
  
\subsubsection{State Prior To This Procedure}
\begin{itemize}
\item DT generator is OFF.
\end{itemize}

\subsubsection{Summary of Procedure}
\begin{itemize}
\item {\bf Make sure you understand the DT Generator shutdown and emergency shutdown  procedures.}
\item Secure DT generator area.
\item Turn on spare NCD counter.
\item Turn on  DT generator.
\item Check operation of DT generator at full NOC setting.
\item Record maximum and minimum neutron flux.

\end{itemize}

\newpage
\begin{tabular}{|l|l|}
\hline
\multicolumn{2}{|l|}{\large\bf DT Generator Turn On Procedure}\\
\hline
 & \\
Operator(s):~~~~~~~~~~~~~~~~~~~~~~~~~~~~~~~~~~~~ & Date: ~~~~~~~~~~~~~~~~~~~~\\
 & \\
\hline
\end{tabular} 



\begin{enumerate}

\checkitem Read and  make sure you { \bf understand } the DT Generator
Turn Off and Emergency Turn Off procedures (also referred to as the DT Generator Shutdown
procedures).


\checkitem If it is not already on, turn on the gas computer.

{\em The use of the Gas Computer is optional }

\checkitem If it is not already running, start the Gas/DT monitoring
  program on the gas computer by:
  \begin{enumerate}
  \item start labview from the programs menu.
  \item from the labview file menu item select {\tt open}.
  \item go to {\tt c:/labview/develop/}
  \item select {\bf febmain.vi}.
  \item After the VI has loaded, click on the run (arrow) button.
  \end{enumerate}
\checkitem If it is not already running and selected, on the labview
  monitoring program, click on the {\bf fraser monitor} check box.

\item\checkbox
 Secure the DT area : ensure that the shielding blocks ( the ``donuts'' )
 are in place and that the grey interlock boxes are in place and connected.

\checkitem
  Turn on the main instument panel, the flow controller and the stepper
motor controller.

\item\checkbox
 Set the Interlock Override on the main instrument panel
 to ``On'' for manual control (normal setting) or ``Computer'' for
 Labview control.


\item\checkbox
 Check that the High Voltage (HV) and neutron pulser
 switches on the DT control panel are in the OFF position,

\item\checkbox
   Set the Neutron Pulse Control (NPC) to 5 and the Neutron
   Output Control (NOC) to 0. 



\item\checkbox Make sure the HV supply for the spare NCD counter is turned
off and the dial setting is at 0.

\item\checkbox Turn on the NIM bin with the Fast Neutron Flux Meter (F.N.F.M)

\item\checkbox Turn on the low voltage powersupply for the spare NCD counter.


\item\checkbox Turn on the F.N.F.M high voltage supply and set the
voltage to 800 V.

\item\checkbox Connect an oscilloscope to the spare NCD counter amplifier. Set
timebase to approximately 1 us/div and vertical scale to a small fraction of a
volt per division.


\item\checkbox Turn on the HV for the spare NCD counter. While watching the scope
for pulses slowly ( about 1 minute ) increase the voltage to 1850 V.  

\item\checkbox
   Connect an oscilloscope to the SYNC output. Set timebase to 
approximately 2 ms full sweep.

\item\checkbox
  Insert the DT generator key and turn on the power.

\item\checkbox
   At this time, the main power light on the DT control panel, and
   the {\bf D.T. GEN ON} light located behind the DT Pit should be on.
   {\bf If either of these lights does not come on, proceed to}
   {\bf step five in the shutdown procedure. }


\item\checkbox
  Switch the neutron pulser on.  Make sure that the neutron pulser light
comes on.  {\bf If the neutron pulser light does not come on, \
proceed to step number four of the shutdown procedure.} 


\item\checkbox
 Make sure that the neutron pulser light comes on. 
 {\bf If the neutron pulser light does not come on, }
 {\bf proceed to step number four of the shutdown procedure
 immediately.} 


\item\checkbox
 Verify that the DT generator is pulsing at  1.8 ms and that the pulse width
is 180 ${\mu}$s.  

\checkitem Leave the pulser running for approximately one minutes.

\item\checkbox
  When the HV is turned on the ``normal''  (green) and the ``high current'' (red)
 will likely flicker for a few  seconds after which the green light will be off for
a while before it comes on steady. The red light should be off .
 { \bf There is now a built in timedelay in the DT generator source circuit - it may
take up to 1 minute for the green light to be steady. As long as the red light does
not  come on don't panic. If the green light continues to flicker past 20 seconds or the 
  red light comes on steady proceed to step number two in the
 shutdown procedure immediately } 


\item\checkbox
 Record the high voltage turn on time below,
 as well as in the DT Generator Log Book. 
     \begin{center}
     \begin{tabular}{|l|}
     \hline
      \\
     HV Turn on Time:~~~~~~~~~~~~~~~~~~~~~~~~\\
      \\
     \hline
     \end{tabular}
     \end{center}
 

\item\checkbox
 Record Initial value of the F.N.F. Meter, 
     \begin{center}
     \begin{tabular}{|l|}
     \hline
      \\
     FNFM Meter Reading:~~~~~~~~~~~~~~~~~~~~~~~~\\
      \\
     \hline
       \\
     FNFM Computer Reading:~~~~~~~~~~~~~~~~~~~~~~~~\\
      \\
     \hline
     \end{tabular}
     \end{center}
\item\checkbox
 Increase the NOC to 10 (slowly over a period of 20 sec) 
 

\item\checkbox The NCD counter should be counting at a few Hz


\item\checkbox
 Record the final (max) value of the F.N.F. Meter, 
     \begin{center}
     \begin{tabular}{|l|}
     \hline
      \\
     FNFM Meter Reading:~~~~~~~~~~~~~~~~~~~~~~~~\\
      \\
     \hline
       \\
     FNFM Computer Reading:~~~~~~~~~~~~~~~~~~~~~~~~\\
      \\
     \hline
     \hline
     \end{tabular}
     \end{center}


\item\checkbox
 Reset the NCD counter scaler.

\checkitem Record start time and initial flux on DT Generator Log Sheet.

\checkitem Turn the NOC setting back down to zero until the source
 is deployed.


\end{enumerate}


%------------------------------------------------------------
\newpage
\subsection{DT Generator Turn Off Procedure}
\shwlabel{procn16start}


\newprocedure{CalProcDTTurnOn}
             {DT Generator Turn Off Procedure}
         {F. Duncan/P. Skensved}
         {Jun. 2002}{3}


  This procedure describes how to turn off the DT generator.
A trained and authorized DT generator operator is required to
perform this procedure except {\bf in an emergency}.


\vspace*{0.25in}
~\\
\begin{tabular}{|l|l|}
\hline
\multicolumn{2}{|l|}{\large\bf DT Generator Turn Off Procedure}\\
\hline
 & \\
Operator:~~~~~~~~~~~~~~~~~~~~~~~~~~~~~~~~~~~~~ & Date: ~~~~~~~~~~~~~~~~~~~~\\
 & \\
\hline
\end{tabular} \\
\begin{enumerate}

\checkitem Decrease the NOC from 10 to 0 (Slowly over a period of 20 sec.)
   Leave the Neutron Pulse Control at 5.
  
\checkitem Turn the HV Off.  Record this time in the DT Logbook.
     \begin{center}
     \begin{tabular}{|l|}
     \hline
      \\
     HV Turn off Time:~~~~~~~~~~~~~~~~~~~~~~~~\\
      \\
     \hline
     \end{tabular}
     \end{center}
 
\checkitem Wait 1 minute.
  
\checkitem Switch the neutron pulser off.
  
\checkitem Turn the DT Generator off and reurn 
  keys to their proper location in the N$^{16}$ cabinet.

\checkitem Record NCD counter total :
     \begin{center}
     \begin{tabular}{|l|}
     \hline
      \\
     Dose:~~~~~~~~~~~~~~~~~~~~~~~~\\
      \\
     \hline
     \end{tabular}
     \end{center}
  
\checkitem Turn down the HV for the NCD counter ( slowly ! )

\checkitem Turn off the HV for the NCD counter

\checkitem Turn off the low voltage powersupply for the NCD counter

\item\checkbox Turn off HV for FNFM (just turn it off, don't ramp down).

\checkitem Turn off the NIM bin

\checkitem Fill Out DT Generator Log Sheet.


\end{enumerate}

\newpage

%------------------------------------------------------------
\newpage
\begin{center}
{\bf DT Generator Log Sheet}
\end{center}
\begin{tabular}{|l|}
\hline
      \\
Previous Total Minutes:~~~~~~~~~~~~~~~~~~~~~~~~\\
      \\
Previous Total Hours:~~~~~~~~~~~~~~~~~~~~~~~~\\
      \\
\hline
\end{tabular}

\begin{center}
\begin{tabular}{|c|c|c|c|c|c|c|c|c|c|}
\hline
\multicolumn{3}{|c|}{start} &      &\multicolumn{4}{|c|}{stop} & duration & total\\
\hline
yy/mm/dd & hh:mm & FNFM & operator  & yy/mm/dd & hh:mm & FNFM & NCD counter   &time &total\\
         &       & (hz) &           &         &       & (hz) & (counts) & (min)    &~~~minutes~~~~\\
\hline
  &  &  &  & & & & & &\\
\hline
  &  &  &  & & & & & &\\
\hline
  &  &  &  & & & & & &\\
\hline
  &  &  &  & & & & & &\\
\hline
  &  &  &  & & & & & &\\
\hline
  &  &  &  & & & & & &\\
\hline
  &  &  &  & & & & & &\\
\hline
  &  &  &  & & & & & &\\
\hline
  &  &  &  & & & & & &\\
\hline
  &  &  &  & & & & & &\\
\hline
  &  &  &  & & & & & &\\
\hline
  &  &  &  & & & & & &\\
\hline
  &  &  &  & & & & & &\\
\hline
  &  &  &  & & & & & &\\
\hline
  &  &  &  & & & & & &\\
\hline
  &  &  &  & & & & & &\\
\hline
  &  &  &  & & & & & &\\
\hline
  &  &  &  & & & & & &\\
\hline
  &  &  &  & & & & & &\\
\hline
  &  &  &  & & & & & &\\
\hline
  &  &  &  & & & & & &\\
\hline
  &  &  &  & & & & & &\\
\hline
  &  &  &  & & & & & &\\
\hline
  &  &  &  & & & & & &\\
\hline
  &  &  &  & & & & & &\\
\hline
  &  &  &  & & & & & &\\
\hline
  &  &  &  & & & & & &\\
\hline
  &  &  &  & & & & & &\\
\hline
  &  &  &  & & & & & &\\
\hline
  &  &  &  & & & & & &\\
\hline
  &  &  &  & & & & & &\\
\hline
  &  &  &  & & & & & &\\
\hline
  &  &  &  & & & & & &\\
\hline
  &  &  &  & & & & & &\\
\hline
  &  &  &  & & & & & &\\
\hline
  &  &  &  & & & & & &\\
\hline
  &  &  &  & & & & & &\\
\hline
  &  &  &  & & & & & &\\
\hline
\end{tabular}
\end{center}

\noindent
\begin{tabular}{|l|}
\hline
      \\
Total Minutes:~~~~~~~~~~~~~~~~~~~~~~~~\\
      \\
Total Hours:~~~~~~~~~~~~~~~~~~~~~~~~~~~~~~~~\\
      \\
\hline
\end{tabular}



%------------------------------------------------------------
\newpage
\subsection{N16 Source Startup Procedure}
\shwlabel{procn16start}~\\

\newprocedure{CalProcN16SourceStart}
          {N16 Source Startup Procedure}
          {F. Duncan/P. Skensved}{Sept. 2004}{2}

 
  This procedure describes how to turn on the N16 source.
An authorized DT and N16 operator is required to perform this 
procedure.
  
\subsubsection{State Prior To This Procedure}
\begin{itemize}
\item DT generator is OFF.
\item Gas flow to N16 gas board and chamber is OFF.
\item N16 chamber is assembled and connected to gas board.
%%\item N16 PMT is wired in to SNO trigger system.
\end{itemize}

\subsubsection{Summary of Procedure}
\begin{itemize}
\item Turn on DT generator.
\item Set up the N16 gas board.
\item Turn on the CO$_2$ gas flow to the gas board.
\item Adjust to desired flow rate.
\item Check pressure in N16 source chamber.
\item Turn on N16 PMT.
\item Adjust NOC and target position to desired $^{16}$N rate.
\end{itemize}


\begin{figure}
\begin{center}
\leavevmode
%\epsfysize=0.85\textheight
\epsfxsize=7in
\epsfbox{./figures/n16_gas_panel.ps}
~\\
\caption[N16 Gas Panel]
        {N16 Gas Panel
         \shwlabel{fign16gaspanel}
        } 
        
\end{center}
\end{figure}


\newpage
\subsubsection{Procedure}
~\\
\begin{tabular}{|l|l|}
\hline
\multicolumn{2}{|l|}{\bf N16 Source Startup Procedure}\\
\hline
 & \\
Operator:~~~~~~~~~~~~~~~~~~~~~~~~~~~~~~~~~~~~~ & Date: ~~~~~~~~~~~~~~~~~~~~\\
 & \\
\hline
\end{tabular} \\

\begin{itemize}
\checkitem
  Complete DT Generator startup procedure.

\checkitem
  Complete Gas Board startup procedure.

\checkitem
  Complete N16 PMT Turn On procedure.

\checkitem
  Complete $^{16}$N Detector Setup procedure.

\end{itemize}







\newpage
\subsubsection{Procedure}
~\\
\begin{tabular}{|l|l|}
\hline
\multicolumn{2}{|l|}{\bf Gas Board Startup Procedure}\\
\hline
 & \\
Operator:~~~~~~~~~~~~~~~~~~~~~~~~~~~~~~~~~~~~~ & Date: ~~~~~~~~~~~~~~~~~~~~\\
 & \\
\hline
\end{tabular} \\


\begin{enumerate}

\checkitem Verify all valves on the gas board are closed.
\checkitem Verify the N$_2$ / CO$_2$ flow control (side B) is off (fully CCW).
\checkitem Verify the Solenoid Valve II switch is in the OFF position.

\checkitem Enter the DCR using standard entry procedure and 
verify that 
\begin {itemize}
 \checkitem the gas input line is connected to the ``dry'' end of the umbilical
 \checkitem the ``blue'' valve is in the open position. 
 \checkitem the gas return line is connected to the ``dry'' end of the umbilical.
  \checkitem the pressure transducer is powered up and that a voltmeter is connected to it.
  \checkitem the voltmeter is reading the correct ( 18 psi )  ``pressure''.
\end{itemize}
\small
{ \em The pressure transducer is hooked up in a temporary way for the time being. Consult
OCE for details.}

\normalsize

\item \checkbox Turn on the power on the main instrument panel
 and the dual flow controller box. 
  
\item\checkbox Flip the switch on flow controller box to ``B'' (for
 N$_2$ and/or CO$_2$).  The reading takes
  a few minutes to equilibriate to zero or near zero.

\item \checkbox Ensure that the CO$_2$ bottle is hooked up to the input
input line at valve VA2 . 

\item\checkbox Check that VA2 is closewd.

\checkitem Close the needle valve on the CO$_2$ bottle. 

\checkitem Open the main CO$_2$ bottle valve.

\checkitem Set the regulator to 80 PSIG. Note that once the gas starts
 flowing the regulator and bottle starts to cool off and you will have 
to readjust the setting.

\checkitem  Record the bottle pressure:
  \begin{center}
  \begin{tabular}{|c|c|}
  \hline
  Transducer & Reading\\
  \hline
    CO$_2$ Bottle Pressure & \\
  (PSIG) & \\
  \hline 
  \end{tabular}
  \end{center}
  {\em The pressure of a full bottle is 850 to 900 psiG. The CO$_2$ in
  the bottle is in liquid or solid form and the pressure will stay
  relatively high until gas only and will drop rapidly thereafter.
  The CO$_2$ regulator has a small white plastic insert ( disc with small
hole ) in order to seal to the bottle. Do't damage or lose it.  }

\checkitem Open the inline needle valve after the CO$_2$ regulator.

\checkitem Open SVII by setting the switch on the main control panel
  to {\bf MAN}.

\item\checkbox Open the manual shut-off valve VB2.

\item\checkbox Direct flow to the CO$_2$/N$_2$ flow meter using valve VC1.
  

\item\checkbox Set the CO$_2$/N$_2$ flow meter to manual and turn the
control knob one turn so that some gas will flow.

\item \checkbox Direct flow towards the CO$_2$/N$_2$ flow meter using valve VD1.

\item\checkbox Direct flow to the N16 target chamber using valve VF1. Make
 sure you are   not sending it to the oven !

\item\checkbox Open valve VF2 to accept the returning CO$_2$.

\item\checkbox Open valve VC2.

\item\checkbox Direct the flow towards the exhaust using valve VB4.

\item\checkbox Open the exhaust needle valve VA4 fully.

\item\checkbox {\bf Slowly } open valve VA2 and look to see if gas is flowing.

\item\checkbox {\bf Slowly} increase the flow until you reach around a 200-280
  reading. Make sure P1 never goes above 100 PSIA. Remember that
  the time constants are long.


\checkitem 
  Record Gas Presures and Flow.
  \begin{center}
  \begin{tabular}{|c|c|}
  \hline
  Transducer & Reading\\
  \hline
    P1 & \\
  (PSIA) & \\
  \hline 
    P2 & \\
     (PSIA)  & \\
  \hline 
    P3 & \\
     (PSIA)  & \\
  \hline 
    P4 & \\
     (PSIA)  & \\
  \hline 
    Flow & \\
    (cc/s x(1/1000)) & \\
  \hline 
    Target   & \\
    Position & \\
  \hline 
  \end{tabular}
  \end{center}


\checkitem Go back to the DCR and verify that the pressure in the N16 source chamber is within
the acceptable upper limit.
\small
{\em This is done by checking the voltage reading from the pressure transducer. The transducer gets
its power from the monitor PMT supply. Contact the OCE for details }


\normalsize


\end{enumerate}




%------------------------------------------------------------
\newpage
\subsection{N16 Source Shutdown Procedure}
\shwlabel{procn16start}

\newprocedure{CalProcN16SourceShut}
        {N16 Source Shutdown Procedure}
             {F. Duncan/P. Skensved}{Jul. 2003}{2}


  This procedure describes how to shut down the
N16 source.  An authorize DT operator and an authorized N16 operator
is required to perform this procedure.
  
\subsubsection{State Prior To This Procedure}
\begin{itemize}
\item DT generator is ON.
\item Gas is flowing through the N16 board to the Chamber.
\item N16 PMT is on.
\end{itemize}

\subsubsection{Summary of Procedure}
\begin{itemize}
\item Complete DT Generator Shutdown procedure.

\item Complete Gas Board Shutdown procedure.

\item Complete N16 PMT turn off procedure.

\end{itemize}


\newpage
\subsubsection{Procedure}
~\\
\begin{tabular}{|l|l|}
\hline
\multicolumn{2}{|l|}{\bf N16 Source Shutdown Procedure}\\
\hline
 & \\
Operator:~~~~~~~~~~~~~~~~~~~~~~~~~~~~~~~~~~~~~ & Date: ~~~~~~~~~~~~~~~~~~~~\\
 & \\
\hline
\end{tabular} \\

\begin{enumerate}

\item\checkbox Close the main CO$_2$ gas bottle
  valve.
  
\item\checkbox Wait for pressure on P1, P2, P3, P4 to reach atmosphere
  (approximately 18 PSIA).
  
\item\checkbox Turn off regulator on CO$_2$ bottle.
  
\item\checkbox Close inline valve near CO$_2$ regulator.
  
\item\checkbox Close VA2.

\checkitem Close SVII by flipping the switch on the main controller
panel to {\bf Off }.

\checkitem Close VB2.

\checkitem Close VC1.

\checkitem Turen the flow control knob to zero.

\checkitem VD1.

\checkitem VF1.

\checkitem VF2.

\checkitem VC2.

\checkitem VB4.

\checkitem VA4.


  
\item\checkbox Turn OFF power to Flowmeter, Stepper Motor panel and
main instrument panel.
  


\end{enumerate}




%------------------------------------------------------------
\newpage
\subsection{Adjusting Neutron Generator Target Position}
\shwlabel{proctargetladder}


\newprocedure{CalProcTargPos}
     {Adjusting Neutron Generator Target Position}
             {F. Duncan}
             {Aug. 2000}{1}

 
  This procedure describes how to change the position of
the neutron generator target ladder position.  This procedure
is used to move from the N16 target to the Li8 target or to
adjust the neutron capture efficiency on a target (for example if
it is desired to decrease the N16 rate).  The nominal target
positions for the N16 and Li8 are listed in table \ref{tabladder}.

\begin{table}[htbn]
\begin{center}
\begin{tabular}{|l|c|}
\hline
target & Position \\
\hline
N16    & 36.4 \\
\hline
Li8    & 101 \\
\hline
\end{tabular}
\caption[Neutron Generator Target Positions]
        {Neutron Generator Target Positions
        \shwlabel{tabladder}
        }
\end{center}
\end{table}

\subsubsection{Procedure}
 
\begin{center}
\begin{tabular} {|l|l|l|l|}
\hline
\multicolumn{4}{|c|}{\bf Adjusting Neutron Generator Target Position}\\
\hline
     &         &           &                   \\
Date & Initial & Procedure ~~~~~~~~~~~~~~~~~~~~~~~~~~~~~~~~~~~~~~~~~~~~&
 Data and Comments ~~~~~~~~~~~~~~~~~\\
     &         &           &                   \\
\hline
&& & \\
&& Turn on Power & \\
&& & \\
\hline
&& Run switch in down position & \\
&& & \\
\hline
&& & \\
&& Stop switch in down position & \\
\hline
&& & \\
&& in1 in up position & \\
&& & \\
\hline
&& & \\
&& in2 in up position & \\
&& & \\
\hline
&& & \\
&& in3 in up position & \\
&& & \\
\hline
&& & \\
&& speed set to HIGH & \\
&& & \\
\hline
&& Use jog forward/reverse to change position & \\
&& to desired position& \\
&& & \\
\hline
&& & \\
&& Turn off controller box & \\
&& & \\
\hline
\end{tabular}
\end{center}


\newpage


\subsection{N16 PMT Turn On Procedure}

\newprocedure{CalProcPMTTurnOn}
             {N16 PMT Turn On Procedure}
        {P. Skensved}{Jul. 2003}{1}



 This procedure describes how to turn on the N16 monitoring PMT.

\begin{enumerate}

\checkitem Verify that the that the HV controller is turned off and that the dial is turned fully CCW.

\checkitem Verify that the controller cable is plugged into the HV controller.
\checkitem Verify that the controller cable is connected to the ``dry'' end of the umbilical.

\checkitem Connect a fast oscilloscope to the linear PMT output at the ``dry'' end
of the umbilical. Terminate with ( real ) 50 ohm resistor. 
\checkitem Set scope to something like 
10 mV per division vertical, 10-20 ns per division horizontal, trigger on negative edge. Use
AUTO trigger at first to verify presence of baseline/noise.

\checkitem Turn on the NIM bin.

\checkitem Turn on HV switch.

\checkitem Start dialling ( slowly ! ) up the HV while watching the scope. You should see the glitches and changes in
baseline on the scope as you do so. If not stop immediately and check that the scope is setup correctly.

\checkitem Real PMT signals should be evident at a voltage setting at or above 2.00

\checkitem Slowly increase the setting to 2.37 while watching for signs of breakdown. If you see any 
{\bf stop immediately}  and reduce the voltage. Contact OCE.
\small
{\em There may be a tag attached to the power supply with a different voltage than the one 
listed above. Always use the value on the tag if there is one}
\normalsize 


\checkitem Disconnect the scope.


  The next steps need only be done if this procedure is executed as part of a general N16 procedure.

\checkitem Verify that the FECD input at 17/15/4 is disconnected. Leave the `tee'' and 50 ohm in place.


\checkitem Use the N16 custom cable to connect the N16 PMT to the correct spigot near the door.
Cable tie this cable  to the URM. Don't cable tie to the ``dry'' end part of the N16 cable !


\small
{\em It is important that this cable be as short as possible to get the correct trigger timing. Don't
just use any random cable you find. Use the correct cable.

  Also, the ``dry'' end of the N16 PMT cable is fragile. Do {\bf not} cable tie to it. Cable tie to the 
long cable instead}

\normalsize


\checkitem Do not connect 17/15/4 until permission has been obtained from the detector operator.


\end{enumerate}


\newpage

\subsection{N16 PMT Turn Off Procedure}

\newprocedure{CalProcPMTTurnOff}
             {N16 PMT Turn Off Procedure}
        {P. Skensved}{Jul. 2003}{1}


 This procedure describes how to turn off the N16 monitoring PMT.

\begin{enumerate}

\checkitem Disconnect the cable at 17/15/4. Leave the ``tee'' and the 50 ohm in place.


\checkitem Disconnect the cable going from the ``dry'' end to the patch  panel near the door.
Coil it up and hang it on the cable rack.

\checkitem Turn down ( slowly ! ) the HV control to a dial setting of zero,

\checkitem Turn off the HV supply

\checkitem Turn off the NIM bin
\small
{\em This also turns off the power to the pressure transducer. You may have to delay the last
two steps until  a later time.}

\normalsize

\end{enumerate}

