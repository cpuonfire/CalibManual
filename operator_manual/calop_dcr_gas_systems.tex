
\markright{CalOp: Calibration Gas Systems}

%========================================================================
%========================================================================
%========================================================================



\chapter{Calibration Gas Systems}
\shwlabel{ChapterCalGas}
  
  The SNO Detector calibration systems utilize several different
sources of gas in the DCR.  Shown in figure \ref{FigDcrGas},
these are:
\begin{description}

\item[Vacuum] There is a vacuum pump located in the Junction with
  a 2 inch (check) line terminating at a valve in the cable tray above
  the pipe box in the DCR.

\item[High Pressure N$_2$] gas derived from a 150 PSIG LN$_2$ dewar
  located in the Junction.  This N$_2$ gas has no significant amounts of
  oxygen or radon and is used to feed the Calibration N$_2$ laser
  and to supply (relatively) high pressure flushes of the 
  URMs.

\item[Low Pressure N$_2$] gas derived from the boil off of the
  Detector Cover Gas Wessington Dewar located in the Junction.  This
  gas has no significant amounts of oxygen or radon and is used
  to maintain a low rate flush of the URMs and the side rope units
  (located on the roof of the DCR).

\item[Instrument Air] compressed air taken from the laboratory 
  house air supply.  This is derived from the INCO compressed air
  with a booster compressor located outside the  car wash.  This 
  is compressed air containing the normal concentrations of oxygen
  and radon.  It is used to pressurize the tensioning cylinders in
  the URMs which provide the tension on the URM umbilicals.

\item[Radioactive Gas] There is a radioactive gas handling system
  located in the Junction.  This primarily used for the  $^{16}$N
  gamma ray source but is also used for the $^{8}$Li $\beta$
  source and the $^{17}$N neutron source.

\end{description}

 
\begin{figure}[htb]
\begin{center}
\leavevmode
\epsfxsize=7in
\epsfbox{figures/dcr_gas_systems.ps}
~\\
\caption[DCR Gas Systems]
        {DCR Gas Systems
         \shwlabel{FigDcrGas}
        }
\end{center}
\end{figure}


\section{(Proposed)Calibration Gas System Nomenclature}

  To distinguish the different gas sys handling systems for
the SNO Calibration equipment and to distinguish the calibration
gas systems from the other SNO systems a unique naming convention
is proposed for the calibration systems.  All Calibration 
gas components will start with the letter ``C''.


\begin{table}
\begin{center}
\begin{tabular}{|ll|}
\hline
  CMV  & Mechanical valve \\
  CSV  & Solenoid valve \\
  CRV  & Relief valve \\
  CPR  & Pressure Regulator \\
  CFM  & Flow meter \\
  CPG  & Pressure gauge (mechanical) \\
  CPT  & Pressure transducer (electronic readout) \\
  CP   & Pump\\
  CVP  & Vacuum Pump\\
  CCV  & Check valve\\
\hline
  100 series & Vacuum \\
  200 series & House Air \\
  300 series & Low Pressure N$_2$ \\
  400 series & High Pressure N$_2$ \\
  600 series & Calibration Laser \\
  800 series & Radioactive Gas \\
\hline
\end{tabular}
\caption[Proposed numbering scheme for gas system components]
  {Proposed numbering scheme for gas system components
   \shwlabel{TabGasNumbering}
  }
\end{center}
\end{table}

  
%========================================================================
\clearpage
\section{Nitrogen Flush System}
\shwlabel{SecN2Flush}
  
\subsection{Introduction}  

  The gas board is designed to allow the URMs and the side rope motor boxes
to be flushed with dry  radon free N$_2$. In normal mode  the gas flow to the
motorboxes and the URMs is restricted to a few liters per minute in order
not to perturb the D$_2$O covergas system and to conserve LN$_2$. For initial
flushing the flow may be increased to the URMs provided the gatevalve is closed.
This is referred to as `bypass mode' below.

 The gas enters the board on the left side and leaves through one or more exits
at the bottom. There are individual lines to each of the side rope motor boxes
located on the roof of the DCR and a common line to the URMs. This line is 
in series with a flow meter and a needle valve ( both located at the south 
east corner of the pipebox ). From there it branches out to the two URMs.


 
 {\em Note : The maximum allowed pressure for the flush system is 10 psi.
Do not under any circumstances exceed this pressure ! }

 


\begin{figure}[htb]
\begin{center}
\leavevmode
\epsfxsize=7in
\epsfbox{figures/flush_gas_panel.ps}
~\\
\caption[DCR Flush System]
        {N$_2$ Flush Gas Board
         \shwlabel{FigFlushGasBoard}
        }
\end{center}
\end{figure}

\subsection{Operation}

  N$_2$ to the board comes from one of two sources :

\begin{itemize}

\item Low pressure supply.

  In this mode the gas is supplied from the Wessington dewar located in 
the junction area.  The gas enters through the lower left input line and 
the pressure is controlled by a fixed regulator on the dewar.


\item High pressure supply.

 Here the gas supply is a 160 psi dewar located in the junction. This dewar
is also used for the laser. Gas enters the board through the top 
left line and flows through the regulator on the board which should be 
set at { \em no more than 10 psi}. Note that there is a checkvalve in  the 
low pressure line which prevents gas flowing  back to the Wessington. This 
particular valve appears to be `missing' on the board (~in case you were 
wondering ...~).


\end{itemize}



 As indicated above the gas board can be operated in one of two modes :

\begin{itemize}

\item ``Normal mode''

  In this mode the gas flows through small restrictions to the devices 
and the flow is limited to approximately 1 liter per minute. 


\item ``Bypass mode'' 

  This mode is used  for rapid flush of a URM.   Direct the gas to the bypass
line and set the three-way valve at the URM line appropriately.


\end{itemize}
  



  
%========================================================================
\clearpage
\section{Radioactive Gas System}
\shwlabel{SecRadioactiveGas}

 
\begin{figure}[htb]
\begin{center}
\leavevmode
\epsfxsize=7in
\epsfbox{figures/radioactive_gas_system.ps}
~\\
\caption[Radioactive Gas system]
        {Radioactive Gas Handling System
         \shwlabel{FigRadioactiveGas}
        }
\end{center}
\end{figure}



