




%-------------------------------------------------------------
%-------------------------------------------------------------
\newpage
\section{Calibration Tube \# 4 Deployment of N16 Source}
  
\begin{center}
\begin{tabular}{|l|l|}
\hline
Version    & 1.1 \\
\hline
Date       & 7 Dec 1999\\
\hline
Written by & F. Duncan\\
\hline
\end{tabular}
\end{center}
 
\noindent
{\bf Overview}

 
  This is a temporary procedure for the deployment of the N16
source down guide tube \# 4.  In is intended for the first deployment
of the source down this tube into the light water volume between
the AV and the PSUP.  Since this has not been done before, this procedure
will proceed cautiously.
  

  The only deployment of a source in a calibration guide tube
has been the lowering of the laserball into CT4 during air fill.
That deployment was done, guiding the source by hand with the
detector off and while visually verifying that the source was
centred in the tube.  That deployment showed that the source tube
was in fact not vertically aligned with the bottom of the tube
being approximately 2'' out of true wrt the top.  After water fill,
all source tubes were realigned with the condition that a 5 1/8''
source could pass through the tube without touching the sides.

  
  Ideally, a ``dry run'' of the deployment down the guide tube would
consist of lowering a dummy source with tapered ends down the guide 
tube without the source tube (the 4' tube that contains the
source when it is retracted) attached.  This would allow observation
of the dummy source and would help diagnose any unexpected problems.  
  Deployment of the N16 source without a dry run complicates the
the situation if there is a problem but should not pose additional
risks.
  
  The procedure will be to deploy the N16 source with a polypropolyne
cone attached above the source.  This will help guide the source
back into the guide tube in case of misalignment.  In the event that
the source gets stuck during the deployment, it will be necessary to
turn off the detector, unbolt the source tube from the gate valve
and raise the URM and source tube up approximately 2 ft to allow
an inspection of the source within the guide tube.

\begin{figure}
\begin{center}
\leavevmode
%\epsfysize=0.85\textheight
\epsfxsize=7in
\epsfbox{./figures/n16_guide_tube.ps}
~\\
\caption[N16 source in Source Tube with Cone]
        {N16 Source in Source Tube with Cone
         \shwlabel{fign16guidetube}
        } 
        
\end{center}
\end{figure}
  
 
\noindent
{\bf Risks}

  The largest risk with the deployment down the guide tube is
that the source gets stuck.  In the worst case we would not be
able to get it out and it would remain in the tube, possibly blocking
part of the PSUP indefinately.  This is probably a relatively low
risk.  The likely problems are:
\begin{description}
\item[Source hangs up on way down]~\\
  The source does not hang vertically from the rope since it is off centre.
  Since the source is not square in the guide tube and since the guide
  tubes are not exactly aligned, there is a chance that the source will
  catch on the way down. An indication that the source has hung up is
  if the tensions in either rope or umbilical drop.
  
  If this happens we will try putting most of the source load on the
  umbilical to straighten it in the tube.  If this does not get the source
  past the obstructions, then we will abort the run.

\item[Source gets stuck while being retracted]~\\
  While retracting source it hangs in the guide tube well below the
  deck.  
  
  The intent of the umbilical cone is to prevent such an event.  However,
  if it does happen, we will have to shut down the detector, open the
  guide tube and try to dislodge the source by hand.
  
\item[Source does not retract completely into source tube]~\\
  The source is deployed single axis and is prone to having the
  rope and umbilical twist.  This is prevented from being a problem
  in the detector by having a rotating collar for the rope attachment.
  However, the umbilical cone negates this, locking the rope and umbilical
  together.  If there is significant twist, we will not be able to lift
  the source all the way up and close the gate valve.
  
  If this happens, we will have to shut the detector off, open up
  and lift the source out by hand.  This should be straight forward
  but does require detector off.
\end{description}



%------------------------------------------
\newpage
~
\vspace*{0.5in}
\begin{center}
\begin{tabular}{|l|l|}
\hline
 & \\
DCR Floor Z position     & 1322.07 cm\\
 & \\
\hline
 & \\
Fiducial Mark to DCR floor & 140.7 cm \\
 & \\
\hline
& \\
Top of Cone to Pivot & 18.5 cm\\
& \\
\hline
Fiducial Calibration & \\
at Cone top          & 1466.27 cm\\
 & \\
\hline
 & \\
Bottom of Source entering PSUP & $z_{pivot}$=643.44 cm \\
 & \\
\hline
 & \\
Minimum Z position &  $z_{pivot}$= -465 cm \\
 & \\
\hline
 & \\
source centre at PSUP &  $z_{pivot}$= 635.82 cm \\
 & \\
\hline
 & \\
cone just below PSUP &  $z_{pivot}$= 583.14 cm \\
 & \\
\hline
 & \\
{\bf Minimum Safe Height to Close Gate Valve} &  $z_{pivot}$= 1451.6 cm \\
 & \\
\hline
\end{tabular}
\end{center}
  
%--------------------------------------
\newpage
%\vspace*{0.2in}
\noindent
{\bf State Prior To This Procedure}
 
\begin{center}
\begin{tabular} {|l|l|l|l|}
\hline
\multicolumn{4}{|c|}{\bf Prior to CT4 N16 Source Deployment}\\
\hline
     &         &           &                   \\
Date & Initial & State ~~~~~~~~~~~~~~~~~~~~~~~~~~~~~~~~~~~~~~~~~~~~&
 Data and Comments ~~~~~~~~~~~~~~~~~\\
     &         &           &                   \\
\hline
&& DCR Lights are on.  Owl tube light monitor is on and being & \\
&& monitored by detector operator.& \\
&& & \\
\hline
&& & \\
&& DCR has been cleaned.  Area around CT4 has been cleared. & \\
&& & \\
\hline
&& URM2 unmounted from glovebox and suspended over DCR floor & \\
&& using the lifting straps and cart.& \\
&& & \\
\hline
&& & \\
&& N16 source mounted on URM & \\
&& & \\
\hline
&& & \\
&& URM rope and umbilical calibrations ok. & \\
&& & \\
\hline
\end{tabular}
\end{center}
 
  
%--------------------------------------
\vspace*{0.2in}
\noindent
{\bf Summary of Procedure}
\begin{itemize}
\item Determine lowest allowed point to put source and
  the minimum height to raise source to close gate valve.
  Put tube coordinates in the manip program.
\item Lower N16 source out of source tube.
\item Attach guide cone to umbilical above manipulator carriage.
\item Retract N16 source into source tube
\item Disassemble lifting rails and support URM-2 with lifting cart.
\item Move URM2 over to guide tube \# 4.
\item Clean gate valve.
\item Attach source tube to gate valve.
\item Turn off DCR lights.  Open gate valve do a light check.
\item Lower source 10 cm, raise again.
\item Deploy source into detector.
\item Take data runs.
\item Retract source
\item Close gate valve.



\end{itemize}

 
%--------------------------------------
\newpage
\begin{center}
\begin{tabular} {|l|l|l|l|}
\hline
\multicolumn{4}{|c|}{\bf CT4 Deployment of N16 Source}\\
\hline
     &         &           &                   \\
Date & Initial & Procedure ~~~~~~~~~~~~~~~~~~~~~~~~~~~~~~~~~~~~~~~~~~~~&
 Data and Comments ~~~~~~~~~~~~~~~~~\\
     &         &           &                   \\
\hline
&& & \\
&& Determine Fiducial mark calibration of Source Tube 
 & $z_{fiducial}=$\\
&& & \\
\hline
&& & \\
&& Determine Minimum Safe height to close gate valve
 & $z_{safe}=$\\
&& & \\
\hline
&& Enter CT4 coordinates into {\bf axis.dat} & \\
&&  x= ????  y = ???? z = ???? &\\
&& & \\
\hline
&& & \\
&& Lower N16 Source from Source Tube. & \\
&& & \\
\hline
&& & \\
&& Attach guide cone to umbilical. & \\
&& & \\
\hline
&& & \\
&& Verify source and cone attachments & \\
&& & \\
\hline
&& & \\
&& Retract Source into Source tube. & \\
&& & \\
\hline
&& & \\
&& Set Source Clamps on Source tube into the {\bf HOLD} position & \\
&& & \\
\hline
&& & \\
&& Detatch URM from lifting straps, supporting it from lifting cart. & \\
&& & \\
\hline
&& & \\
&& Position URM and source tube over CT4. & \\
&& & \\
\hline
&& & \\
&& Clean CT4 gate valve with U/P water and lint free rag. &\\
&& & \\
\hline
&& Attach Source Tube to CT4 gate valve. & \\
&& Verify Tube is vertical in both N/S and E/W planes. & \\
&& & \\
\hline
&& {\bf Set Source Clamps on source tube into the} & \\
&& {\bf RELEASE position.} &\\
&& & \\
\hline
&& & \\
&& Verify all light seals on URM2 and source tube. & \\
&& & \\

\hline
\end{tabular}
\end{center}

 
%----------------------------------------------------------
\newpage
\begin{center}
\begin{tabular} {|l|l|l|l|}
\hline
\multicolumn{4}{|c|}{\bf CT4 Deployment of N16 Source (cont'd)}\\
\hline
     &         &           &                   \\
Date & Initial & Procedure ~~~~~~~~~~~~~~~~~~~~~~~~~~~~~~~~~~~~~~~~~~~~&
 Data and Comments ~~~~~~~~~~~~~~~~~\\
     &         &           &                   \\
\hline
&& & \\
&& Turn off DCR lights. &\\
&& & \\
\hline
&& & \\
&& Open CT4 gate valve. & \\
&& & \\
\hline
&& & \\
&& Perform light check on URM2 and source tube. & \\
&& & \\
\hline
&& & \\
&& Turn DCR lights back on. & \\
&& & \\
\hline
&& Lower Source into CT4 in 10 cm steps& \\
&& until the source is completely below the deck. & \\
&& (See  {\em Possible Problems}.) & \\
\hline
&& Lower source into the detector volume. & \\
&& $z_{pivot}$= 643.44& \\
&& & \\
\hline

\end{tabular}
\end{center}
 
 
%----------------------------------------------------------
\newpage
\begin{center}
\begin{tabular} {|l|l|l|l|}
\hline
\multicolumn{4}{|c|}{\bf CT4 Retraction of N16 Source (cont'd)}\\
\hline
     &         &           &                   \\
Date & Initial & Procedure ~~~~~~~~~~~~~~~~~~~~~~~~~~~~~~~~~~~~~~~~~~~~&
 Data and Comments ~~~~~~~~~~~~~~~~~\\
     &         &           &                   \\
\hline
&& Raise Source to beginning of PSUP. &\\
&&   $z_{pivot}$=540.0  & \\
&& & \\
\hline
&& Set URM2ROPE Speed to 0.5 cm/s &\\
&&   {\tt urm2rope maxspeed 0.5}  & \\
&& & \\
\hline
&& Raise Source in 10 cm steps to  & \\
&&   $z_{pivot}$=725cm & \\
&& & \\
\hline
&& Set URM2ROPE Speed to 3 cm/s &\\
&&   {\tt urm2rope maxspeed 3.0}  & \\
&& & \\
\hline
&& Raise Source to just below deck & \\
&&   $z_{pivot}$=1300.0 & \\
&& & \\
\hline
&& In 10cm or smaller steps raise source & \\
&& into source tube.& \\
&& {\bf Minimum safe height  $z_{pivot}$= 1451.6} \\
\hline
&& & \\
&& Close Gate valve & \\
&& & \\
\hline
&& & \\
&& Remove handle from gate valve & \\
&& & \\
\hline

\end{tabular}
\end{center}
 
\vspace*{0.2in}
\noindent
{\bf State At Completion Of This Procedure}\\
The N16 source is retracted from detector, gate valve closed.
  
%-----------------------------------------------------------------
\newpage
\vspace*{0.2in}
\noindent
{\bf Possible Problems}
\begin{description}
\item[Source does not enter CT4 from the source tube.]~\\
  The last deployment of the N16 source on the glove box found
it difficult to get the source out of the source tube into the
glovebox.  It is believed that the problem was caused by two
things:
\begin{itemize}
\item The source is suspended with most of its load on a rope
  that is not centred above the source's centre of mass.
\item The source tube was not put on the gate valve square.
\end{itemize}
These two problems resulted in the source being off centre and
the flat bottom of the source hitting the protruding edge of the
gate valve.  
  The solution was to transfer most of the load of the source to
the umbilical which {\em is} in line with the centre of mass.
This let the source hang vertically and it then passed through
the gate valve.
  
If this problem is encountered during the CT4 deployment:
\begin{enumerate}
\item transfer load from the rope to the umbilical.  Try 90 N on
  the umbilical, 40N on the rope.  Do this by putting the rope
  in tension mode with the command:
  \begin{verbatim}
       manip> urm2rope tension 40
  \end{verbatim}
\item Lower the source 10 cm into the CT by using the command:
  \begin{verbatim}
       manip> urm2umbilical down 10
  \end{verbatim}
\end{enumerate}  

\item[Source cannot be retracted into source tube above gate valve]~\\
  This is most likely caused by the rope twisting around the umbilical.
  Attempts can be made to remove the source by lowering it and raising
  it again --- perhaps at a lower speed.  If these attempts fail, the
  only option is:
  \begin{enumerate}
  \item Turn off the detector.
  \item Disconnect all HV cables.
  \item Lower source 1m.
  \item unbolt URM from CT4.
  \item Raise URM with lifting cart approximately 1m.
  \item raise source until it is in DCR.  Hand feed it into
        source tube.
  \item When source is clear of gate valve.  Close gate valve.
  \end{enumerate}
  



\end{description}




