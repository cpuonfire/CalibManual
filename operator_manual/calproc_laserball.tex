

%------------------------------------------------------------------------
%------------------------------------------------------------------------
%------------------------------------------------------------------------
\section{Laser Procedures}
\shwlabel{secproclaserball}
   
\begin{figure}[htb]
\begin{center}
\leavevmode
%\epsfysize=0.85\textheight
\epsfxsize=7in
\epsfbox{./figures/lasergas.eps}
~\\
\caption[Laser Gas System]
        {Laser Gas System
         \shwlabel{figlasergas}
        }
\end{center}
\end{figure}
  
  
\subsection{Starting the Laser}
\begin{enumerate}
\item Open MV1 the manual valve on the LN$_2$ dewar.
\item Open MV2 the pressure builder valve on the LN$_2$ dewar.
\item Note pressure on PG1, the pressure gauge on the dewar.
      The pressure should be at least 120 psig.
\item Note time.  {\bf The laser must NOT be turned on until 20 min
       after gas flow has started.}
\item Enter DCR and note values on PG2 and PG3 and FG1.
  PG2 should be at least 120 psig, PG3 should be at least 100 PSIG
  FG1 should be at least 40.
\item Test control of the laser by changing the mirror position
\item Using either manip computer or manmon determine PT4 and PT5.
  PT4 should be at least 80 psi.
\end{enumerate}
  
\subsection{Calibrating the Filter Wheels}
The control system can sometimes loose track
of where the filter wheels are positioned.
To fix this follow this procedure.  This example
is for {\tt filterwheela}, replace with 
{\tt filterwheelb} for the 2nd filterwheel.
\begin{enumerate}
\item Reinitialize the filterwheel
  \begin{verbatim}
  filterwheela init
  \end{verbatim}
\item Find the tab on the wheel
  \begin{verbatim}
  filterwheela findtab
  \end{verbatim}
\item select the desired filterwheel position using
  the {\tt n2laser setnd} command.
\end{enumerate}

\subsection{Calibrating the Dye Cell Mirror}
\begin{enumerate}
\item Reinitialize the mirror
  \begin{verbatim}
   dyelaser init
  \end{verbatim}
\item Find the zero of the mirror
  \begin{verbatim}
  dyelaser findzero
  \end{verbatim}
\item Select desired wavelength
  \begin{verbatim}
  dyelaser cell <0-9>
  \end{verbatim}
\end{enumerate}


%---------------------------------------
%\newpage 
%\vspace*{0.2in}
%\noindent
%{\bf State At Completion Of This Procedure}\\
%Laserball is assembled and ready for retraction into source tube.


%------------------------------------------------------------------------
%------------------------------------------------------------------------
%------------------------------------------------------------------------




\newpage

\subsection{Assembly of Mark III Laserball}
  
\begin{center}
\begin{tabular}{|l|l|}
\hline
Version    & 0.9 \\
\hline
Date       & Nov 2002\\
\hline
Written by & P. Skensved\\
\hline
\end{tabular}
\end{center}
 
\subsubsection{Overview}

  This is a procedure describing the assembly of the
Mark III laserball. In order to ensure that there are no leaks and
that the laserball is securely attached to the umbilical
it is important that correct size o-rings are used everywhere. The
assembly uses small screws and nuts in many places. These are easily
stripped and damaged. {\bf Do not overtighten ! And do not re-tighten
just to make ``sure'' !}  If you do not know what the proper torque 
should be contect an expert. 

  If there are holes drilled through the screws secrure them with wire.
Do not twist or flex the wire more than necessary and make sure you do not
leave any weakened pieces of wire  as
they may end up in the detector. Re-do the wire instead. Bend the ends so
that they will not poke holes in the gloves.

  Do not ever attempt to remove the shroud around the laserball or undo
the 2 inch cajon fitting attached to the glass ball itself. You will disturb
the alignment of the fiber. 

\begin{figure}[p]
\begin{center}
\leavevmode
\epsfxsize=6.0in
%%%%%%%%%\epsfbox{./figures/laserball_exploded3.ps}
\caption[View of Mark III laserball]{
  View of Mark III laserball.
  \shwlabel{figexploded3}
  }
\end{center}
\end{figure}


%-------------------------------------------------------------------------
%-------------------------------------------------------------------------
%-------------------------------------------------------------------------
\subsubsection{Prior To This Procedure}
  \begin{itemize}
  \item The rope may or may not be attached to the rotating bearing.
  \item The carriage may or may not be attached to weight cylinder.
  \item The spool piece may or may not be attached to weight cylinder.
  \item The can may or may not be attached to the spool piece.
  \item The can may or may not be assembled.
  \item The laserball may or may not be attached to the can
  \item The salt probe may or may not be installed and connected.
  \end{itemize}

\newpage

\subsection{Assembly}

\begin{enumerate}
\checkitem Tie the rope to the rotating bearing. {\bf This is to be done by an expert only !}
\checkitem Slide the rotating bearing on the umbilical.
\checkitem Slide the o-ring on the umbilical. Use correct size o-ring.
\checkitem Attach the carriage to the weight cylinder. Secure the nuts with wire.
\checkitem Attach the spool piece to the weight cylinder. Secure the nuts with wire.
\checkitem Slide the carriage, weight cylinder and spool piece  on the umbilical.
\checkitem Slide a spacer with groove facing down on the umbilical.
\checkitem Slide the o-ring on the umbilical.
\checkitem Slide the stainless steel lid on the umbilical.
\checkitem Slide the o-ring on the umbilical.
\checkitem Slide the acrylic lid on the umbilical.
\checkitem Tie the wires in a  `Hallinian' knot and secure with cable ties. {\bf This is to
be done by an expert only !}
\checkitem Pull the knot up against the lid ( Gently !!! )
\checkitem Make sure the acrylic plate o-ring is in place.

     The can is designed to be used with a salt probe or a dummy plug. Use the
appropriate section for the next few steps. The can also holds a blue LED which
is used to index the laserball in the H$_2$O. Note that only one electrical
device can be connected at any given time. 

\checkitem Assemble saltprobe.
\begin{enumerate}

\item Mount the inner clamp around the salt probe.
\item Put the salt probe into the can.
\item Slide the o-ring up atround the saltprobe.
\item Connect the wires to the saltprobe. The colour codes are listed in the log book.
item Make an entry in the logbook stating that the saltprobe is now connected to the umbilical.
\end{enumerate}

\checkitem Assemble plug.
\begin{enumerate}
\item Put the plug in the bottom of the can.
\item Slide the o-ring up around the plug.
\end{enumerate}

\checkitem LED assembly.
\begin{enumerate}
\item Press the LED into place. Be careful not to break anything.
\item Connect the wires to the LED. The colour codes are listed in the logbook.
\item Make an entry in the logbook stating that the LED is now connected to the umbilical
\end{enumerate}

\checkitem Feed the fibers through the can and through the hole in the bottom of the can.
\checkitem Make sure the wires and the fibers are placed correctly inside the can. The prefered
routing of the fiber is in a helix so which is free to move up and down with the optical
coupling.
\checkitem Attach the acrylic lid to the can with 4 screws. Ensure that the o-ring is in place.
Make sure you're using the correct holes in the plate and make sure you're using the correct
length  screws.
\checkitem Slide the umbilical o-ring up against the acrylic lid.
\checkitem Check that everything looks ok inside the can
\checkitem Slide the stainless steel lid into place. Attach with screws. Use correct screws.
\checkitem Slide the second o-ring into place.
\checkitem Slide the pressure plate into place.
\checkitem Attach the can to the spool piece using 5 screws. Use correct screws.
\checkitem Make sure the o-rings  are in place ( one for the saltprobe / plug
and another for the center ).
\checkitem Push the fiber in to the CAJON  fitting on the laserball. Tighten the fitting ligthly. 
Make sure you don't twist or otherwise disturb the solid fiber on the laserball itself and make sure the fiber end is through
the o-ring.
\checkitem Mount the laserball on the can. Make sure the fiber is not bent or pinched
in any way. Use correct screws.
\checkitem Slide clamp up around saltprobe or plug. Make sure it is oriented correctly ( so that it fits ) and
tighten screw appropriately. ( {\bf Do not overtigthen  } ! )

\checkitem Slide the o-ring for the rotating bearing into place.
\checkitem Attach the rotating bearing to the carriage with 4 screws and secure them
with wire. Do not overtighten !!!  Do not pinch or damage the o-ring.


\end{enumerate}



%------------------------------------------------------------------------
%------------------------------------------------------------------------
%------------------------------------------------------------------------
\newpage
\subsection{Disassembly of Mark III Laserball}
\shwlabel{procpca}~\\
\noindent
\begin{tabular}{|l|l|}
\hline
Version              & 0.9 \\
\hline
Written/Revised by   & P. Skensved \\
\hline
Date Written/Revised & Nov 2002\\
\hline
\end{tabular}
 
