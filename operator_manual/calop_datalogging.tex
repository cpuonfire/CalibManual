
%--------------------------------------------------------------
%--------------------------------------------------------------
%--------------------------------------------------------------
\chapter{Data Logging}
\shwlabel{ChapterDataLogging}
  
  There are several means by which data from the manipulator
system are logged.
\begin{itemize}
\item {\bf manip\_logger} web based logging tool
\item CMA logging of AV rope lengths
\item Command and Error logging on {\bf manip}
\item Explicit logging of source data through {\bf manmon}
\end{itemize}
   

\section{manip\_logger web based data logging}

\begin{verbatim}

Fraser,

We found a network card and got the MANIP test computer
running here again.  I've tested the new logging feature
and it seems to work well (at least for the salt probe).

The new manip_logger is now running on polaris.

The new feature is very flexible.  You can configure polling
objects dynamically without restarting the logger.  All you
do is edit the file "extra_log.dat" in the manip_logger
directory (on polaris in the online private web pages).

Here is my documentation:

--------------------------------------------------------------- 

This is the current "extra_log.dat" file:

salt {
state = off;
delay = 60; // delay time (secs)
entry = "TESTROPELOADCELL", "Tension:", 1;
entry = "LASERBALL", "Position:", 3;
}

You can add up to 20 entries similar to this one.

To enable the above polling, simply edit this file and change
the state from "off" to "on".

This entry produces a log file called "salt_YYMM.log2" with
the following format:

2001 7 17 12 49 43 94.12 9.00 -25.00 0.00
2001 7 17 12 50 41 94.12 9.00 -25.00 0.00
2001 7 17 12 51 41 94.12 9.00 -25.00 0.00
2001 7 17 12 52 41 94.12 9.00 -25.00 0.00

The first 6 lines are year, month, day, hour, minute, second,
as with the other log2 files.  The next lines are those
specified by the "entry" lines in "extra_log.dat".  Here,
it has logged the testropeloadcell tension and the laserball
position.

Here are more details about the format of "extra_log.dat":

state - "on" or "off" to enable or disable logging for this item.

delay - nominal delay between log entries in seconds (actual
delay implemented is the nearest multiple of 10 seconds).

entry - The first argument is the object to poll (case
insensitive).  The 2nd argument is the case-sensitive string
to match the specific parameter of the "monitor" command to
be logged.  This must be an exact match of all characters
up to and including the colon.  The 3rd argument is the number
of values to log for this parameter (ie.  x,y,z position is
3 values).

The manip_logger checks every logging cycle (10 seconds) to
see if this file has been modified.  If it has, the new
settings are loaded.

---------------------------------------------------------------

Hopefully this makes sense.  And hopefully it will work OK
for you.  If you have any problems, you can always go back
to the old version which I've moved to private/manip_logger/old.

I'll be back in 3 weeks.  Good luck.

 - Phil


\end{verbatim}




