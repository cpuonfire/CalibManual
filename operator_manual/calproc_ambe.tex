

  
%------------------------------------------------------------------------
%------------------------------------------------------------------------
%------------------------------------------------------------------------
\subsection{AmBe Source Procedures}
\shwlabel{secprocAmBe}


 The AmBe Neutron Source is an Americium-Berylium encapsulated source
of neutrons used to calibrate the detector neutron capture efficiency.
It consists of a comercial Amercium Source on a small wafer that produces
$\alpha$ particles which capture on a Berylium disk producin neutrons
in a ($\alpha$,n) interaction on the Be.  


\begin{table}[htb]
\begin{center}
\begin{tabular}{|l|c|}
\hline
Assembled Weight & ~~~~~~~~ 64N ~~~~~~~~\\
\hline
Volume           & ~~~~~~~~~~~~~~~~~~~~~\\
(including weight and carriage) & \\
\hline
Pivot Centre Offset & 76.6 cm\\
\hline
Pivot Bottom Offset & 77.3 cm\\
\hline
\end{tabular}
\caption[AmBe Neutron Source]
  {AmBe Neutron Source
   \shwlabel{TabSourceAmBe}
  }
\end{center}
\end{table}



\clearpage
\begin{figure}
\begin{center}
\epsfxsize=7in
\epsfbox{./figures/ambe_exploded.ps}
~\\
\caption[Assembly drawing of the manipulator mounted AmBe Source]
        {Assembly drawing of the manipulator mounted AmBe Source 
         showing from the Laserball Canister down.
         \shwlabel{figAmBe}
        } 
        
\end{center}
\end{figure}




\clearpage

\begin{figure}[t]
\begin{center}
\leavevmode
\epsfxsize=5.0in
\epsfbox{figures/Photo_AmBe_Assembled.eps}
\caption[AmBe Source Assembled]{
  \shwlabel{PhotoAmBeAssembled}}
  The Assembled AmBe Source
\end{center}
\end{figure}
\begin{figure}[b]
\begin{center}
\leavevmode
\epsfxsize=5.0in
\epsfbox{figures/Photo_AmBe_SealPlate.eps}
\caption[AmBe Source Sealplate]{
  \shwlabel{PhotoAmBeSealPlate}}
  The AmBe Sealplate.  The Cajon fitting, the fibre and
  the o-rings (around salt probe and around base of the stud)
  can be seen.
\end{center}
\end{figure}



%------------------------------------------------------------
\clearpage

\subsection{AmBe Source Assembly Procedure}
\newprocedure{CalProcAmBeAssembly}
             {AmBe Source Assembly Procedure}
             {Fraser Duncan}{2002/11/08}{1}

  The AmBe is similar to the acrylic encapsulated sources in that
it is mounted on a stem that mounts on the Laserball source's canister.
instead of the Laserball itself.  Thus there are two paths to
the assembly of the AmBe source.  
\begin{enumerate}
\item The Laserball is mounted on a URM.  The Laserball is removed and
  the AmBe source is mounted in it's place.
\item Full assembly of all the source components for the AmBe on the
  URM.
\end{enumerate}
While the AmBe source mounting is similar to the acrylic source mounts,
there are two important differences:
\begin{itemize}
\item The mounting stem for the AmBe source is made from Teflon to
  (low hydrogen content) to minimize neutron capture on the mounting
   hardware.
\item A stainless steel disk is sandwitched between the stem and the
  Laserball canister to form the water seal on the canister. This is
  because the teflon stem does not make a good water seal.  It is
  very important to ensure the canister is in place.
\end{itemize}



{\em Temporary Note:  This procedure describes the assembly of the
  source starting with the assembled Laserball and ending with the
  assembled AmBe source.}






\noindent
{\bf State Prior To This Procedure:}
\begin{enumerate}
\item URM2 has been unmounted from the glovebox and rolled back
\item Source has been lowered out of the URM source tube.
\item The laserball canister is mounted on URM2 but the
  Laserball stem and ball have been removed.
\item The fibre for the Laserball is protruding from the bottom
  of the laserball cansister.
\end{enumerate}




\noindent
{\bf Procedure:}
~\\
\begin{tabular}{|l|l|}
\hline
 & \\
Operator(s):~~~~~~~~~~~~~~~~~~~~~~~~~~~~~~~~~~~~~~~~~~~~~ 
 & Date: ~~~~~~~~~~~~~~~~~~~~~~~~~~~~~~~~\\
 & \\
\hline
\end{tabular} 
~\\
~\\
\begin{enumerate}

\checkitem Put o-ring in the {\em Seal Plate} around the stud.

\checkitem Verify that the o-ring around the salt probe is present
  in the bottom of the Laserball Cansiter.

\checkitem Put the {\em cajun fitting}  
  on to the {\em stud} on the {\em seal plate}.

\checkitem Tighten the {\em cajun fitting} onto the {\em stud}.

\checkitem Loosen the top of the {\em cajun fitting}

\checkitem Push the {\em Laserball fibre} into the top of the
  {\em cajun fitting}.\\
  {\em There will be resistance as the fibre passed through the
    o-ring inside the  cajun fitting.}

\checkitem Tighten the {\em cajun fitting} finger tight.\\
  {\small\em At this point the assembly should look like the
   photograph in figure \ref{PhotoAmBeSealPlate}.}

\checkitem Slide the {\em seal plate} up against the 
   {\em Laserball canister}.\\
  {\em As you do this make sure that both the o-ring on the seal plate
    around the stud and the oring on the canister around the salt probe
    remain seated.}

\checkitem Slide AmBe {\em stem} over the salt probe and ``snug'' it up
  against the {\em seal plate}.

\checkitem Slide the {\em compression plate} over the {\em stem} and
  the {\em salt probe} and ``snug'' it up against the base of the {\em stem}.

\checkitem Put the 4 1'' screws through the {\em compression plate},
   base of the {\em stem} and {\em seal plate} and thread them into
   the {\em Laserball cansiter}.

\checkitem Align the  hole in the salt probe to be tangential to
  the {\em stem}.

\checkitem Tighten the 4 1'' screws holding on the {\em stem}.

\checkitem Slide the salt probe {\em clamp} down the salt probe as ``snug''
  it up against the {\em compression plate}. \\
  {\em There is a flat spot on the clamp that fits against the stem.}

\checkitem Tighten the set screw on the {\em clamp}.

\checkitem Remove the AmBe {\em source container} from the Radioactive Source
  cabinet.

\checkitem Sign out the AmBe source in the {\em Radioactive Source Log Book}.

\checkitem  Inspect the three screws on the side of the
  AmBe {\em source container}.  All three should be present and
  fully threaded in.

\checkitem One at a time install the three mounting screws that hold the
  AmBe source to the stem.
  \begin{enumerate}
  \item Place the nut in the slot on the lid of the source.
  \item Push the cap screw with washer through the stem and source lid
    into the nut.
  \item partially tighten the cap screw into the nut.\\
    {\em It is probably necessary to use a small screw driver to hold
      the nut in place while tightening the screw.}
  \end{enumerate}
  
\checkitem Tighten all three mounting screws.

\checkitem Inspect all screws on the source assembly:
  \begin{itemize}
  \item The horizontal sealing screws in the source container.
  \item The 3 mounting screws in the holding the source to the stem.
  \item The 4 screws holding the stem to the Laserball Canister.
  \item The screws on the Laserball Canister, Weight Cylinder and
     Carriage.
  \end{itemize}

\checkitem Inspect the knot tying the URM rope to the carriage.

\checkitem Check the zero tension in the URM rope.  If it is not zero,
  recalibrate the zero offset.

\checkitem Retract the rope and umbilical until the source is hanging
  from the URM.
 
\checkitem Note the weight of the source.  It should agree with the
  value in table \ref{TabSourceAmBe}.

\checkitem Retract the source back into the URM.

\checkitem Calibrate the central rope length.




\end{enumerate}



{\small
~\\
~\\
\noindent
{\bf Revision History:}\\
\begin{tabular}{llll}
Rev. & Date & Author & Comments\\

0             & 
2002/11/08    & 
Fraser Duncan &
\parbox[t]{3.0in}{
  First Draft
}
\end{tabular}
}





%========================================================================
%========================================================================
%========================================================================

\newpage
\markright{\standardheader}



