

  
%------------------------------------------------------------------------
%------------------------------------------------------------------------
%------------------------------------------------------------------------
\section{Rn Spike Source Procedures}
\shwlabel{secprocRnSpike}



  The Rn Spike Source is a device for injecting Rn loaded D$_2$O 
into the detector. A small pump is used to transfer the activated water from 
an acrylic cylinder ( the resevoir ) through a teflon tube located
inside the umbilical down to the`source' itself.




\begin{table}[htb]
\begin{center}
\begin{tabular}{|l|c|}
\hline
Assembled Weight & ~~~~~~~~ 64N ~~~~~~~~\\
\hline
Volume           & ~~~~~~~~~~~~~~~~~~~~~\\
(including weight and carriage) & \\
\hline
Pivot Centre Offset & 80 cm\\
\hline
Pivot Bottom Offset & 80 cm\\
\hline
\end{tabular}
\caption[Rn Spike Source]
  {Rn Spike Source
   \shwlabel{TabSourceRnSpike}
  }
\end{center}
\end{table}



\clearpage
\begin{figure}
\begin{center}
\epsfxsize=7in
\epsfbox{./figures/rn_spike_exploded.ps}
~\\
\caption[Assembly drawing of the manipulator mounted Rn Spike Source]
        {Assembly drawing of the manipulator mounted Rn Spike Source 
         showing from the Canister down.
         \shwlabel{figRnSpike}
        } 
        
\end{center}
\end{figure}




\clearpage

\begin{figure}[t]
\begin{center}
\leavevmode
\epsfxsize=5.0in
\epsfbox{figures/Photo_RnSpike_Assembled.eps}
\caption[Rn Spike Source Assembled]{
  \shwlabel{PhotoRnSpikeAssembled}}
  The Assembled Rn Spike Source
\end{center}
\end{figure}


%------------------------------------------------------------
\clearpage

\subsection{Rn Spike Source Assembly Procedure}
\newprocedure{CalProcRnSpikeAssembly}
             {Rn Spike Source Assembly Procedure}
             {Peter Skensved}{2003/01/31}{1}

  The Rn Spike Source is similar to the acrylic encapsulated sources in that
the umbilical terminates in a stainless steel can sealed with o-rings.
The inner capilliary carrying the Rn loaded water is coupled to a thin
stainless tube exiting through the bottom of the can. The couplings and the
seal is done with Cajon fittings. A screen at the bottom of the tube acts
as a diffuser. The top of the can attaches to the weight cylinder identically
to all the other sources.




\noindent
{\bf State Prior To This Procedure:}
\begin{enumerate}
\item URM1 has been unmounted from the glovebox and rolled back
\item The umbilical has been lowered out of the URM source tube.
\end{enumerate}




\noindent
{\bf Procedure:}
~\\
\begin{tabular}{|l|l|}
\hline
 & \\
Operator(s):~~~~~~~~~~~~~~~~~~~~~~~~~~~~~~~~~~~~~~~~~~~~~ 
 & Date: ~~~~~~~~~~~~~~~~~~~~~~~~~~~~~~~~\\
 & \\
\hline
\end{tabular} 
~\\
~\\
\begin{enumerate}

\checkitem Slide the rotating bearing over the umbilical.

\checkitem Slide the o-ring over the umbilical.

\checkitem Attach the carriage to the top of the weight cylinder. Secure the
nuts with locking wire.

\checkitem Slide the carriage and weight cylinder over the umbilical.

\checkitem Slide the spool piece over the umbilical. Attach to weight 
cylinder and secure the nuts with locking wire.

\checkitem Slide the top stainless steel spacer over the umbilical.

\checkitem Slide the o-ring over the umbilical

\chekkitem Slide the bottom stainless steel spacer over the 
umbilical, grooveside up.

\checkitem Slide the bottom o-ring over the umbilical.

\checkitem Slide the lid over the umbilical.






\checkitem Put o-ring in the {\em Seal Plate} around the stud.

\checkitem Verify that the o-ring around the salt probe is present
  in the bottom of the Laserball Cansiter.

\checkitem Put the {\em cajun fitting}  
  on to the {\em stud} on the {\em seal plate}.

\checkitem Tighten the {\em cajun fitting} onto the {\em stud}.

\checkitem Loosen the top of the {\em cajun fitting}

\checkitem Push the {\em Laserball fibre} into the top of the
  {\em cajun fitting}.\\
  {\em There will be resistance as the fibre passed through the
    o-ring inside the  cajun fitting.}

\checkitem Tighten the {\em cajun fitting} finger tight.\\
  {\small\em At this point the assembly should look like the
   photograph in figure \ref{PhotoAmBeSealPlate}.}

\checkitem Slide the {\em seal plate} up against the 
   {\em Laserball canister}.\\
  {\em As you do this make sure that both the o-ring on the seal plate
    around the stud and the oring on the canister around the salt probe
    remain seated.}

\checkitem Slide AmBe {\em stem} over the salt probe and ``snug'' it up
  against the {\em seal plate}.

\checkitem Slide the {\em compression plate} over the {\em stem} and
  the {\em salt probe} and ``snug'' it up against the base of the {\em stem}.

\checkitem Put the 4 1'' screws through the {\em compression plate},
   base of the {\em stem} and {\em seal plate} and thread them into
   the {\em Laserball cansiter}.

\checkitem Align the  hole in the salt probe to be tangential to
  the {\em stem}.

\checkitem Tighten the 4 1'' screws holding on the {\em stem}.

\checkitem Slide the salt probe {\em clamp} down the salt probe as ``snug''
  it up against the {\em compression plate}. \\
  {\em There is a flat spot on the clamp that fits against the stem.}

\checkitem Tighten the set screw on the {\em clamp}.

\checkitem Remove the AmBe {\em source container} from the Radioactive Source
  cabinet.

\checkitem Sign out the AmBe source in the {\em Radioactive Source Log Book}.

\checkitem  Inspect the three screws on the side of the
  AmBe {\em source container}.  All three should be present and
  fully threaded in.

\checkitem One at a time install the three mounting screws that hold the
  AmBe source to the stem.
  \begin{enumerate}
  \item Place the nut in the slot on the lid of the source.
  \item Push the cap screw with washer through the stem and source lid
    into the nut.
  \item partially tighten the cap screw into the nut.\\
    {\em It is probably necessary to use a small screw driver to hold
      the nut in place while tightening the screw.}
  \end{enumerate}
  
\checkitem Tighten all three mounting screws.

\checkitem Inspect all screws on the source assembly:
  \begin{itemize}
  \item The horizontal sealing screws in the source container.
  \item The 3 mounting screws in the holding the source to the stem.
  \item The 4 screws holding the stem to the Laserball Canister.
  \item The screws on the Laserball Canister, Weight Cylinder and
     Carriage.
  \end{itemize}

\checkitem Inspect the knot tying the URM rope to the carriage.

\checkitem Check the zero tension in the URM rope.  If it is not zero,
  recalibrate the zero offset.

\checkitem Retract the rope and umbilical until the source is hanging
  from the URM.
 
\checkitem Note the weight of the source.  It should agree with the
  value in table \ref{TabSourceAmBe}.

\checkitem Retract the source back into the URM.

\checkitem Calibrate the central rope length.




\end{enumerate}



{\small
~\\
~\\
\noindent
{\bf Revision History:}\\
\begin{tabular}{llll}
Rev. & Date & Author & Comments\\

0             & 
2002/11/08    & 
Fraser Duncan &
\parbox[t]{3.0in}{
  First Draft
}
\end{tabular}
}





%========================================================================
%========================================================================
%========================================================================

\newpage
\markright{\standardheader}



