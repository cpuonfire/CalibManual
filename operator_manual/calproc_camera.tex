

  
%------------------------------------------------------------------------
%------------------------------------------------------------------------
%------------------------------------------------------------------------
\section{Manipulator Camera Procedures}
\shwlabel{secprocCamera}

  These procedures describe assembly and operation of the manipulator
mounted camera.

\begin{table}[htb]
\begin{center}
\begin{tabular}{|l|r|}
\hline
Assembled Weight & ~~~~~~~~~~~~~~~~~~~~~\\
\hline
Volume           & ~~~~~~~~~~~~~~~~~~~~~\\
(including weight and carriage) & \\
\hline
Pivot Centre Offset & \\
\hline
Pivot Bottom Offset & \\
\hline
\end{tabular}
\caption[Calibration Device:Manipulator Camera]
  {Calibration Device: Manipulator Camera
   \shwlabel{CalDevCamera}
  }
\end{center}
\end{table}



\clearpage
\begin{figure}
\begin{center}
\framebox{\Huge\bf Drawing of camera}
%\leavevmode
%\epsfxsize=7in
%\epsfbox{../figures/n16_partial_exploded.ps}
~\\
\caption[Assembly drawing of the manipulator mounted camera and mount]
        {Assembly drawing of the manipulator mounted camera and
         it's mount.
         \shwlabel{figcamera}
        } 
        
\end{center}
\end{figure}




%------------------------------------------------------------
\clearpage

\subsection{Manipulator Camera Assembly Procedure}
\newprocedure{CalProcCameraAssembly}
             {Manipulator Camera Assembly Procedure}
             {Fraser Duncan}{2002/09/26}{1}

\begin{enumerate}

\checkitem If it does not already exist, create a {\bf camera}
  polyaxis object in {\tt polyaxis.dat}.

\checkitem Mount spool piece on weight cylinder

\checkitem Mount carriage on weight cylinder

\checkitem Slide rotating bearing, o-ring and weight cylinder onto umbilical

\checkitem Slide top pressure plate, o-ring, bottom pressure 
           plate and o-ring onto umbilical

\checkitem Slide top cover onto umbilical

\checkitem Tie `Hallinian' knot on umbilical

\checkitem Slide o-rings, pressure plates and spool piece into place

\checkitem Do up associated nuts

\checkitem Slide o-ring and rotating bearing into place

\checkitem Do up screws

\checkitem Secure them with wire

\checkitem Tie central rope to rotating bearing ( standard figure 8 knot )

\checkitem Push camera into housing ( make sure both o-rings are in place )

\checkitem Slide o-ring and top cover over wires

\checkitem Fasten top cover

\checkitem Slide tube over wires

\checkitem Feed wires though bottom hole on can

\checkitem Mount camera housing on can

\checkitem Secure knurled nuts with locking wire

\checkitem Secure tubing with cable ties

\checkitem Do up connection inside can ( make sure o-ring is in place )

\checkitem Test camera:
  \begin{enumerate}
  \checkitem Connect umbilical to camera controller.
  \checkitem Connect camera controller to monitor and VCR.
  \checkitem Turn on camera and attempt to image appropriate object
     below URM.
  \checkitem Record image on video tape.
  \checkitem Play back video tape to verify recording.
  \checkitem Disconnect umbilical from camera.
  \end{enumerate}

\checkitem {\bf Do Not Operate}
  Status tag put dry end of umbilical to prevent connection
  of power to camera lights.

\checkitem Fasten lid on can

\checkitem Measure:
     \begin{center}
     \begin{tabular}{|l|}
     \hline
      \\
     carriage pivot to bottom of camera:~~~~~~~~~~~~~~~~~~~~~~~~\\
      \\
     \hline
      \\
     carriage pivot to camera pivot:~~~~~~~~~~~~~~~~~~~~~~~~\\
      \\
     \hline
     \end{tabular}
     \end{center}
\checkitem Record above distances in log book.

\checkitem Set the pivot-pivot offset in the {\bf camera} polyaxis
  object in {\tt polyaxis.dat}.

\checkitem Set the pivot-bottom distance in the {\bf camera} polyaxis
  object in {\tt polyaxis.dat} to 2cm greater than the above
  pivot-bottom distance.

\checkitem Clean and Inspect everything

\checkitem Mount guide tube cone

\checkitem Measure
     \begin{center}
     \begin{tabular}{|l|}
     \hline
      \\
     top of cone to bottom of camera:~~~~~~~~~~~~~~~~~~~~~~~~\\
      \\
      \\
     \hline
     \end{tabular}
     \end{center}
\checkitem Record above measurement in log book.

\checkitem Clean and inspect everything

\checkitem Adjust the camera to the desired orientation.\\
  {\em This is probably straight down for the first deployment
   to look for the Berkely Blob.}

\checkitem Retract the camera into the URM and calibrate the central
  rope.


\end{enumerate}



{\small
~\\
~\\
\noindent
{\bf Revision History:}\\
\begin{tabular}{llll}
Rev. & Date & Author & Comments\\
0           & 
2002/09/26  & 
Peter Skensved &
\parbox[t]{3.0in}{
  First draft
}\\

1             & 
2002/09/26    & 
Fraser Duncan &
\parbox[t]{3.0in}{
  Slight format changes.  Fleshed out camera test procedure.
}
\end{tabular}
}




%------------------------------------------------------------
\clearpage

\subsection{Manipulator Camera Deployment Down Guidetube \# 1}
\newprocedure{CalProcCameraDeployment}
             {Manipulator Camera Deployment Down Guidetube \# 1}
             {Fraser Duncan}{2002/09/26}{1}


\noindent
  The deployment is like any other source in the guide tube :

Since the clearance bewteen the bottom of the camera and the gate valve
special care must be exercised so as not to decapitate the source ie.
distances in z has to be double checked before the valve is closed.

Special care must be while driving the source in the guide tube since there
is a non-zero chance of getting the source caught especially at the entrance
to the guide tube, at the `knee' and at the valve. An experienced operator
has to be present at all times.


Since we will be driving the camera close to the acrylic vessel we will have to
override the minimum distances set in MANIP. It is important that the 
defaults be re-established after this deployment. This is an `expert only' 
task.


\begin{table}
\begin{center}
\begin{tabular}{|l|r|}
\hline
                  &   Z (cm) \\
\hline
  & \\
Entrance to PSUP  &  734.4  \\
  & \\
\hline
  & \\
Top of AV         &  446.9  \\
  & \\
\hline
\end{tabular}
\caption[Camera deployment down GT1]
  {Z positions necessary for deployment down guidetube \# 1.
   \shwlabel{TabCameraGuidetube}
  }
\end{center}
\end{table}

\noindent
{\bf Prior to deployment:}
\begin{enumerate}
\item Camera was assembled according to procedure \ref{CalProcCameraAssembly}
  in URM.
\item URM is mounted on guide tube.

\item Umbilical is disconnected from the camera controller.  This should
  be indicated by a status tag.

\item Sufficient blank video tapes are available to record the 
  entire deployment.

\end{enumerate}

{\bf 
  This device emitts light and could damage either phototubes
  or the trigger if used in the detector while HV is on.
}

\noindent
{\bf Procedure:}

\begin{enumerate}

\checkitem Verify guidetube gate valve is closed.

\checkitem Flush the URM with nitrogen until O$_2$ level is below
  0.5\%.

\checkitem Calibrate central rope using cross hairs or bottom flange
           ( there should be enough clearance to see the top of the
           cone through the window - if not use `high tension point'
           instead ).

\checkitem Remove {\bf Do Not Operate} status tag from umbilical end.

\checkitem Connect umbilical to camera controller.

\checkitem Turn on camera and attempt to image top of gate valve.  If unable
  to image the gatevalve, investigate before proceeding.

\checkitem Stop SNO detector data run.

\checkitem Ramp down detector HV and turn off HV supplies on all 
  crates.

\checkitem Do HV status from {\em HV Master} and verify all HV supplies
  are off.

\checkitem Do CMOS read on all crates to verify HV is off.

\checkitem Disconnect HV cables from all crates including OWL tubes.

\checkitem Verify that the Analog Light Monitor has HV off.

\checkitem Verify that the Analog Light Monitor has HV power supply
  unplugged.

\checkitem Verify that the Analog Light Monitor has the HV cables unplugged
  from the power supply.

\checkitem Start recording on the VCR.  Use best quality recording
  level.

\checkitem Open the guide tube gate valve.

\checkitem Deploy the source slowly down to just above the water
  surface.  attempt to image the surface of the water, looking
  for any surface film or debri.

\checkitem Deploy camera to 1 m above bottom of guide tube.\\
  ( Z = 834 cm)

\checkitem Attempt to image the opening of the guide tube into the
  the PSUP.  

\checkitem Deploy the camera assembly into the PSUP.
  
\checkitem Deploy camera assembly to 1 m above AV ( Z = 550).\\
  Attempt to image AV.

\checkitem Deploy camera to 50 cm above AV ( Z = 497).\\
  Attempt to image AV.

\checkitem Deploy camera to 30 cm above AV (Z = 477).\\
  Attempt to image AV.

\checkitem Deploy camera to 20 cm above AV (Z = 467).\\
  Attempt to image AV.

\checkitem  Deploy camera to 10 cm above AV (z = 457).\\
   Watch rope and umbilical tensions for indications of 
   contact.

\checkitem If a closer deployment to the AV is required,
  set the maximum speed of the rope and umbilical to 0.5 cm/s.

\checkitem After the imaging of the AV, retract the  camera from the
  detector.

\checkitem Verify the source is above gate valve visually with camera.

\checkitem Close the gate valve.

\checkitem Turn off the camera lights

\checkitem Disconnect the umbilical from the camera controller.

\checkitem Put {\bf Do Not Operate} status tag on the dry end of the
  umbilical.

\checkitem Reconnect HV cables to PMT crates and ramp up the detector.

\end{enumerate}



{\small
~\\
~\\
\noindent
{\bf Revision History:}\\
\begin{tabular}{llll}
Rev. & Date & Author & Comments\\
0           & 
2002/09/26  & 
Peter Skensved &
\parbox[t]{3.0in}{
  First Draft
}\\

1             & 
2002/09/26    & 
Fraser Duncan &
\parbox[t]{3.0in}{
 More detailed procedure.
}
\end{tabular}
}



\newpage
\markright{\standardheader}



