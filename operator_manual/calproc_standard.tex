
%------------------------------------------------------------------------
%------------------------------------------------------------------------
%------------------------------------------------------------------------
\section{Standard Calibration Procedures}
\shwlabel{secprocstandard}
 
  This chapter contains ``start to finish'' procedures for
standard calibrations.
  
%------------------------------------------------------------------------
%------------------------------------------------------------------------
%------------------------------------------------------------------------
\newpage
\subsection{PCA Calibration}

\newprocedure{CalProcPCA}
             {PCA Calibration Procedure}
             {Peter Skensved}{2003/01/06}{2}

 

\subsubsection{Introduction}
  
  This procedure describes step by step the process of performing a
laserball PCA calibration of the SNO detector.  It is intended as
a guideline for operators who have been trained on the manipulator
and laser.  It assumes that the laserball is mounted on URM2 which
is located on the 10'' gatevalve on the glovebox.
  

  Supplementing this procedure are the documents available
from the SNO calibration home page,
\begin{verbatim}
  http://www.sno.phy.queensu.ca/private/calibration/index.html
\end{verbatim}
In particular look at:
\begin{description}
\item[Online Manipulator Documentation] contains online manual
  for all commands and operations with the manip program running
  on the manip computer.  This is the reference source for commands
  done from the manip console.
\item[Manipulator User Manual] contains an overview of the manipulator
  system and descriptions of sources and some procedures.  In particular
  it describes how to start the manmon program for controling and monitoring
  the manipulator.
\item[Manipulator Reference Manual] contains technical information on
  the manipulator.
\end{description}
 
\noindent 
  The outline of the procedure is:
\begin{enumerate}
\item Prepare the laser and URM for use (turn on N$_2$ supply etc).
\item Flush URM2 with N$_2$ gas to remove O$_2$ and Rn.
\item Calibrate URM2 central rope.
\item Lower source into glovebox.
\item Connect side ropes to source. (if not single axis mode)
\item Deploy source into detector.
\item Take PCA data
\item Retract source to glovebox.
\item Remove side ropes. (if not single axis mode)
\item Retract source into source tube above gatevalve.
\item Shutdown laser and gas flow to laser and URM.
\end{enumerate}
  



%----------------------------------------------------------------------
\newpage
\subsubsection{Procedure}
~\\
\begin{tabular}{|l|l|}
\hline
 & \\
Operator(s):~~~~~~~~~~~~~~~~~~~~~~~~~~~~~~~~~~~~~~~~~~~~~ 
 & Date: ~~~~~~~~~~~~~~~~~~~~~~~~~~~~~~~~\\
 & \\
\hline
\end{tabular} 
~\\
~\\
  The procedures in this section are intended to be followed
sequentually for the PCA calibration except where it is noted
that a following procedure can be skipped.  Specifically,
if the PCA is to be done in {\em single axis} mode, the side
ropes do not need to be attached or detached from the source.
Procedures supplementary to the main PCA calibration are found
in section \ref{secproclaserball}.  

\begin{center}
                     {\bf Prior to PCA}
\end{center}
\begin{enumerate}
\item\checkbox Permission for procedure and confirmation of equipment readiness
  has been received from Head of Calibration Group.

\item\checkbox Laserball is mounted in URM2 which is
  mounted on 10'' valve on glovebox.

\item\checkbox 10'' gatevalve is closed and locked.

  

\begin{center}
                  {\bf Readying Laser and URM for Operation}
\end{center}

\item\checkbox Verify that the LN$_2$ dewar in the junction is
  at least 1/4 full.  If not, swap it out with another dewar.
  Record liquid level of Dewar,
     \begin{center}
     \begin{tabular}{|l|}
     \hline
      \\
     LN$_2$ Level:~~~~~~~~~~~~~~~~~~~~~~~~\\
      \\
     \hline
     \end{tabular}
     \end{center}

\item\checkbox Verify that the dewar gas pressure is approximately
  130 to 150 psig. If not, swap it out with another dewar.

\item\checkbox Turn on N$_2$ Flow to laser from dewar at junction 
  (Marked {\bf Gas Use} on dewar).
     \begin{center}
     \begin{tabular}{|l|}
     \hline
      \\
     Note Time:~~~~~~~~~~~~~~~~~~~~~~~~\\
      \\
     \hline
     \end{tabular}
     \end{center}

\item\checkbox Turn on pressure builder valve (Marked {\bf Pressure Builder} 
  on dewar).\\
  %------------------------
  \small
  {\em The pressure builder valve opens a controlled leak on the dewar
       to maintain the 150 psi pressure head.  If the valve is not
       opened, the gas pressure to the laser will eventually
       drop below the operating level.}
  \normalsize
  %------------------------


\item Once the URM is flushed the N$_2$ supply may be switched from the high
pressure dewar to the Wessington dewar. Check with the Operations Group first
before switching. Do not use the Wessington if a transfer is in progress.
Consult the gasboard section for details on how to switch.


\item\checkbox Contact Detector Operator and get permission to enter DCR. Make sure
that the DCR activity bit is set.
  

\item\checkbox Turn on lights in DCR following standard procedure. ( See Detector Operator 
Manual )

\item\checkbox Remove the flush return line on the URM.
  %------------------------
  \small
  {\em The presence of the buffer line makes it difficult to measure the O$_2$ from
the URM.
  }
  \normalsize
  %------------------------


  
\item\checkbox Check that flush inlet line is connected to URM2.  If not
  connect it. Open the valve on the source tube.\\
  %------------------------
  \small
  {\em It may be necessary to valve off other URMs to get sufficient flow.
  }
  \normalsize
  %------------------------

\item\checkbox Set up Gas Board in `bypass mode' for `URM flush' only. If you are
using the high pressure feed {\bf do not exceed } 10 psi on the regulator.
  %------------------------
  \small
  {\em Bypass mode maximizes the flow to the URM.
  }
  \normalsize
  %------------------------

\item\checkbox Check that flow meter ( located at South-East corner of
pipe box ) is railed. If not, open needle valve near the flowmeter fully.

   {\bf Flush should continue until O$_2$ reading at the rear of the URM is less than 0.8\%.}
   %--------------------------
   \small
   {\em
     This may take up to an hour depending on when the URM was last
     flushed.
   }
   \normalsize



\item\checkbox Check that the source clamps are in the OUT position.  
{\bf Both } knobs have to be in the extreme {\bf OUT} position. 
{\bf 
     WARNING:  If the source is moved with the clamps in the {\bf IN} position,
       the source, umbilical,
     and manipulator may be severely damaged ! 
   }
  %--------------------------------
  \small
  {\em
   The clamps are used to secure the source while the URM is being moved
   on and off the glovebox.
  }
  \normalsize
  %--------------------------------


\item\checkbox Check the pressure on the air cylinder for the umbilical
takeup mechanism. It should be between 45 and 55 psig. 
   {\bf Do not operate the URM if the pressure is below 40 psig. }
 If the pressure falls below 10 psig at any point ( even momentarily ) call the OCE. 
An internal inspection of the URM is mandatory before operating the unit again.
   %-------------------------------
   \small
   {\em
     The pressure cylinder on the URM maintains tension on the umbilical
     takeup reel.  A low gas pressure can result in the umbilical falling
     off the takeup reel and getting caught or jammed leading to destruction
     of the umbilical.
   }
   \normalsize
   %--------------------------------
  
   
\item\checkbox Verify that the 10'' gatevalve  is locked in the  closed position.\\
   %-------------------------------
   \small
   {\em
     The valve is CLOSED when the handle points towards the pipebox and the slot
      on the handle stem points AWAY from the source tube.
   }
   \normalsize
   %--------------------------------
 
\item \checkbox Calibrate Central Rope Length\\
      (see procedure  \ref{seccalcentre} 
       {\em Central Rope Position Calibration}).
      Record changes in length of central rope and umbilical,
      The current fiducial mark for URM2  on the 10'' gatevalve
      is
      \[
               z_{mark} = 1559.9
      \]
       Note : the fiducial mark is written on the source tube. If it
       differs from the above number use it instead. 

     \begin{center}
     \begin{tabular}{|l|}
     \hline
      \\
     $\Delta$l rope:~~~~~~~~~~~~~~~~~~~~~~~~\\
      \\
     \hline
      \\
     $\Delta$l umbilical:~~~~~~~~~~~~~~~~~~~~~~~~\\
      \\
     \hline
     \end{tabular}
     \end{center}


\item\checkbox Check that all seals are in place on URM.  Including:
   \begin{itemize}
      \item\checkbox flush inlet line
      \item\checkbox window on front of URM  motorbox
      \item\checkbox window on rear of URM motorbox
      \item\checkbox umbilical feedthrough on rear of motorbox
      \item\checkbox view port window cover on source tube
			\item\checkbox window on rear of stretcher box
   \end{itemize}

\item\checkbox Check that you are familiar with the section on operating the laser 
( section \ref{ChapterLaser} ) especially the {\bf Emergency Shutdown Procedure}. Also, be
aware that UV absorbing safety glasses {\bf MUST} be worn while
the laser is running unless {\bf ALL } covers on the laser are in place.


\item\checkbox Plug in the powercord to the laser.

\item\checkbox Check that the POWER  switch on the laser is to {\bf remote}

\item\checkbox Check that the CONTROL switch on the laser is set to {\bf remote}

\item\checkbox Check that the manual shutoff valve on the right of MV5 is open

\item\checkbox Check that the manual shutoff valve MV9 is open.

\item\checkbox  Reset the `Kill Switch' by pushing the red reset button.

\item\checkbox Type {\tt n2laser poweron } on the manip computer.
   \small
   {\em
 This turns on the low voltage power and  energizes the N$_2$ gas valve.

   }
   \normalsize
   %--------------------------------


  
\item\checkbox Verify that N$_2$ is flowing through flow gauge FG5 to the laser head.
If there is no flow consult an expert. {\bf Running the laser without sufficient N$_2$ flow
causes serious damage to the laser head !}

\item\checkbox Record observed laser gas pressure  and
  flow values.

\small
{\em Note that the expected values listed below may be superseded by ones
listed on one or more tags attached to the valves or flow meters. Always use the values found on the tags.}
\normalsize

  \begin{center}
  \begin{tabular}{|l|c|c|}
  \hline
  Transducer & Expect & Observed \\
  \hline
         & & \\
     PG2 & 110--150 psig &\\
         & & \\
  \hline
         & & \\
     PG3 & 90--110 psig & \\
         & & \\
  \hline
         & & \\
     PT4 & 100--110 psig & \\
         & & \\
  \hline
         & & \\
     FG5 & $\approx$ 50 (bottom of ball) & \\
         & & \\
  \hline
         & & \\
     PT6 & $\approx$ 90 psig & \\
         & & \\
  \hline
  \end{tabular}
  \end{center}
  PG2, PG3 and FG5 are read off the gas panel on the end of the
  laser cabinet.  PT4 and PT6 are read from the manipulator computer
  either from the {\tt manmon} laser window or using the commands,
  \begin{tabbing}
   aaaaaaaaaaaaaaaaaaa\=aaaaaaaaaaaaaaaaaaaaaaaaaaaaaaaaaaa\=aaaa\kill
         \>{\tt n2laser hipressure}  \> (for PT4) \\
         \>{\tt n2laser lowpressure}  \> (for PT6) \\
   \end{tabbing}

  
   Gas must flow through the laser for $\approx$ 10 min before turning
  the laser high voltage on.
  
\item\checkbox Block the light by setting the neutral density to 20 
\begin{verbatim}
 n2laser setd 20
on the manip computer.
\end{verbatim}

\item\checkbox Select  dyecell 4 ( 500 nm ) by typing
\begin{verbatim}
  dyelaser cell 4
\end{verbatim}

\item\checkbox Check the state of the laser by issuing a 
\begin{verbatim}
n2laser monitor 
\end{verbatim}
command on the manip console. It will tell you what the general state of the laser is.
\begin{itemize}
\item\checkbox Check that all 4 stirmotors are on.
\item\checkbox Check that there is 120V to the laser
\item\checkbox Check that the filterwheels do not report any problems.
\end{itemize}

\item \checkbox Wait until the O$_2$ level in the URM is at or below 0.8\%



%--------------------------------------------------------------
\begin{center}
            {\bf Deploying Source from Source Tube Into Glovebox}
\end{center}

 \item\checkbox Verify that the URM is below 0.8\% O$_2$.


\item\checkbox Check that flush return line is connected to
  URM2.  If not, connect it.\\
  %------------------------
  \small
  {\em It may be necessary to move it from another URM.
  }
  \normalsize
  %------------------------
  

 \item\checkbox Turn off DCR lights.

 \checkitem Record the Cover Gas O$_2$ level
     \begin{center}
     \begin{tabular}{|l|}
     \hline
      \\
     Cover Gas O$_2$ Reading:~~~~~~~~~~~~~~~~~~~~~~~~\\
      \\
     \hline
     \end{tabular}
     \end{center}

 \item\checkbox Verify OWL light monitor is on.  Establish communications
  with person watching light monitor.
  %-------------------------
  \small
  {\em
    Suggestion:  Station the person watching the OWL monitor at
    the Deck Mac.  Then he/she can shout through the  wall of the
    DCR and you don't need to use the phones which slow communications
    down.
  }
  \normalsize
  %-------------------------
 

\item\checkbox Open gate valve ( {\bf Slowly !} ).\\
  Record the time the valve is opened.
     \begin{center}
     \begin{tabular}{|l|}
     \hline
      \\
     Time Gate Valve Opened:~~~~~~~~~~~~~~~~~~~~~~~~\\
      \\
     \hline
     \end{tabular}
     \end{center}

 \item\checkbox Lock gate valve open.

 \item\checkbox With flashlight perform light leak check on URM.  In particular
   check the seal of the source tube window and around the base of the source tube.
   Also, check around any inspection panel which may have been removed in the recent past.

 \item\checkbox Using the dimmer switch, { \bf slowly } bring up breaker 9 lights in
   the DCR.  Person still watching owl monitor.


 \item\checkbox DAQ is connected to the {\bf manip} computer.

 \item \checkbox In DAQ, source type is set to {\bf LASERBALL}.

 \item\checkbox DAQ is in a {\bf source transitional run}.

 \item \checkbox Verify that {\bf manip\_logger} on {\bf crag1}
                 is running and logging the {\bf Laserball}  source.

 \item\checkbox Check movement of laserball down:
  \begin{center}
  \begin{tabular}{|l|l|}
  \hline
  console & {\tt manip$>$ laserball by 0 0 -5} \\
  \hline
  manmon  & in laserball window: \\
          & set x = 0, y = 0, z = -5\\
          & click on {\bf move by} \\
  \hline
  \end{tabular}
  \end{center}
  %--------------------
  \small
  {\em 
    The laserball should move down 5 cm.  The tension on the rope
    should be 40-60 N.  The tension on the umbilical should be
    10-30N.
  }
  \normalsize
  %--------------------

 \item\checkbox Check that the source offset and orientation is set correctly.
   At the console type 
 {\tt laserball sourceoffset } \\
  The current laserball has an offset of -64.4 cm. If the reported number is
different contact the OCE. For single axis deployment mode the orientation should
be 0

  {\tt laserball orientation 0 } \\
 If deployed with sideropes the orientation depends on what direction the slot
faces. If in doubt contact the OCE. 

 \item\checkbox Deploy source into the glovebox:
  \begin{center}
  \begin{tabular}{|l|l|}
  \hline
  console & {\tt manip$>$ laserball to 0 0 1380} \\
  \hline
  manmon  & in laserball window: \\
          & set x = 0, y = 0, z = 1380\\
          & click on {\bf move to} \\
  \hline
  \end{tabular}
  \end{center}



%-------------------------------------------------------------
\begin{center}
  {\bf Deploying Manipulator into Centre of 
            Detector from Glovebox}
\end{center}
\shwlabel{sectocentre}
 
 \item\checkbox Contact Water Supervisor and advise him/her that the source is
   being lowered into the D2O.  \\
   %--------------------
   \small
   {\em
     The water group maintains a very small differential pressure
     between the light and heavy water.  The volume of the source
     is enough to disrupt this differential pressure.
   }
   \normalsize
   %---------------------

 \item\checkbox Check tensions on urm2rope and urm2umbilical.  Rope tension
   should be approximately 30-50 N.  Umbilical tension should
   be between 15-40 N. Note that the tensions are reduced once the
source is submerged.
  
 \item\checkbox Move laserball to centre of detector.
  \begin{center}
  \begin{tabular}{|l|l|}
  \hline
  console & {\tt manip$>$ laserball to 0 0 66.4} \\
  \hline
  manmon  & in laserball window: \\
          & click on {\bf Position the source}\\
          & set x = 0, y = 0, z = 0\\
          & click on {\bf move to} \\
  \hline
  \end{tabular}
  \end{center}

%-------------------------------------------------------------
\begin{center}
                 {\bf Turning On the Laser}
\end{center}

 \item\checkbox Verify on the console that the control power on the laser is on :
\begin{verbatim}
 n2laser monitor
\end{verbatim} 
  All voltages should be on, all stir motors should should be ON,
 both filterwheels should be IDLE, the dye cell motor should be IDLE and
 gas should be flowing. If not contact OCE.

     If using manmon, the 
  power lights for 120VAC and 40 VDC should turn green
       the lights next to each dye cell indicating the stir motor
       status should turn green and the status boxes should indicate
  IDLE.



 \item\checkbox Verify that the N2 gas has been flowing through the laser for
    $\approx$ 10 min. Note that the gas flow is turned on and off with the 
    n2laser poweron/poweroff commands.



%   It will take about 5 minutes while the laser warms up.  A status
%   message should indicate this waiting period.
%   After the laser is warmed up, it will indicate that it is ready.

 \item\checkbox  Select desired wavelength or dye cell.
  \begin{center}
  \begin{tabular}{|l|l|}
  \hline
  console & {\tt manip$>$ dyelaser cell $<$0-9$>$} \\
          & or \\
          & {\tt manip$>$ dyelaser wavelength $<$wavelength$>$}\\
  \hline
  manmon  & in laserwindow \\
          & click on button above desired dye cell\\
  \hline
  \end{tabular}
  \end{center}
 \item\checkbox Set ND = 6.0 or higher.
  \begin{center}
  \begin{tabular}{|l|l|}
  \hline
  console & {\tt manip$>$ n2laser setnd 6.0} \\
  \hline
  manmon  & laserwindow-$>$Windows-$>$Neutral Density Settings \\
          & click on desired neutral density\\
  \hline
  \end{tabular}
  \end{center}
  %--------------------
  \small
  {\em
    The filter wheels are set to a large attenuation when first turning
    on the laser to prevent a large amount of light being introduced 
    into the detector and overwhelming the data aquisition.  Once
    the laser is on, the rate and intensity can be adjusted to the
    desired level.
  }
  \normalsize
  %--------------------


  \item\checkbox Wait for the laser to return status READY

 \item\checkbox Turn on laser light.
  \begin{center}
  \begin{tabular}{|l|l|}
  \hline
  console & {\tt manip$>$ n2laser start} \\
  \hline
  manmon  & in laserwindow \\
          & click on {\bf light on}\\
  \hline
  \end{tabular}
  \end{center}
   Now wait 90 seconds while the trigger is delayed.
   Laser will come on at 10 Hz trigger rate.
  
\checkitem Plug the Laserball Trigger signal into the EXTA input on the
  MTCD.
  
%-------------------------------------------------------------

\begin{center}
                 {\bf Taking PCA Data}
\end{center}
\item {\bf PCA Runs}
  The exact nature of the PCA runs will vary.  The ``caonical''
  run tends to be 
  \begin{enumerate}
  \checkitem Long Low Occupancy Run.
    \begin{itemize}
    \item 500nm
    \item The centroid for the raw TAC is usually at 1800.
    \item 5-8\% occupancy (ND setting at 500nm is approximately 5.5).
    \item 40 Hz laser trigger rate (or best you can do without buffer
          overflow).
    \item Run Type: PCA 
    \item 10 subruns (2-3 hours).
    \end{itemize}
  \checkitem Short Medium Occupancy Run.
    \begin{itemize}
    \item 500nm
    \item The centroid for the raw TAC is usually at 1800.
    \item 20-25\% occupancy (ND setting at 500nm is approximately 5.0).
    \item 5-10 Hz laser trigger rate (or best you can do without buffer
          overflow).
    \item Run Type: PCA 
    \item 3 subruns (20 minutes).
    \end{itemize}
  \end{enumerate}
%-------------------
\small
{\em
  The laser can be run up to 45Hz but the current DAQ looks like
it sometimes has trouble keeping up.  Generally you want to keep
the data rate to no more than 300kB/s.
}
\normalsize
%------------------

%-------------------------------------------------------------

\begin{center}
                 {\bf Turning Off Laser}
\end{center}

\checkitem Unplug the  EXTA at the MTCD.

 \item\checkbox Turn off laser light
  \begin{center}
  \begin{tabular}{|l|l|}
  \hline
  console & {\tt manip$>$ n2laser stop} \\
  \hline
  manmon  & in laserwindow \\
          & click on {\bf light off}\\
  \hline
  \end{tabular}
  \end{center}
 \item\checkbox Turn off laser power
  \begin{center}
  \begin{tabular}{|l|l|}
  \hline
  console & {\tt manip$>$ n2laser poweroff} \\
  \hline
  manmon  & in laserwindow \\
          & click on {\bf power off}\\
  \hline
  \end{tabular}
  \end{center}
 \item\checkbox Unplug laser power cord from wall outlet\\
  %-------------------
  \small
  {\em
    The laser is unplugged when it is not intended to be used for
    extended periods.  This is because it has been observed that
    on several occasions after an unscheduled power outage that the
    laser has come up in a funny state.
  }
  \normalsize
  %-------------------
  


%-------------------------------------------------------------
\begin{center}
                   {\bf Retracting Manipulator to glovebox}
\end{center}

\item\checkbox Contact Water Supervisor.  Inform him/her that the source is
   about to be removed from the D$_2$O. 

\item\checkbox Retract laserball from AV into glovebox.
  \begin{center}
  \begin{tabular}{|l|l|}
  \hline
  console & {\tt manip$>$ laserball to 0 0 1300} \\
  \hline
  manmon  & in laserball window: \\
          & click on {\bf Position the pivot}\\
          & set x = 0, y = 0, z = 1300\\
          & click on {\bf move to} \\
  \hline
  \end{tabular}
  \end{center}
\item\checkbox Retract laserball to position to disconnect side ropes.
  \begin{center}
  \begin{tabular}{|l|l|}
  \hline
  console & {\tt manip$>$ laserball to 0 0 1380} \\
  \hline
  manmon  & in laserball window: \\
          & click on {\bf Position the pivot}\\
          & set x = 0, y = 0, z = 1380\\
          & click on {\bf move to} \\
  \hline
  \end{tabular}
  \end{center}
  %------------------------
  \small
  {\em
    When moving the laserball to 1380, it is important to make sure
    you are moving with respect to the { \bf pivot } and  { \bf not } the
   centre of the source which is
    approximately 64 cm  below the pivot.  This is especially important if the sideropes
are attached ! 

  }
  \normalsize
  %-----------------------

  


  
%-------------------------------------------------------------
\begin{center}
           {\bf Retracting source above gate valve.  Side ropes NOT attached.}
\end{center}
\shwlabel{secabovegv}
\item\checkbox move laserball to 1530
  \begin{center}
  \begin{tabular}{|l|l|}
  \hline
  console & {\tt manip$>$ laserball to 0 0 1530} \\
  \hline
  \end{tabular}
  \end{center}  
\item\checkbox move laserball to 1540
  \begin{center}
  \begin{tabular}{|l|l|}
  \hline
  console & {\tt manip$>$ laserball to 0 0 1540} \\
  \hline
  \end{tabular}
  \end{center}  
\item\checkbox move laserball to 1550
  \begin{center}
  \begin{tabular}{|l|l|}
  \hline
  console & {\tt manip$>$ laserball to 0 0 1550} \\
  \hline
  \end{tabular}
  \end{center}
{\bf
 NOTE:\\
   MINIMUM SAFE HEIGHT TO CLOSE GATEVALVE IS 1535cm.\\
   If unable to get above this height, contact expert.
}
\item\checkbox Retrieve the gatevalve key from the DCR lock box.
\item\checkbox Unlock the gatevalve.
\item\checkbox Carefully close the gate valve by rotating the handle {\em clockwise}.
  {\em Expect resistance when the handle is about 3/4 of the way to
  the closed position.  This is the normal overcentering of the
  valve mechanism.} {\bf If resistance is felt before this or 
  if any sounds are heard that might be caused by valve hitting the source,
  STOP and contact an expert.}
  Record the time the valve is closed.
     \begin{center}
     \begin{tabular}{|l|}
     \hline
      \\
     Time Gate Valve Closed:~~~~~~~~~~~~~~~~~~~~~~~~\\
      \\
     \hline
     \end{tabular}
     \end{center}
 
\item\checkbox Lock the gatevalve in the {\bf CLOSED} position.
\item\checkbox Return the gatevalve key to the DCR lock box.
 
 \checkitem Record the Cover Gas O$_2$ level
     \begin{center}
     \begin{tabular}{|l|}
     \hline
      \\
     Cover Gas O$_2$ Reading:~~~~~~~~~~~~~~~~~~~~~~~~\\
      \\
     \hline
     \end{tabular}
     \end{center}

\item\checkbox Close the URM flush valve if the soure does not need drying out.
\small
{\em It is desirable to leave a minute flow of N$_2$ through the URM in order to dry
out the source and the umbilical. Contact OCE for instructions.}

\normalsize
\item\checkbox Turn off the URM flush regulator ( if the source does not need drying out ).

\item\checkbox IF the laser is off,
   turn off gas flow at the LN$_2$ dewar in the junction:
   \begin{enumerate}
   \item close {\bf Gas Use} valve
   \item close {\bf Pressure Building} valve
   \end{enumerate}


 \item\checkbox If the source is retracted, and gate valve closed,
   turn off gas flow at the high pressure LN$_2$ dewar in the junction:
   \begin{enumerate}
   \item close {\bf Gas Use} valve
   \item close {\bf Pressure Building} valve
   \end{enumerate}






%-------------------------------------------------------------
\begin{center}
           {\bf After Calibration}
\end{center}
\item\checkbox Source is above gate valve.
\item\checkbox Gate valve is closed and locked.
\item\checkbox Laser is off.
\item\checkbox Manual shutoff valve to the right of MV5 is closed
\item\checkbox Laser power cord is unplugged from wall outlet.
\item\checkbox High pressure LN$_2$ dewar is turned off (both {\bf Gas Use} valve and 
  {\bf Pressure Building} valve).
\item\checkbox Flush return line is disconnected from rear of URM2
\item\checkbox Gas board is set up to provide sufficient flow to dry out the inside
of the URM.

\end{enumerate}


   

{\small
~\\
~\\
\noindent
{\bf Revision History:}\\
\begin{tabular}{llll}
Rev. & Date & Author & Comments\\

0             & 
?    & 
Fraser Duncan &
\parbox[t]{3.0in}{
  First Draft
}\\

1             & 
?    & 
Fraser Duncan &
\parbox[t]{3.0in}{
  Many earlier drafts
}\\


2             & 
2003/01/06 & 
Peter Skensved &
\parbox[t]{3.0in}{
  Many updates
}\\


\end{tabular}
}


%------------------------------------------------------------------------
%------------------------------------------------------------------------
%------------------------------------------------------------------------
\newpage
\subsection{N16 Calibration}

\newprocedure{CalProcN16}
             {N16 Calibration Procedure}
             {Fraser Duncan/Peter Skensved}{2003/09/20}{2}

  
  This procedure describes step by step the process of performing an
N16  calibration at the centre of the detector in single axis mode.  It is intended as
a guideline for operators who have been trained on the manipulator
and N16.  It assumes that the N16 source is mounted on URM3,
located on gatevalve 3 on the glovebox (gatevalve 3 is the 4''
valve located at the northwest corner.
  


\noindent 
  The outline of the procedure is:
\begin{enumerate}
\item Prepare the URM for use (turn on N$_2$ supply etc).
\item Flush URM3 with N$_2$ gas to remove room air and radon.
\item Calibrate URM3 central rope.
\item Lower source into glovebox.
\item Deploy source into detector.
\item Take N16 data
\item Retract source to glovebox.
\item Retract source into source tube above gatevalve.
\item Shutdown N16 gas board and gas flow to laser and URM.
\end{enumerate}
  
\vspace*{0.25in}
\noindent 
You will need the following procedures to complete an N16
calibration.
\begin{itemize}
\item {\bf N16 Calibration} (this procedure) 
\item {\bf Central Rope Calibration Procedure}
\item {\bf DT Generator Turn On Procedure}
\item {\bf N16 Source Startup Procedure}
\item {\bf N16 Source Shutdown Procedure}
\item {\bf DT Generator Turn Off Procedure}

\end{itemize}

%----------------------------------------------------------------------
\newpage
\subsubsection{Procedure}
~\\

\noindent
\begin{tabular}{|l|l|}
\hline
 & \\
Operator(s):~~~~~~~~~~~~~~~~~~~~~~~~~~~~~~~~~~~~~~~~~~~~~ 
 & Date: ~~~~~~~~~~~~~~~~~~~~~~~~~~~~~~~~\\
 & \\
\hline
\end{tabular} 
~\\
~\\

  The procedures in this section are intended to be followed
sequentually for the N16 calibration except where it is noted
that a following step can be skipped.  Specifically,
if the N16 is to be done in {\em single axis} mode, the side
ropes do not need to be attached or detached from the source.
%%Procedures supplementary to the main N16 calibration are found
%%in section \ref{secsupp}.  

\begin{center}
                    {\bf Prior to N16}
\end{center}
\begin{enumerate}
\item \checkbox Permission for procedure and confirmation of equipment readiness
  has been received from Head of Calibration Group.

\item \checkbox N16 is mounted in URM3 which is
  mounted on glovebox gatevalve 3.

\item \checkbox Gatevalve 3 is closed and locked ( or handle is removed ).

  

\begin{center}
                  {\bf Readying N16 Source and URM for Operation}
\end{center}
 

\item \checkbox Contact Detector Operator and verify that the DCR activity bit is set.
  

\item \checkbox Turn on lights in DCR following standard procedure.
  

\item \checkbox Verify that the high pressure N$_2$ to the laser is valved
off ( end of laser, upper right hand corner )


\item \checkbox Verify that all valves on the gasboard are closed and the regulator
is set to zero.



\item\checkbox Remove the flush return line on the URM.
  %------------------------
  \small
  {\em The presence of the buffer line makes it difficult to measure the O$_2$ from
the URM.
  }
  \normalsize
  %------------------------



\item\checkbox Check that flush inlet line is connected to URM3.  If not
  connect it. Open the valve at the source tube.\\
  %------------------------
	\small
  {\em It may be necessary to valve off the other URMs to get sufficient flow.
  }
  %------------------------
  \normalsize



\item \checkbox Verify that the LN$_2$ dewar in the junction is
  at least 1/4 full.  If not, swap it out with another dewar.
  Record liquid level of Dewar,
     \begin{center}
     \begin{tabular}{|l|}
     \hline
      \\
     LN$_2$ Level:~~~~~~~~~~~~~~~~~~~~~~~~\\
      \\
     \hline
     \end{tabular}
     \end{center}

\item \checkbox Verify that the dewar gas pressure is above
  20 psig. If not, swap it out with another dewar. Unlike the laser the 16N
does not require a minimum of 100 psig N$_2$. The N$_2$ is only used for flushing.
Note that the dewar won't last long if the pressure is below 100 psig

\item \checkbox Turn on the high pressure N$_2$ flow from the dewar at junction 
  (Marked {\bf Gas Use} on dewar).
     \begin{center}
     \begin{tabular}{|l|}
     \hline
      \\
     Note Time:~~~~~~~~~~~~~~~~~~~~~~~~\\
      \\
     \hline
     \end{tabular}
     \end{center}

\item \checkbox Turn on pressure builder valve (Marked {\bf Pressure Builder} 
  on dewar).\\
  %------------------------
  \small
  {\em The pressure builder valve opens a controlled leak on the dewar
       to maintain the 150 psig pressure head.  If the valve is not
       opened, the gas pressure to the laser and URM will eventually
       drop below the operating level ( 10psig ). }
  \normalsize
  %------------------------



\item\checkbox Set up Gas Board in `bypass mode' for `URM flush' only. If you are
using the high pressure feed {\bf do not exceed } 10 psi on the regulator.
  %------------------------
  \small
  {\em Bypass mode maximizes the flow to the URM. ( Gas enters at top left hand corner,
flows through regulator along outside right hand side line
down to the URM with all other valves closed - see GasBoard Section for details ).

  }
  \normalsize
  %------------------------

\item\checkbox Check that flow meter ( located at east side of
pipe box ) is railed. If not, open needle valve near the flowmeter fully.

   {\bf Flush should continue until O$_2$ reading at the rear of the URM is less than 0.8\%.}
   %--------------------------
   \small
   {\em
     This may take up to an hour depending on when the URM was last
     flushed.
   }
   \normalsize



\item Once the URM is flushed the N$_2$ supply may be switched from the high
pressure dewar to the Wessington dewar. Check with the Operations Group first
before switching. Do not use the Wessington if a transfer is in progress.
Consult the gasboard section for details on how to switch. If the flush is continued
from the high pressure dewar reduce the flow using the needle valve. A setting of
2 full turns above the point where the flowmeter rails is sufficient.



\item\checkbox Verify ( blue ) valve on N16 Umbilical gas feed line (the translucent
               line) is OPEN.  If not, open it.


\item\checkbox Execute the N16 PMT Turn On Procedure



\item\checkbox Do not connect the PMT trigger cable to channel 4, slot 15, crate 17
before obtaining permission from the detector operator.

\item \checkbox Check that the source clamps are in the RELEASE position.  
   There are two clamps.  Check both.\\
  {\bf
    WARNING:  The clamp positions RELEASE and HOLD are 90 deg apart.
      Rotate the clamp in the SHORT direction (90deg) from the HOLD
      to RELEASE position.       The clamp must {\bf not}  be rotated the 
other way !

  }
  %--------------------------------
  \small
  {\em
   The clamps are used to secure the source while the URM is being moved
   on and off the glovebox.  If the source is moved with the clamps in
   the HOLD position, it may severely damage the manipulator and the umbilical.
  }
  \normalsize
  %--------------------------------
 
\item \checkbox Check the pressure on the air cylinder for the umbilical
takeup mechanism. It should be between around 55 psig.
   {\bf Do not operate the URM if the pressure is below 40 psig. }
 If the pressure falls below 10 psig at any point ( even momentarily ) call the OCE. 
An internal inspection of the URM is mandatory before operating the unit again.
   %-------------------------------
   \small
   {\em
     The pressure cylinder on the URM maintains tension on the umbilical
     takeup reel.  A low gas pressure can result in the umbilical falling
     off the takeup reel and getting caught or jammed leading to destruction
     of the umbilical.
   }
   \normalsize
   %--------------------------------


  
\item\checkbox Verify that Gate Valve 3 is closed and locked ( or handle is removed ).
   %-------------------------------
   \small
   {\em
     The valve is CLOSED when the handle points towards the pipebox and the slot
      on the handle stem points AWAY from the source tube.
   }
   \normalsize
   %--------------------------------


\item \checkbox Calibrate Central Rope Length\\
      (see procedure  
       {\em Central Rope Position Calibration}).
      Record changes in length of central rope and umbilical.
       The current fiducial mark for URM3 on gatevalve 3 is 
   \[
         z_{mark} = 1558.5 
       \]
       Note : the fiducial mark is written on the source tube. If it
       differs from the above number use what is written on the source tube.

     \begin{center}
     \begin{tabular}{|l|}
     \hline
      \\
     $\Delta$l rope:~~~~~~~~~~~~~~~~~~~~~~~~\\
      \\
     \hline
      \\
     $\Delta$l umbilical:~~~~~~~~~~~~~~~~~~~~~~~~\\
      \\
     \hline
     \end{tabular}
     \end{center}

 
\item \checkbox Check that all seals are in place on URM.  Including:
   \begin{itemize}
      \item\checkbox flush inlet line
      \item\checkbox window on front of URM  motorbox
      \item\checkbox window on rear of URM motorbox
      \item\checkbox umbilical feedthrough on rear of motorbox
      \item\checkbox view port window cover on source tube
       \item\checkbox window on rear of stretcher box
   \end{itemize}
\small
{ \em
 Note that the flush outlet is small enough and is shadowed well enough that it does not
pose a threat to the detector unless light is shone directly at into it. 


} 

\normalsize

\end{enumerate}



%--------------------------------------------------------------
\begin{center}
            {\bf Deploying Source from Source Tube Into Glovebox}
\end{center}


\begin{enumerate}

 \item\checkbox Verify that the URM is below approximately 0.8\% O$_2$.


\item\checkbox Check that flush return line is connected to
  URM3.  If not, connect it.\\
  %------------------------
  \small
  {\em It may be necessary to move it from another URM.
  }
  \normalsize
  %------------------------
  

 \item\checkbox Turn off DCR lights.


 \checkitem Record the Cover Gas O$_2$ level
     \begin{center}
     \begin{tabular}{|l|}
     \hline
      \\
     Cover Gas O$_2$ Reading:~~~~~~~~~~~~~~~~~~~~~~~~\\
      \\
     \hline
     \end{tabular}
     \end{center}

 \item\checkbox Verify OWL light monitor is on.  Establish communications
  with person watching light monitor.
  %-------------------------
  \small
  {\em
    Suggestion:  Station the person watching the OWL monitor at
    the Deck Mac.  Then he/she can shout through the  wall of the
    DCR and you don't need to use the phones which slow communications
    down. Or contact the detector operator by phone.
  }
  \normalsize
  %-------------------------
 

\item\checkbox Open gate valve ( {\bf Slowly !} ).\\
  Record the time the valve is opened.
     \begin{center}
     \begin{tabular}{|l|}
     \hline
      \\
     Time Gate Valve Opened:~~~~~~~~~~~~~~~~~~~~~~~~\\
      \\
     \hline
     \end{tabular}
     \end{center}

 \item\checkbox Remove handle and key for gatevalve 3 and place {\bf Gatevalve Open}
sign on valve.

 \item\checkbox With a flashlight perform light leak check on URM.  In particular
   check the seal of the source tube window and around the base of the source tube.
   Also, check around any inspection panel which may have been removed in the recent past.

 \item\checkbox Using the dimmer switch, { \bf slowly } bring up breaker 9 lights in
   the DCR.  Person still watching owl monitor. If everything is ok bring up the other
    breakers. {\bf If there is any sign of a lightleak turn off the DCR lights
immediately, close the gatevalve and contact the OCE.}




 \item\checkbox DAQ is connected to the {\bf manip} computer.

 \item \checkbox In DAQ, source type is set to {\bf N16}.

 \item\checkbox DAQ is in a {\bf source transitional run}.

 \item \checkbox Verify that {\bf manip\_logger} on {\bf crag1}
                 is running and logging the {\bf N16}  source.





 \item \checkbox Check movement of N16 down:
  \begin{center}
  \begin{tabular}{|l|l|}
  \hline
  console & {\tt manip$>$ n16 by 0 0 -5} \\
  \hline
  manmon  & in n16 window: \\
          & set x = 0, y = 0, z= -5\\
          & click on {\bf move by} \\
  \hline
  \end{tabular}
  \end{center}
  %--------------------
  \small
  {\em 
    The N16 should move down 5 cm.  The tension on the rope
    should be 90-110 N.  The tension on the umbilical should be
    20-40N.
  }
  \normalsize
  %--------------------

 \item \checkbox Deploy source into the glovebox:
  \begin{center}
  \begin{tabular}{|l|l|}
  \hline
  console & {\tt manip$>$ n16 to 0 0 1370} \\
  \hline
  manmon  & in n16 window: \\
          & set x = 0, y = 0, z= 1370\\
          & click on {\bf move to} \\
  \hline
  \end{tabular}
  \end{center}

\end{enumerate}


%-------------------------------------------------------------
\begin{center}
              {\bf Deploying Manipulator into Centre of 
            Detector from Glovebox}
\end{center}
\shwlabel{sectocentre}
 

\begin{enumerate}



 \item \checkbox Contact Water Supervisor and advise him/her that the source is
   being lowered into the D$_2$O.  \\
   %--------------------
   \small
   {\em
     The water group maintains a very small differential pressure
     between the light and heavy water.  The volume of the source
     is enough to disrupt this differential pressure and could potentially
     result in an SDS trip.
   }
   \normalsize
   %---------------------

 \item \checkbox Check tensions on URM3ROPE and URM3UMBILICAL.  Rope tension
   should be approximately 90-110 N.  Umbilical tension should
   be between 20-40 N.
  
 \item \checkbox Move N16 to centre of detector.
  \begin{center}
  \begin{tabular}{|l|l|}
  \hline
  console & {\tt manip$>$ n16 to 0 0 71} \\
  \hline
  manmon  & in n16 window: \\
          & click on {\bf Position the source}\\
          & set x = 0, y = 0, z= 0\\
          & click on {\bf move to} \\
  \hline
  \end{tabular}
  \end{center}
  As the source goes into the water, the rope tension will decrease to
  approximately 60 N and the umbilical to approximately 20 N.

\end{enumerate}
%-------------------------------------------------------------
\begin{center}
                {\bf Turning On N16 Source}
\end{center}
  

\begin{enumerate}

\item\checkbox Check that you are familiar with the section on operating the DT generator
and the associated gasboard especially the {\bf Emergency Shutdown Procedure}.


\item \checkbox Execute procedure \ref{procdton}, {\em Turning on DT Generator}.

\item \checkbox Execute procedure \ref{procn16start}.
  
%-------------------------------------------------------------

\begin{center}
             {\bf Taking N16 Data}
\end{center}
\item\checkbox Taking N16 Data.
  The exact nature of the N16 runs will vary.  The ``canonical''
  run tends to be 
  \begin{itemize}
  \item NOC setting of 10
  \item Target setting of 36.4
  \item Flow rate 280-300
  \end{itemize}

\end{enumerate}

%-------------------------------------------------------------

\begin{center}
             {\bf Turning Off N16 Source}
\end{center}

\begin{enumerate}

\item \checkbox Execute procedure {\em Shutting Down N16 Gas System}
\item \checkbox Execute procedure {\em Turning off DT Generator}

\end{enumerate}


%-------------------------------------------------------------
\begin{center}
              {\bf Retracting Manipulator to glovebox}
\end{center}

\begin{enumerate}

\item \checkbox Contact Water Supervisor.  Inform him/her that the source is
   about to be removed from the Heavy water. 

\item \checkbox Retract N16 from AV into glovebox.
  \begin{center}
  \begin{tabular}{|l|l|}
  \hline
  console & {\tt manip$>$ n16 to 0 0 1300} \\
  \hline
  manmon  & in N16 window: \\
          & click on {\bf Position the pivot}\\
          & set x = 0, y = 0, z= 1300\\
          & click on {\bf move to} \\
  \hline
  \end{tabular}
  \end{center}

\small
{\em If sideropes are attached move the source to approximately z = 1370 and
disconnect the sideropes before retracting the source any further. See the
siderope procedures.}


  {\em
    When moving the N16 to 1370, it is important to make sure
    you are moving the {\bf pivot}, not the centre of the source which is
    approximately 71 cm below the pivot.
    This is especially important if the sideropes
    are attached ! 


  }
  \normalsize
  %-----------------------

  
\end{enumerate}
  
%-------------------------------------------------------------
\begin{center}
            {\bf Retracting source above gate valve.  Side ropes NOT attached.}
\end{center}
\shwlabel{secabovegv}

\begin{enumerate}

\item \checkbox move N16 to 1530
  \begin{center}
  \begin{tabular}{|l|l|}
  \hline
  console & {\tt manip$>$ n16 to 0 0 1530} \\
  \hline
  \end{tabular}
  \end{center}  
\item \checkbox move n16 to 1540
  \begin{center}
  \begin{tabular}{|l|l|}
  \hline
  console & {\tt manip$>$ n16 to 0 0 1540} \\
  \hline
  \end{tabular}
  \end{center}  
\item \checkbox move N16 to 1550
  \begin{center}
  \begin{tabular}{|l|l|}
  \hline
  console & {\tt manip$>$ n16 to 0 0 1550} \\
  \hline
  \end{tabular}
  \end{center}
{\bf
 NOTE:\\
   MINIMUM SAFE HEIGHT TO CLOSE GATEVALVE IS 1530cm.\\
   If unable to get above this height ( due for example to high tension ), contact OCE
immediately.
}
%%\item \checkbox Retrieve the gatevalve key from the DCR lock box.

%%\item \checkbox Unlock the gatevalve.

\item Remove `Gatevalve open' sign.

\item \checkbox Carefully close the gate valve by rotating the handle {\em clockwise}.
  {\em Expect resistance when the handle is about 3/4 of the way to
  the closed position.  This is the normal overcentering of the
  valve mechanism.} {\bf If resistance is felt before this or 
  if any sounds are heard that might be caused by valve hitting the source,
  STOP and contact the OCE.}
  Record the time the valve is closed.
     \begin{center}
     \begin{tabular}{|l|}
     \hline
      \\
     Time Gate Valve Closed:~~~~~~~~~~~~~~~~~~~~~~~~\\
      \\
     \hline
     \end{tabular}
     \end{center}


\item \checkbox Remove handle and key.

 
 \checkitem Record the Cover Gas O$_2$ level
     \begin{center}
     \begin{tabular}{|l|}
     \hline
      \\
     Cover Gas O$_2$ Reading:~~~~~~~~~~~~~~~~~~~~~~~~\\
      \\
     \hline
     \end{tabular}
     \end{center}



\item\checkbox Execute N16 PMT Turn Off Procedure

\item  It is advisable to leave a slow N$_2$ flush of the URM in order to dry out
the source and the inside of the URM. Contact the OCE to organize who will turn
off the flow and when it should be done. If it is decided not to leave the
flush on follow the shutoff procedure listed below :



\item  Turn off gas flow at the LN$_2$ dewar in the junction:
   \begin{enumerate}
   \item \checkbox close {\bf Gas Use} valve
   \item \checkbox close {\bf Pressure Building} valve
   \end{enumerate}

\item \checkbox Turn off the flush on the gasboard.
\item \checkbox Close the URM flush valve.


\end{enumerate}

%-------------------------------------------------------------
\begin{center}
                 {\bf After Calibration}
\end{center}

\begin{enumerate}

\item \checkbox Source is above gate valve.
\item \checkbox Gate valve is closed and handle plus key removed.
\item \checkbox N16 PMT Power Supply set to  0V, Turned off, NIM BIN turned off.
\item \checkbox N16 gas board is OFF.
\item \checkbox DT Generator is OFF.


%\item \checkbox LN$_2$ dewar is turned off (both {\bf Gas Use} valve and 
%  {\bf Pressure Building} valve).

\end{enumerate}








{\small
~\\
~\\
\noindent
{\bf Revision History:}\\
\begin{tabular}{llll}
Rev. & Date & Author & Comments\\

0             & 
?    & 
Fraser Duncan &
\parbox[t]{3.0in}{
  First Draft
}\\

1             & 
2000/07/30 & 
Fraser Duncan &
\parbox[t]{3.0in}{
  Many earlier drafts
}\\

2     &
2003/08/20 &
Peter Skensved & 
Revisions to reflect new realities  \\
2004/08/10 &
Peter Skensved & 
More revisions to reflect even newer realities ...  \\

\end{tabular}
}



%------------------------------------------------------------------------
%------------------------------------------------------------------------
%------------------------------------------------------------------------
\newpage
\subsection{Acrylic Source Calibration Procedure}
\shwlabel{procpca}~\\


\newprocedure{CalProcAcryl}
             {Acrylic Source Calibration Procedure}
             {Fraser Duncan/Peter Skensved}{Oct. 2004}{3}




\subsubsection{Introduction}
   
  This procedure describes step by step the deployment process
for a acrylic source into the AV.  It is basically identical
to the Laserball or N16 deployment procedure.
  It is intended as
a guideline for operators who have been trained on the manipulator
and laser.  It assumes that the laserball is mounted on URM2 which
is located on the 10'' gatevalve on the glovebox.


  Supplementing this procedure are the documents available
from the SNO calibration home page,
\begin{verbatim}
  http://www.sno.phy.queensu.ca/private/calibration/index.html
\end{verbatim}
In particular look at:
\begin{description}
\item[Online Manipulator Documentation] contains online manual
  for all commands and operations with the manip program running
  on the manip computer.  This is the reference source for commands
  done from the manip console.
\item[Manipulator User Manual] contains an overview of the manipulator
  system and descriptions of sources and some procedures.  In particular
  it describes how to start the manmon program for controling and monitoring
  the manipulator.
\item[Manipulator Reference Manual] contains technical information on
  the manipulator.
\end{description}

\noindent
  The outline of the procedure is:
\begin{enumerate}
\item Prepare the URM for use (turn on N$_2$ supply etc).
\item Flush URM2 with N$_2$ gas to remove O$_2$ and Rn.
\item Calibrate URM2 central rope.
\item Lower source into glovebox.
\item Connect side ropes to source. (if not single axis mode)
\item Deploy source into detector.
\item Take data
\item Retract source to glovebox.
\item Remove side ropes. (if not single axis mode)
 \item Retract source into source tube above gatevalve.
\item Shutdown gasflow to  URM.
\end{enumerate}






%----------------------------------------------------------------------
\newpage
\subsubsection{Procedure}
~\\
\begin{tabular}{|l|l|}
\hline
 & \\
Operator(s):~~~~~~~~~~~~~~~~~~~~~~~~~~~~~~~~~~~~~~~~~~~~~
 & Date: ~~~~~~~~~~~~~~~~~~~~~~~~~~~~~~~~\\
 & \\
\hline
\end{tabular}
~\\
~\\
  The procedures in this section are intended to be followed
sequentually for the source calibration except where it is noted
that a following procedure can be skipped.  Specifically,
if the source run is to be done in {\em single axis} mode, the side
ropes do not need to be attached or detached from the source.

\begin{center}
                     {\bf Prior to Calibration Run}
\end{center}

\begin{enumerate}
\item\checkbox Permission for procedure and confirmation of equipment readiness
  has been received from Head of Calibration Group.

\item\checkbox Source is mounted in URM2 which is
  mounted on 10'' valve on glovebox.

\item\checkbox 10'' gatevalve is closed and locked.



\begin{center}
                  {\bf Readying URM for Operation}
\end{center}

\item\checkbox Verify that the LN$_2$ dewar in the junction is
  at least 1/4 full.  If not, swap it out with another dewar.
  Record liquid level of Dewar,
     \begin{center}
     \begin{tabular}{|l|}
     \hline
      \\
     LN$_2$ Level:~~~~~~~~~~~~~~~~~~~~~~~~\\
      \\
     \hline
     \end{tabular}
     \end{center}

\item\checkbox Verify that the dewar gas pressure is approximately
  130 to 150 psig. If not, swap it out with another dewar.

\item\checkbox Turn on N2 Flow to laser from dewar at junction
  (Marked {\bf Gas Use} on dewar).
     \begin{center}
     \begin{tabular}{|l|}
     \hline
      \\
     Note Time:~~~~~~~~~~~~~~~~~~~~~~~~\\
      \\
     \hline
     \end{tabular}
     \end{center}

\item\checkbox Turn on pressure builder valve (Marked {\bf Pressure Builder}
  on dewar).\\
  %------------------------
  \small
  {\em The pressure builder valve opens a controlled leak on the dewar
       to maintain the 150 psi pressure head.  If the valve is not
       opened, the gas pressure to the laser will eventually
       drop below the operating level.}
  \normalsize
  %------------------------


\item Once the URM is flushed the N$_2$ supply may be switched from the high
pressure dewar to the Wessington dewar. Check with the Operations Group first
before switching. Do not use the Wessington if a transfer is in progress.
Consult the gasboard section for details on how to switch.


\item\checkbox Contact Detector Operator and get permission to enter DCR. Make sure
that the DCR activity bit is set.


\item\checkbox Turn on lights in DCR following standard procedure. ( See Detector Operator
Manual )

\item\checkbox Remove the flush return line on the URM.
  %------------------------
  \small
  {\em The presence of the buffer line makes it difficult to measure the O$_2$ from
the URM.
  }
  \normalsize
  %------------------------


\item\checkbox Check that flush inlet line is connected to URM2.  If not
  connect it. Open the valve on the source tube.\\
  %------------------------
  \small
  {\em It may be necessary to valve off other URMs to get sufficient flow.
  }
  \normalsize
  %------------------------

\item\checkbox Set up Gas Board in `bypass mode' for `URM flush' only. If you are
using the high pressure feed {\bf do not exceed } 10 psi on the regulator.
  %------------------------
  \small
  {\em Bypass mode maximizes the flow to the URM.
  }
  \normalsize
  %------------------------


\item\checkbox Check that flow meter ( located at South-East corner of
pipe box ) is railed. If not, open needle valve near the flowmeter fully.

   {\bf Flush should continue until O$_2$ reading at the rear of the URM is less than 0.8\%.}
   %--------------------------
   \small
   {\em
     This may take up to an hour depending on when the URM was last
     flushed.
   }
   \normalsize



\item\checkbox Check that the source clamps are in the OUT position.
{\bf Both } knobs have to be in the extreme {\bf OUT} position.
{\bf
     WARNING:  If the source is moved with the clamps in the {\bf IN} position,
       the source, umbilical,
     and manipulator may be severely damaged !
   }
  %--------------------------------
  \small
  {\em
   The clamps are used to secure the source while the URM is being moved
   on and off the glovebox.
  }
  \normalsize
  %--------------------------------


\item\checkbox Check the pressure on the air cylinder for the umbilical
takeup mechanism. It should be between 45 and 55 psig.
   {\bf Do not operate the URM if the pressure is below 40 psig. }
 If the pressure falls below 10 psig at any point ( even momentarily ) call the OCE.
An internal inspection of the URM is mandatory before operating the unit again.
   %-------------------------------
   \small
   {\em
     The pressure cylinder on the URM maintains tension on the umbilical
     takeup reel.  A low gas pressure can result in the umbilical falling
     off the takeup reel and getting caught or jammed leading to destruction
     of the umbilical.
   }
   \normalsize
   %--------------------------------


\item\checkbox Verify that the 10'' gatevalve  is locked in the  closed position.\\
   %-------------------------------
   \small
   {\em
     The valve is CLOSED when the handle points towards the pipebox and the slot
      on the handle stem points AWAY from the source tube.
   }
   \normalsize
   %--------------------------------

\item \checkbox Calibrate Central Rope Length\\
      (see procedure  \ref{seccalcentre}
       {\em Central Rope Position Calibration}).
      Record changes in length of central rope and umbilical,
      The current fiducial mark for URM2  on the 10'' gatevalve
      is
      \[
               z_{mark} = 1559.9
      \]
       Note : the fiducial mark is written on the source tube. If it
       differs from the above number use it instead.

     \begin{center}
     \begin{tabular}{|l|}
     \hline
      \\
     $\Delta$l rope:~~~~~~~~~~~~~~~~~~~~~~~~\\
      \\
     \hline
      \\
     $\Delta$l umbilical:~~~~~~~~~~~~~~~~~~~~~~~~\\
      \\
     \hline
     \end{tabular}
     \end{center}


\item\checkbox Check that all seals are in place on URM.  Including:
   \begin{itemize}
      \item\checkbox flush inlet line
      \item\checkbox window on front of URM  motorbox
      \item\checkbox window on rear of URM motorbox
      \item\checkbox umbilical feedthrough on rear of motorbox
      \item\checkbox view port window cover on source tube
      \item\checkbox window on rear of stretcher box
   \end{itemize}


\item \checkbox Wait until the O$_2$ level in the URM is at or below 0.8\%


%--------------------------------------------------------------
\begin{center}
            {\bf Deploying Source from Source Tube Into Glovebox}
\end{center}

 \item\checkbox Verify that the URM is below 0.8\% O$_2$.


\item\checkbox Check that flush return line is connected to
  URM2.  If not, connect it.\\
  %------------------------
  \small
  {\em It may be necessary to move it from another URM.
  }
  \normalsize
  %------------------------


 \item\checkbox Turn off DCR lights.

 \checkitem Record the Cover Gas O$_2$ level
     \begin{center}
     \begin{tabular}{|l|}
     \hline
      \\
      Cover Gas O$_2$ Reading:~~~~~~~~~~~~~~~~~~~~~~~~\\
      \\
     \hline
     \end{tabular}
     \end{center}

 \item\checkbox Verify OWL light monitor is on.  Establish communications
  with person watching light monitor.
  %-------------------------
  \small
  {\em
    Suggestion:  Station the person watching the OWL monitor at
    the Deck Mac.  Then he/she can shout through the  wall of the
    DCR and you don't need to use the phones which slow communications
    down.
  }
  \normalsize
  %-------------------------


\item\checkbox Open gate valve ( {\bf Slowly !} ).\\
  Record the time the valve is opened.
     \begin{center}
     \begin{tabular}{|l|}
     \hline
      \\
     Time Gate Valve Opened:~~~~~~~~~~~~~~~~~~~~~~~~\\
      \\
     \hline
     \end{tabular}
     \end{center}

 \item\checkbox Lock gate valve open.

 \item\checkbox With flashlight perform light leak check on URM.  In particular
   check the seal of the source tube window and around the base of the source tube.
   Also, check around any inspection panel which may have been removed in the recent past.

 \item\checkbox Using the dimmer switch, { \bf slowly } bring up breaker 9 lights in
   the DCR.  Person still watching owl monitor.


 \item\checkbox DAQ is connected to the {\bf manip} computer.

 \item \checkbox In DAQ, source type is set to {\bf ACRYLIC}.

 \item\checkbox DAQ is in a {\bf source transitional run}.

 \item \checkbox Verify that {\bf manip\_logger} on {\bf crag1}
                 is running and logging the {\bf Acrylic}  source.

 \item\checkbox Check movement of acrylic source down:
  \begin{center}
  \begin{tabular}{|l|l|}
  \hline
  console & {\tt manip$>$ acrylic by 0 0 -5} \\
  \hline
  manmon  & in acrylic window: \\
          & set x = 0, y = 0, z = -5\\
          & click on {\bf move by} \\
  \hline
  \end{tabular}
  \end{center}

  %--------------------
  \small
  {\em
    The source should move down 5 cm.  The tension on the rope
    should be 40-60 N.  The tension on the umbilical should be
    10-30N.
  }
  \normalsize
  %--------------------


 \item\checkbox Check that the source offset is set correctly.
   At the console type
 {\tt acrylic sourceoffset } \\
  The actual offset depends on the exact configuration ( canned / un-canned, spacer present etc. )  

 \item\checkbox Deploy source into the glovebox:
  \begin{center}
  \begin{tabular}{|l|l|}
  \hline
  console & {\tt manip$>$ acrylic to 0 0 1380} \\
  \hline
  manmon  & in acrylic window: \\

          & set x = 0, y = 0, z = 1380\\
          & click on {\bf move to} \\
  \hline
  \end{tabular}
  \end{center}



%-------------------------------------------------------------
\begin{center}
  {\bf Deploying Manipulator into Centre of
            Detector from Glovebox}
\end{center}
%%%%\shwlabel{sectocentre}


 \item\checkbox Contact Water Supervisor and advise him/her that the source is
   being lowered into the D$_2$O.  \\
   %--------------------
   \small
   {\em
     The water group maintains a very small differential pressure
     between the light and heavy water.  The volume of the source
     is enough to disrupt this differential pressure.
   }
   \normalsize
   %---------------------

 \item\checkbox Check tensions on urm2rope and urm2umbilical.  Rope tension
   should be approximately 30-50 N.  Umbilical tension should
   be between 15-40 N. Note that the tensions are reduced once the
source is submerged.


 \item\checkbox Move acrylic source to centre of detector ( assuming source offset is 70 cm ).
  \begin{center}
  \begin{tabular}{|l|l|}
  \hline
  console & {\tt manip$>$ acrylic to 0 0 70} \\
  \hline
  manmon  & in acrylic window: \\
          & click on {\bf Position the source}\\
          & set x = 0, y = 0, z = 0\\
          & click on {\bf move to} \\
  \hline
  \end{tabular}
  \end{center}

%-------------------------------------------------------------

\checkitem Take data. The exact configuration will vary. 


%-------------------------------------------------------------
\begin{center}
                   {\bf Retracting Manipulator to glovebox}
\end{center}

\item\checkbox Contact Water Supervisor.  Inform him/her that the source is
   about to be removed from the D$_2$O.

\item\checkbox Retract acrylic source from AV into glovebox.
  \begin{center}
  \begin{tabular}{|l|l|}
  \hline
  console & {\tt manip$>$ acrylic to 0 0 1300} \\
  \hline
  manmon  & in acrylic window: \\
           & click on {\bf Position the pivot}\\
          & set x = 0, y = 0, z = 1300\\
          & click on {\bf move to} \\
  \hline
  \end{tabular}
  \end{center}
\item\checkbox Retract acrylic source to position to disconnect side ropes.
  \begin{center}
  \begin{tabular}{|l|l|}
  \hline
  console & {\tt manip$>$ acrylic to 0 0 1380} \\
  \hline
  manmon  & in acrylic window: \\
          & click on {\bf Position the pivot}\\
          & set x = 0, y = 0, z = 1380\\
          & click on {\bf move to} \\
  \hline
  \end{tabular}
  \end{center}
  %------------------------
  \small
  {\em
    When moving the acrylic source to 1380, it is important to make sure
    you are moving with respect to the { \bf pivot } and  { \bf not } the
   centre of the source which is
    approximately 64 cm  below the pivot.  This is especially important if the sideropes
are attached !
}
  \normalsize
  %-----------------------





%-------------------------------------------------------------
\begin{center}
           {\bf Retracting source above gate valve.  Side ropes NOT attached.}
\end{center}
\shwlabel{secabovegv}
\item\checkbox move acrylic to 1530
  \begin{center}
  \begin{tabular}{|l|l|}
  \hline
  console & {\tt manip$>$ acrylic to 0 0 1530} \\
  \hline
  \end{tabular}
  \end{center}
\item\checkbox move acrylic to 1540
  \begin{center}
  \begin{tabular}{|l|l|}
  \hline
  console & {\tt manip$>$ acrylic to 0 0 1540} \\
  \hline
  \end{tabular}
  \end{center}
\item\checkbox move acrylic to 1550
  \begin{center}
  \begin{tabular}{|l|l|}
  \hline
  console & {\tt manip$>$ acrylic to 0 0 1550} \\
  \hline
  \end{tabular}
  \end{center}
{\bf
 NOTE:\\
   MINIMUM SAFE HEIGHT TO CLOSE GATEVALVE IS 1535cm.\\
   If unable to get above this height, contact expert.
}
\item\checkbox Retrieve the gatevalve key from the DCR lock box.
\item\checkbox Unlock the gatevalve.
\item\checkbox Carefully close the gate valve by rotating the handle {\em clockwise}.
  {\em Expect resistance when the handle is about 3/4 of the way to
  the closed position.  This is the normal overcentering of the
  valve mechanism.} {\bf If resistance is felt before this or
  if any sounds are heard that might be caused by valve hitting the source,
  STOP and contact an expert.}
  Record the time the valve is closed.
     \begin{center}
     \begin{tabular}{|l|}
     \hline
      \\
     Time Gate Valve Closed:~~~~~~~~~~~~~~~~~~~~~~~~\\
      \\
     \hline
     \end{tabular}
     \end{center}

\item\checkbox Lock the gatevalve in the {\bf CLOSED} position.
\item\checkbox Return the gatevalve key to the DCR lock box.

 \checkitem Record the Cover Gas O$_2$ level
     \begin{center}
     \begin{tabular}{|l|}
     \hline
      \\
     Cover Gas O$_2$ Reading:~~~~~~~~~~~~~~~~~~~~~~~~\\
      \\
     \hline
     \end{tabular}
     \end{center}

\item\checkbox Close the URM flush valve if the soure does not need drying out.
\small
{\em It is desirable to leave a minute flow of N$_2$ through the URM in order to dry
out the source and the umbilical. Contact OCE for instructions.}

\normalsize
\item\checkbox Turn off the URM flush regulator ( if the source does not need drying out ).

\item\checkbox IF the laser is off,
   turn off gas flow at the LN$_2$ dewar in the junction:
   \begin{enumerate}
   \item close {\bf Gas Use} valve
   \item close {\bf Pressure Building} valve
   \end{enumerate}


 \item\checkbox If the source is retracted, and gate valve closed,
   turn off gas flow at the high pressure LN$_2$ dewar in the junction:
   \begin{enumerate}
   \item close {\bf Gas Use} valve
   \item close {\bf Pressure Building} valve
   \end{enumerate}






%-------------------------------------------------------------
\begin{center}
           {\bf After Calibration}
\end{center}
\item\checkbox Source is above gate valve.
\item\checkbox Gate valve is closed and locked.
\item\checkbox High pressure LN$_2$ dewar is turned off (both {\bf Gas Use} valve and
  {\bf Pressure Building} valve) if source is not to be dryed out.
\item\checkbox Flush return line is disconnected from rear of URM2
\item\checkbox Gas board is set up to provide sufficient flow to dry out the inside
of the URM.

\end{enumerate}



%------------------------------------------------------------

\newpage
\section{Laserball to/from Acrylic Procedures}

\newprocedure{CalProcSwitch}
   {Laserball to/from Acrylic Procedures}
   {P. Skensved}{Oct. 2004}{3}


\subsection{Procedure for Changing from Laserball to Acrylic Source.}


\subsubsection{Procedure}
~\\
\begin{tabular}{|l|l|}
\hline
 & \\
Operator(s):~~~~~~~~~~~~~~~~~~~~~~~~~~~~~~~~~~~~~~~~~~~~~
 & Date: ~~~~~~~~~~~~~~~~~~~~~~~~~~~~~~~~\\
 & \\
\hline
\end{tabular}
~\\
~\\

This procedure describes going from the Mark III laserball to a
generic 
acrylic source.  The acrylic source stem replaces the laserball and
it's stem where it attaches to the laserball ``can''.



\begin{enumerate}
      
\checkitem Obtain permission to proceed from calibration group. 
\checkitem Remove salt probe / plug clamp. 
\checkitem While supporting both laserball and can remove screws which hold the laserball to the can.
\checkitem Gently slide the laserball away from the can until CAJON fitting is exposed. Be sure to support both
can and laserball and make sure they're aligned. 
\checkitem Carefully unscrew top end of CAJON fitting and pull out the fiber end. Do NOT let it retract into the can.
\checkitem Store laserball in toolbox
\checkitem Ensure o-ring is in place on stem.
\checkitem Ensure dummy CAJON fitting is pin place on stem.
\small
{\em  
  Note : The teflon uses a steel spacer between the can and the stem ( the teflon stem is too soft to
seal properly. Make sure the o-ring is present on the top of the spacer. It has a pin which holds
the CAJON fitting. The o-ring between the stem and the plate is optional.
}
\normalsize
\checkitem Carefully insert fiber end into stem CAJON fitting and tighten lightly.
\checkitem Verify the saltprobe / plug o-ring is still in place.
\checkitem Push the fiber in to the cajon fitting on the stem. Tighten the fitting ligthly.
\checkitem Push the stem up against the can. Make sure the fiber is not bent or pinched
in any way.
\checkitem Slide the steel plate into place and bolt plate plus stem to can. Use correct screws.
\checkitem Slide clamp up around saltprobe or plug. Make sure it is oriented correctly ( so that
 it fits ) and
tighten screw appropriately. ( {\bf Do not overtigthen  } ! )

  The next few steps depend on the desired source configuration ( canned, un-canned, 
un-canned with spacer, sealed can ).


\checkitem Canned source
 \begin{enumerate}
 \checkitem Retrieve the appropriate source from the source cabinet. Make an entry in the log.
 \checkitem Place source in delrin can
 \checkitem Ensure o-ring is in place on lid. Ensure all three captured nuts are in place.
 \checkitem Gently press lid into place. Make sure it is oriented correctly.
 \small
  {\em Use something soft like a cable tie to check that the holes line up.
  Delrin is very soft and it is extremely easy to strip the threads. { \bf  Make absolutely sure that the
 holes are line up properly before inserting
  the screws ! } Use something soft like a cable tie to check the alignment. }
 \normalsize

 \checkitem Put in the three screws.  Do not use any tools - use your fingers. If there is {\bf any }
 friction at all chances are that the holes aren't lined up. Stop immediately, remove the screws and
 check the alingment again.

 \checkitem Tighten the screws infinitesimally with and allen key. {\bf Do not overtighten ! }

 \checkitem Secure srcews with stainless steel wire.

 \checkitem Ensure small o-ring is in place on top of lid. 

 \checkitem Attach can to stem with three screws. Use proper length screws. 

 \small
 { \em It is very easy to cross thread the screws in the captured nuts. Make {\bf absolutely} sure
 they are in correctly and { \bf do not overtighten ! } }

 \checkitem Secure screws with wire.
 \normalsize

 \end{enumerate}




\checkitem Uncanned source
 \begin{enumerate}
 \checkitem Mount source with three screws. Use spacer if so desired. Use proper length screws.
 \small
 {\em Acrylic is brittle and easy to damage. Use extreme care when threading the
 screws into the source. {\bf Do not overtighten !}  And do not cross thread the screws.  }
 \normalsize
 \checkitem Secure with wire.
 \end{enumerate}

\checkitem Sealed can

   Note : A sealed can is { \bf not } to be opened anywhere in the lab. The can is an essential
part onf the containment system for the source. At present we have two sealed sources ( both AmBe sources ).
One is in a teflon can the other is in a stainless can. The latter differs from the standard steel can
in that it has a small top cover. {\bf If in doubt contact the OCE !}.


 \begin{enumerate}
  
  \checkitem 
 \checkitem Attach can to stem with three screws. Use proper length screws. Do {\bf NOT } open can.
    The sealed stainless steel can has a small top cover plate and requires a ``donut''
    to fit the stem properly. If in doubt contact the OCE. 



 \small
 { \em It is very easy to cross thread the screws in the captured nuts. Make {\bf absolutely} sure
 they are in correctly and { \bf do not overtighten ! } }
 \normalsize
 \checkitem Secure screws with wire.
 \end{enumerate}




\end{enumerate}




%------------------------------------------------------------


%------------------------------------------------------------

\newpage


\subsection{Procedure for Changing from Acrylic Source to Laserball.}


\subsubsection{Procedure}
~\\
\begin{tabular}{|l|l|}
\hline
 & \\
Operator(s):~~~~~~~~~~~~~~~~~~~~~~~~~~~~~~~~~~~~~~~~~~~~~
 & Date: ~~~~~~~~~~~~~~~~~~~~~~~~~~~~~~~~\\
 & \\
\hline
\end{tabular}
~\\
~\\

This procedure describes going from the generic Acrylic Source to the Mark III Laserball.
  The laserball replaces the acrylic source and stem.



\begin{enumerate}

\checkitem Obtain permission to proceed from the calibration group.
\checkitem Remove salt probe / plug clamp.
\checkitem Remove the 3 screws holding the can containing the source ( or the source if uncanned configuration ).

\checkitem Contact OCE for instructions to verify what steps are to be followed regarding the can and source.
   {\bf Do not proceed unless authorized to do so !}

\small
{\em  The can may be part of the containment system for the source. {\bf Do NOT open the can unless directed to do so
by the OCE !   }    }
\normalsize 

  \checkitem Uncanned source

\begin{enumerate}

 \checkitem Place source in the proper bag and box in the source cabinet. Fill out logbook.

\end{enumerate}

\checkitem Canned source  ( {\bf NOT SEALED source }

 \begin{enumerate}

 \checkitem Remove 3 screws on the side.

  \checkitem Gently pry off the top.

  \checkitem Remove acrylic source from container. Fill out logbook and place source in proper bag and box in the source cabinet.

\end{enumerate}


\checkitem Sealed source.

\small
{\em  For the sealed source the can is an essential part of the containment system for the source.
 {\bf Do NOT open the can  !   }    }
\normalsize 

\begin{enumerate}

 \checkitem Place sealed source and ``donut'' ( if present ) in a bag and fill out the source logbook.

\end{enumerate}


\checkitem While supporting both source stem and the rest of the assembly remove screws which hold the stem to the can.
\checkitem Remove pressure plate.
\checkitem Gently slide the stem away from the can until CAJON fitting is exposed. Be sure to support both
can and stem and make sure they're aligned. 
\checkitem Carefully unscrew top end of CAJON fitting and pull out the fiber end. Do not let it retract into the can.
\checkitem Make sure the o-rings  are in place ( one for the saltprobe / plug
and another for the center ).
\checkitem Push the fiber in to the CAJON  fitting on the laserball. Tighten the fitting ligthly.
Make sure you don't twist or otherwise disturb the solid fiber on the laserball itself and make
 sure the fiber end is through
the o-ring.
\checkitem Mount the laserball on the can. Make sure the fiber is not bent or pinched
in any way. Use correct screws.
\checkitem Slide clamp up around saltprobe or plug. Make sure it is oriented correctly ( so that it fits ) and
tighten screw appropriately. ( {\bf Do not overtigthen  } ! )


\end{enumerate}


\newpage

\section{Other Source  Procedures}

\newprocedure{CalProcOther}
   {Other Source  Procedures}
   {P. Skensved}{Sept. 2004}{1}


   The following sections describe procedues for assembling and disassembling various sources.

 In order to ensure that there are no leaks and
that the source is securely attached to the umbilical
it is important that correct size o-rings are used everywhere. The
assembly uses small screws and nuts in many places. These are easily
stripped and damaged. {\bf Do not overtighten ! And do not re-tighten
just to make ``sure'' !}  If you do not know what the proper torque
should be contect an expert.
  If there are holes drilled through the screws secrure them with wire.
Do not twist or flex the wire more than necessary and make sure you do not
leave any weakened pieces of wire  as
they may end up in the detector. Re-do the wire instead. Bend the ends so
that they will not poke holes in the gloves.

