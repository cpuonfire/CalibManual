
%------------------------------------------------------------------------
%------------------------------------------------------------------------
%------------------------------------------------------------------------
  
\chapter{Manipulator Geometry}
\shwlabel{ChapterGeometry}
  
\section{Global Coordinate System}
  When positioning the manipulator the user is in fact positioning the
carraige pivot of the manipulator in the {\bf global coordinate system}
which is located at the designed centre of the PSUP and the AV.  Be
warned that neither the PSUP nor the AV are actually expected to be centred
on the global origin. In fact it is known that the PSUP has shifted by
at least 2'' and the AV shifts depending on the load.  Because the exact
location of the AV is needed to position the manipulator (need to know the
location of the anchor blocks in the AV) there are neck monitors used
to measure the position of the top of the AV neck and thus infer the position
of the centre of the AV.
  
  From the construction drawings the deck of the DCR
floor is located at
\begin{verbatim}
        100' = 1200 in.
\end{verbatim}
by definition.  The nominal centre of the AV 
and thus the origin of the global coordinate system is located at a height
of 
\begin{verbatim}
           56' 7 1/2 `` = 679.5 in.
\end{verbatim}
Therefore the distance from the global origin to the deck is
\[
        520.5 in. = 1322.07 cm
\]
by design.
  
\begin{table}[htbp]
\begin{center}
\begin{tabular}{|l|l|l|l|}
\hline
measurement                  & dim (in)     & dim (cm)        & from
\\ \hline
height of DCR floor          & 100'= 1200'' & 3048.00         & definition \\
height of global origin      & 56' 7 1/2'' = 679.5'' & 1725.93 & design \\
d(global origin to DCR floor)& 520.5''               & 1322.07 & calc \\
\hline
\end{tabular}
\caption[Global Coordinate System]
        {Global Coordinate System
         \shwlabel{tabglobaldim}
        }
\end{center}
\end{table}
  
  
\begin{table}
\begin{center}
\begin{tabular}{|l|l|l|l|} \hline
                               & coord system & location & source of measurement \\
DCR Floor                      & global &  z = 1318.47 cm &  \\
\hline
Bottom of Tube Flange on URM-1 & global & z= 1504.43 & \\
\hline
Height of side rope feedthroughs & global & z = 1424.22 & \\
on glovebox                      & & & \\
\hline
\end{tabular}
\caption[Manipulator Geometry]
        {Manipulator Geometry
         \shwlabel{tabmangeo}
         }
\end{center}
\end{table}
  
  
\section{Glovebox and Universal Interface}
  
  The dais of the universal inteface is located 17 and 1/8 `` above
the floor of the cleanroom.  Taking the nominal height of the DCR floor
as 1322.07 cm, the UI Dais is located at 1384.30 cm and the top of the
glovebox is located at 1427.80 cm in global coordinates.
\begin{table}[htbp]
\begin{center}
\begin{tabular}{|l|l|l|l|}
\hline
measurement                      & dim (in)   & dim (cm) & from \\ \hline
DCR floor to UI Dais             & 17 1/8 ``  & 43.50    & measurement \\
UI Dais to top plate of glovebox & 24 1/2 ``  & 62.23    & measurement \\
Nominal height of UI Dais        &            & 1384.30  & measurement \\
Nominal height of glovebox top   &            & 1427.80  & measurement \\
\hline
westrope feedthrough  x          & -21.000''  & -53.340  & Drawing \\
westrope feedthrough  y          &   0.750''  &   1.905  & Drawing \\
\hline
eastrope feedthrough  x          &  21.000''  &  53.340  & Drawing \\
eastrope feedthrough  y          &   0.750''  &   1.905  & Drawing \\
\hline
northrope feedthrough  x         &   0.000''  &   0.000  & Drawing \\
northrope feedthrough  y         &  15.750''  &  40.005  & Drawing \\
\hline
southrope feedthrough  x         &   0.000''  &   0.000  & Drawing \\
southrope feedthrough  y         & -20.250''  & -51.435  & Drawing \\
\hline
10'' gate valve        x         &   0.000''  &   0.000  & Drawing \\
10'' gate valve        y         &  -8.500''  & -21.590  & Drawing \\
\hline
 6'' gate valve        x         &   6.656''  &  16.906 & Drawing \\
 6'' gate valve        y         &   9.250''  &  23.495 & Drawing \\
\hline
 4'' gate valve        x         &  -6.313''  & -16.035 & Drawing \\
 4'' gate valve        y         &   9.250''  &  23.495 & Drawing \\
\hline
\end{tabular}
\caption[Glovebox and UI dimensions]
        {Glovebox and UI dimensions
         \shwlabel{tabgbdim}
        }
\end{center}
\end{table}
  
  
\section{Acrylic Vessel}
The thermal expansion coefficient for the acrylic is,
\[
        6 \times 10^{-5} C^{-1}
\]
The design specs for the AV give the 
distance from top of chimmney to centre of vessel at 23 C
\begin{verbatim}
        42' 2 3/8''
\end{verbatim}
which is 506.375 cm. and the
nominal outside radius
\begin{verbatim}
        236.6''
\end{verbatim} which is 600.964 cm.
with a nominal thickness of 2.15'' (5.461cm).

This can be compared to the results found in SNO-STR-98-003 (R. Komar) 
for actual measurements of the AV.
\begin{table}[htbp]
\begin{center}
\begin{tabular}{|l|l|l|}
\hline
measurement                    & design     & as built \\ \hline
Vessel Inner Radius            &  236.43''  & 236.38$\pm$0.23'' \\ 
Top of Chimney to AV centre    & 506.375''  & 506.16$\pm$0.12'' \\
Top of Chimney to bottom of AV & 742.59''   & 742.59$\pm$0.05'' \\
\hline
\end{tabular}
\end{center}
\end{table}
Using the Komar measurements, the nominal dimensions of the AV
are given in table \ref{tabavdim}.
\begin{table}[htbp]
\begin{center}
\begin{tabular}{|l|l|l|l|}
\hline
measurement                      & dim (in)   & dim (cm)& from \\ \hline
Average Vessel Inner Radius &  236.38$\pm$0.23'' & 600.41$\pm$0.58 
                            & measurement\\
Top of Chimney to AV bottom &  742.59$\pm$0.05''  & 1886.18$\pm$0.13
                            & measurement\\
Top of Chimney to AV centre &  506.16$\pm$0.12''  & 1285.65$\pm$0.30 
                            & measured?\\
Neck Ring gasket            &   1/8''             & 0.3175 & measured\\
Neck Ring plate             &   3/8''             & 0.9525 & measured \\
AV Centre to AV top plate   &  506.66''           & 1286.92 & calculated \\
DCR floor to AV top plate   &  12.4375       & 31.59 & measured/calculated\\
\hline
\end{tabular}
\caption[Acrylic Vessel Dimensions]
        {Acrylic Vessel Dimensions
         \shwlabel{tabavdim}
        }
\end{center}
\end{table}
On top of the AV neck flange is a gasket (1/8'') and a stainless steel
top plate (3/8'').  This gives a distance from the centre of the
AV to the top plate of,
\[
     506.16 + 1/8 + 3/8 = 506.66 in = 1286.92 cm
\]
The distance from the AV top plate to the UI flange on the DCR floor
was measured before the UI was installed.  (This flange is no longer
accessable since the UI has been installed.)  This was a measurement
after the final installation of the AV at nominal lab temperature before
any water in the AV.
The distance measured on 3 April 1998 was
28 9/16''.  The distance of the flange from the DCR floor was measured
to be 16 1/8''.  Therefore the distance of the AV topplate from the
DCR floor
\[
           28 9/16 - 16 1/8 = 12.4375'' = 31.59 cm
\]
The distance from the DCR floor to the centre of the AV from the 
measurement of the topplate location and the Komar measurements of the
AV are therefore,
\[
          1286.92 cm + 31.59 cm  = 1318.51 cm
\]
which corresponds to the AV being located above the nominal position
by 
\[
           1322.07 - 1318.51 = 3.56 cm
\]
{\bf What was Chris's AV measurement at this time?}

\section{URM-1 (Jury Rig)}
URM-1 is located on the 10'' port in a jury rig.  It is physically  mounted
32 3/16'' (81.756 cm) above the glovebox.  The source tube flange is 1''
thick so the tube flange is located 79.216 cm above the glovebox.  The
nominal height of the top of the glovebox is (table \ref{tabgbdim})
1427.80 cm so the location of the URM-1 tube flange is
\[
        1427.80 + 79.216 = 1507.02 cm
\]


  
\section{Guidetubes}

\begin{verbatim}
Hi John,
  I am currently looking at how consistant our measurement of the location
of the centre of the AV is with respect to the Deck and UI and will send
out a report when I have some numbers.  Mostly I have no knowledge of the
location of the PSUP.  However I do have one datum.
  
  When we put the source down guide tube number 5 (the one in the north
west corner of the DCR) we found that the exit of the guide tube into 
the PSUP was out of true with the top of the guide tube in the DCR.  
I am told that when the guide tubes were installed that they were aligned
by a plumb bob to better than 1/4".  When we put the source down last 
spring, we found that exit of the guide tube into the PSUP had shifted
approximately 2".  This number was determined by centring a 4" diameter
cone at the exit of the tube and noting where the string holding it was
located at the top of the tube.  The bottom of the tube was shifted
in the southwest direction (-x, -y) by about 2".  I have numbers for
the nominal location of the guide tube as being at 
        x = -586.6,  y = 222.2
The guide tube is supposed to enter the PSUP at z = 560.  Please note
that I have no idea if this misalignment means that the PSUP was out
of true when the guide tube was installed or if it became out of true
when AFTER the guide tube was installed.  
  
  Cheers,
     Fraser
\end{verbatim}

  
\section{other}
The design specs for the AV indicate,
\begin{verbatim}
  distance from top of chimmney to centre of vessel at 23 C
        42' 2 3/8''
  nominal outside radius
        236.6''
  nominal thickness 2.15''
\end{verbatim}
This can be compared to the results found in SNO-STR-98-003 (R. Komar) 
for actual measurements of the AV.
\begin{table}[htbp]
\begin{center}
\begin{tabular}{|l|l|l|}
\hline
measurement                    & design     & as built \\ \hline
Vessel Inner Radius            &  236.43''  & 236.38$\pm$0.23'' \\ 
Top of Chimney to AV centre    & 506.375''  & 506.16$\pm$0.12'' \\
Top of Chimney to bottom of AV & 742.59''   & 742.59$\pm$0.05'' \\
\hline
\end{tabular}
\end{center}
\end{table}
  
  
  
\begin{table}
\begin{center}
\begin{tabular}{lrrrrrr}
  & CT4 \\
\hline
x &  230.75 in \\
  & -586.11 cm \\

y & 81.875 in \\
  & 207.96 cm \\
  
r & 586.12 cm \\
  
Enters PSUP & 564.64 cm \\

\end{tabular}
\caption[Guide Tubes]
        {Guide Tubes
         \shwlabel{tabguidetubes}
        }
\end{center}
\end{table}
   
\begin{figure}
\begin{center}
\leavevmode
%\epsfysize=0.85\textheight
\epsfxsize=6.5in
\epsfbox{figures/guide_tubes.ps}
~\\\
\caption[Guide Tube Layout In DCR]
        {Layout of the Guide Tubes in the DCR
         \shwlabel{figguidetubeplan}
        }
\end{center}
\end{figure} 
 
  
   
\begin{figure}
\begin{center}
\leavevmode
%\epsfysize=0.85\textheight
\epsfxsize=6.5in
\epsfbox{figures/guide_tubes_elevation.ps}
~\\\
\caption[Guide Tube Elevation View]
        {Guide Tube Elevation View
         \shwlabel{figguidetubeelevation}
        }
\end{center}
\end{figure} 
 
  
