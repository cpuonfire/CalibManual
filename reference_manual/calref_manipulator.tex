 
  

%------------------------------------------------------------------------
%------------------------------------------------------------------------
%------------------------------------------------------------------------


\chapter{Manipulator}
\shwlabel{ChapterManipulator}

  
\section{Manipulator Systems}
  
 
\begin{figure}[htbp]
\begin{center}
\leavevmode
%\epsfysize=0.85\textheight
\epsfxsize=6in
\epsfbox{figures/MANCTRL.eps}
\caption[SNO manipulator control hardware]{SNO manipulator control
  \shwlabel{figmanctrl}}
\end{center}
\end{figure}
  
  
%-------------------------------------------------------------------------
\newpage
\subsection{Umbilical Retrival Mechanism (URM)}
The Umbilical Retrival Mechanism (or URM for short) is a combination
of an umbilical stretcher box and a central rope in one gas tight
housing.
 
\subsubsection{Wiring}
The wiring for URM-2 is shown in figure \ref{figurm2wiring}.
\begin{figure}[htbp]
\begin{center}
\leavevmode
%\epsfysize=0.85\textheight
\epsfxsize=12in
\epsfbox{figures/urm2wire.eps}
\caption[URM-2 wiring]{URM-2 wiring.  {\bf The wiring of the encoders
  on this figure is incorrect (pins 1-4 on the 9pin D connectors are
  reversed.  The correct wiring is shown in table \ref{taburm2wiring}}.
  \shwlabel{figurm2wiring}}
\end{center}
\end{figure}

\begin{table}
\begin{center}
\begin{tabular}{|l|l|l|l|} 
\hline
Connector & pin & colour& function \\
j1 & 1 & red   & rope motor-1 \\
   & 2 & black & rope motor-2 \\
   & 3 & white & rope motor-3 \\
   & 4 & green & rope motor-4 \\
\hline
j2 & 1 & red   & umbilical motor-1 \\
   & 2 & black & umbilical motor-2 \\
   & 3 & white & umbilical motor-3 \\
   & 4 & green & umbilical motor-4 \\
\hline
j3 & 1 & red   & rope encoder GND \\
   & 2 & blue  & rope encoder A \\
   & 3 & green & rope encoder B \\
   & 4 & black & rope encoder +5VDC \\
   & 5 &       & NC \\
   & 6 & brown  & rope loadcell GND \\
   & 7 & yellow & rope loadcell +8VDC \\
   & 8 & white  & rope loadcell + \\
   & 9 & orange & rope loadcell - \\
\hline
j4 & 1 & red    & umbilical encoder GND \\
   & 2 & blue   & umbilical encoder A \\
   & 3 & green  & umbilical encoder B \\
   & 4 & black  & umbilical encoder +5VDC \\
   & 5 &        & NC \\
   & 6 & brown  & umbilical loadcell GND \\
   & 7 & yellow & umbilical loadcell +8VDC \\
   & 8 & white  & umbilical loadcell + \\
   & 9 & orange & umbilical loadcell - \\
\hline
j5 & 1 &        & rope tension limit switch \\
   & 2 &        & rope tension limit switch \\
   & 3 &        & umbilical travel limit switch 1\\
   & 4 &        & umbilical travel limit switch 1\\
   & 5 &        & umbilical travel limit switch 2\\
   & 6 &        & umbilical travel limit switch 2\\
   & 7 &        &        \\
   & 8 &        &        \\
   & 9 &        &        \\
\hline
\end{tabular}
\end{center}
\caption[URM-2 Wiring]
  {URM-2 Wiring.   {\bf Note the encoder wiring.  This is correct.
    The wiring on figure \ref{figurm2wiring} is incorrect.}
   \shwlabel{taburm2wiring}}
\end{table}
  

%-------------------------------------------------------------------------
\newpage
\subsection{Side Rope Motor Mounts}
 
\subsubsection{Wiring}

\begin{table}
\begin{center}
\begin{tabular}{|l|l|l|l|} 
\hline
Connector & pin & colour& function \\
\hline
j3 & 1 & red   & rope encoder GND \\
   & 2 & blue  & rope encoder A \\
   & 3 & green & rope encoder B \\
   & 4 & black & rope encoder +5VDC \\
   & 5 &       & NC \\
   & 6 & brown  & rope loadcell GND \\
   & 7 & yellow & rope loadcell +8VDC \\
   & 8 & white  & rope loadcell + \\
   & 9 & orange & rope loadcell - \\
\hline
\end{tabular}
\end{center}
\caption[Side Rope Motor Mount Wiring]
  {Side Rope Motor Mount Wiring
   \shwlabel{tabsideropewiring}}
\end{table}
  

%------------------------------------------------------------------------
%------------------------------------------------------------------------
%------------------------------------------------------------------------
  
\section{Control Hardware}
 
  
%\begin{figure}[htbp]
%\begin{center}
%\leavevmode
%%\epsfysize=0.85\textheight
%\epsfxsize=6in
%\epsfbox{figures/control1.eps}
%\caption[Manipulator Control Computer and Hardware]
%        {Manipulator Control Computer and Hardware
%         \shwlabel{figcontrol1}}
%\end{center}
%\end{figure}
  
  
  Each of the ropes is referred to as an ``axis'' and each axis has a
\begin{itemize}
\item stepping motor to wind rope in or out.
\item shaft encoder to measure length of rope.
\item load cell to measure tension of rope.
\end{itemize}
The stepping motors are controlled from a National Instruments TIO 10 card
in the control PC that has many clock signals.  These signals are fanned
out through the motor fanout box to the individual motor controllers.
The readback from the axis is the load cell measuring the tension of the
rope and the shaft encoder measuring the length of the rope.  The input
signals go into the counter boards which are gray boxes 
which are daisy chained together and are indvidually 
addressable.    The address for the counters are set with jumpers on 
the boards as is the address for the analog circuit.  Note that the
analog circuit has a different address from the counter circuit.
These boxes are read out by the data concentrator which is
read by the computer.
  
The motors currently assigned in the control system are given in
table \ref{tabMCCmotors}.  
\begin{table}[htb]
\begin{center}
\begin{tabular}{|l|l|l|}
\hline
channel &  motor  & system\\
\hline
 1 & dyeLaser mirror   & Laser\\
 2 & filterwheela      & Laser\\
 3 & filterwheelb      & Laser\\
 4 & laser trigger     & Laser\\
 5 & centralrope       & URM-1\\
 6 & laserumbilical    & URM-1\\
 7 & westropemotor     & \\
 8 & eastropemotor     & \\
\hline
 9 & centralrope3      & URM-2\\
10 & gasumbilical      & URM-2\\
11 & &\\
12 & &\\
13 & &\\
14 & &\\
15 & &\\
16 & &\\
\hline
\end{tabular}
\caption[Manipulator Control Computer motors  in Use]
        {Manipulator Control Computer motors in Use
         \shwlabel{tabMCCmotors}
        }
\end{center}
\end{table}
The counter boards currently in use are shown in table \ref{tabMCCcounters}
along with the analog channels used on each board.
\begin{table}[htb]
\begin{center}
\begin{tabular}{|l|l|l|l|l|}
\hline
DC Slot & CB address & analog address & Signal & system\\
\hline
 1 & 0x80   &     &  SROPE1  & \\
 1 & 0x94   &     &  SROPE6  & \\
 1 & 0x98   &     &  SROPE7  & \\
 1 &        &     &          & \\
\hline
 2 & 0x84   &     &  SROPE2  & \\
 2 & 0x88   &     &  SROPE3  & \\
 2 & 0x8c   &     &  SROPE4  & \\
 2 & 0x90   &     &  SROPE5  & \\
\hline
 3 &        &     &          & \\
 3 &        &     &          & \\
 3 &        &     &          & \\
 3 &        &     &          & \\
\hline
 4 &        &     &          & \\
 4 &        &     &          & \\
 4 &        &     &          & \\
 4 &        &     &          & \\
\hline
 5 &  0xD8  &     &          & \\
 5 &        &     &          & \\
 5 &  0xD8  & 3   & centralrope3 &  URM-2\\
 5 &  0xDC  & 4   & gasumbilical &  URM-2\\
\hline
 6 &  0xA0  & 1   & westrope       &  \\
 6 &  0xF8  & 2   & laserumbilical & URM-1 \\
 6 &  0xA8  & 3   & centralrope    & URM-1 \\
 6 &  0xAC  & 4   & eastrope       &  \\
\hline
 7 &        &     &          & \\
 7 &        &     &          & \\
 7 &        &     &          & \\
 7 &        &     &          & \\
\hline
 8 &        & 1   & N2LASERLOWPRESSURE       & Laser\\
 8 &        & 3   & N2LASERHIPRESSURE        & Laser\\
 8 &  0xE4  &     & DYELASERENCODER          & Laser \\
 8 &  0xE8  &     & FILTERWHEELENCODER       & Laser \\
 8 &  0xEC  &     & N2 laser and fileter wheel status bits     & Laser\\
\hline
\end{tabular}
\caption[Manipulator Control Computer counter boards in Use]
        {Manipulator Control Computer counter boards in Use
         \shwlabel{tabMCCcounters}
        }
\end{center}
\end{table}


 
%-------------------------------
\clearpage
\newpage
\subsection{Counter Boards}
  
  The Encoder counter boards contain circuitry for a digital shaft encoder
and an analog circuit.  The digital encoder circuitry consists of a debounce
circuit and a 16 bit up/down counter to count the encoder pulses.  The
encoder is addressed with 8 address lines, 
$A_0$ --- $A_7$.  Bits $A_0$ and $A_1$
are not used and addresses 7C through 7F are a master reset that resets
all counters on all boards.
\begin{verbatim}
             A: 7 6 5 4 3 2 1 0
  master reset  0 1 1 1 1 1 X X
\end{verbatim}
  
    The analog circuit consists of a 2 stage amplifier.  The amplified
signal is sent to the Data concentrator where a muliplexed ADC selects
one of 4 possible channels as the data input.  The set of jumpers at J2
on the circuit board allow selection of which analog channel is assigned
to each counter board.

\begin{table}[htb]
\begin{center}
\begin{tabular}{|l|l|}
\hline
\multicolumn{2}{|c|}{Counter Board J1/J2}\\
\hline
pin & assignment\\
\hline
 1 & +5 VDC\\
 2 & +12 VDC\\
 3 & -12 VDC\\
 4 & CLOCK +\\
 5 & ANALOG 1\\
 6 & ANALOG 2\\
 7 & ANALOG 3\\
 8 & ANALOG 4\\
 9 & ANALOG 5\\
10 & ANALOG 6\\
11 & ANALOG 7\\
12 & SHIELD\\
13 & SHIELD\\
14 & DIG GND\\
15 & ANL GND\\
16 & ANL GND\\
17 & CLOCK -\\
18 & \\
19 & \\
20 & \\
21 & \\
22 & \\
23 & \\
24 & \\
25 & \\
\hline
\end{tabular}
\caption[Optical Encoder Counter Board Connectors J1/J2]
        {Optical Encoder Counter Board Connectors J1 and J2.
         Analog Interface Daisy chain connectors.
         Connectors have identical connections for daisy chaining.
         \shwlabel{tabCBj1} 
        }
\end{center}
\end{table}
  
  
\begin{table}[htb]
\begin{center}
\begin{tabular}{|l|l|}
\hline
\multicolumn{2}{|c|}{Counter Board J3/J4}\\
\hline
pin & assignment\\
\hline
1 & D0\\
2 & D2\\
3 & D4\\
4 & D6\\
5 & D8\\
6 & D10\\
7 & D12\\
8 & D14\\
9 & GND\\
10 & A7\\
11 & A5\\
12 & A3\\
13 & A1\\
14 & GND\\
15 & FLAG 0\\
16 & FLAG 2\\
17 & \\
18 & \\
19 & \\
20 & D1\\
21 & D3\\
22 & D5\\
23 & D7\\
24 & D9\\
25 & D11\\
26 & D13\\
27 & D15\\
28 & BUSEN\\
29 & A6\\
30 & A4\\
31 & A2\\
32 & A0\\
33 & GND\\
34 & FLAG 1\\
35 & FLAG 3\\
36 & \\
37 & \\
\hline
\end{tabular}
\caption[Optical Encoder Counter Board Connectors J3/J4]
        {Optical Encoder Counter Board Connectors J3 and J4.
         Digital interface daisychain connectors.
         Connectors have identical connections for daisy chaining.
         \shwlabel{tabCBj3} 
        }
\end{center}
\end{table}

  
\begin{table}[htb]
\begin{center}
\begin{tabular}{|l|l|}
\hline
\multicolumn{2}{|c|}{Counter Board J5}\\
\hline
pin & assignment\\
\hline
1 & GND\\
2 & A input, shaft encoder\\
3 & B input, shaft encoder\\
4 & +5 VDC\\
5 & \\
6 & GND to load cell\\
7 & +8 VDC to load cell\\
8 & neg load cell signal\\
9 & pos load cell signal\\
\hline
\end{tabular}
\caption[Optical Encoder Counter Board Connector J5]
        {Optical Encoder Counter Board Connectors J5.
         Shaft Encoder and Analog channel input connector
         \shwlabel{tabCBj5} 
        }
\end{center}
\end{table}
  
\clearpage
  
%-------------------------------
\newpage
\subsection{Data Concentrator}
  The data concentrator consists of chasis and backplane that hold up to
8 data concentrator cards.  Each concentrator card contains 4 single stage
amplifiers to amplify the 4 allowed analog signals from the counter boards
and a set of buffers to drive the digital signals from the counter boards.
Up to 4 counter boards are connected to the data concentrator card via
2 daisy chain cables.
    
    The signals to/from the data concentrator card go through an interface
card in the data concentrator chasis and connect to the I/O card in
the Manipulator Control Computer PC.
  
%-------------------------------
\newpage
\subsection{Motor Fanout and Watchdog Timer Box}
This box takes fans out the TIO-10 card outputs to the stepper motor
controls and has a watchdog timer circuit to detect a dead computer
and shut off the stepper motors.
The fanout to each motor goes through a 4 pin connector,
\begin{table}[htb]
\begin{center}
\begin{tabular}{|l|l|}
\hline
pin & assignment \\
\hline
1 & +5 VDC\\
2 & PULSE \\
3 & DIR \\
4 & AWO \\
\hline
\end{tabular}
\caption[Motor connections on Motor Fanout Box]
        {Motor Connections on Motor Fanout Box
         \shwlabel{tabFOmotor}
        }
\end{center}
\end{table}
The pinouts to a motor are shown in table \ref{tabFOmotor} where,
\begin{description}
\item[PULSE] a pulse to step the motor one step
\item[DIR] sets direction of motion
\item[AWO] All Windings Off --- warning, this means no holding torque.
\end{description}  
The TIO-10 signals used to control each motor channel are listed in
table \ref{tabFOtio10} along with the pins on each connector associated
with the signal.

\begin{table}[htb]
\begin{center}
\begin{tabular}{|l|l|l|l|l|}
\hline
motor & motor pin & motor function& TIO-10 pin & TIO-10 funtion \\
\hline
1 & J2-1 & +5 VDC         &       & \\
  & J2-2 & PULSE          & J1-3  & OUT1\\
  & J2-3 & DIR            & J1-35 & A0\\
  & J2-4 & AWO            & J1-36 & A1\\
\hline
2 & J3-1 & +5 VDC         &       & \\
  & J3-2 & PULSE          & J1-6  & OUT2\\
  & J3-3 & DIR            & J1-37 & A2\\
  & J3-4 & AWO            & J1-38 & A3\\
\hline
3 & J4-1 & +5 VDC         &       & \\
  & J4-2 & PULSE          & J1-9  & OUT3\\
  & J4-3 & DIR            & J1-41 & A6\\
  & J4-4 & AWO            & J1-42 & A7\\
\hline
4 & J5-1 & +5 VDC         &       & \\
  & J5-2 & PULSE          & J1-17 & OUT6\\
  & J5-3 & DIR            & J1-39 & A4\\
  & J5-4 & AWO            & J1-40 & A5\\
\hline
5 & J6-1 & +5 VDC         &       & \\
  & J6-2 & PULSE          & J1-20 & OUT7\\
  & J6-3 & DIR            & J1-45 & B2\\
  & J6-4 & AWO            & J1-46 & B3\\
\hline
6 & J7-1 & +5 VDC         &       & \\
  & J7-2 & PULSE          & J1-23 & OUT8\\
  & J7-3 & DIR            & J1-47 & B4\\
  & J7-4 & AWO            & J1-48 & B5\\
\hline
7 & J8-1 & +5 VDC         &       & \\
  & J8-2 & PULSE          & J1-26 & OUT9\\
  & J8-3 & DIR            & J1-43 & B0\\
  & J8-4 & AWO            & J1-44 & B1\\
\hline
8 & J9-1 & +5 VDC         &       & \\
  & J9-2 & PULSE          & J1-28 & OUT10\\
  & J9-3 & DIR            & J1-49 & B6\\
  & J9-4 & AWO            & J1-50 & B7\\
\hline
\end{tabular}
\caption[TIO-10 signals used for motor control]
        {TIO-10 signals used for motor control
         \shwlabel{tabFOtio10}
        }
\end{center}
\end{table}
  
\clearpage
 
%-------------------------------
\newpage
\subsection{Stepping Motors and Stepping Motor Controllers}
The stepping motor controllers used for the manipulator are Superior Electric
SLO-SYN Model SS2000MD4 Tranlator/Drive which is ``a bipolar, adjustable
speed, two-phase PWM drive which uses hybrid power devices''.
\begin{table}[htb]
\begin{center}
\begin{tabular}{l|l} 
pin & Assignment \\ \hline
1  &  OPTO \\
2  & PULSE \\
3  & DIR \\
4  & AWO \\
\end{tabular}
\caption[Stepping Motor Controller IO terminal block]
        {Stepping Motor Controller IO terminal block
         \shwlabel{tabSMIO}
        }
\end{center}
\end{table}  
The IO connections are shown in table \ref{tabSMIO}
where the signals are:
\begin{description}
\item[OPTO] Opto-Isolator Supply\\
    User supplied power for the opto-isolators.
\item[PULSE] Pulse input\\
    A low to high transition on this terminal advances the motor one
    step.  The step size is determined by the Step Resolution switch
    settings.
\item[DIR] Direction Input\\
    When this signal is high, motor rotation will be clockwise.  Rotation
    will be counterclockwise when this signal is low.
    Clockwise and counterclockwise directions are properly oriented
    when viewing the motor from the end opposite the mounting flange.
\item[AWO] All windings Off Input\\
  When this signal is low, AC and DC current to the motor will be
  zero.  {\bf Caution:  There will be no holding torque when the AWO signal
  is low.}
\end{description}
\begin{tabbing}
\=aaaaa\=aa\=aaaa\=aa\kill
\>OPTO \\
\>  \>Voltage --- 4.5 to 6.0 VDC\\
\>  \>Current --- 16 mA per signal used\\
\>Other Signals\\
\>  \>Votage\\
\>  \>   \>Low:  \>$\leq$0.8 VDC\\
\>  \>   \>      \>$\geq$0.0 VDC\\
\>  \>   \>High: \>$\leq$0.8 VDC\\
\>  \>   \>      \> $\geq$0.0 VDC\\
\>  \>Current\\
\>  \>   \>Low:  \>$\leq$16 mA\\
\>  \>   \>High: \>$\leq$0.2 mA\\

\end{tabbing}
The Motor and power supply connections are given in table\ref{tabSMps}
\begin{table}[htb]
\begin{center}
\begin{tabular}{l|l} 
pin & Assignment \\ \hline
1  & M1 (Phase A+) \\
2  & M3 (Phase A-) \\
3  & M4 (Phase B+) \\
4  & M5 (Phase B-) \\
5  & Vm(+)\\
6  & Vom(-)\\
\end{tabular}
\caption[Stepping Motor Controller Motor and Power Supply terminal block]
        {Stepping Motor Controller Motor and Power Supply terminal block
         \shwlabel{tabSMps}
        }
\end{center}
\end{table}
where Vm(+) is a 24VDC unregulated supply voltage and Vom(-) is it's return.
The motor current is determined by the dip switch 1-7 settings shown
in table \ref{tabSMdip}
\begin{table}[htb]
\begin{center}
\begin{tabular}{c|l} 
switch & current \\ \hline
none & 0.5 A \\
1  & 0.75 A \\
2  & 1.0 A \\
3  & 1.5 A \\
4  & 2.0 A \\
5  & 2.5 A \\
6  & 3.0 A \\
7  & 3.5 A \\
\end{tabular}
\caption[Stepping Motor Controller current settings]
        {Stepping Motor Controller Motor current settings.  Only one
         switch should be set at a time.
         \shwlabel{tabSMdip}
        }
\end{center}
\end{table}
The number of pulses per revolution is selected using dip switch 8.  The
settings are given in table \ref{tabSMstep}
\begin{table}[htb]
\begin{center}
\begin{tabular}{c|l|l} 
switch position 8 & Step Resolution & Pulses per Rev \\ \hline
0 (off) & Full-Step & 200 \\
1 (on)  & Half-Step & 400 \\
\end{tabular}
\caption[Stepping Motor Controller step size]
        {Stepping Motor Controller Motor step size
         \shwlabel{tabSMstep}
        }
\end{center}
\end{table}
 
%-----------------------------------------------------
\clearpage 
\newpage
\subsubsection{Stepping Motor Connections}
  
\begin{table}
\begin{center}
\begin{tabular}{|l|l|l|l|}
\hline
motor type &  wire  & drive pin & function \\
\hline
4-lead     &  red         & 1 & phase A\\
           &  white/red   & 2 & phase A\\
           &  black       & 3 & phase B\\
           &  white/black & 4 & phase B\\
\hline
\end{tabular}
\caption[Stepping Motor Connections]
        {Stepping Motor Connections
         \shwlabel{tabSMconnections}
        }
\end{center}
\end{table}
  
%---------------------------------------------------  
\clearpage
\newpage
\subsection{PC-TIO-10 Card}
The National Instruments PC-TIO-10 Card is located in the calibration 
computer and is used to control the manipulator stepping motors (it
is also used to control stepping motors on the N$_2$ laser used for the
laserball).
\begin{table}
\begin{center}
\begin{tabular}{|r|l|}
\hline
pin & signal\\ \hline
1  & SOURCE1\\
2  & GATE1  \\
3  & OUT1   \\
4  & SOURCE2\\
5  & GATE2  \\
6  & OUT2   \\
7  & SOURCE3\\
8  & GATE3  \\
9  & OUT3   \\
10 & SOURC4 \\
11 & GATE4  \\
12 & OUT4   \\
13 & GATE5  \\
14 & OUT5   \\
15 & SOURCE6\\  
16 & GATE6  \\
17 & OUT6   \\
18 & SOURC7 \\
19 & GATE7  \\ 
20 & OUT7   \\
21 & SOURCE8\\
22 & GATE8  \\
23 & OUT8   \\
24 & SOURCE9\\
25 & GATE9  \\
\hline
\end{tabular}
\caption[PC-TIO-10 card connector pin assignments (pins 1-25)]
        {PC-TIO-10 card connector pin assignments (pins 1-25)
         \shwlabel{tabTIO10pins}
        }
\end{center}
\end{table}
  
  
\begin{table}
\begin{center}
\begin{tabular}{|r|l|}
\hline
pin & signal\\ \hline
26 & OUT9 \\ 
27 & GATE10 \\
28 & OUT10\\
29 & FOUT1 \\
30 & FOUT2\\
31 & EXTIRQ1\\
32 & EXTIRQ2\\
33 & GND\\
34 & +5V\\
35 & A0\\
36 & A1\\
37 & A2\\
38 & A3\\
39 & A4\\
40 & A5\\
41 & A6\\
42 & A7\\
43 & B0\\
44 & B1\\
45 & B2\\
46 & B3\\
47 & B4\\
48 & B5\\
49 & B6\\
50 & B7\\
\hline
\end{tabular}
\caption[PC-TIO-10 card connector pin assignments cont'd (pins 26-50)]
        {PC-TIO-10 card connector pin assignments cont'd (pins 26-50)
         For the 8 IO lines A0-A7, the MSB is A7.  for the 8 IO lines
         B0-B7, the MSB is B7.
         \shwlabel{tabTIO10pinscontd}
        }
\end{center}
\end{table}
  
%-------------------------------------
\newpage
\subsection{Optical Encoders}
The optical encoders used to measure the rope lengths in the system
are made by Bourns.  Part number?  We use both the 128 steps per
rotation and the 256 steps per rotation encoders.  No index marks are
on the encoders so no absolute zero can be determined.  The encoder
is driven by 5VDC and and supplies two TTL level quadrature signals,
A and B.  The pin assignment for the encoders is shown in 
table \ref{tabOEpins}.
\begin{table}[htb]
\begin{center}
\begin{tabular}{l|l}
pin & assignment\\
\hline
1 & GND\\
2 & N.C.\\
3 & A output \\
4 & VCC \\
6 & B output \\
\end{tabular}
\caption[Bourns Optical Shaft Encoders Pin assignment]
        {Bourns Optical Shaft Encoders Pin assignment.
         Pins numbered from left to right with the encoder
         shaft oriented away from viewer and pins at top.
         \shwlabel{tabOEpins}
        }
\end{center}
\end{table}
The shaft encoders are read out by the {\em counter boards} which
were designed inhouse at Queen's.  
   
  
%-------------------------------------------------------------------------
\clearpage
\newpage
\subsection{Load Cells}
The load cells used in the manipulator are made by,
\begin{verbatim}
  Tranducer Techniques
\end{verbatim}
and are model MLP-100 (100 lb loads)
Maximum allowed excitation is 12 VDC, calibration is done at 10 VDC.
The connections for the load cells are shown in table \ref{tabLCconnections}.
\begin{table}[htb]
\begin{center}
\begin{tabular}{l|l}
wire & function\\ \hline
red  & + excitation\\
black & - excitation\\
green & + signal\\
white & - signal\\
shield & ground\\
\end{tabular}
\caption[Load Cells]
        {Load Cell Connections
         \shwlabel{tabLCconnections}
        }
\end{center}
\end{table}
We drive the load cells at 8 VDC.  The reponse of the load cells is,
\begin{center}
\begin{tabular}{lll}
type        &  load   & reponse \\
            &  (lb)   & mV/V    \\
\hline
100 lb cell & 50    & 1 \\
(MLP-100) & 100   & 2 \\
\hline
300 lb cell & 150 & 1 \\
(MLP-300)   & 300 & 2 \\
\hline
\end{tabular}
\end{center}
where the reponse in mV/V means the output voltage (in mV) for a given
driving voltage (in V).
For an 8 volt driving voltage as used in the SNO manipulator, the
output reponse is,
\begin{center}
\begin{tabular}{|rr|rr|}
\hline
\multicolumn{2}{|c|}{MLP-100} &\multicolumn{2}{|c|}{MLP-300} \\ 
\hline
load & signal &  load & signal \\
(N)  & (mV)   &  (N)  & (mV)  \\
\hline
 1   &  0.036 &    1  &  0.012 \\
 5   &  0.180 &    5  &  0.060 \\
10   &  0.360 &   10  &  0.120 \\
50   &  1.780 &   50  &  0.599 \\
100  &  3.596 &  100  &  1.199 \\
500  & 17.978 &  500  &  5.993 \\
     &        & 1000  & 11.985 \\
\hline
\end{tabular}
\end{center}
%-------------------------------------------------------------------------
%-------------------------------------------------------------------------
%-------------------------------------------------------------------------
  
\section{Software}
  
  
\subsection{Program Description}
\subsection{Data files}
\begin{verbatim}
On PC
-----
c:\motors\manip\
  
   wiring.dat  -- TIO 10 wiring map
                   counter board wiring map
                   motor fanout wiring map
  
       -- both the manipulator and the AV position sensors
  
   motor.dat    -- physical parameters for motors
  
   encoder.dat  -- physical parameters for encoders
  
   loadcell.dat -- physical parameters for load cells
  
   axis.dat     -- combines motor loadcell and encoder infor plus other
                   stuff (i.e. wire tension etc) to form info on axis
  
   polyaxis.dat -- combines 3 axes into the manipulator
  
   av.dat       -- information on acrylic vesel geometry

\end{verbatim}

  



